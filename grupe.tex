\documentclass[11pt]{book}
\usepackage[slovene]{babel}
\usepackage[utf8]{inputenc}
\usepackage{amsmath}
\usepackage[dvipsnames]{xcolor}
\usepackage{titlesec}
% \usepackage{amssymb}
\usepackage{pstricks,pst-plot,pst-math}
\usepackage{pstricks-add}
\usepackage{graphicx}
\usepackage{enumerate}
\usepackage{color}
\usepackage{fouriernc}
\usepackage{microtype}
\usepackage{MnSymbol}
\usepackage{tikz-cd}
\usetikzlibrary{backgrounds}
\usepackage{wrapfig}
\usepackage{geometry}
\geometry{
    a4paper,
    left=45mm,
    right=45mm,
    top=20mm,
    bottom=20mm
    }
    \usepackage{comment}

\usepackage{amsthm}
\renewcommand{\llangle}{\langle\langle}
\renewcommand{\rrangle}{\rangle\rangle}
\renewcommand{\wreath}{\; wr. \;}


\usetikzlibrary{calc,math}

\usepackage{changepage}   % for the adjustwidth environment
\usepackage{hyperref}
\hypersetup{
    colorlinks=true,
    linkcolor=cyan,
    filecolor=magenta,      
    urlcolor=cyan
    }

\usepackage[backgroundcolor=svetlosiva,linecolor=siva,textsize=footnotesize]{todonotes}
%\usepackage{todonotes}

\pagestyle{plain}
% \counterwithout{footnote}{chapter} % continuous counter numbering

\usepackage{enumitem}
\setlist[description]{leftmargin=\parindent,labelindent=\parindent, font=\normalfont\itshape\textbullet\space}


\def\NN{\mathbb{N}}
\def\ZZ{\mathbb{Z}}
\def\QQ{\mathbb{Q}}
\def\RR{\mathbb{R}}
\def\CC{\mathbb{C}}
\def\one{\mathbf{1}}
\def\two{\mathbf{2}}
\def\three{\mathbf{3}}
\def\four{\mathbf{4}}
\def\five{\mathbf{5}}
\def\six{\mathbf{6}}
\def\seven{\mathbf{7}}
\def\FF{\mathcal{F}}
\def\field{\mathbb{F}}
\def\GG{\mathcal{G}}
\def\PP{\mathcal{P}}
\def\UU{\mathcal{U}}
\def\ZR{\mathsf{ZZR}}
\DeclareMathOperator\image{im}
\DeclareMathOperator\Spec{Spec}
\def\prenos{\pi}
\def\sestavi{\sigma}
\def\razstavi{\rho}
\DeclareMathOperator\Aut{Aut}
\DeclareMathOperator\GL{GL}
\DeclareMathOperator\SL{SL}
\DeclareMathOperator\PSL{PSL}
\DeclareMathOperator\Gal{Gal}
\DeclareMathOperator\Sym{Sym}
\DeclareMathOperator\Homeo{Homeo}
\DeclareMathOperator\Cay{Cay}
\DeclareMathOperator\Isom{Isom}
\DeclareMathOperator\id{id}
\DeclareMathOperator\supp{supp}
\DeclareMathOperator\End{End}
\DeclareMathOperator\Mat{Mat}
\DeclareMathOperator\Cone{Cone}
\DeclareMathOperator\diam{diam}

\def\definicija{\color{rdeca}\bf\em}
\def\vprasanje{\color{oranzna}}
\def\literatura{\color{modra}}
\def\vaje{{\literatura (glej vaje)}}
% \def\kljuka{$\checkmark\kern-0.7em\checkmark$}
\def\kljuka{$\checkmark$}

\theoremstyle{definition}

\newtheoremstyle{zgled}
 {}{}%
 {\color{zelena}}
 {}%
 {\color{zelena}\bfseries}%
 {\color{zelena}.}%
 { }{}

\theoremstyle{zgled}
\newtheorem*{zgled}{Zgled}

\newtheoremstyle{odprtproblem}
 {}{}%
 {\color{oranzna}}
 {}%
 {\color{oranzna}\bfseries}%
 {\color{oranzna}.}%
 { }{}

\theoremstyle{odprtproblem}
\newtheorem*{odprtproblem}{Odprt problem}

\newtheoremstyle{domacanaloga}
 {}{}%
 {\color{vijolicna}}
 {}%
 {\color{vijolicna}\bfseries}%
 {\color{vijolicna}.}%
 { }{}

\theoremstyle{domacanaloga}
\newtheorem*{domacanaloga}{Domača naloga}

\newenvironment{dokaz}
    {\color{siva}\begin{proof}}
    {\end{proof}}

% \newenvironment{imenovandokaz}[1]
%     {\smallskip \color{siva} \noindent {\bf {#1}.} 
%     }
%     {\hfill$\square$\smallskip
%     }

\newtheoremstyle{izrek}
 {}{}% above, below 
 {\color{black}\itshape}
 {}% indent
 {\color{black}\bfseries}%
 {\color{black}.}%
 { }{}

\theoremstyle{izrek}
\newtheorem*{izrek}{Izrek}

% \newenvironment{izrek}
%     {\smallskip\begin{center}\noindent\begin{tabular}{|p{0.9\textwidth}|}
%     \hline{\noindent\bf Izrek.}
%     }
%     { 
%     \\\hline
%     \end{tabular}\end{center}\smallskip
%     }

\newtheorem*{trditev}{Trditev}
\newtheorem*{pomoznatrditev}{Pomožna trditev}

% \newenvironment{trditev}
%     {\smallskip\begin{center}\noindent\begin{tabular}{|p{0.9\textwidth}|}
%     \hline{\noindent\bf Trditev.}
%     }
%     { 
%     \\\hline
%     \end{tabular}\end{center}\smallskip
%     }

\newtheorem*{lema}{Lema}

% \newenvironment{lema}
%     {\smallskip\begin{center}\noindent\begin{tabular}{|p{0.9\textwidth}|}
%     \hline{\noindent\bf Lema.}
%     }
%     { 
%     \\\hline
%     \end{tabular}\end{center}\smallskip
%     }

\newtheorem*{posledica}{Posledica}

% \newenvironment{posledica}
%     {\smallskip\begin{center}\noindent\begin{tabular}{|p{0.9\textwidth}|}
%     \hline{\noindent\bf Posledica.}
%     }
%     { 
%     \\\hline
%     \end{tabular}\end{center}\smallskip
%     }

\newenvironment{povzetek}
    {
\smallskip
\begin{center}
\color{svetlosiva}
\begin{tabular}{|p{0.7\textwidth}}
    }
    {
\end{tabular}
\end{center}
\smallskip
    }


\definecolor{rdeca}{rgb}{0.62, 0.16, 0.10}
\definecolor{zelena}{rgb}{0.15, 0.4, 0.20}
\definecolor{oranzna}{rgb}{0.72, 0.38, 0.082}
\definecolor{rjava}{rgb}{0.7490196078431373, 0.3686274509803922, 0.1843137254901961}
\definecolor{modra}{rgb}{0.2784313725490196, 0.5411764705882353, 0.8392156862745098}
\definecolor{vijolicna}{rgb}{0.48627450980392156, 0.2980392156862745, 0.792156862745098}
\definecolor{siva}{rgb}{0.5, 0.5, 0.5}
\definecolor{svetlosiva}{rgb}{0.7, 0.7, 0.7}


\titleformat{\section}
  {\color{rdeca}\LARGE\bf}{\thesection}{1em}{}
\renewcommand{\thesubsection}{}
\titleformat{\subsection}
  {\Large\bf}{}{1em}{}

\title{\bf Teorija grup}
\author{Urban Jezernik}

% za generiranje html dokumenta s stilom mystyle.css uporabi:
% pandoc grupe.tex --toc --toc-depth=2 --metadata date="`date -u "+%d. %m. %Y"`" --template template.html -c mystyle.css -s --mathjax -o index.html


\begin{document}

\baselineskip=14pt

\maketitle

\setcounter{tocdepth}{1}
\tableofcontents

\newpage

\subsection*{Kratek opis predmeta}

Grupe so {\em osrednji objekt abstraktne algebre}. Njihova ključna lastnost je, da v njih lahko {\em množimo}, hkrati pa operacija množenja zadošča ravno dovolj aksiomom, da je mogoče o grupah povedati veliko s {\em strukturnega stališča}. Pri tem predmetu bomo govorili o različnih strukturnih aspektih grup, pri čemer nas bo vodila {\em rast}.

Za dano grupo $G = \langle S \rangle$, generirano z množico $S \subseteq G$, je njena {\em funkcija rasti} enaka
\[
\texttt{rast} \colon \NN \to \NN, \quad n \mapsto |\{ s_1 s_2 \cdots s_n \mid s_i \in S \cup S^{-1} \cup \{ 1 \} \}|,
\]
in sicer izračuna število elementov grupe $G$, ki jih dobimo z množenjem največ $n$ elementov množice $S$ in njihovih inverzov.

\begin{zgled}
\begin{itemize}
\item Naj bo $\ZZ = \langle 1 \rangle$. V tem primeru dobimo
\[
\texttt{rast}(n) = |\{ -n, -n+1, \dots, n \}| = 2n+1.
\]
Rast je torej {\vprasanje linearna}.

\item Naj bo $\ZZ^2 = \langle (1,0), (0,1) \rangle$. V tem primeru dobimo
\[
\texttt{rast}(n) = |\{ (x,y) \in \ZZ^2 \mid |x| + |y| \leq n \}| = 2 n^2 + 2 n + 1.
\]
Rast je torej {\vprasanje kvadratna}.

\item Naj bo $F_2 = \langle x,y \rangle$ prosta grupa.\footnote{Vse osnovne definicije v zvezi z grupami bomo ponovili, tudi konstrukcijo proste grupe.} V tem primeru dobimo
\begin{align*}
\texttt{rast}(n) &= \text{število besed v $x,y,x^{-1},y^{-1}$ dolžine največ $n$} \\ 
&= 1 + 4 + 4 \cdot 3 + 4 \cdot 3 \cdot 3 + \cdots + 4 \cdot 3^{n-1} \\
&= 2 \cdot 3^n - 1.
\end{align*}
Rast je torej {\vprasanje eksponentna}.

\end{itemize}
\end{zgled}

Pri predmetu bomo raziskovali {\em interakcijo} med asimptotskim obnašanjem rasti grup in njihovo algebraično strukturo. Podrobneje si bomo pogledali naslednje teme:

\begin{itemize}
\item {\em osnovni pojmi teorije grup}, končno prezentirane grupe;
\item {\em delovanja}, Cayleyjev graf, prosta delovanja na drevesih;
\item {\em struktura grup}, rešljive in nilpotentne grupe, razširitve;
\item {\em rast}, tipi rasti, izračunljivost;
\item {\em polinomska rast}, izrek Gromova;
\item {\em srednja rast}, Grigorčukova grupa, amenabilnost;
\item {\em eksponentna rast}, enakomernost, Breuillardova domneva.
\end{itemize}

\subsection*{Literatura}

Osnovne pojme teorije grup in geometrijska delovanja črpamo iz Löhine knjige, strukturo grup pa iz Mannove. Teorijo razširitev povzamemo iz Robinsonove monografije. Pri rasti se zopet naslonimo na Mannovo knjigo, s katero pokrijemo osnove in raziščemo grupe polinomske rasti. Pri dokazu izreka Gromova neenakost med topološko in Hausdorffovo dimenzijo pridobimo iz monografije Hurewicza in Wallmana. Obravnavo grup srednje rasti naslonimo na ekspozicijo Grugorčuka in Paka, povezavo z amenabilnostjo prikažemo kot v Mannovi knjigi, paradoksalne dekompozicije pa kot v Löhini. Pri grupah eksponentne rasti predstavimo zgled iz Osinovega članka in Bartholdijevo grupo, redukcijo hitro rastočih množic iz neskončnih v končne grupe pa prikažemo kot Breuillard, Green in Tao.

\begin{itemize}
\item {\literatura C. Löh, {\em Geometric Group Theory -- An Introduction}, Springer, 2017.} 
\item {\literatura A. Mann, {\em How Groups Grow}, London Mathematical Society Lecture Note Series 395, 2012.}
\item {\literatura D. J. S. Robinson, {\em A Course in the Theory of Groups}, Springer, 1995.}
\item {\literatura W. Hurewicz in H. Wallman, {\em Dimension Theory}, Princeton University Press, 1941.}
\item {\literatura R. Grigorčuk in I. Pak, {\em Groups of intermediate growth: an introduction for beginners}, 2006.}
\item {\literatura D. V. Osin, {\em The entropy of solvable groups}, Ergodic Theory Dynam. Systems 23 (2003), 907--918.}
\item {\literatura L. Bartholdi, {\em A Wilson group of non-uniformly exponential growth}, Comptes Rendus Mathematique 336.7 (2003), 549--554.}
\item {\literatura E. Breuillard, B. Green, T. Tao, {\em The structure of approximate groups}, Publications mathématiques de l'IHÉS 116 (2012), 115-–221.}
\end{itemize}

\chapter{Osnovni pojmi}

\section{Kategorija grup}

\subsection{Abstraktne grupe}

{\definicija Grupa} je množica $G$ z binarno operacijo $\cdot \colon G \times G \to G$, ki zadošča aksiomom:

\begin{itemize}
	\item asociativnost: $\forall g_1, g_2, g_3 \in G \colon (g_1 g_2) g_3 = g_1 (g_2 g_3)$;
	\item obstoj enote: $\exists 1 \in G \ \forall g \in G \colon 1g = g1 = g$;
	\item obstoj inverzov: $\forall g \in G \ \exists g^{-1} \in G \colon g g^{-1} = g^{-1} g = 1$.
\end{itemize}

Množica $H \subseteq G$ je {\definicija podgrupa}, če je grupa glede na zožitev $\cdot |_{H \times H}$. V tem primeru dobimo množico {\definicija odsekov} $G/H = \{ gH \mid g \in G \}$, kjer je $gH = \{ g \cdot h \mid h \in H \}$. Kardinalnost množice odsekov je {\definicija indeks} $|G:H|$.

Za dani grupi $G,H$ je preslikava $\varphi \colon G \to H$ {\definicija homomorfizem}, če velja $\forall g_1, g_2 \in G \colon \varphi(g_1 g_2) = \varphi(g_1) \varphi(g_2)$. Bijektivni homomorfizem je {\definicija izomorfizem}, v tem primeru pišemo $G \cong H$. {\definicija Jedro} je $\ker \varphi = \{ g \in G \mid \varphi(g) = 1 \}$, {\definicija slika} je $\image \varphi = \{ \varphi(g) \mid g \in G \}$.

\subsection{Konkretne grupe}

Zgornja aksiomatizacija koncepta grupe izhaja iz konkretnih modelov simetrij geometrijskih ali algebraičnih objektov.

\begin{zgled}
Naj bo $X$ množica. {\em Simetrična grupa} je $\Sym(X) = \{ f \colon X \to X \mid f \text{ bijekcija} \}$. Za $X = \{ 1,2,\dots, n\}$ dobimo $\Sym(X) = S_n$.
\end{zgled}

Vsaka abstraktna grupa izhaja iz simetrične grupe.

\begin{trditev}[Cayleyjev izrek]
Vsaka grupa je izomorfna podgrupi neke simetrične grupe.
\end{trditev}

\begin{dokaz}
$G$ je izomorfna podgrupi $\Sym(G)$:
\[
f \colon G \to \Sym(G), \quad f(g) \colon x \mapsto g \cdot x \quad (x \in G).
\]
Preslikava $f$ je injektiven homomorfizem.
\end{dokaz}

Nasploh imajo množice preslikav, ki ohranjajo neko strukturo, ponavadi strukturo grupe. V naravi se grupe pojavljajo v taki obliki.

\begin{zgled}
\begin{itemize}
	\item Naj bo $G$ grupa. Priredimo ji njeno {\em grupo avtomorfizmov} $\Aut(G) = \{ f \colon G \to G \mid f \text{ izomorfizem} \}$.

    \item Naj bo $X$ metrični prostor. Priredimo mu {\em grupo izometrij\footnote{{\definicija Izometrija} $f \colon X \to X$ je bijektivna preslikava, za katero velja $\forall x,y \in X \colon d(x,y) = d(f(x), f(y))$.}} $\Isom(X) = \{ f \colon X \to X \mid f \text{ izometrija}\}$.

    \item Naj bo $V$ vektorski prostor nad poljem $k$. Priredimo mu {\em matrično grupo} $\Aut(V) = \{ f \colon V \to V \mid f \text{ linearen izomorfizem} \} = \GL(V)$.

    \item Naj bo $L/K$ Galoisjeva razširitev polj. Priredimo ji {\em Galoisjevo grupo} $\Gal(L/K) = \{ \sigma \in \Aut(L) \mid \sigma |_K = \id_K \}$.
\end{itemize}
\end{zgled}

Obstajajo pa tudi druge pojavitve grup v matematiki.

\begin{zgled}
Topološkemu prostoru $X$ priredimo {\em fundamentalno grupo} $\pi_1(X)$ in {\em homološke grupe} $H_i(X)$.
\end{zgled}

\subsection{Edinke in kvocienti}

Naj bo $G$ grupa. Podgrupa $N \leq G$ je {\definicija edinka}, če je invariantna za konjugiranje:
\[
\forall n \in N \ \forall g \in G \colon g n g^{-1} \in N.
\]
Pišemo $N \unlhd G$. V tem primeru ima množica odsekov $G/N$ strukturo grupe: $g_1 N \cdot g_2 N = g_1g_2 N$. To je {\definicija kvocientna grupa}. {\definicija Kvocientna projekcija} je homomorfizem
\[
\pi \colon G \to G/N, \quad g \mapsto gN \qquad (g \in G).
\]

\begin{zgled}
Grupa celih števil $\ZZ$ za seštevanje je edinka v grupi realnih števil $\RR$ za seštevanje. Kvocientna grupa je $\RR/\ZZ \cong S^1 = \{ z \in \CC \mid |z| = 1 \}$, izomorfizem podaja preslikava $r + \ZZ \mapsto e^{2 \pi i r}$.
\end{zgled}

\section{Generatorji in relacije}

Kadar želimo najti kakšno grupo {\em s posebnimi algebraičnimi lastnostmi}, je včasih težko najti objekt, ki nam bo prek avtomorfizmov dal želeno grupo. V tem primeru velikokrat uporabimo abstraktno konstrukcijo grupe z {\em generatorji in relacijami}.

\subsection{Generatorji grup}

Naj bo $G$ grupa in $S \subseteq G$ njena podmnožica. {\definicija Podgrupa, generirana s $S$}, je najmanjša podgrupa grupe $G$, ki vsebuje $S$. Oznaka: $\langle S \rangle$. Velja 
\[
\langle S \rangle = \{ s_1^{\epsilon_1} \cdots s_n^{\epsilon_n} \mid n \in \NN, \ s_i \in S, \ \epsilon_i \in \{ 1, -1 \} \}.
\]
Rečemo, da {\definicija $S$ generira $G$}, če velja $\langle S \rangle = G$. Če $G$ vsebuje kakšno {\em končno} podmnožico $S$, ki generira $G$, je $G$ {\definicija končno generirana} grupa.

\begin{zgled}
$\ZZ = \langle 1 \rangle = \langle 2,3 \rangle > \langle 2 \rangle = 2 \cdot \ZZ$.
\end{zgled}

\subsection{Proste grupe}

Grupa $F$ je {\definicija prosto generirana} s $S \subseteq F$, če ima naslednjo {\em univerzalno lastnost}: za vsako grupo $H$ in vsako preslikavo $\varphi \colon S \to H$ obstaja natanko en homomorfizem $\bar \varphi \colon F \to H$, ki razširja $\varphi$, to je $\bar \varphi(s) = \varphi(s)$ za vsak $s \in S$.

% \[
%     \begin{tikzcd}
%     S \arrow{r}{\varphi} \arrow[d, hookrightarrow] & H \\
%     F \arrow[ur, dashrightarrow]
%     \end{tikzcd}
% \]

Grupa je {\definicija prosta}, če je prosto generirana s kakšno podmnožico.

\begin{zgled}
\begin{itemize}
    \item $\ZZ$ je prosto generirana z $\{ 1 \}$, ne pa z $\{ 2, 3 \}$.
    \item $\ZZ/2\ZZ$ {\em ni} prosta.
\end{itemize}
\end{zgled}

Iz vsake množice $S$ lahko prosto generiramo grupo na naslednji način.

\begin{izrek}
Naj bo $S$ množica. Tedaj obstaja grupa, prosto generirana z $S$. Ta grupa je enolična do izomorfizma natančno.
\end{izrek}

\begin{dokaz}
Dokažimo le obstoj, enoličnost sledi iz univerzalne lastnosti prosto generiranih grup.

Vpeljimo abecedo $A = S \cup \hat S$, kjer je $\hat S = \{ \hat s \mid s \in S \}$ disjunktna kopija množice $S$ (njeni elementi igrajo vlogo $s^{-1}$ za $s \in S$).

Naj bo $A^*$ množica vseh zaporedij ({\em besed}) simbolov iz $A$. Ta vsebuje prazno besedo, ki jo označimo z $\epsilon$. Besede lahko množimo s konkatenacijo: $x \cdot y = xy$ za $x,y \in A^*$.

Uvedimo ekvivalenčno relacijo $\sim$ na $A^*$ ({\em krajšanje}): za vse $x,y \in A^*$ in $s \in S$ zahtevajmo $x s \hat s y \sim xy$ in $x \hat s s y \sim xy$.

Naj bo $F(S) = A^*/\sim$. Konkatenacija inducira množenje na $F(S)$: $[x] \cdot [y] = [xy]$.

\underline{$F(S)$ je grupa:} Enota je $[\epsilon]$, inverze izračunamo induktivno z $[s \cdot x]^{-1} = [x]^{-1} \cdot [\hat s]$ in $[\hat s \cdot x]^{-1} = [x]^{-1} \cdot [s]$ za $s \in S, x \in A^*$. \kljuka

\underline{$F(S)$ je prosto generirana s $S$:} Imamo naravno preslikavo $S \to F(S), s \mapsto [s]$. Naj bo $\varphi \colon S \to G$ neka preslikava za neko grupo $G$. Potem $\varphi$ lahko razširimo do homomorfizma $\bar \varphi \colon F(S) \to G$ induktivno: $\bar \varphi([s \cdot x]) = \varphi(s) \cdot \bar \varphi([x])$, $\bar \varphi([\hat s \cdot x]) = \varphi(s)^{-1} \cdot \bar \varphi([x])$ za $s \in S, x \in A^*$. \kljuka
\end{dokaz}

Elementi grupe $F(S)$ so torej {\em okrajšane besede} s črkami iz $S \cup S^{-1}$, množimo pa jih s konkatenacijo. Tukaj uporabljamo oznako $S^{-1} = \{ s^{-1} \mid s \in S \}$.

{\definicija Prosta grupa ranga $n$} je $F_n = F(\{ 1, \dots, n \})$. Enolična je do izomorfizma natančno. 

Grupa je končno generirana, če in samo če je kvocient proste grupe ranga $n$ za nek $n$ (sledi iz univerzalne lastnosti).

\subsection{Generatorji in relacije} 

Naj bo $G$ grupa in $S \subseteq G$. {\definicija Edinka, generirana s $S$}, je najmanjša edinka v $G$, ki vsebuje $S$. Oznaka: $\llangle S \rrangle$. Velja:
\[
\llangle S \rrangle = \left\{ g_1 s_1^{\epsilon_1} g_1^{-1} g_2 s_2^{\epsilon_2} g_2^{-1} \cdots g_n s_n^{\epsilon_n} g_n^{-1} \mid n \in \NN, \ g_i \in G, \ s_i \in S, \ \epsilon_i \in \{ 1, -1 \} \right\}.
\]

Naj bo $F(S)$ prosto generirana s $S$. Naj bo $R \subseteq F(S)$ neka podmnožica. Naj bo $\langle S \mid R \rangle = F(S)/\llangle R \rrangle$, to je {\definicija grupa, generirana s $S$ z relacijami $R$}. Če za neko grupo $G$ velja $G \cong \langle S \mid R \rangle$, je par $(S,R)$ {\definicija prezentacija grupe $G$ z generatorji $S$ in relacijami $R$}.

Grupa $\langle S \mid R \rangle$ ima naslednjo {\em univerzalno lastnost}: za vsako grupo $H$ in vsako preslikavo $\varphi \colon S \to H$, katere enolična razširitev $\varphi^* \colon (S \cup S^{-1})^* \to H$ zadošča $\varphi^*(r) = 1$ za vsak $r \in R$, obstaja natanko en homomorfizem $\bar \varphi \colon \langle S \mid R \to H$, ki razširja $\varphi$, to je $\bar \varphi(s) = \varphi(s)$ za vsak $s \in S$. 

% \[
%     \begin{tikzcd}
%     S \arrow{r}{\varphi} \arrow[d, hookrightarrow] & H \\
%     F(S) \arrow{d} & \\
%     \langle S \mid R \rangle = F(S) /  \llangle R \rrangle  \arrow[uur, dashrightarrow]
%     \end{tikzcd}
% \]

Ta lastnost sledi neposredno iz univerzalne lastnosti proste grupe in univerzalne lastnosti kvocientne projekcije $\pi \colon F(S) \to \langle S \mid R \rangle$.

\begin{zgled}
\begin{itemize}
    \item $\ZZ/n\ZZ \cong \langle x \mid x^n \rangle$. Izomorfizem izhaja iz $F(\{ x \}) \cong \ZZ$, $\llangle \{ x^n \} \rrangle \cong \langle n \rangle \leq \ZZ$.

    \item Naj bo $X_n$ pravilni $n$-kotnik v ravnini. Grupo izometrij $Isom(X_n)$ imenujemo {\definicija diedrska grupa} in označimo z $D_{2n}$. Trdimo, da velja:
    \[
    D_{2n} \cong \langle s, t \mid s^n, t^2, t s t^{-1} = s^{-1} \rangle.
    \]
    Tukaj zlorabljamo oznako, $S = \{ s, t \}$, $R = \{ s^n, t^2, t s t^{-1} \cdot s \}$. 

    Označimo $G = \langle S \mid R \rangle$. Naj bo $\sigma \in D_{2n}$ rotacija za kot $2 \pi / n$ okoli središča $X_n$, in naj bo $\tau \in D_{2n}$ zrcaljenje $X_n$ preko enega vozlišča. Velja $\sigma^n = \tau^2 = 1$ in $\tau \sigma \tau^{-1} = \sigma^{-1}$. Iz univerzalne lastnosti $G$ dobimo homomorfizem $\bar \varphi \colon G \to D_{2n}$, ki slika $s \mapsto \sigma, t \mapsto \tau$.

    Velja $D_{2n} = \{ \sigma^i \mid 0 \leq i \leq n-1 \} \cup  \{ \tau \sigma^i \mid 0 \leq i \leq n-1 \}$, torej je $|D_{2n}| = 2n$.

    Definirajmo preslikavo $\psi \colon D_{2n} \to G$, ki preslika $\sigma^i \mapsto s^i, \tau \sigma^i \mapsto t s^i$. Preprost račun pokaže, da je $\psi$ homomorfizem, ki je inverzen $\bar \varphi$. Res torej velja $D_{2n} \cong G$.

    \item Kateri znani grupi je izomorfna grupa 
    \[
    G = \langle x, y \mid x y x^{-1} = y^2, \ y x y^{-1} = x^2 \rangle?
    \]
    V grupi $G$ velja račun: $x y x^{-1} = y^2$, zato $x \cdot x y x^{-1} = x y^2$, od koder sledi $y x y^{-1} \cdot y x^{-1} = x y^2$, to pa implicira $y = x y^2$, zato je nujno $x = y^{-1}$. Iz prve relacije v definiciji $G$ zdaj sledi $y = 1$, s tem pa tudi $x = 1$. Ker $\{ x, y \}$ generira $G$, sklepamo $G \cong 1$, torej je $G$ trivialna grupa.
\end{itemize}
\end{zgled}

{\vprasanje Ali lahko iz prezentacije ugotovimo (na primer z računalniškim programom) ali je dana grupa trivialna?} V splošnem je odgovor {\em ne}! {\definicija Besedni problem} za grupo $\langle S \mid R \rangle$ sprašuje, če se lahko ob vnosu besede $x \in (S \cup S^{-1})^*$ odločimo (z algoritmom), ali je $x = 1$ v $\langle S \mid R \rangle$.\footnote{Tu imamo v resnici dve vprašanji. 1. Ali obstaja algoritem, ki se ob vnosu $x$ ustavi, če je le $x = 1$ v grupi? 2. Ali obstaja algoritem, ki se ob vnosu $x$ ustavi, če je le $x \neq 1$ v grupi? Kasneje bomo videli, da je odgovor na prvo vprašanje vedno pritrdilen, težave pa so z drugim vprašanjem.} Ta problem je za nekatere grupe preprosto rešljiv, na primer za grupo $\langle 1 \mid \emptyset \rangle \cong \ZZ$. Obstajajo pa grupe $\langle S \mid R \rangle$ (celo z $|S|, |R| < \infty$), za katere je besedni problem {\definicija neodločljiv} \href{http://www.mathnet.ru/php/archive.phtml?wshow=paper&jrnid=tm&paperid=1180&option_lang=eng}{(Novikov 1955)} -- takega algoritma ni!\footnote{Turingov stroj lahko zakodiramo v grupo, pravila zakodiramo v relacije, črke in stanja zakodiramo v generatorje.} 

\subsection{Končno prezentirane grupe}

Grupa $G$ je {\definicija končno prezentirana}, če ima kakšno prezentacijo $G \cong \langle S \mid R \rangle$ z $|R|, |S| < \infty$.

Končno prezentirane grupe so vedno končno generirane, obratno pa v splošnem ne drži.

\begin{zgled}
Naj bo $L = \left\langle a, t \mid a^2, [t^k a t^{-k}, t^j a t^{-j}] \text{ za vse $j,k \in \ZZ$} \right\rangle$.\footnote{Oznaka $[x,y]$ stoji za {\definicija komutator} $x^{-1} y^{-1} x y$.} Grupa $L$ se imenuje {\definicija svetilkarjeva grupa}.\footnote{Angleško {\em lamplighter group}.}

Grupa $L$ {\em ni} končno prezentirana \vaje.

Grupo $L$ lahko uresničimo na bolj konkreten način kot podgrupo matrik preko homomorfizma
\[
\varphi \colon L \to \GL_2(\ZZ/2\ZZ[x, x^{-1}])
\]
iz $L$ v matrike nad kolobarjem Laurentovih polinomov\footnote{Element tega kolobarja je na primer $x^{-2} + 1 + x^{10}$.} nad $\ZZ/2\ZZ$, ki preslika generatorja na naslednji način:
\[
\varphi \colon 
a \mapsto 
\begin{pmatrix}
1 & 1 \\
0 & 1
\end{pmatrix},
\quad
t \mapsto 
\begin{pmatrix}
x & 0 \\
0 & 1
\end{pmatrix}.
\]
S to definicijo velja $\varphi(a^2) = 1$ in $\varphi([t^k a t^{-k}, t^l a t^{-l}]) = 1$, zato po univerzalni lastnosti res dobimo homomorfizem $\varphi$. 

Lahko je preveriti, da velja
\[
\image \varphi = \left\{ 
\begin{pmatrix}
x^k & P \\
0 & 1
\end{pmatrix} \mid
k \in \ZZ, \ P \in \ZZ/2\ZZ[x, x^{-1}]
\right\}.
\]
Z nekaj truda se da napisati tudi inverzno preslikavo $\psi \colon \image \varphi \to L$ \vaje. Zato velja $L \cong \image \varphi$, s čimer smo abstraktno grupo $L$ uresničili kot podgrupo matrik.
\end{zgled}

\section{Nove grupe iz starih}

Pogledali si bomo nekaj standardnih konstrukcij novih grup iz starih.

\subsection{Produkti}

Naj bo $I$ indeksna množica in $(G_i)_{i \in I}$ družina grup. Tvorimo njihov {\definicija direktni produkt} $\prod_{i \in I} G_i$: to je množica $\prod_{i \in I} G_i$, opremljena z operacijo po komponentah, se pravi $(g_i)_{i \in I} \cdot (h_i)_{i \in I} = (g_i h_i)_{i \in I}$. {\definicija Direktna vsota} $\bigoplus_{i \in I} G_i$ je podgrupa direktnega produkta, ki sestoji iz elementov $(g_i)_{i \in I}$, za katere velja $g_i = 1$ za vse razen končno mnogo komponent.

\subsection{Razširitve}

Grupe lahko združimo tudi na bolj zakomplicirane načine. Za dani grupi $N,Q$ je {\definicija razširitev $Q$ z $N$} eksaktno zaporedje grup
\[
1 \to N \xrightarrow{\iota} G \xrightarrow{\pi} Q \to 1,
\]
torej $\ker \iota = 1$, $\image \pi = Q$, $\image \iota = \ker \pi$.

\begin{zgled}
\begin{itemize}
    \item $0 \to \ZZ/2\ZZ \xrightarrow{\iota} \ZZ/2\ZZ \times \ZZ/2\ZZ \xrightarrow{\pi} \ZZ/2\ZZ \to 0$, kjer je $\iota(x) = (x,0)$ in $\pi(x,y) = y$.

    V tem primeru imamo trivialno razširitev, kjer je srednja grupa $G$ kar direktni produkt grup $Q$ in $N$.

    \item $0 \to \ZZ  \xrightarrow{\iota} \ZZ \xrightarrow{\pi} \ZZ/2\ZZ \to 0$, kjer je $\iota(n) = 2 n$ in $\pi(n) = n \mod{2}$.

    Srednja grupa v razširitvi v tem primeru {\em ni} kar direktni produkt grup $Q$ in $N$.
\end{itemize}
\end{zgled}

\subsection{Semidirektni produkti}

Obstajajo razširitve, ki so le malo deformirani direktni produkti. Za dani grupi $Q,N$ in dan homomorfizem\footnote{Torej $Q$ deluje na $N$. O delovanjih bomo podrobneje govorili v naslednjem poglavju.} $\varphi \colon Q \to \Aut(N)$ je {\definicija semidirektni produkt $Q$ z $N$ glede na $\varphi$} grupa $N \rtimes_\varphi Q$, ki je kot množica $N \times Q$, operacija pa je definirana kot $(n_1, q_1) \cdot (n_2, q_2) = (n_1 \varphi(q_1)(n_2), q_1 q_2)$. V primeru, ko za vsak $q \in Q$ velja $\varphi(q) = \id_N$, dobimo $N \rtimes_\varphi Q = N \times Q$, torej kar običajen direktni produkt grup.

\begin{zgled}
\begin{itemize}
    \item Vemo že, da lahko diedrsko grupo predstavimo kot $D_{2n} = \langle s, t \mid s^n, t^2, t s t^{-1} = s^{-1} \rangle$. Zdaj jo bomo predstavili še kot semidirektni produkt.

    Naj bo $\varphi \colon \ZZ/2\ZZ \to \Aut(\ZZ/n\ZZ), \ 1 \mapsto (x \mapsto -x)$. Tvorimo $G = \ZZ/n\ZZ \rtimes_\varphi \ZZ/2\ZZ$. Preslikava $f \colon D_{2n} \to G$, dana s predpisom $s \mapsto (1,0), t \mapsto (0,1)$, je homomorfizem. Velja $\image f = G$ in $|D_{2n}| = |G| = 2n$, zato je $f$ izomorfizem.

    Podobno velja v limiti $n \to \infty$. Naj bo $D_{\infty} = \langle s, t \mid t^2, t s t^{-1} = s^{-1} \rangle$. Naj bo $\varphi \colon \ZZ/2\ZZ \to \Aut(\ZZ), \ 1 \mapsto (x \mapsto -x)$. Tedaj imamo izomorfizem $D_{\infty} \cong \ZZ \rtimes_\varphi \ZZ/2\ZZ$, ki izhaja iz prirejanja $s \mapsto (1,0), t \mapsto (0,1)$.

    \item Spomnimo se svetilkarjeve grupe 
    \[
    L = \left\langle a, t \mid a^2, [t^k a t^{-k}, t^j a t^{-j}] \text{ za vse $j,k \in \ZZ$} \right\rangle.
    \]

    Element $a \in L$ je reda $2$, torej imamo eno kopijo $\ZZ/2\ZZ$ v $L$.
    Element $t \in L$ je neskončnega reda, kar nam da eno kopijo $\ZZ$ v $L$. Element $t^k$ konjugira $a$ v $t^k a t^{-k}$. Za vsak $k \in \ZZ$ dobimo torej eno kopijo $\ZZ/2\ZZ$ v $L$. Vse te različne kopije med sabo komutirajo, torej v $L$ najdemo $\bigoplus_{i \in \ZZ} \ZZ/2\ZZ$.

    Definirajmo $\varphi \colon \ZZ \to \bigoplus_{i \in \ZZ} \ZZ/2\ZZ$, $i \mapsto ((g_n)_{n \in \ZZ} \mapsto (g_{n+i})_{n \in \ZZ})$. Tvorimo grupo $\left( \bigoplus_{i \in \ZZ} \ZZ/2\ZZ \right) \rtimes_{\varphi} \ZZ$. Trdimo, da je ta semidirektni produkt izomorfen grupi $L$. Generatorjema $a$ in $t$ bosta pri tem ustrezala elementa $A = (e_0, 0)$ in $T = (0, 1)$, kjer smo z $e_i \in \bigoplus_{i \in \ZZ}\ZZ/2\ZZ$ označili bazni vektor, ki je pri $i$-ti komponenti enak $1$, sicer pa $0$.

    Definirajmo najprej preslikavo $\alpha \colon L \to \left( \bigoplus_{i \in \ZZ} \ZZ/2\ZZ \right) \rtimes_{\varphi} \ZZ$, $a \mapsto A, t \mapsto T$. S tem res dobimo induciran homomorfizem grup po univerzalni lastnosti, saj se relatorji $L$ preslikajo v enoto. Velja namreč $A^2 = (2 e_0, 0) = (0,0)$, $T A T^{-1} = (0,1)(e_0,0)(0,-1) = (e_{-1},0)$, $T^k A T^{-k} = (e_{-k}, 0)$ in zatorej $[T^k A T^{-k}, T^j A T^{-j}] = (0,0)$.

    Definirajmo zdaj še inverzni homomorfizem $\beta \colon \left( \bigoplus_{i \in \ZZ} \ZZ/2\ZZ \right) \rtimes_{\varphi} \ZZ \to L$. Ta naj po definiciji slika $(e_0, n) \mapsto a t^n$, $(e_{-k},0) \mapsto t^k a t^{-k}$. Od tod izpeljemo kandidata za splošno formulo za $\beta$: $(\sum_{k \in \ZZ} a_{-k} e_{-k}, n) \mapsto \prod_{k \in \ZZ} (t^k a t^{-k})^{a_{-k}} \cdot t^n$. Ker so skoraj vsi koeficienti $a_k$ enaki $0$, je zadnji produkt končen. Lahko je preveriti \vaje, da je $\beta$ res homomorfizem, ki je inverzen $\alpha$. 

    Torej je res $\left( \bigoplus_{i \in \ZZ} \ZZ/2\ZZ \right) \rtimes_{\varphi} \ZZ \cong L$.

    {\em Zdaj lahko pojasnemo ime svetilkarjeva grupa.} Element $L$ je prek izomorfizma zgoraj določen z elementom iz $\bigoplus_{i \in \ZZ} \ZZ/2\ZZ$ (= končno mnogo izbir $1$ iz množice $\ZZ$) in z elementom iz $\ZZ$. 

    Predstavljajmo si, imamo za vsak element iz $\ZZ$ svetilko, ki je bodisi prižgana bodisi ugasnjena. Svetilkar je na začetku službe na poziciji $0 \in \ZZ$. Prvi element pri izbiri elementa v $L$ nam pove, katere svetilke je svetilkar prižgal. Drugi element nam pove, kje je svetilkar na koncu službe.

    Element grupe $L$ je beseda v $a$ in $t$, na primer $t a t^3 a t^{-2}$. To s svetilkarjem interpretiramo takole: svetilkar začne v $0 \in \ZZ$, potem se premakne za $1$ v levo, prižge svetilko, se premakne za $3$ v levo, prižge svetilko, se premakne za $2$ v desno in se ustavi. Na koncu sta svetilki prižgani na pozicijah $-1, -4$, svetilkar sam pa je na poziciji $-2$. To ustreza naslednjemu računu v $L$:
    \[
    t a t^3 a t^{-2} = t a t^{-1} \cdot t^4 a t^{-4} \cdot t^2 \equiv (e_{-1} + e_{-4}, 2).
    \]
\end{itemize}
\end{zgled}

    \begin{figure}[t]
    \centering
    \begin{tikzpicture}
    \tikzstyle{vertex}=[circle, fill = oranzna, minimum size = 5pt, inner sep = 0pt]

    \foreach \n in {-6,...,1}
        \draw[thick] (\n, 0) -- (\n+1, 0);

    \foreach \n in {-5,...,1}
        \node[vertex, label = below : $\n$] at (\n, 0) {};

    \foreach \n in {-4,-1}
        \node[vertex, fill = rdeca, minimum size = 10pt] at (\n, 0) {};

    \node[vertex, minimum size = 10pt, fill = vijolicna] at (-2,0) {};

    \end{tikzpicture}
    \caption{Svetlikar se sprehaja in prižiga svetilke}
    \end{figure}

Semidirektni produkti se pojavijo v posebnih vrstah razširitev. {\definicija Razcepna razširitev} je razširitev $1 \to N \to G \xrightarrow{\pi} Q \to 1$, v kateri obstaja {\em homomorfizem} $s \colon Q \to G$, ki zadošča $\pi \circ s = \id_Q$. Takemu homomorfizmu rečemo {\definicija prerezni homomorfizem}. 

Za dani grupi $N,Q$ in delovanje $\varphi \colon Q \to \Aut(N)$ imamo razcepno razširitev
\[
1 \to N \to N \rtimes_{\varphi} Q \to Q \to 1
\]
z vložitvijo $n \mapsto (n,1)$, projekcijo $(n,q) \mapsto q$ in prereznim homomorfizmom $q \mapsto (1,q)$.

Velja tudi obratno. Naj bo $1 \to N \to G \to Q \to 1$ razcepna razširitev s prereznim homomorfizmom $s$. Tedaj dobimo delovanje $Q$ na $N$: $\varphi \colon Q \to \Aut(N), \ q \mapsto (n \mapsto s(q) \cdot n \cdot s(q)^{-1})$. Element $q$ torej s $s$ dvignemo v $G$ in konjugiramo element $n$. Pri tem velja $G \cong N \rtimes_{\varphi} Q$, izomorfizem izhaja iz $(n,q) \mapsto n \cdot s(q)$ in $g \mapsto (g \cdot (s \circ \pi(g))^{-1},\pi(g))$. 

S tem smo premislili vsebino naslednje trditve.

\begin{trditev}
Semidirektni produkti so natanko grupe, ki se pojavijo kot razširitve v razcepnih razširitvah.
\end{trditev}

Več o razširitvah si bomo pogledali nekoliko kasneje.

\chapter{Delovanja}

{\em Bistvo} grup je, da delujejo -- na različnih objektih.

\section{Delovanja grup}

\subsection{Osnovno o delovanjih}

Bodi $G$ grupa, $\mathcal{C}$ kategorija in $X$ objekt v $\mathcal{C}$. {\definicija Delovanje} grupe $G$ na $X$ je homomorfizem $G \xrightarrow{\rho} \Aut_{\mathcal{C}}(X)$.

\begin{zgled}
\begin{itemize}
    \item Za $\mathcal{C}$ vzemimo kategorijo množic. Naj bo $X$ množica. Delovanje $G$ na $X$ je homomorfizem $G \xrightarrow{\rho} \Sym(X)$ iz $G$ v grupo bijekcij množice $X$.

    \item Za $\mathcal{C}$ vzemimo kategorijo topoloških prostorov. Naj bo $X$ topološki prostor. Delovanje $G$ na $X$ je homomorfizem $G \xrightarrow{\rho} \Homeo(X)$ iz $G$ v grupo homeomorfizmom prostora $X$.

    \item Za $\mathcal{C}$ vzemimo kategorijo vektorskih prostorov. Naj bo $V$ vektorski prostor. Delovanje $G$ na $V$ je homomorfizem $G \xrightarrow{\rho} \GL(V)$ iz $G$ v grupo linearnih avtomorfizmov prostora $V$. Ta zadnji primer je še posebej pomemben, saj grupo $G$ uresničimo kot grupo linearnih transformacij vektorskega prostora. V tem primeru rečemo, da je $\rho$ {\definicija upodobitev} grupe $G$.
\end{itemize}
\end{zgled}

\subsection{Cayleyjevo delovanje}

Cayleyjevo delovanje je najpomembnejši primer delovanja. Izhaja iz Cayleyjevega izreka:
\[
G \to \Sym(G), \quad g \mapsto (x \mapsto g \cdot x) \quad (g,x \in G).
\]
To delovanje grupe $G$ na množici $G$ si lahko vizualiziramo z grafom na naslednji način.

Dana je grupa $G$ in $S \subseteq G$ neka njena generirajoča podmnožica. {\definicija Cayleyjev graf} grupe $G$ glede na $S$ je graf {\definicija $\Cay(G,S)$} z:
\begin{itemize}
    \item vozlišči $V = G$,\footnote{Za vsak element $g \in G$ imamo torej eno vozlišče.}
    \item povezavami $E = \{ \{ g, g \cdot s \} \mid g \in G, \ s \in S \cup S^{-1} \backslash \{ 1 \} \}$.
\end{itemize}

\begin{zgled}
\begin{itemize}
    \item $\Cay(\ZZ, \{ 1 \})$ je pot neskončne dolžine. 

    \item $\Cay(\ZZ, \{ 2, 3 \})$ je $4$-regularen graf, ki pa ni drevo.

    \item $\Cay(\ZZ^2, \{ (1,0), (0,1) \})$ je ravninska mreža.

    \item $\Cay(\ZZ/6\ZZ, \{ 1 \})$ je $6$-cikel.

    \item $\Cay(S_3, \{ \tau, \sigma \})$, kjer je $\tau = (1\ 2)$ in $\sigma = (1\ 2\ 3)$, je tristrana prizma z eno {\em križno} povezavo.

    \item $\Cay(F_2, \{ x, y \})$ je $4$-regularno drevo.
\end{itemize}
\end{zgled}

    \begin{figure}[t]
    \centering
    \begin{tikzpicture}
    \tikzstyle{vertex}=[circle, fill = oranzna, minimum size = 5pt, inner sep = 0pt]

    \foreach \n in {-5,...,4}
        \draw[line width = 0.25mm] (\n, 0) -- (\n+1, 0);

    \foreach \n in {-4,...,4}
        \node[vertex, label = below : $\n$] at (\n, 0) {};
    \end{tikzpicture}
    \caption{$\Cay(\ZZ, \{ 1 \})$}
    \end{figure}

    \begin{figure}[t]
    \centering
    \begin{tikzpicture}
    \tikzstyle{vertex}=[circle, fill = oranzna, minimum size = 5pt, inner sep = 0pt]

    \foreach \n in {-5,...,3}
        \draw[line width = 0.25mm] (\n, 0) to[bend left=30] (\n + 2, 0);

    \foreach \n in {-5,...,2}
        \draw[line width = 0.25mm] (\n, 0) to[bend left=60] (\n + 3, 0);

    \foreach \n in {-4,...,4}
        \node[vertex, label = below : $\n$] at (\n, 0) {};
    \end{tikzpicture}
    \caption{$\Cay(\ZZ, \{ 2, 3 \})$}
    \end{figure}

    \begin{figure}[t]
    \centering
    \begin{tikzpicture}[scale = 1.3]
    \tikzstyle{vertex}=[circle, fill = oranzna, minimum size = 5pt, inner sep = 0pt]

    \foreach \n in {-2,...,1}
        \foreach \m in {-1,...,1}
            \draw[thick] (\n, \m) to (\n + 1, \m);

    \foreach \n in {-1,...,1}
        \foreach \m in {-2,...,1}
            \draw[thick] (\n, \m) to (\n, \m + 1);

    \foreach \n in {-1,...,1}
        \foreach \m in {-1,...,1}
            \node[vertex, label = -100 : {$(\n , \m)$}] at (\n, \m) {};
    \end{tikzpicture}
    \caption{$\Cay(\ZZ^2, \{ (1,0), (0,1) \})$}
    \end{figure}

    \begin{figure}[t]
    \centering
    \begin{tikzpicture}
    \tikzstyle{vertex}=[circle, fill = oranzna, minimum size = 5pt, inner sep = 0pt]

    \foreach \n in {0,...,5}
        \draw[line width = 0.25mm] (60*\n:1) to (60*\n + 60:1);

    \foreach \n in {0,...,5}
        \node[vertex, label = below : {$\n$}] at (60*\n:1) {};

    \end{tikzpicture}
    \caption{$\Cay(\ZZ/6\ZZ, \{ 1 \})$}
    \end{figure}

    \begin{figure}[t]
    \centering
    \begin{tikzpicture}
    \tikzstyle{vertex}=[circle, fill = oranzna, minimum size = 5pt, inner sep = 0pt]

    \foreach \n in {0,...,2}
        \draw[line width = 0.25mm] (120*\n:1) to (120*\n + 120:1);


    \foreach \n in {0,...,2}
        \draw[line width = 0.25mm] (120*\n:2) to (120*\n + 120:2);


    \draw[line width = 0.25mm] (0:2) to (0:1);
    \draw[line width = 0.25mm] (120:2) to (240:1);
    \draw[line width = 0.25mm] (240:2) to (120:1);

    \node[vertex, label = below : {$1$}] at (0:1) {};
    \node[vertex, label = right : {$\sigma$}] at (120:1) {};
    \node[vertex, label = below : {$\sigma^2$}] at (240:1) {};

    \node[vertex, label = below : {$\tau$}] at (0:2) {};
    \node[vertex, label = right : {$\tau \sigma$}] at (120:2) {};
    \node[vertex, label = below : {$\tau \sigma^2$}] at (240:2) {};

    \end{tikzpicture}
    \caption{$\Cay(S_3, \{ \tau, \sigma \})$}
    \end{figure}

    \begin{figure}[t]
    \centering
    \begin{tikzpicture}
    \tikzstyle{vertex}=[circle, fill = oranzna, minimum size = 5pt, inner sep = 0pt]

    \def\distance{2.5}
    \def\factor{2}
    \def\tinyedge{\distance/(\factor * \factor)}

    \node[vertex, label = below left : {$1$}] (1) at (0,0) {};

    \node[vertex, label = below left : {$x$}] (x) at (\distance,0) {};
    \draw[line width = 0.25mm] (1) -- (x);
    \node[vertex, label = below left : {$x^{-1}$}] (x-1) at (-\distance,0) {};
    \draw[line width = 0.25mm] (1) -- (x-1);

    % x2
        \draw[line width = 0.25mm] ({\distance * (1 + 1/\factor)},0) -- ({\distance * (1 + 1/\factor)},{\tinyedge});
        \draw[line width = 0.25mm] ({\distance * (1 + 1/\factor)},0) -- ({\distance * (1 + 1/\factor)},{-\tinyedge});
        \draw[line width = 0.25mm] ({\distance * (1 + 1/\factor)},0) -- ({\distance * (1 + 1/\factor) + \tinyedge},0);

    \node[vertex, label = below left : {$x^2$}] (x2) at ({\distance * (1 + 1/\factor)},0) {};
    \draw[line width = 0.25mm] (x) -- (x2);

    % x-2
        \draw[line width = 0.25mm] ({-\distance * (1 + 1/\factor)},0) -- ({-\distance * (1 + 1/\factor)},{-\tinyedge});
        \draw[line width = 0.25mm] ({-\distance * (1 + 1/\factor)},0) -- ({-\distance * (1 + 1/\factor)},{\tinyedge});
        \draw[line width = 0.25mm] ({-\distance * (1 + 1/\factor)},0) -- ({-\distance * (1 + 1/\factor) - \tinyedge},0);

    \node[vertex, label = below left : {$x^{-2}$}] (x-2) at ({-\distance * (1 + 1/\factor)},0) {};
    \draw[line width = 0.25mm] (x-1) -- (x-2);

    \node[vertex, label = below left : {$y$}] (y) at (0,\distance) {};
    \draw[line width = 0.25mm] (1) -- (y);
    \node[vertex, label = below left : {$y^{-1}$}] (y-1) at (0,-\distance) {};    
    \draw[line width = 0.25mm] (1) -- (y-1);

    %y2
        \draw[line width = 0.25mm] (0,{\distance * (1 + 1/\factor)}) -- (0,{\distance * (1 + 1/\factor) + \tinyedge});
        \draw[line width = 0.25mm] (0,{\distance * (1 + 1/\factor)}) -- ({\tinyedge},{\distance * (1 + 1/\factor)});
        \draw[line width = 0.25mm] (0,{\distance * (1 + 1/\factor)}) -- ({-\tinyedge},{\distance * (1 + 1/\factor)});

    \node[vertex, label = below left : {$y^2$}] (y2) at (0,{\distance * (1 + 1/\factor)}) {};
    \draw[line width = 0.25mm] (y) -- (y2);

    % y-2
        \draw[line width = 0.25mm] (0,{-\distance * (1 + 1/\factor)}) -- ({-\tinyedge},{-\distance * (1 + 1/\factor)});
        \draw[line width = 0.25mm] (0,{-\distance * (1 + 1/\factor)}) -- ({\tinyedge},{-\distance * (1 + 1/\factor)});
        \draw[line width = 0.25mm] (0,{-\distance * (1 + 1/\factor)}) -- (0,{-\distance * (1 + 1/\factor) - \tinyedge});

    \node[vertex, label = below left : {$y^{-2}$}] (y-2) at (0,{-\distance * (1 + 1/\factor)}) {};
    \draw[line width = 0.25mm] (y-1) -- (y-2);

    % xy

    \node[vertex, label = below left : {$xy$}] (xy) at (\distance,\distance * 1/\factor) {};
    \draw[line width = 0.25mm] (x) -- (xy);
        \draw[line width = 0.25mm] (xy) -- ($(xy) + (0,{\tinyedge})$);
        \draw[line width = 0.25mm] (xy) -- ($(xy) + ({\tinyedge},0)$);
        \draw[line width = 0.25mm] (xy) -- ($(xy) + ({-\tinyedge},0)$);

    % xy-1
    \node[vertex, label = below left : {$xy^{-1}$}] (xy-1) at (\distance,-\distance * 1/\factor) {};
    \draw[line width = 0.25mm] (x) -- (xy-1);
        \draw[line width = 0.25mm] (xy-1) -- ($(xy-1) + (0,{-\tinyedge})$);
        \draw[line width = 0.25mm] (xy-1) -- ($(xy-1) + ({\tinyedge},0)$);
        \draw[line width = 0.25mm] (xy-1) -- ($(xy-1) + ({-\tinyedge},0)$);

    % x-1y
    \node[vertex, label = below left : {$x^{-1}y$}] (x-1y) at (-\distance,\distance * 1/\factor) {};
    \draw[line width = 0.25mm] (x-1) -- (x-1y);
        \draw[line width = 0.25mm] (x-1y) -- ($(x-1y) + (0,{\tinyedge})$);
        \draw[line width = 0.25mm] (x-1y) -- ($(x-1y) + ({\tinyedge},0)$);
        \draw[line width = 0.25mm] (x-1y) -- ($(x-1y) + ({-\tinyedge},0)$);

    % x-1y-1
    \node[vertex, label = below left : {$x^{-1}y^{-1}$}] (x-1y-1) at (-\distance,-\distance * 1/\factor) {};
    \draw[line width = 0.25mm] (x-1) -- (x-1y-1);
        \draw[line width = 0.25mm] (x-1y-1) -- ($(x-1y-1) + (0,{-\tinyedge})$);
        \draw[line width = 0.25mm] (x-1y-1) -- ($(x-1y-1) + ({\tinyedge},0)$);
        \draw[line width = 0.25mm] (x-1y-1) -- ($(x-1y-1) + ({-\tinyedge},0)$);

    % yx
    \node[vertex, label = below left : {$yx$}] (yx) at (\distance * 1/\factor,\distance) {};
    \draw[line width = 0.25mm] (y) -- (yx);
        \draw[line width = 0.25mm] (yx) -- ($(yx) + (0,{\tinyedge})$);
        \draw[line width = 0.25mm] (yx) -- ($(yx) + (0,{-\tinyedge})$);
        \draw[line width = 0.25mm] (yx) -- ($(yx) + ({\tinyedge},0)$);

    % yx-1
    \node[vertex, label = below left : {$yx^{-1}$}] (yx-1) at (-\distance * 1/\factor,\distance) {};
    \draw[line width = 0.25mm] (y) -- (yx-1);
        \draw[line width = 0.25mm] (yx-1) -- ($(yx-1) + (0,{\tinyedge})$);
        \draw[line width = 0.25mm] (yx-1) -- ($(yx-1) + (0,{-\tinyedge})$);
        \draw[line width = 0.25mm] (yx-1) -- ($(yx-1) + ({-\tinyedge},0)$);

    % y-1x
    \node[vertex, label = below left : {$y^{-1}x$}] (y-1x) at (\distance * 1/\factor,-\distance) {};    
    \draw[line width = 0.25mm] (y-1) -- (y-1x);
        \draw[line width = 0.25mm] (y-1x) -- ($(y-1x) + (0,{\tinyedge})$);
        \draw[line width = 0.25mm] (y-1x) -- ($(y-1x) + (0,{-\tinyedge})$);
        \draw[line width = 0.25mm] (y-1x) -- ($(y-1x) + ({\tinyedge},0)$);

    % y-1x-1
    \node[vertex, label = below left : {$y^{-1}x^{-1}$}] (y-1x-1) at (-\distance * 1/\factor,-\distance) {}; 
    \draw[line width = 0.25mm] (y-1) -- (y-1x-1);
        \draw[line width = 0.25mm] (y-1x-1) -- ($(y-1x-1) + (0,{\tinyedge})$);
        \draw[line width = 0.25mm] (y-1x-1) -- ($(y-1x-1) + (0,{-\tinyedge})$);
        \draw[line width = 0.25mm] (y-1x-1) -- ($(y-1x-1) + ({-\tinyedge},0)$);
    \end{tikzpicture}  
    \caption{$\Cay(F_2, \{ x, y \})$}
    \end{figure}

Proste grupe so natanko grupe, katerih Cayleyjev graf je {\em lahko} drevo.\footnote{Trditev nam podaja {\em geometrijsko} karakterizacijo prostih grup. V naslednjem razdelku jo bomo še razširili.}

\begin{trditev}
Naj bo $G$ grupa in $S \subseteq G$ z lastnostjo $S \cap S^{-1} = \emptyset$. Če je $\Cay(G,S)$ drevo, potem $S$ prosto generira $G$.
\end{trditev}

\begin{dokaz}
Po univerzalni lastnosti proste grupe dobimo homomorfizem $\varphi \colon F(S) \to G$. Ker $S$ generira $G$, je $\varphi$ surjektiven. Privzemimo, da $\varphi$ {\em ni} injektiven. Naj bo $1 \neq w = s_1 \cdots s_n \in F(S)$ ($s_i \in S \cup S^{-1}$) okrajšana beseda minimalne dolžine, ki je v jedru $\ker \varphi$. Ločimo več primerov glede na dolžino besede.

\begin{description}
    \item[Če je $n = 1$:] V tem primeru je $w = s_1 \in S \cup S^{-1}$. Toda $\varphi|_{S}$ je injektivna. Protislovje s predpostavko $\varphi(w) = 1$.

    \item[Če je $n = 2$:] V tem primeru je $w = s_1 s_2$ in velja $\varphi(w) = 1$. Torej $1 = \varphi(s_1 s_2) = s_1 \cdot s_2 \in G$. Torej je $s_1 \neq s_2^{-1}$ v $F(S)$,\footnote{Beseda $w$ je okrajšana.} medtem ko je $s_1 = s_2^{-1}$ v $G$. Od tod sledi $s_1 \in S \cap S^{-1} \subseteq G$. Protislovje s predpostavko $S \cap S^{-1} = \emptyset$.

    \item[Če je $n \geq 3$:] Zaporedje elementov $1, s_1, s_1 s_2, \dots, s_1 s_2 \cdots s_n$ tvori cikel v $\Cay(G,S)$, saj je $s_1 s_2 \cdots s_n = \varphi(w) = 1 \in G$. Vsa vozlišča tega cikla so {\em različna} med sabo: če je namreč $s_1 \cdots s_i = s_1 \cdots s_j \in G$ za nek $i < j$, potem je beseda $w^\prime = s_1 s_2 \cdots s_i s_{j+1} \cdots s_n \in F(S)$ krajše dolžine kot $w$, hkrati pa velja $\varphi(w^\prime) = s_1 s_2 \cdots s_i s_{j+1} \cdots s_n = s_1 s_2 \cdots s_j s_{j+1} \cdots s_n = \varphi(w) = 1$. Protislovje z minimalnostjo dolžine $w \in \ker \varphi$. Tako res dobimo $n$-cikel v grafu $\Cay(G,S)$. Protislovje s predpostavko, da je $\Cay(G,S)$ drevo.
\end{description}
\end{dokaz}

\subsection{Prosta delovanja}

Cayleyjevo delovanje je posebni primer {\definicija prostega delovanja} -- to je delovanje grupe $G$ na množici $X$, za katero velja\footnote{Pogoj pomeni, da noben netrivialen element grupe nima fiksnih točk.}
\[
\forall g \in G - \{ 1 \} \ \forall x \in X \colon \ g \cdot x \neq x.
\]

\begin{zgled}
\begin{itemize}
    \item Cayleyjevo delovanje je prosto. Element $x \in G$ je namreč fiksna točka delovanja elementa $g \in G$, če in samo če velja $g \cdot x = x$, kar je mogoče le v primeru $g = 1$.

    \item Naj $\ZZ$ deluje na krožnici $S^1$ s predpisom
    \[
    \ZZ \to \Isom(S^1), \quad n \mapsto (z \mapsto z \cdot e^{2 \pi i n \alpha})
    \]
    za nek izbrani parameter $\alpha \in \RR$.
    \begin{description}
        \item[Če je $\alpha \in \QQ$:] Naj bo $\alpha = \frac{a}{b}$. Tedaj $n = b$ deluje kot $\id_{S^1}$, torej delovanje {\em ni} prosto.

        \item[Če je $\alpha \in \RR - \QQ$:] Število $z \in S^1$ je fiksna točka delovanja nekega $n \in \ZZ$, če in samo če je $2 \pi i n \alpha \in 2 \pi i \ZZ$, kar pomeni ravno $\alpha \in \QQ$. Torej je za iracionalne izbire $\alpha$ to delovanje prosto.
    \end{description}
\end{itemize}
\end{zgled}

Ko grupe delujejo na grafih, topoloških prostorih ali ostalih objektih različnih kategorij, pri definiciji prostega delovanja upoštevamo dodatno strukturo. Na primer, rečemo, da grupa $G$ {\definicija prosto deluje na grafu} $\Gamma = (V,E)$, če velja:
\begin{itemize}
    \item obstaja homomorfizem $\rho \colon G \to \Aut(\Gamma)$, kjer grupa $\Aut(\Gamma)$ sestoji iz bijekcij $f \colon V \to V$, za katere velja
    \[
    \forall u,v \in V \colon \ \{ u,v \} \in E \Leftrightarrow \{ f(u), f(v) \} \in E,
    \]

    \item za vsak $g \in G - \{ 1 \}$ velja
    \[
    \forall v \in V \colon \ \rho(g)(v) \neq v \quad \text{in} \quad
    \forall \{u,v\}\in E \colon \ \{ \rho(g)(u), \rho(g)(v) \} \neq \{ u,v \}.
    \]
\end{itemize}

\begin{zgled}
Grupa $G$ deluje na grafu $\Cay(G,S)$:
\[
G \to \Aut \Cay(G,S), \quad g \mapsto (x \mapsto g \cdot x).
\]
Tukaj je pomembno, da smo v definiciji Cayleyjevega delovanja uporabili množenje z {\em leve} z elementom $g \in G$, v definiciji Cayleyjevega grafa pa smo uporabili množenje z {\em desne} z elementi $s \in S$.

Elementi $g \in G$ res delujejo kot avtomorfizmi grafa, ker za $x,y \in G$ velja:\footnote{Oznaka $x \sim y$ pomeni, da je $x$ povezan z $y$ v grafu $\Cay(G,S)$.}
\[
x \sim y \Leftrightarrow \exists s \in S \cup S^{-1} \colon \ x = y \cdot s \Leftrightarrow \exists s \in S \cup S^{-1} \colon \ g \cdot x = g \cdot y \cdot s \Leftrightarrow gx \sim gy.
\]
To delovanje {\em ni} nujno vedno prosto. Na primer, če vzamemo $G = \ZZ/2\ZZ = \{ 0, 1 \}$, $S = \{ 1 \}$, potem je $\Cay(G,S)$ pot dolžine $1$, na kateri element $1 \in G$ deluje tako, da zamenja vozlišči $0$ in $1$ med sabo in {\em ohranja povezavo} med njima.
\end{zgled}

Izkaže se, da je edina ovira za prostost delovanja grupe na svojem Cayleyjevem grafu obstoj elementov reda $2$ v $S$.

\begin{trditev}
Bodi $G$ grupa, $G = \langle S \rangle$. Cayleyjevo delovanje $G$ na $\Cay(G,S)$ je prosto, če in samo če $S$ {\em ne} vsebuje elementov reda $2$.
\end{trditev}

\begin{dokaz}
Cayleyjevo delovanje je vselej prosto na vozliščih. Preveriti moramo le, kaj se dogaja s povezavami.
\begin{description}
\item[Če obstaja $s \in S, s^2 = 1$:] V tem primeru $s$ deluje na povezavi $\{ 1, s \}$ tako, da jo preslika v povezavo $\{ s, s^2 \} = \{ 1, s \}$. Torej $s$ fiksira to povezavo, zato delovanje {\em ni} prosto. \kljuka

\item[Če delovanje ni prosto:] Naj bo $g \in G - \{ 1 \}$, ki fiksira povezavo $\{ u, v \}$. Velja torej $\{ u, v \} = \{ g u, g v \}$. Ker je $u \sim v$, lahko pišemo $v = u s$ za nek $s \in S \cup S^{-1} - \{ 1 \}$. Ločimo dva primera.
    \begin{description}
        \item[Če je $u = gu$ in $v = gv$:] Sledi $g = 1$. Protislovje.
        \item[Če je $u = gv$ in $v = gu$:] V tem primeru velja $u = gv = gus = vs = u s^2$, zato sledi $s^2 = 1$. \kljuka
    \end{description}
\end{description}
\end{dokaz}

\subsection{Orbite in stabilizatorji}

Vsako delovanje $G$ na množici $X$ lahko razgradimo na {\em poddelovanja} na podmnožicah $X$. Osnovne enote so orbite. {\definicija Orbita} elementa $x \in X$ je množica $G.x = \{ g.x \mid g \in G \}$.\footnote{Grupa $G$ deluje na $X$ prek homomorfizma $\rho \colon G \to \Sym(X)$. Oznaka $g.x$ pomeni $\rho(g)(x)$.} {\definicija Množica orbit} je kvocientna množica $G \backslash X = \{ G.x \mid x \in X \}$. Kadar ima $X$ kakšno dodatno strukturo, ima ponavadi tudi množica orbit naravno dodatno strukturo.

\begin{zgled}
Naj grupa $S^1$ deluje na $\CC$ s predpisom
\[
S^1 \to \Isom(\CC), \ z \mapsto (x \mapsto z \cdot x).
\]
Velja $S^1.0 = \{ 0 \}$, $S^1.i = S^1$. V splošnem je $S^1.x$ ravno krožnica s polmerom $|x|$. Kvocientno množico $S^1 \backslash \CC$ lahko naravno identificiramo z množico $\RR_{\geq 0}$.
\end{zgled}

{\definicija Stabilizator} elementa $x \in X$ je $G_x = \{ g \in G \mid g.x = x \}$. To je vselej podgrupa grupe $G$. Dualen objekt stabilizatorju je {\definicija množica fiksnih točk} dane podmnožice $H \subseteq G$, to je $X^H = \{ x \in X \mid \forall h \in H \colon \ h.x = x \}$.

\begin{zgled}
Naj bo $K = [0,1] \times [0,1] \subseteq \RR^2$ kvadrat. Grupa $G = \Isom(K)$ je izomorfna diedrski grupi $D_8$, generirana je z rotacijo $s \colon K \to K$ (vrtenje za $2\pi/4$) in zrcaljenjem $t \colon K \to K$ (zrcaljenje prek glavne diagonale $(0,0)-(1,1)$).

Fiksne točke generatorjev so $K^t = \{ (x,x) \mid x \in [0,1] \}$, $K^s = \{ (1/2, 1/2 )\}$. Velja še $G.(0,0) = \{ (0,0), (1,0), (0,1) , (1,1) \}$, $G_{(0,0)} = \{ 1, t \}$, $G_{(1/3, 0)} = 1$, $K^G = \{ (1/2, 1/2) \}$. Točka $(1/2, 1/2)$ je torej globalna fiksna točka delovanja.
\end{zgled}

\begin{figure}[t]
\centering
\begin{tikzpicture}
    \coordinate (y) at (0,4);
    \coordinate (x) at (4,0);
    \draw[<->, thick] (y) node[above] {$y$} -- (0,0) --  (x) node[right] {$x$};
    \draw[thick, fill = vijolicna!10, draw = vijolicna] (0,0) -- (0,3) -- (3,3) -- (3,0) -- (0,0);
    \draw[->,thick, draw = vijolicna] (2.5,3.5) to [bend right = 30]  node[midway,above] {$\color{vijolicna} s$} (0.5,3.5);
    \draw[thick, draw = vijolicna, <->] (4.3,3.2) -- node[midway,below] {$\color{vijolicna} t$} (3.2, 4.3);
    \draw[thick, dashed] (-0.3,-0.3) -- (4,4);
\end{tikzpicture}
\caption{Delovanja generatorjev diedrske grupe $D_8$ na kvadratu}
\end{figure}

Stabilizatorji in orbite so povezani na naslednji način.

\begin{trditev}
Naj $G$ deluje na množici $X$. Naj bo $x \in X$. Tedaj je preslikava
\[
A_x \colon G/G_x \to G.x, \quad g G_x \mapsto g.x
\]
bijektivna, zato velja $|G/G_x| = |G.x|$. V primeru $|X|, |G| < \infty$ velja tudi
\[
|G \backslash X| = \frac{1}{|G|} \sum_{g \in G} |X^g|.
\]
\end{trditev}

\begin{dokaz}
\underline{$A_x$ je dobro definirana:} Če je $g_1 G_x = g_2 G_x$, potem $g_2^{-1} g_1 \in G_x$, zato $g_2^{-1} g_1 .x = x$, torej $g_2.x = g_1.x$. \kljuka

\underline{$A_x$ je surjektivna:} Velja po definiciji. \kljuka

\underline{$A_x$ je injektivna:} Če je $g_1.x = g_2.x$, potem $g_2^{-1}g_1.x=x$, zato $g_2^{-1}g_1 \in G_x$, torej je $g_2G_x = g_1G_x$. \kljuka

Za zadnji del opazujmo množico $F = \{ (g, x) \mid g \in G, \ x \in X,\ g.x = x \} \subseteq G \times X$. Velja $|F| = \sum_{g \in G} |X^g| = \sum_{x \in X} |G_x| = \sum_{x \in X} |G|/|G.x| = |G| \sum_{x \in X} 1/|G.x| = |G| \sum_{Y \in G \backslash X} \sum_{x \in Y} 1/|Y| = |G| \cdot |G \backslash X|$. \kljuka
\end{dokaz}

\begin{zgled}
Naj bo $p$ praštevilo, $C = \{ x_1, \dots, x_a \}$ za nek $a \in \NN$ in naj bo $\mathcal{N} = C^p$ množica vseh nizov dolžine $p$ iz črk $C$. Ciklična grupa $\ZZ/p\ZZ$ deluje na $\mathcal{N}$ s cikličnim zamikom:
\[
\ZZ/p\ZZ \to \Sym(\mathcal{N}), \quad
i \mapsto (c_1 \cdots c_p \mapsto c_{p - i + 1} \cdots c_p c_1 \cdots c_{p-i}).
\]
Število orbit tega delovanja je po zadnji trditvi enako
\[
|\ZZ/p\ZZ \backslash \mathcal{N}| = \frac{1}{p} \sum_{i \in \ZZ/p\ZZ} |\mathcal{N}^i|.
\]
Velja: $\mathcal{N}^0 = \mathcal{N}$ in za $i \neq 0$ je $\mathcal{N}^i = \{ x_1 \cdots x_1, x_2 \cdots x_2, \dots, x_a \cdots x_a \}$. Od tod sledi
\[
|\ZZ/p\ZZ \backslash \mathcal{N}| = \frac{1}{p} (|\mathcal{N}^0| + (p-1)|\mathcal{N}^1|) = \frac{1}{p} (a^p + (p-1)a).
\]
Ker velja $|\ZZ/p\ZZ \backslash \mathcal{N}| \in \NN$, sledi $p \mid a^p + (p-1)a$, torej
\[
\forall a \in \ZZ \colon \ a^p \equiv a \mod{p}.
\]
To je natanko mali Fermatov izrek.
\end{zgled}

\subsection{Tranzitivna delovanja}

Delovanje grupe $G$ na množici $X$ je {\definicija tranzitivno}, če je $|G \backslash X| = 1$, torej če obstaja ena sama orbita.

\begin{zgled}
Opazujmo Cayleyjevo delovanje grupe $G$ na $\Cay(G,S)$, kjer $\langle S \rangle = G$. To delovanje je vselej {\em tranzitivno na vozliščih}: za $x,y \in G$ element $y x^{-1}$ preslika vozlišče $x$ v $y$. To delovanje je tudi {\em prosto na vozliščih}: če je $g x = x$, potem sledi $g = 1$.
\end{zgled}

Izkaže se, da tranzitivnost in prostost {\em karakterizirata} Cayleyjeve grafe.

\begin{trditev}
Bodi $G$ grupa, ki deluje na {\em povezanem} grafu $X = (V, E)$. Če je to delovanje tranzitivno in prosto na množici $V$, potem za vsak $x \in V$ velja, da množica $S = \{ s \in G \mid \{ x, s.x \} \in E \}$ generira $G$ in $X \cong \Cay(G, S)$.
\end{trditev}

\begin{dokaz}
Definirajmo preslikavo $\varphi \colon G \to V, g \mapsto g.x$. Zaradi predpostavke o tranzitivnost in prostosti je $\varphi$ bijekcija. Torej lahko identificiramo $G$ in $V$ prek $\varphi$.

\underline{$S$ generira $G$:} Naj bo $g \in G$. Opazujmo neko pot v grafu $X$ od $x$ do $g.x$:
\[
x_0 = 0, \ x_1 = g_1.x, \ x_2 = g_2.x, \dots, x_n = g_n.x = g.x.
\]
za neke $g_i \in G$. Naj bo $s_j = g_j^{-1} \cdot g_{j+1}$. 

\begin{figure}[ht]
\centering
\begin{tikzpicture}
\tikzstyle{vertex}=[circle, fill = oranzna, minimum size = 5pt, inner sep = 0pt]

\coordinate (x0) at (0,0);
\coordinate (x1) at (1,1);
\coordinate (x2) at (2,0);
\coordinate (xn) at (5,0);

\draw[thick] (x0) -- (x1) -- (x2) -- (2.5,0.5);
\draw[thick, dotted] (2.5,0.5) -- (4.5,0.5);
\draw[thick] (4.5,0.5) -- (xn);
\node[vertex, label = below : {$x_0 = x$}] at (x0) {};
\node[vertex, label = above : {$x_1 = g_1.x$}] at (x1) {};
\node[vertex, label = below : {$x_2 = g_2.x$}] at (x2) {};
\node[vertex, label = below : {$x_n = g_n.x = g.x$}] at (xn) {};
\end{tikzpicture}
\end{figure}

Premislimo, da za vsak $j$ velja $s_j \in S$: vemo $x_j \sim x_{j+1}$, zato je $g_j^{-1}.x_j \sim g_j^{-1}.x_{j+1}$, kar se poenostavi v $x \sim g_j^{-1}g_{j+1}.x \sim s_j.x$, torej je res $x \sim s_j.x$.

Zdaj velja $g = g_n = g_0 g_0^{-1} g_1 g_1^{-1} \cdots g_{n-1}^{-1} g_n = g_0 s_1 s_2 \cdots s_{n-1} = s_1 \cdots s_{n-1}$. Torej res $S$ generira $G$. \kljuka

\underline{$X \cong \Cay(G,S)$:} Imamo že bijekcijo med vozlišči obeh grafov, $\varphi \colon G \to V$. Za $g,h \in G$ velja $\varphi(g) \sim \varphi(h)$, če in samo če je $g.x \sim h.x$, kar je enakovredno $x \sim g^{-1}h.x$, torej $g^{-1}h \in S$, to pa je nazadnje enakovredno $g \sim h$ v $\Cay(G,S)$. Torej $\varphi$ inducira izomorfizem grafov $X$ in $\Cay(G,S)$. \kljuka
\end{dokaz}

\section{Proste grupe in drevesa}

\subsection{Prosto delovanje proste grupe na drevesu}

Bodi $F$ prosta grupa, prosto generirana z množico $S \subseteq F$. Tedaj je $\Cay(F,S)$ drevo in $F$ deluje na tem drevesu s Cayleyjevim delovanjem. 

{\vprasanje Ali je to delovanje prosto?} Po trditvi iz prejšnjega razdelka vemo, da je temu tako, če in samo če $S$ ne vsebuje elementov reda $2$. Če obstaja $s_0 \in S$ reda $2$, potem lahko definiramo preslikavo $S \to \ZZ$, ki preslika $s_0$ v $1$, vse ostale elemente množice $S$ pa v $0$. Po univerzalni lastnosti proste grupe dobimo homomorfizem $\varphi \colon F \to \ZZ$ in velja račun $0 = \varphi(1) = \varphi(s_0^2) = \varphi(s_0)^2 = 2$. Protislovje. Torej tak $s_0$ ne obstaja in prosta grupa vselej deluje prosto na drevesu.

Dokazali bomo, da velja tudi obratno.

\begin{izrek}{\sc (prosta $\Leftrightarrow$ prosto na drevesu)}
Grupa je prosta, če in samo če deluje prosto na drevesu.
\end{izrek}

V eno smer smo izrek že premislili. Predpostavimo zdaj, da grupa $G$ {\em prosto} deluje na (morda neskončnem) drevesu $T$.

\begin{zgled}
Grupa $\ZZ$ deluje s pomikom v desno na drevesu, ki ga dobimo tako, da v Cayleyjevem grafu $\Cay(\ZZ, \{ 1 \})$ vsakemu vozlišču dodamo dva lista.
\end{zgled}

\begin{figure}[t]
\centering
\begin{tikzpicture}
\tikzstyle{vertex}=[circle, fill = oranzna, minimum size = 5pt, inner sep = 0pt]

\draw[thick] (-5, 0) -- (-4, 0);
\foreach \n in {-4,...,4}
    {
    \draw[thick] (\n, 0) -- (\n+1, 0);
    \draw[thick] (\n, 0) -- (\n, 1);
    \draw[thick] (\n, 0) -- (\n, -1);
    };

\foreach \n in {-4,...,4}
    {
    \node[vertex, label = below left : $\n$] at (\n, 0) {};
    \node[vertex] at (\n, 1) {};
    \node[vertex] at (\n, -1) {};
    };
\end{tikzpicture}
\caption{Drevo, na katerem prosto deluje grupa $\ZZ$ s pomikom v desno}
\end{figure}

Kot vidimo v zadnjem zgledu, drevo $T$ ni nujno Cayleyjev graf grupe $G$. Bomo pa pokazali, da lahko $T$ vselej kontraktiramo oziroma skrčimo do Cayleyjevega grafa grupe $G$, in sicer tako, da kontraktiramo določena poddrevesa v $T$.

\subsection{Vpeta drevesa delovanj}

Naj grupa $G$ deluje na grafu $X$ z avtomorfizmi. {\definicija Vpeto drevo delovanja} $G$ na $X$ je podgraf $X$, ki je drevo in ki vsebuje natanko eno vozlišče v vsaki od $G$-orbit na vozliščih $X$.

\begin{zgled}
\begin{itemize}
    \item Naj $\ZZ$ deluje na drevesu s pomikom v desno kot v zadnjem zgledu. Imamo $3$ orbite $\ZZ$ na vozliščih drevesa. Vpeto drevo je pot dolžine $2$ od enega od novih listov prek vozlišča na Cayleyjevem grafu do drugega od listov.
    
    \item Grupa $\ZZ/n\ZZ$ deluje z rotacijo na diskretnem torusu $\Cay(\ZZ/n\ZZ \times \ZZ/m\ZZ, \{ (1,0), (0,1) \})$. Število orbit na vozliščih je $m$. Vpeto drevo delovanja je v tem primeru $m$-cikel brez ene povezave.
\end{itemize}
\end{zgled}

\begin{figure}[t]
\centering
\begin{tikzpicture}
\tikzstyle{vertex}=[circle, fill = oranzna, minimum size = 5pt, inner sep = 0pt]

\draw [rounded corners=1em, draw = vijolicna, dashed, thick] (-4.4, 1.4) rectangle (4.4,0.6);
\draw [rounded corners=1em, draw = vijolicna, dashed, thick] (-4.4, 0.4) rectangle (4.4,-0.4);
\draw [rounded corners=1em, draw = vijolicna, dashed, thick] (-4.4, -0.6) rectangle (4.4,-1.4);

% \draw [rounded corners=1em, draw = zelena, thick, fill = zelena!50] (-0.4,1.4) rectangle (0.4,-1.4);

\draw[thick] (-5, 0) -- (-4, 0);
\foreach \n in {-4,...,4}
    {
    \draw[thick] (\n, 0) -- (\n+1, 0);
    \draw[thick] (\n, 0) -- (\n, 1);
    \draw[thick] (\n, 0) -- (\n, -1);
    };

\draw[line width = 3pt, draw = vijolicna] (0,-1) -- (0,0) -- (0,1);

\foreach \n in {-4,...,4}
    {
    \node[vertex, label = below left : $\n$] at (\n, 0) {};
    \node[vertex] at (\n, 1) {};
    \node[vertex] at (\n, -1) {};
    };

% \foreach \y in {-1,0,1}
    % \node[vertex, fill = vijolicna, minimum size = 10pt] at (0, \y) {};
\end{tikzpicture}
\caption{Vpeto drevo delovanja $\ZZ$ in tri orbite na vozliščih drevesa}
\end{figure}

\begin{figure}[t]
\centering
\begin{tikzpicture}[scale=1.5]
\tikzstyle{vertex}=[circle, fill = oranzna, minimum size = 5pt, inner sep = 0pt]

\foreach \n in {0,...,4}
    \foreach \m in {0,...,5}
        \draw[thick] ($(72*\n:1.5) + (60*\m:0.4)$) -- ($(72*\n:1.5) + (60*\m + 60:0.4)$);

\foreach \m in {0,...,5}
    \foreach \n in {0,...,4}
        \draw[thick] ($(72*\n:1.5) + (60*\m:0.4)$) -- ($(72*\n + 72:1.5) + (60*\m:0.4)$);

\draw[line width = 3pt, draw = vijolicna] ($(0:1.5) + (0:0.4)$) -- ($(0:1.5) + (60:0.4)$) -- ($(0:1.5) + (120:0.4)$) -- ($(0:1.5) + (180:0.4)$) -- ($(0:1.5) + (240:0.4)$) -- ($(0:1.5) + (300:0.4)$);

\foreach \n in {0,...,4}
    \foreach \m in {0,...,5}
        \node[vertex] at ($(72*\n:1.5) + (60*\m:0.4)$) {};


% \foreach \m in {0,...,5}
%     \node[vertex, fill = vijolicna, minimum size = 10pt] at ($(0:1.5) + (60*\m:0.4)$) {};

\draw[->,thick, draw = vijolicna] (0:0.6) to [bend right = 30]  node[midway,below left] {$\color{vijolicna} (1,0)$} (72:0.6);
\end{tikzpicture}
\caption{Vpeto drevo delovanja grupe $\ZZ/5\ZZ$ na diskretnem torusu $\Cay(\ZZ/5\ZZ \times \ZZ/6\ZZ, \{ (1,0), (0,1) \})$}
\end{figure}

\begin{trditev}
Vsako delovanje grupe na povezanem grafu ima vpeto drevo delovanja.
\end{trditev}

\begin{dokaz}
Naj $G$ deluje na grafu $X$. Naj bo $\mathcal{T}$ množica vseh poddreves v $X$, ki vsebujejo največ eno vozlišče iz vsake $G$-orbite. Velja $\mathcal{T} \neq \emptyset$, na primer $\mathcal{T}$ vsebuje drevo iz enega samega vozlišča. Množica $\mathcal{T}$ je delno urejena z relacijo inkluzije. Vsaka veriga v $\mathcal{T}$ ima zgornjo mejo (lahko vzamemo kar unijo dreves v verigi). Po Zornovi lemi zato obstaja maksimalen element v $\mathcal{T}$: izberimo enega in mu recimo $T$.

% \begin{figure}[h]
% \centering
% \begin{tikzpicture}
% \tikzstyle{vertex}=[circle, fill = oranzna, minimum size = 5pt, inner sep = 0pt]

% \node at (-1, 1) {$T:$};
% \node[vertex] (O) at (0,0) {};
% \node[vertex] (u) at (60:1) {};
% \draw[thick] (O) -- (u);
% \draw[thick] (u) -- ($(60:1) + (60:1)$);
% \draw[thick] (u) -- ($(60:1) + (-60:1)$);
% \draw[thick] (u) -- ($(60:1) + (0:1)$);
% \draw[thick] ($(60:1) + (0:1)$) -- ($(60:1) + (0:1) + (60:1)$);
% \draw[thick] ($(60:1) + (0:1)$) -- ($(60:1) + (0:1) + (-60:1)$);
% \node[vertex] at ($(60:1) + (60:1)$) {};
% \node[vertex] at ($(60:1) + (-60:1)$) {};
% \node[vertex] at ($(60:1) + (0:1)$) {};
% \node[vertex] at ($(60:1) + (0:1) + (60:1)$) {};
% \node[vertex] at ($(60:1) + (0:1) + (-60:1)$) {};
% \end{tikzpicture}
% \end{figure}

\underline{$T$ je vpeto drevo delovanja:} Recimo, da ni. Torej obstaja $v \in V(X)$, da je $T \cap G.v = \emptyset$. Premislimo, da obstaja taka izbira $v$, da je $v$ soseden nekemu vozlišču v $T$. Res, ker je $X$ povezan, obstaja pot $p$ od nekega vozlišča $u \in T$ do $v$. Naj bo $v^\prime$ prvo vozlišče na $p$, ki ni v $T$. 

\begin{figure}[h]
\centering
\begin{tikzpicture}
\tikzstyle{vertex}=[circle, fill = oranzna, minimum size = 5pt, inner sep = 0pt]

\draw[thick, dashed] (-0.25,-0.25) rectangle ($(60:1) + (0:1) + (60:1) + (0.25, 0.25)$);
\node at ($(60:1) + (0:1) + (-60:1) + (0.75, 0)$) {$T$};

\draw[thick] (0,0) -- (-1, 0);
\draw[thick] (-1,0) -- ($(-1,0) + (120:1)$);
\draw[thick] ($(-1,0) + (120:1)$) -- ($(-1,0) + (120:1) + (60:1)$);

\coordinate (O) at (0,0) {};
\coordinate (u) at (60:1) {};
\draw[thick] (O) -- (u);
\draw[thick] (u) -- ($(60:1) + (60:1)$);
\draw[thick] (u) -- ($(60:1) + (-60:1)$);
\draw[thick] (u) -- ($(60:1) + (0:1)$);
\draw[thick] ($(60:1) + (0:1)$) -- ($(60:1) + (0:1) + (60:1)$);
\draw[thick] ($(60:1) + (0:1)$) -- ($(60:1) + (0:1) + (-60:1)$);

\draw[line width = 3pt, draw = vijolicna] (60:1) -- (0,0) -- (-1,0) -- node[midway, below left] {\color{vijolicna} $p$} ($(-1,0) + (120:1)$) -- ($(-1,0) + (120:1) + (60:1)$);

\node[vertex] at (O) {};
\node[vertex, label = above : $u$] at (u) {};
\node[vertex, label = below : $v^\prime$] at (-1, 0) {};
\node[vertex] at ($(-1,0) + (120:1)$) {};
\node[vertex, label = right : $v$] at ($(-1,0) + (120:1) + (60:1)$) {};

\node[vertex] at ($(60:1) + (60:1)$) {};
\node[vertex] at ($(60:1) + (-60:1)$) {};
\node[vertex] at ($(60:1) + (0:1)$) {};
\node[vertex] at ($(60:1) + (0:1) + (60:1)$) {};
\node[vertex] at ($(60:1) + (0:1) + (-60:1)$) {};
\end{tikzpicture}
\end{figure}

\begin{description}
\item[Če $T \cap G.v^\prime = \emptyset$:] V tem primeru lahko $v$ nadomestimo z $v^\prime$. \kljuka 
\item[Sicer:] V tem primeru obstaja $g \in G$, da je $g.v^\prime \in T$. Naj bo $p^\prime$ odsek poti $p$ od $v^\prime$ do $v$. Torej je $g.p^\prime$ pot od $g.v^\prime \in T$ do $g.v$, pri čemer velja $G.g.v \cap T = G.v \cap T = \emptyset$ po izbiri $v$. Torej je $g.p^\prime$ krajša pot kot $p$ z enako lastnostjo (povezuje $T$ z vozliščem, katerega orbita ne seka $T$). Postopek ponavljamo in na koncu dobimo vozlišče z želeno lastnostjo. \kljuka
\end{description}

K drevesu $T$ dodajmo izbrano vozlišče $v$ in povezavo, ki ga povezuje z drevesom $T$. Dobimo novo drevo v $\mathcal{T}$, ki vsebuje $T$, kar je protislovje z maksimalnostjo drevesa $T \in \mathcal{T}$. Torej je res $T$ vpeto drevo delovanja. \kljuka
\end{dokaz}

\subsection{Kontrakcija in Cayleyjevo drevo}

\begin{dokaz}[Dokaz izreka {\sc (prosta $\Leftrightarrow$ prosto na drevesu)}]
Naj $G$ deluje prosto na drevesu $T$. Dokazujemo, da je $G$ prosta grupa. 

Naj bo $T^\prime$ vpeto drevo delovanja $G$ na $T$. Kontraktirajmo drevo $T^\prime$ v vozlišče, prav tako naredimo z orbitami $T^\prime$. Na ta način dobimo nov graf $X$ z:
\begin{itemize}
\item vozlišči $V = \{ g.T^\prime \mid g \in G \}$,
\item povezavami $E = \{ \{ g.T^\prime, h.T^\prime \} \mid g,h \in G, \ \text{$g.T^\prime$ in $h.T^\prime$ sta sosednji} \}$,
\end{itemize}
kjer sosednost $g.T^\prime$ in $h.T^\prime$ pomeni, da sta ti dve drevesi različni in da obstajata $u \in g.T^\prime$, $v \in h.T^\prime$ z $u \sim_T v$. 


\begin{figure}[h]
\centering
\begin{tikzpicture}
\tikzstyle{vertex}=[circle, fill = oranzna, minimum size = 5pt, inner sep = 0pt]

\def\offset{-4};
\node at ($(-0, 1) + (\offset, 0)$) {$T^\prime$};
\node[vertex] (O) at ($(0,0) + (\offset, 0)$) {};
\node[vertex] (u) at ($(60:1) + (\offset, 0)$) {};
\draw[thick] (O) -- (u);
\draw[thick] (u) -- ($(60:1) + (60:1) + (\offset, 0)$);
\draw[thick] (u) -- ($(60:1) + (-60:1) + (\offset, 0)$);
\draw[thick] (u) -- ($(60:1) + (0:1) + (\offset, 0)$);
\draw[thick] ($(60:1) + (0:1) + (\offset, 0)$) -- ($(60:1) + (0:1) + (60:1) + (\offset, 0)$);
\draw[thick] ($(60:1) + (0:1) + (\offset, 0)$) -- ($(60:1) + (0:1) + (-60:1) + (\offset, 0)$);
\node[vertex] at ($(60:1) + (60:1)+ (\offset, 0)$) {};
\node[vertex] at ($(60:1) + (-60:1)+ (\offset, 0)$) {};
\node[vertex] at ($(60:1) + (0:1)+ (\offset, 0)$) {};
\node[vertex] at ($(60:1) + (0:1) + (60:1)+ (\offset, 0)$) {};
\node[vertex] at ($(60:1) + (0:1) + (-60:1)+ (\offset, 0)$) {};

\draw[line width = 3pt, draw = vijolicna] ($(60:1) + (0:1) + (-60:1)$) to[bend left=30] ($(60:1) + (60:1)+ (4, 0)$);

\node at (-0, 1) {$g.T^\prime$};
\node[vertex] (O) at (0,0) {};
\node[vertex] (u) at (60:1) {};
\draw[thick] (O) -- (u);
\draw[thick] (u) -- ($(60:1) + (60:1)$);
\draw[thick] (u) -- ($(60:1) + (-60:1)$);
\draw[thick] (u) -- ($(60:1) + (0:1)$);
\draw[thick] ($(60:1) + (0:1)$) -- ($(60:1) + (0:1) + (60:1)$);
\draw[thick] ($(60:1) + (0:1)$) -- ($(60:1) + (0:1) + (-60:1)$);
\node[vertex] at ($(60:1) + (60:1)$) {};
\node[vertex] at ($(60:1) + (-60:1)$) {};
\node[vertex] at ($(60:1) + (0:1)$) {};
\node[vertex] at ($(60:1) + (0:1) + (60:1)$) {};
\node[vertex, label = right : $u$] at ($(60:1) + (0:1) + (-60:1)$) {};

\def\offset{4};
\node at ($(-0, 1) + (\offset, 0)$) {$h.T^\prime$};
\node[vertex] (O) at ($(0,0) + (\offset, 0)$) {};
\node[vertex] (u) at ($(60:1) + (\offset, 0)$) {};
\draw[thick] (O) -- (u);
\draw[thick] (u) -- ($(60:1) + (60:1) + (\offset, 0)$);
\draw[thick] (u) -- ($(60:1) + (-60:1) + (\offset, 0)$);
\draw[thick] (u) -- ($(60:1) + (0:1) + (\offset, 0)$);
\draw[thick] ($(60:1) + (0:1) + (\offset, 0)$) -- ($(60:1) + (0:1) + (60:1) + (\offset, 0)$);
\draw[thick] ($(60:1) + (0:1) + (\offset, 0)$) -- ($(60:1) + (0:1) + (-60:1) + (\offset, 0)$);
\node[vertex, label = right : $v$] at ($(60:1) + (60:1)+ (\offset, 0)$) {};
\node[vertex] at ($(60:1) + (-60:1)+ (\offset, 0)$) {};
\node[vertex] at ($(60:1) + (0:1)+ (\offset, 0)$) {};
\node[vertex] at ($(60:1) + (0:1) + (60:1)+ (\offset, 0)$) {};
\node[vertex] at ($(60:1) + (0:1) + (-60:1)+ (\offset, 0)$) {};
\end{tikzpicture}
\end{figure}

Za graf $X$ veljajo naslednje lastnosti:
\begin{itemize}
    \item Za vsaka različna elementa $g, h \in G$ je $g.T^\prime \cap h.T^\prime = \emptyset$. Ta lastnost sledi iz prostosti delovanja in izbire $T^\prime$. 
    \item Če sta $g.T^\prime, h.T^\prime$ sosednji, potem med njima obstaja natako ena povezava v $T$. Ta lastnost velja, ker v $T$ ni ciklov.
    \item Graf $X$ je drevo. Cikel v $X$ bi nam namreč porodil cikel v $T$.
    \item Grupa $G$ inducirano deluje na $X$, to delovanje je tranzitivno in prosto na vozliščih $X$.
\end{itemize}
Iz zadnje lastnosti po dokazani trditvi sledi, da obstaja $\tilde S \subseteq G$, da je $X \cong \Cay(G,\tilde S)$. 

\underline{$G$ nima elementov reda $2$:} Če je $s \in G$ reda $2$, potem je $\langle s \rangle$ končna grupa, ki prosto deluje na drevesu $T$. To ni mogoče, saj ima vsako delovanje končne grupe na drevesu globalno fiksno točko \vaje. \kljuka

Ker v $\tilde S$ ni elementov reda $2$, lahko izberemo $S \subseteq \tilde S$, za katero velja $S \cap S^{-1} = \emptyset$ in $\langle S \rangle = \langle \tilde S \rangle = G$. Še vedno je $\Cay(G,S)$ drevo. Iz trditve v razdelku o Cayleyjevem delovanju zdaj sledi, da $S$ prosto generira $G$.
\end{dokaz}

\subsection{Uporaba: podgrupe prostih grup}

\begin{posledica}{\sc (Nielsen--Schreier)}
Podgrupa proste grupe je prosta.
\end{posledica}

\begin{dokaz}
Naj bo $F$ prosta grupa in $G \leq F$. Grupa $F$ prosto deluje na drevesu, zato tudi $G$ prosto deluje na drevesu. Iz zadnjega izreka sledi, da je $G$ prosta.
\end{dokaz}

\begin{zgled}
Naj bo $F_2$ prosta grupa ranga $2$, generirana z elementoma $x,y$. Opazujmo preslikavo $\{ x,y \} \to S_3$, $x \mapsto (1 \ 2), y \mapsto (1 \ 2 \ 3)$. Po univerzalni lastnosti obstaja razširitev te preslikave do homomorfizma $\varphi \colon F_2 \to S_3$. Ta je surjektiven, saj velja $S_3 = \langle (1 \ 2), (1 \ 2 \ 3) \rangle$. Jedro $\ker \varphi$ je torej podgrupa indeksa $6$ v $F_2$ in je zato sama prosta. {\vprasanje Kateri prosti grupi je izomorfna grupa $\ker \varphi$?}
\end{zgled}
 
\begin{posledica}{\sc (kvantitativni Nielsen--Schreier)}
Naj bo $F$ prosta grupa ranga $n$ in naj bo $G \leq F$ pogrupa indeksa $k$. Potem je $G$ prosta ranga $k \cdot (n-1) + 1$.
\end{posledica}

\begin{dokaz}
Naj $S$ prosto generira $F$. Naj bo $T = \Cay(F,S)$ drevo, na katerem $F$ deluje prosto s Cayleyjevem delovanjem. 

Opazujmo zožitev tega delovanja na podgrupo $G \leq F$. Naj bo $T^\prime$ vpeto drevo delovanja $G$ na $T$. Za $f \in V(T) = F$ je orbita $G.f$ ravno desni odsek $G$ v $F$, torej je število vozlišč v $T^\prime$ enako $|F:G| = k$.

Po kontrakciji poddreves $g.T^\prime$ za $g \in G$ dobimo\footnote{Kot v dokazu izreka {\sc (prosta $\Leftrightarrow$ prosto na drevesu)}.} drevo $X \cong \Cay(G,S^\prime)$ za nek $S^\prime \subseteq G$ in grupa $G$ je prosta ranga $|S^\prime|$. Pri tem je $|S^\prime|$ enako polovici\footnote{V Cayleyjevem grafu imamo namreč dve povezavi za vsak $s \in S^\prime$: $s$ in $s^{-1}$.} števila povezav od $T^\prime$ do $g.T^\prime$ za vse $g \in G$.

\begin{figure}[h]
\centering
\begin{tikzpicture}
\tikzstyle{vertex}=[circle, fill = oranzna, minimum size = 5pt, inner sep = 0pt]

\draw[line width = 3pt, draw = vijolicna] ($(60:1) + (1,0) + (60:1)$) to[bend left=30] ($(60:1) + (60:1)+ (4, 0)$);
\draw[line width = 3pt, draw = vijolicna] ($(60:1) + (0:1) + (-60:1)+ (-4, 0)$) to[bend left=30] ($(0,0)$);

\def\offset{-4};
\node at ($(-0, 1) + (\offset, 0)$) {$g_1.T^\prime$};
\node[vertex] (O) at ($(0,0) + (\offset, 0)$) {};
\node[vertex] (u) at ($(60:1) + (\offset, 0)$) {};
\draw[thick] (O) -- (u);
\draw[thick] (u) -- ($(60:1) + (60:1) + (\offset, 0)$);
\draw[thick] (u) -- ($(60:1) + (-60:1) + (\offset, 0)$);
\draw[thick] (u) -- ($(60:1) + (0:1) + (\offset, 0)$);
\draw[thick] ($(60:1) + (0:1) + (\offset, 0)$) -- ($(60:1) + (0:1) + (60:1) + (\offset, 0)$);
\draw[thick] ($(60:1) + (0:1) + (\offset, 0)$) -- ($(60:1) + (0:1) + (-60:1) + (\offset, 0)$);
\node[vertex] at ($(60:1) + (60:1)+ (\offset, 0)$) {};
\node[vertex] at ($(60:1) + (-60:1)+ (\offset, 0)$) {};
\node[vertex] at ($(60:1) + (0:1)+ (\offset, 0)$) {};
\node[vertex] at ($(60:1) + (0:1) + (60:1)+ (\offset, 0)$) {};
\node[vertex] at ($(60:1) + (0:1) + (-60:1)+ (\offset, 0)$) {};



\node at (-0, 1) {$T^\prime$};
\node[vertex] (O) at (0,0) {};
\node[vertex] (u) at (60:1) {};
\draw[thick] (O) -- (u);
\draw[thick] (u) -- ($(60:1) + (60:1)$);
\draw[thick] (u) -- ($(60:1) + (-60:1)$);
\draw[thick] (u) -- ($(60:1) + (0:1)$);
\draw[thick] ($(60:1) + (0:1)$) -- ($(60:1) + (0:1) + (60:1)$);
\draw[thick] ($(60:1) + (0:1)$) -- ($(60:1) + (0:1) + (-60:1)$);
\node[vertex] at ($(60:1) + (60:1)$) {};
\node[vertex] at ($(60:1) + (-60:1)$) {};
\node[vertex] at ($(60:1) + (0:1)$) {};
\node[vertex] at ($(60:1) + (0:1) + (60:1)$) {};
\node[vertex] at ($(60:1) + (0:1) + (-60:1)$) {};

\def\offset{4};
\node at ($(-0, 1) + (\offset, 0)$) {$g_2.T^\prime$};
\node[vertex] (O) at ($(0,0) + (\offset, 0)$) {};
\node[vertex] (u) at ($(60:1) + (\offset, 0)$) {};
\draw[thick] (O) -- (u);
\draw[thick] (u) -- ($(60:1) + (60:1) + (\offset, 0)$);
\draw[thick] (u) -- ($(60:1) + (-60:1) + (\offset, 0)$);
\draw[thick] (u) -- ($(60:1) + (0:1) + (\offset, 0)$);
\draw[thick] ($(60:1) + (0:1) + (\offset, 0)$) -- ($(60:1) + (0:1) + (60:1) + (\offset, 0)$);
\draw[thick] ($(60:1) + (0:1) + (\offset, 0)$) -- ($(60:1) + (0:1) + (-60:1) + (\offset, 0)$);
\node[vertex] at ($(60:1) + (60:1)+ (\offset, 0)$) {};
\node[vertex] at ($(60:1) + (-60:1)+ (\offset, 0)$) {};
\node[vertex] at ($(60:1) + (0:1)+ (\offset, 0)$) {};
\node[vertex] at ($(60:1) + (0:1) + (60:1)+ (\offset, 0)$) {};
\node[vertex] at ($(60:1) + (0:1) + (-60:1)+ (\offset, 0)$) {};
\end{tikzpicture}
\end{figure}

Jasno velja
\[
\sum_{v \in V(T^\prime)} \deg_T(v) = |V(T^\prime)| \cdot 2n = k \cdot 2n.
\]
Po drugi strani pa lahko zadnjo vsoto izračunamo tako, da najprej upoštevamo notranje povezave v $T^\prime$, ki jih je natanko $2 \cdot (k-1)$ (vsako od $k-1$ povezav v $T^\prime$ preštejemo dvakrat), in za tem še povezave iz $T^\prime$ do drugih $g.T^\prime$ za $g \in G$, ki jih je natanko $2 \cdot |S^\prime|$. Sledi
\[
|S^\prime| = \frac12 (2kn - 2(k-1)) = k(n-1) + 1.
\]
\end{dokaz}

\begin{zgled}
Nadaljujemo zadnji zgled. Ker velja $|F_2:\ker \varphi| = 6$, sledi $\ker \varphi \cong F_7$.
\end{zgled}





\begin{posledica}
Prosta grupa ranga vsaj $2$ vsebuje proste podgrupe poljubno velikega ranga.
\end{posledica}

\begin{dokaz}
Sestavimo surjektiven homomorfizem $\varphi \colon F_r \to S_n = \langle (1 \ 2), (1 \ 2 \ \cdots \ n) \rangle$. Jedro $\ker \varphi$ je prosta grupa ranga $n! \cdot (r-1) + 1$, kar raste čez vse meje ko gre $n \to \infty$. Hkrati $\ker \varphi$ vsebuje proste grupe vsakega ranga manjšega od $n! (r - 1) + 1$.
\end{dokaz}

\begin{posledica}
Podgrupa končnega indeksa v končno generirani grupi je končno generirana.
\end{posledica}

\begin{dokaz}
Naj bo $G$ končno generirana grupa in naj bo $H \leq G$ podgrupa končnega indeksa. Naj bo $G = \langle S \rangle$, $|S| < \infty$.  Naj bo $\pi \colon F(S) \to G$ surjektiven homomorfizem. Naj bo $H^\prime = \pi^{-1}(H) \leq F(S)$. Velja $|F(S) : H^\prime| = |F(S)/\ker \pi : H^\prime/\ker \pi| = |G : H| < \infty$. Torej je $H^\prime$ končno generirana prosta grupa. Njena slika $H = \pi(H^\prime)$ je zato tudi končno generirana.
\end{dokaz}

Premislek pokaže \vaje, da je število podgrup danega indeksa v končno generirani grupi $G$ vselej končno, njihov presek pa je podgrupa edinka končnega indeksa v $G$.

\subsection{Pingpong}

Vzpostavili bomo še en geometrijski kriterij za prepoznavanje prostih grup. 

\begin{izrek}{\sc (pingpong)}
Naj bo $G = \langle a,b \rangle$ grupa z delovanjem na množici $X$. Predpostavimo, da obstajata neprazni podmnožici $A,B \subseteq X$, $B \not\subseteq A$, ki zadoščata {\em pingpong pogoju}:
\[
\forall n \in \ZZ \backslash \{ 0 \} \colon \ a^n.B \subseteq A \text{ in } b^n.A \subseteq B.
\]
Tedaj je $G$ prosto generirana z $\{ a, b \}$.
\end{izrek}

\begin{figure}[t]
\centering
\begin{tikzpicture}[scale=1.7]
\draw[draw = vijolicna, thick, fill = vijolicna!10] (0,0) rectangle (1,1);
\node[label = below : $\color{vijolicna} A$] at (0.5,0) {};
\draw[draw = rdeca, thick, fill = rdeca!10] (2,0) rectangle (3,1);
\node[label = below : $\color{rdeca} B$] at (2.5,0) {};

\draw[->, thick] (1.75,1) to [bend right = 30] node[midway,above] {$a$} (1.25,1);
\draw[->, thick] (1.25,0) to [bend right = 30] node[midway,below] {$b$} (1.75,0);
\end{tikzpicture}
\caption{Pingpong}
\end{figure}

\begin{dokaz}
Naj bo $F = F(\{ \alpha, \beta \})$ prosto generirana s simboloma $\alpha, \beta$. Po univerzalni lastnosti dobimo homomorfizem $\varphi \colon F \to G$, $\alpha \mapsto a, \beta \mapsto b$. Predpostavimo, da $\varphi$ {\em ni} injektiven. Naj bo torej $w \in F$ okrajšana beseda v $\ker \varphi$. Obravnavajmo štiri možnosti glede na začetek in konec besede $w$.

\begin{description}
\item[$w$ se začne in konča z $\alpha$:] V tem primeru je $w = \alpha^{n_0} \beta^{m_1} \alpha^{n_1} \cdots \beta^{m_k} \alpha^{n_k}$ za $n_i, m_i \in \ZZ \backslash \{ 0 \}$. Jasno je $\varphi(w).B = B$, po drugi strani pa po pingpong pogoju velja
\[
\varphi(w).B = a^{n_0} b^{m_1} a^{n_1} \cdots b^{m_k} a^{n_k}.B \subseteq a^{n_0} b^{m_1} a^{n_1} \cdots b^{m_k}.A \subseteq \cdots \subseteq a^{n_0}.B \subseteq A.
\]
Protislovje s predpostavko $B \not\subseteq A$.

\item[$w$ se začne in konča z $\beta$:] V tem primeru je konjugirana beseda $\alpha w \alpha^{-1}$ okrajšana beseda v jedru $\ker \varphi$, ki je enake oblike kot v prvem primeru. Protislovje.

\item[$w$ se začne z $\alpha$ in konča z $\beta$:] V tem primeru je $w = \alpha^n w^\prime \beta^m$ za $n,m \in \ZZ \backslash \{ 0 \}$. Tako je konjugirana beseda $\alpha^k w \alpha^{-k} = \alpha^{k+n} w^\prime \beta^m \alpha^{-k}$ okrajšana beseda v $\ker \varphi$, ki je enake oblike kot v prvem primeru. Protislovje.

\item[$w$ se začne z $\beta$ in konča z $\alpha$:] V tem primeru je $w^{-1}$ okrajšana beseda v $\ker \varphi$, ki je kot v tretjem primeru. Protislovje.
\end{description}
\end{dokaz}

\begin{zgled}
Opazujmo grupo matrik $\SL_2(\ZZ) = \{ A \in \GL_2(\ZZ) \mid \det A = 1 \}$. Izberimo njuna elementa
\[
a = \begin{pmatrix}
1 & 2 \\ 0 & 1
\end{pmatrix},
\quad
b = \begin{pmatrix}
1 & 0 \\ 2 & 1
\end{pmatrix}.
\]
Dokažimo, da je $\langle a, b \rangle$ prosta grupa, prosto generirana z $\{ a, b \}$. V ta namen uporabimo {\sc (pingpong)}. Grupa $\SL_2(\ZZ)$ naravno deluje na ravnini $\RR^2$. Razdelimo jo na dva dela:
\[
A = \{ (x,y) \in \RR^2 \mid |x| > |y| \}, \quad
B = \{ (x,y) \in \RR^2 \mid |x| < |y| \}.
\]
Za vsak $n \in \ZZ \backslash \{ 0 \}$ in $(x,y) \in B$ velja
\[
a^n \begin{pmatrix} x \\ y \end{pmatrix} = \begin{pmatrix} x + 2n y \\ y \end{pmatrix} \in A,
\]
kjer zadnja vsebovanost sledi iz $|x + 2ny| \geq |2ny| - |x| \geq 2 |y| - |x| > 2 |y| - |y| = |y|$. Torej je res $a^n.B \subseteq A$. Podobno preverimo $b^n.A \subseteq B$.

Z nekaj dodatnega premisleka se da pokazati \vaje, da velja celo $|\SL_2(\ZZ) : \langle a,b \rangle| = 12$. Torej $\SL_2(\ZZ)$ vsebuje prosto podgrupo končnega indeksa.
\end{zgled}

\begin{figure}[t]
\centering
\begin{tikzpicture}[scale=0.7]

\draw[thick, dashed, draw = none, fill = vijolicna!10] (4,-4) -- (0,0) -- (4,4);
\draw[thick, dashed, draw = none, fill = vijolicna!10] (-4,-4) -- (0,0) -- (-4,4);
\draw[thick, dashed, draw = none, fill = rdeca!10] (-4,-4) -- (0,0) -- (4,-4);
\draw[thick, dashed, draw = none, fill = rdeca!10] (-4,4) -- (0,0) -- (4,4);

\draw[thick, dashed] (-4,-4) -- (4, 4);
\draw[thick, dashed] (-4,4) -- (4, -4);

\draw[<-, thick] (0,4) node[above] {$y$} -- (0,-4);
\draw[<-, thick] (4,0) node[right] {$x$} -- (-4,0);

\node[label = below : $\color{vijolicna} A$] at (3, 1) {};
\node[label = left : $\color{rdeca} B$] at (1, 3) {};
\end{tikzpicture}
\caption{Pingpong v ravnini $\RR^2$ glede na delovanje grupe $\SL_2(\ZZ)$}
\end{figure}

Naj bo $\mathcal{P}$ grupno-teoretična lastnost.\footnote{Na primer, $\mathcal{P}$ je lahko {\em je abelova grupa} ali {\em je prosta grupa}.} Če ima grupa $G$ podgrupo $H$, ki zadošča $\mathcal{P}$ in je končnega indeksa v $G$, potem rečemo, da $G$ {\definicija virtualno zadošča $\mathcal{P}$}.

\begin{zgled}
\begin{itemize}
    \item $\SL_2(\ZZ)$ je virtualno prosta grupa.
    \item Iz kvantitativnega Nielsen--Schreierjevega izreka sledi, da je prosta grupa ranga $2$ virtualno prosta ranga $k$ za vsak $k \geq 2$.
    \item Vsaka končna grupa je virtualno trivialna.
\end{itemize}
\end{zgled}

\chapter{Struktura grup}

V tem razdelku si bomo ogledali, kako lahko grupe {\em razrežemo} na preproste kose.

\section{Rešljive grupe}

\subsection{Rešljivost}

Grupa $G$ je {\definicija rešljiva}, če ima verigo podgrup
\[
G = G_1 \unrhd G_2 \unrhd \cdots \unrhd G_n = 1,
\]
katerih faktorji $G_i/G_{i+1}$ so abelovi za vsak $i = 1, 2, \dots, n-1$. Ekvivalentno, velja $\forall x,y \in G_i \colon\  [x,y] \in G_{i+1}$. Rešljive grupe so torej grupe, ki jih lahko {\em razrežemo} na abelove kose.

Za podgrupi $H,K \leq G$ uvedimo oznako
\[
[H,K] = \langle [h,k] \mid h \in H, k \in K \rangle.
\]
Pogoj, da je $G_i/G_{i+1}$ abelova, je ekvivalenten $[G_i, G_i] \leq G_{i+1}$.

\subsection{Izvedena vrsta}

V grupi $G$ induktivno definiramo njeno {\definicija izvedeno vrsto podgrup}
\[
G^{(1)} = G, \quad
G^{(i+1)} = [G^{(i)}, G^{(i)}] \text{ za $i \geq 1$.}
\]

Premislimo naprej, da velja $G^{(i+1)} \unlhd G^{(i)}$. Naj bo $w \in G^{(i+1)}$, torej $w = \prod_{j} [a_j, b_j]$ za neke $a_j, b_j \in G^{(i)}$. Za vsak $g \in G^{(i)}$ tedaj velja
\[
g w g^{-1} = \prod_j g [a_j, b_j] g^{-1} = \prod_j [g a_j g^{-1}, g b_j g^{-1}] \in \prod_j [G^{(i)}, G^{(i)}] \subseteq G^{(i+1)}.
\]

Opazimo, da je po definiciji grupa $G^{(i)}/G^{(i+1)}$ vselej abelova. Če torej za nek $n \in \NN$ velja $G^{(n)} = 1$, potem je $G$ rešljiva. Res pa je tudi obratno.

\begin{trditev}
Grupa $G$ je rešljiva, če in samo če velja $G^{(n)} = 1$ za nek $n \in \NN$.
\end{trditev}

\begin{dokaz}
Predpostavimo, da je $G$ rešljiva in da njeno rešljivost opazi veriga $G = G_1 \unrhd G_2 \unrhd \cdots \unrhd G_n = 1$. Velja torej $[G_i, G_i] \leq G_{i+1}$ za vsak $i$. Sledi $G^{(1)} = G_1$, $G^{(2)} = [G_1, G_1] \leq G_2$, $G^{(3)} = [G^{(2)}, G^{(2)}] \leq [G_2, G_2] \leq G_3$, \dots Induktivno sklepamo, da za vsak $i$ velja $G^{(i)} \leq G_i$. Zato mora veljati tudi $G^{(n)} = 1$.
\end{dokaz}

Najmanjše število $n \in \NN$ z lastnostjo $G^{(n+1)} = 1$ je {\definicija izvedena dolžina} grupe $G$.

\begin{zgled}
\begin{itemize}
    \item Abelova grupa je rešljiva izvedene dolžine $1$.
    \item Opazujmo diedrsko grupo $D_{2n} \cong \ZZ/n\ZZ \rtimes \ZZ/2\ZZ \cong \langle s, t \mid s^n, t^2, t s t^{-1} = s^{-1} \rangle$. Velja $[s,t] = s^{-2} \in D_{2n}^{(2)}$, zato $\langle s^2 \rangle \in D_{2n}^{(2)}$. Predpostavimo, da je $n$ sodo število. Tedaj je kvocient $D_{2n}/\langle s^2 \rangle$ izomorfen grupi $\langle s, t \mid s^2, t^2, t s t^{-1} = s^{-1} \rangle = \langle s, t \mid s^2, t^2, st=ts \rangle \cong \ZZ/2\ZZ \times \ZZ/2\ZZ$. Ker je slednja grupa abelova, od tod sledi $D_{2n}^{(2)} \leq \langle s^2 \rangle$ in zato velja $D_{2n}^{(2)} = \langle s^2 \rangle$. Slednja grupa je abelova, zato je nazadnje $D_{2n}^{(3)} = 1$. Torej je $D_{2n}$ rešljiva izvedene dolžine $2$. Če je $n$ liho število, podoben argument pokaže, da je $D_{2n}^{(2)} = \langle s^2 \rangle = \langle s \rangle$ in $D_{2n}^{(3)} = 1$, torej je tudi v tem primeru grupa rešljiva izvedene dolžine $2$.
\end{itemize}
\end{zgled}

S pomočjo izvedene vrste lahko preprosto izpeljemo dednost rešljivosti v osnovnih primerih konstrukcij novih grup.

\begin{trditev}
Podgrupe in kvocienti rešljivih grup so rešljivi. Razširitev rešljive grupe z rešljivo je rešljiva.
\end{trditev}

\begin{dokaz}
Naj bo $G$ rešljiva grupa z $G^{(n)} = 1$. 

Za podgrupo $H \leq G$ preprost induktiven argument dokaže $H^{(i)} \leq G^{(i)}$ za vsak $i$. V posebnem sledi $H^{(n)} = 1$. \kljuka

Za $N \unlhd G$ velja $(G/N)^{(n)} = G^{(n)}N/N = 1$. \kljuka

Naj bo nazadnje $K$ še ena rešljiva grupa z $K^{(m)} = 1$. Naj bo $E$ razširitev $K$ z $G$. Tedaj velja $E^{(m)}G/G = (E/G)^{(m)} = K^m = 1$, zato je $E^{(m)} \leq G$. Od tod induktivno sledi $E^{(m+i)} \leq G^{(i)}$ za vsak $i$, zato v posebnem velja $E^{(m+n)} = 1$. \kljuka
\end{dokaz}

\begin{zgled}
Svetilkarjeva grupa $L = (\bigoplus_{i \in \ZZ} \ZZ/2\ZZ) \rtimes \ZZ$ je razširitev abelove grupe z abelovo, torej je rešljiva.
\end{zgled}

\subsection{Noetherskost}

\begin{trditev}
Bodi $G$ grupa. Naslednje trditve so ekvivalentne:
\vspace{-0.5\baselineskip}
\begin{enumerate}[noitemsep]
    \item Vsaka neprazna množica podgrup grupe $G$ ima maksimalen element.
    \item Vsaka strogo naraščajoča veriga podgrup $G_1 < G_2 < \cdots$ grupe $G$ je končna.
    \item Vsaka podgrupa grupe $G$ je končno generirana. \vspace{-\baselineskip}
\end{enumerate}
\end{trditev}

Grupe, ki zadoščajo tem pogojem, imenujemo {\definicija noetherske}.

\begin{dokaz}
\underline{1. $\Rightarrow$ 2.:} Imejmo verigo $G_1 < G_2 < \cdots$ v $G$. Tedaj ima množica $\{ G_1, G_2, \dots \}$ po predpostavki maksimalen element. Torej je veriga končna. \kljuka

\underline{2. $\Rightarrow$ 1.:} Uporabimo Zornovo lemo. Ker ima vsaka veriga zgornjo mejo (namreč, zadnji člen verige), obstaja maksimalen element katerekoli množice podgrup. \kljuka

\underline{2. $\Rightarrow$ 3.:} Naj bo $H \leq G$. Izberimo neko množico generatorjev $S$ grupe $H$. Naj bo $s_1 \in S$ in $H_1 = \langle s_1 \rangle$. Če je $H_1 = H$, je $H$ končno generirana. Predpostavimo zdaj $H_1 < H$. Naj bo $s_2 \in S \backslash H_1$ in $H_2 = \langle s_1, s_2 \rangle$. Če je $H_2 = H$, je $H$ končno generirana. Predpostavimo zdaj $H_1 < H_2 < H$. Postopek nadaljujemo in dobimo naraščajočo verigo podgrup v $H$. Ker je vsaka veriga v $G$ končna, se mora ta postopek ustaviti. Torej obstaja $i \in \NN$, da je $H_i = H$. S tem je $H = \langle s_1, s_2, \dots, s_i \rangle$. \kljuka

\underline{3. $\Rightarrow$ 2.:} Naj bo $G_1 < G_2 < \cdots$ neskončna veriga v $G$. Naj bo $H = \bigcup_{i \in \NN} G_i$. Velja $H \leq G$, zato je po predpostavki $H = \langle h_1, \cdots, h_n \rangle$ za neke $h_i \in H$. Za vsak generator $h_i$ obstaja člen verige, ki ga vsebuje. Torej obstaja člen verige, recimo $G_k$, ki vsebuje vse $h_1, \dots, h_n$. Torej $G_k \geq H$, zato je $G_i \leq G_k$ za vsak $i \in \NN$. Veriga je torej končna. Protislovje. \kljuka 
\end{dokaz}

Ni se težko prepričati \vaje, da je lastnost noetherskosti zaprta za podgrupe, kvociente in razširitve.

\begin{zgled}
\begin{itemize}
    \item $\ZZ$ je noetherska grupa.
    \item Končne grupe so noetherske.
    \item Iz zgodnjih dveh primerov sledi, da je vsaka končno generirana abelova grupa noetherska.
    \item $\QQ$ ni noetherska grupa. Vsebuje namreč podgrupe $H_k = \{ \frac{a}{2^i} \mid a \in \ZZ, \ i \in \NN_0, \ i \leq k \}$ za $k \in \NN_0$, ki tvorijo neskončno verigo
    \[
    \ZZ = H_0 < H_1 < H_2 < \cdots < \QQ.
    \]
\end{itemize}
\end{zgled}

\begin{trditev}
Rešljiva grupa je noetherska, če in samo če ima verigo
\[
G = G_1 \unrhd G_2 \unrhd \cdots \unrhd G_n = 1,
\]
kjer so vsi kvocienti $G_i/G_{i+1}$ {\em ciklični}.
\end{trditev}

Rešljivim noetherskim grupam pravimo {\definicija policiklične}.

\begin{dokaz}
\underline{($\Leftarrow$):} Ciklične grupe so noetherske. Tako je najprej $G_{n-1}$ noetherska. Ker je $G_{n-2}$ razširitev ciklične grupe $G_{n-2}/G_{n-1}$ z grupo $G_{n-1}$, je tudi $G_{n-2}$ noetherska. Induktivno na ta način sklepamo, da je $G_i$ noetherska za vsak $i$. V posebnem je $G = G_1$ noetherska. \kljuka

\underline{($\Rightarrow$):} Bodi $G$ rešljiva z verigo $G = G_1 \unrhd G_2 \unrhd \cdots \unrhd G_n = 1$ z abelovimi kvocienti. Ker je $G$ noetherska, je noetherski tudi vsak člen verige $G_i$, zato je $G_i/G_{i+1}$ noetherska, torej je nazadnje $G_i/G_{i+1}$ končno generirana abelova grupa. Zapišemo jo lahko kot direktno vsoto cikličnih grup:
\[
G_i/G_{i+1} = C_1 \oplus C_2 \oplus \cdots \oplus C_k.
\]
Naj bo $H_\ell = \left( \bigoplus_{j = 1}^\ell C_j \right)G_{i+1} \leq G_i$ delna vsota cikličnih podgrup, dvignjena v grupo $G_i$. Torej imamo verigo podgrup
\[
G_{i+1} = H_0 \leq H_1 \leq \cdots \leq H_k = G_i
\]
s kvocienti $H_{j+1}/H_j \cong C_{j+1}$. Na ta način lahko v grupi $G$ originalno verigo med vsakima členoma $G_{i+1} \unlhd G_i$ pofinimo z vmesnimi podgrupami $H_j$, tako da so v novi verigi vsi kvocienti ciklični. Torej je $G$ policiklična. \kljuka
\end{dokaz}

\begin{zgled}
\begin{itemize}
    \item $\QQ$ ni noetherska, torej ni policiklična grupa.
    \item Svetilkarjeva grupa $L$ ima podgrupo $\bigoplus_{i \in \ZZ} \ZZ/2\ZZ$, ki {\em ni} končno generirana. Torej $L$ {\em ni} noetherska.
    \item Končne rešljive grupe so policiklične, na primer $D_{2n}$ je policiklična.
    \item Neskončna diedrska grupa $D_{\infty} = \ZZ \rtimes \ZZ/2\ZZ$ je policiklična.
\end{itemize}
\end{zgled}

\subsection{Hirschova dolžina}

\begin{lema}
Naj ima grupa $G$ dve verigi podgrup:
\[
G = G_1 \unrhd G_2 \unrhd \cdots \unrhd G_n = 1 \quad \text{\vprasanje ($G$-veriga)}
\]
in
\[
G = H_1 \unrhd H_2 \unrhd \cdots \unrhd H_m = 1. \quad \text{\vprasanje ($H$-veriga)}
\]
Tedaj obstajata verigi podgrup
\[
G = K_{1,1} \unrhd K_{1,2} \unrhd \cdots \unrhd K_{1,n} \unrhd K_{2,1} \unrhd \cdots \unrhd K_{m,n} = 1 \quad \text{\vprasanje ($K$-veriga)}
\]
in
\[
G = L_{1,1} \unrhd L_{1,2} \unrhd \cdots \unrhd L_{1,m} \unrhd L_{2,1} \unrhd \cdots \unrhd L_{n,m} = 1, \quad \text{\vprasanje ($L$-veriga)}
\]
pri čemer je $K$-veriga pofinitev $H$-verige (= vsi členi $H$-verige se pojavijo v $K$-verigi), $L$-veriga je pofinitev $G$-verige in kvocienti $K$-verige so do permutacije členov natančno enaki kvocientom $L$-verige. 
\end{lema}

\begin{dokaz}
Pofinimo $H$-verigo s pomočjo $G$-verige. Najprej velja
\[
G = H_1 \unrhd H_2 G_2 \unrhd H_2 G_3 \unrhd \cdots \unrhd H_2 G_n = H_2,
\]
za tem imamo verigo
\[
H_2 \unrhd H_3 (G_2 \cap H_2) \unrhd H_3 (G_3 \cap H_2) \unrhd \cdots \unrhd H_3 (G_n \cap H_2) = H_3,
\]
med splošnima členoma $H_i \unrhd H_{i+1}$ pa imamo verigo
\[
H_i \unrhd H_{i+1}(G_2 \cap H_i) \unrhd H_{i+1}(G_3 \cap H_i) \unrhd \cdots \unrhd H_{i+1}(G_n \cap H_i) = H_{i+1}.
\]
Označimo $K_{i,j} = H_{i+1}(G_j \cap H_i)$. Tako dobimo $K$-verigo,
\[
G = K_{1,1} \unrhd K_{1,2} \unrhd \cdots \unrhd K_{m,n} = 1,
\]
ki je pofinitev $H$-verige, saj velja $K_{i,1} = H_i$.

Analogno lahko pofinimo $G$-verigo s pomočjo $H$-verige. Naj bo $L_{j,i} = G_{j+1}(H_i \cap G_j)$. Tako dobimo $L$-verigo,
\[
G = L_{1,1} \unrhd L_{1,2} \unrhd \cdots \unrhd L_{n,m} = 1,
\]
ki je pofinitev $G$-verige, saj velja $L_{j,1} = G_j$.

Obe novi verigi imata enako število členov ($n\cdot m$). Za kvociente $K$-verige velja
\[
K_{i,j}/K_{i,j+1} = \frac{H_{i+1}(G_j \cap H_i)}{H_{i+1}(G_{j+1} \cap H_i)} \cong \frac{H_{i+1}(G_j \cap H_i) / H_{i+1}}{H_{i+1}(G_{j+1} \cap H_i) / H_{i+1}}.
\]
Zdaj uporabimo drugi izrek o izomorfizmu\footnote{Natančneje, za grupe $X,Y,Z$, $Z \leq X$, velja $XY/Y \cong X/(X \cap Y)$, ta izomorfizem pa prenese podgrupo $ZY/Y$ v $Z(X \cap Y)/(X \cap Y)$.}, od koder sledi
\[
K_{i,j}/K_{i,j+1} \cong \frac{(G_j \cap H_i) / (G_j \cap H_{i+1})}{(G_{j+1} \cap H_i) (G_j \cap H_{i+1}) / (G_j \cap H_{i+1})}
\cong \frac{G_j \cap H_i}{(G_{j+1} \cap H_i) (G_j \cap H_{i+1})}.
\]
Zadnja grupa je simetrična glede na menjavo $G$-vrste in $H$-vrste. Torej bi do enakega rezultata prišli z analizo kvocientov $L$-vrste. Od tod sledi
\[
K_{i,j}/K_{i,j+1} \cong L_{j,i} / L_{j, i + 1}.
\]
Dokaz je s tem zaključen.
\end{dokaz}

\begin{trditev}
Bodi $G$ policiklična grupa. Vsaki verigi, ki opazita policikličnost $G$, imata enako število neskončnih kvocientov.
\end{trditev}

\begin{dokaz}
Po lemi imata vsaki dve verigi pofinitev z enakimi cikličnimi kvocienti (do permutacije natančno). Zato je trditev dovolj dokazati za primer, ko je prva veriga pofinitev druge. Tako pofinitev dobimo z zaporednim vstavljanjem enega novega člena med dva obstoječa: $G_i < H < G_{i-1}$. Ločimo dva primera.
\begin{description}
    \item[Če je $|G_{i-1}/G_i| < \infty$:] V tem primeru ima nova veriga (z dodanim členom $H$) enako število neskončnih kvocientov. \kljuka
    \item[Če je $G_{i-1}/G_i \cong \ZZ$:] V tem primeru je nujno $H/G_i \cong \ZZ$ in $|G_{i-1}/H| < \infty$. Torej je število neskončnih kvocientov v stari in novi verigi zopet enako. \kljuka 
\end{description}
\end{dokaz}

Število neskončnih kvocientov v katerikoli verigi policiklične grupe $G$ se imenuje {\definicija Hirschova dolžina}, oznaka $h(G)$.

\begin{zgled}
\begin{itemize}
    \item Hirschova dolžina $D_\infty = \ZZ \rtimes \ZZ/2\ZZ$ je enaka $h(D_\infty) = 1$.
    \item Opazujmo {\definicija Heisenbergovo grupo}
    \[
    H = \left\{ \begin{pmatrix}
    1 & a & b \\ 0 & 1 & c \\ 0 & 0 & 1
    \end{pmatrix} \mid a,b,c \in \ZZ \right\} \leq \GL_3(\ZZ).
    \]
    Ta grupa je policiklična, kar opazi veriga podgrup
    \[
    H \unrhd 
    \left\{ \begin{pmatrix}
    1 & a & b \\ 0 & 1 & 0 \\ 0 & 0 & 1
    \end{pmatrix} \mid a,b \in \ZZ \right\}
    \unrhd
    \left\{ \begin{pmatrix}
    1 & 0 & b \\ 0 & 1 & 0 \\ 0 & 0 & 1
    \end{pmatrix} \mid b \in \ZZ \right\}
    \unrhd 1.
    \]
    Kvocienti te verige so vsi izomorfni $\ZZ$. Torej je $h(H) = 3$.
\end{itemize}
\end{zgled}

Hirschova dolžina je pomembna invarianta policikličnih grup, ki jo lahko uporabimo pri induktivnih sklepih. Primer take uporabe je podan v naslednji trditvi. Rečemo, da je grupa {\definicija brez torzije}, kadar nima netrivialnih elementov končnega reda.

\begin{trditev}
Policiklične grupe so virtualno brez torzije.
\end{trditev}

\begin{dokaz}
Indukcija na Hirschovo dolžino. Naj bo $G = G_1 \unrhd G_2 \unrhd \cdots \unrhd G_n = 1$. Naj bo $G_i/G_{i+1}$ prvi neskončni kvocient te vrste. Torej je $|G:G_i| < \infty$. Dovolj bo zato dokazati, da je policiklična grupa $G_i$ virtualno brez torzije. Brez škode za splošnost lahko torej privzamemo $i = 1$. 

Ker velja $h(G_2) = h(G) - 1$, po indukciji obstaja $H \leq G_2$, za katero je $|G_2 : H| < \infty$ in $H$ je brez torzije. Privzamemo lahko, da je $H \unlhd G$.\footnote{Po potrebi namreč lahko $H$ zamenjamo s presekom $\bigcap_{K \leq G, |G:K|=|G:H|} K$, ki je edinka končnega indeksa v $G$.} Naj bo $G/G_2 = \langle x G_2 \rangle$ za nek $x \in G$. Naj bo $K = \langle x,H \rangle$. 

\begin{figure}[h]
\centering
\begin{tikzpicture}
\tikzstyle{vertex}=[circle, fill = oranzna, minimum size = 5pt, inner sep = 0pt]

\draw[thick] (0, -2) to [bend right = 15] (2, -1) to [bend right = 15] (0,0);
\node[vertex, label = right : $K$] at (2,-1) {}; 

\draw[thick] (0,0) to node [midway,right] {$\ZZ$}(0, -1);
\node[vertex, label = left : $G$] at (0,0) {};
\draw[thick] (0,-1) to node [midway,right] {$< \infty$} (0, -2);
\node[vertex, label = left : $G_2$] at (0, -1) {};
\draw[thick] (0,-2) to node [midway, right] {brez torzije} (0, -4);
\node[vertex, label = left : $H$] at (0, -2) {};
\node[vertex, label = left : $1$] at (0, -4) {};
\end{tikzpicture}
\end{figure}

\underline{$K$ je končnega indeksa v $G$:} Velja $G = K G_2$, zato je $|G:K| = |G_2K : K| = |G_2 : G_2 \cap K| \leq |G_2 : H| < \infty$.

\underline{$K$ je brez torzije:} Vsak element $k \in K$ lahko zapišemo kot $k = x^a h$ za neka $a \in \ZZ$ in $h \in H$. Če je $k$ torzijski element, potem je $k^m = 1$, od koder sledi $H = k^m H = (kH)^m = x^{am} H$, zato je $x^{am} \in H \leq G_2$. V tem primeru je $xG_2$ reda največ $am$ v $G/G_2 \cong \ZZ$. Od tod sledi $a = 0$. Tako je $k = h \in H$. Ker je $H$ brez torzije, je slednje sprto s predpostavko, da je $k$ torzijski element. \kljuka
\end{dokaz}

\section{Nilpotentne grupe}

\subsection{Nilpotentnost}

Grupa $G$ je {\definicija nilpotentna}, če ima verigo podgrup
\[
G = G_1 \geq G_2 \geq \cdots \geq G_n = 1
\]
z lastnostjo $G_i \unlhd G$ in $G_i/G_{i+1} \leq Z(G/G_{i+1})$\footnote{{\definicija Center} grupe $G$ je $Z(G) = \{ x \in G \mid \forall y \in G \colon \ xy = yx \}$.} za vsak $i = 1,2,\dots, n-1$. Taki verigi rečemo {\definicija centralna vrsta}. 

Pogoj $G_i/G_{i+1} \leq Z(G/G_{i+1})$ je ekvivalenten $\forall x \in G_i \ \forall g \in G \colon [x,g] \in G_{i+1}$, kar lahko krajše zapišemo kot $[G_i, G] \leq G_{i+1}$.

Nilpotentne grupe so jasno rešljive.

\subsection{Spodnja in zgornja centralna vrsta}

Kot pri rešljivosti lahko opazimo nilpotentnost z eno posebno verigo. Induktivno definiramo {\definicija spodnjo centralno vrsto} z naslednjim predpisom:
\[
\gamma_1(G) = G, \quad \gamma_{i+1}(G) = [\gamma_i(G), G] \text{ za $i \geq 1$.}
\]
Induktivno hitro preverimo, da za vsako centralno vrsto $G = G_1 \geq G_2 \geq \cdots \geq G_n = 1$ velja $\gamma_i(G) \leq G_i$, zato sledi $\gamma_n(G) = 1$. Res je tudi obratno. Če je $\gamma_c(G) = 1$ za nek $c$, potem je $G = \gamma_1(G) \geq \gamma_2(G) \geq \cdots \geq \gamma_c(G) = 1$ centralna vrsta, ki opazi nilpotentnost grupe $G$. Torej je $G$ nilpotentna, če in samo če velja $\gamma_c(G) = 1$ za nek $c$. Če je $c$ najmanjše število s to lastnostjo, potem rečemo, da je $G$ {\definicija razreda nilpotentnosti} $c-1$.

\begin{zgled}
\begin{itemize}
    \item Grupe razreda nilpotentnosti $1$ so natanko grupe z $\gamma_2(G) = 1$, kar je ekvivalentno $[G,G] = 1$, torej so to natanko abelove grupe.
    \item Naj bo $H$ Heisenbergova grupa. Izračunamo
    \[
    \gamma_2(H) = H^{(2)} = 
    \left\{ \begin{pmatrix}
    1 & 0 & b \\ 0 & 1 & 0 \\ 0 & 0 & 1
    \end{pmatrix} \mid b \in \ZZ \right\}
    \leq Z(H).
    \]
    Torej je $\gamma_3(H) = [\gamma_2(H), H] = 1$. Tako je $H$ nilpotentna razreda $2$.
\end{itemize}
\end{zgled}

Analogno definiramo {\definicija zgornjo centralno vrsto} z naslednjim predpisom:
\[
Z_1(G) = Z(G), \quad Z_{i+1}(G)/Z_i(G) = Z(G/Z_i(G)) \text{ za $i \geq 1$}.
\]
Induktivno hitro sledi, da za vsako centralno vrsto $G = G_1 \geq G_2 \geq \cdots \geq G_n = 1$ velja $Z_i(G) \geq G_{n-i}$, zato sledi $Z_{n-1}(G) = G$. Res je tudi obratno. Če je $Z_c(G) = G$ za nek $c$, potem je $G = Z_c(G) \geq Z_{c-1}(G) \geq \cdots \geq Z_1(G) \geq 1$ centralna vrsta, ki opazi nilpotentnost grupe $G$. Torej je $G$ nilpotentna, če in samo če velja $Z_c(G) = G$ za nek $c$. Najmanjši tak $c$ je ravno razred nilpotentnosti grupe $G$ \vaje.

\begin{zgled}
\begin{itemize}
    \item Če je $G$ abelova grupa, potem je $Z_1(G) = G$. Razred nilpotentnosti $G$ je enak $1$.

    \item Naj bo $H$ Heisenbergova grupa. Izračunamo
    \[
    Z_1(H) = Z(H) =
    \left\{ \begin{pmatrix}
    1 & 0 & b \\ 0 & 1 & 0 \\ 0 & 0 & 1
    \end{pmatrix} \mid b \in \ZZ \right\} = \gamma_2(H).
    \]
    Hkrati je $H/Z(H) \cong \ZZ \times \ZZ$, torej je $Z_2(H) = H$. Razred nilpotentnosti grupe $H$ je enak $2$.

    \item Členi spodnje in zgodnje centralne vrste niso nujno enaki. Naj bo na primer $G = H \times \ZZ$, kjer je $H$ Heisenbergova grupa. Tedaj je $\gamma_2(G) = \gamma_2(H) \times 1$, $Z_1(G) = Z(H) \times \ZZ = \gamma_2(H) \times \ZZ$.
\end{itemize}
\end{zgled}

Kot pri rešljivih grupah ni težko preveriti, da so nilpotente grupe zaprte za podgrupe, kvociente in direktne produkte \vaje.

\begin{zgled}
Nilpotentne grupe {\em niso} nujno zaprte za razširitve. Preprost primer je simetrična grupa $S_3$, katere veriga $S_3 \geq A_3 \cong \ZZ/3\ZZ \geq 1$ s kvocientom $S_3/A_3 \cong \ZZ/2\ZZ$ opazi njeno policikličnost. Ker velja $Z(S_3) = 1$, grupa $S_3$ ni nilpotentna.
\end{zgled}

\begin{zgled}
Velik razred nilpotentnih grup tvorijo končne $p$-grupe.\footnote{Kasneje bomo pojasnili, kaj točno mislimo z izrazom, da končne $p$-grupe tvorijo {\em velik} razred grup.} To so grupe, katerih moč je potenca praštevila $p$. Razlog za njihovo nilpotentost je v tem, da imajo take grupe vselej netrivialen center. V taki grupi $G$ zato velja $Z_1(G) > 1$, $Z_2(G)/Z_1(G) = Z(G/Z_1(G)) > Z_1$, \dots in torej zgornja centralna vrsta strogo raste, če še ni dosegla celotne grupe $G$. Slej kot prej torej velja $Z_c(G) = G$ za nek $c$.
\end{zgled}

\subsection{Komutatorji}

{\definicija Komutator} elementov $x,y \in G$ je $[x,y] = x^{-1} y^{-1} x y$. Preprost račun pokaže, da komutator zadošča naslednjima {\definicija komutatorskima identitetama}:
\begin{align*}
[xy, z] &= [x,z]^y \cdot [y,z], \\
[x,yz] &= [x,z] \cdot [x,y]^z,
\end{align*}
pri čemer smo uporabili oznako $g^x = x^{-1} g x$ za desno konjugiranje.\footnote{V grupi imamo dve konjugiranji: levo in desno. Levo konjugiranje z $x$ je $g \mapsto x g x^{-1} = {}^x g$ in to je levo delovanje, desno konjugiranje z $x$ pa je $g \mapsto x^{-1} g x = g^x$ in to je desno delovanje.}

V nilpotentni grupi $G$ razreda $2$ velja $\gamma_2(G) \leq Z(G)$, zato je v taki grupi res $[xy,z] = [x,z][y,z]$ in $[x,yz] = [x,y][x,z]$, torej je komutator {\em bilinearen} v taki grupi.

{\definicija Komutator uteži $c$} je element $[x_1, x_2, \dots, x_c]$, definiran induktivno kot 
\[
[x_1, x_2, \dots, x_c] = [[x_1, x_2, \dots, x_{c-1}], x_c].
\]
Komutatorji uteži $1$ so običajni grupni elementi. Komutatorji uteži $2$ so običajni komutatorji dveh grupnih elementov. Komutator uteži $3$ pa je na primer element $[x,y,z] = [[x,y],z] = [x,y]^{-1} z^{-1} [x,y] z$.

Preprost račun pokaže, da za komutatorje uteži $3$ v vsaki grupi velja {\definicija Hall-Wittova identiteta}:\footnote{To je analog Jacobijeve identitete iz Liejevih algeber.}
\[
[x, y^{-1}, z]^y \cdot [y, z^{-1},x]^z \cdot [z, x^{-1}, y]^x = 1.
\]

Analogno oznako kot za komutatorje višjih uteži uporabimo za podgrupe. Za podgrupe $H,K,L \leq G$ imamo torej $[H,K] = \langle [h,k] \mid h \in H, k \in K \rangle$ in $[H,K,L] = [[H,K], L]$. Pri tem moramo biti previdni, saj je slednja grupa definirana induktivno in torej v splošnem {\em ni} nujno kar enaka grupi, generirani s komutatorji uteži $3$ oblike $[h,k,l]$ za $h \in H, k \in K, l \in L$.

\begin{lema}
Naj bosta $H,K \unlhd G$, $H = \langle X \rangle$, $K = \langle Y \rangle$. Tedaj je
\[
[H,K] = \llangle [x,y] \mid x \in X, y \in Y \rrangle.
\]
\vspace{-1.5\baselineskip}
\end{lema}

\begin{dokaz}
Grupa $[H,K]$ vsebuje vse komutatorje $[x,y]$ za $x \in X, y \in Y$, zato vsebuje tudi podgrupo edinko, generirano s temi elementi: $N := \llangle [x,y] \mid x \in X, y \in Y \rrangle$. Velja tudi obratna inkluzija. V kvocientu $[H,K]/N$ namreč odsek $xN$ komutira z $yN$ za $x \in X, y \in Y$, zato v tem kvocientu celotna podgrupa $HN/N$ komutira s podgrupo $KN/N$. Od tod sledi $[H,K]N = N$, torej je res tudi $[H,K] \leq N$.
\end{dokaz}

% S pomočjo Hall-Wittove identitete ni težko preveriti \vaje, da velja:

% \begin{lema}(o treh podgrupah)
% Naj bodo $H,K,L \unlhd G$. Tedaj je
% \[
% [H,K,L] \leq [K,L,H] \cdot [L,H,K].
% \]
% \vspace{-1.5\baselineskip}
% \end{lema}

\subsection{Komutatorji in spodnja centralna vrsta}

% \begin{posledica}
% V vsaki grupi $G$ velja $[\gamma_i(G), \gamma_j(G)] \leq \gamma_{i+j}(G)$ za vse $i,j$.
% \end{posledica}

% \begin{dokaz}
% Indukcija na $j$.
% \begin{description}
%     \item[j = 1:] $[\gamma_i(G), G] = \gamma_{i+1}(G)$ po definiciji. \kljuka
%     \item[$j-1 \to j$:] Velja račun
%     \begin{align*}
%     [\gamma_i(G), \gamma_j(G)] 
%     &= [\gamma_i(G), [\gamma_{j-1}(G), G]] \\
%     &= [\gamma_{j-1}(G), G, \gamma_i(G)] \\
%     &\leq [G, \gamma_i(G), \gamma_{j-1}(G)] \cdot [\gamma_i(G), \gamma_{j-1}(G), G],
%     \end{align*}
%     kjer smo uporabili lemo o treh podgrupah. Slednje je po indukciji enako
%  $[\gamma_{i+1}(G), \gamma_{j-1}(G)] \cdot [\gamma_{i+j-1}(G), G] \leq \gamma_{i+j}(G)$. \kljuka
% \end{description}
% \end{dokaz}

Spodnja centralna vrsta je vselej generirana s komutatorji ustreznih uteži.

\begin{lema}
V vsaki grupi $G$ velja $\gamma_i(G) = \langle [x_1, x_2, \dots, x_i] \mid x_1, x_2, \dots, x_i \in G \rangle$.
\end{lema}

\begin{dokaz}
Indukcija na $i$.
\begin{description}
    \item[$i = 1$:] \kljuka

    \item[$i-1 \to i$:] Velja $\gamma_i(G) = [\gamma_{i-1}(G), G]$, zato iz zadnje leme in indukcije sledi
    \[
    \gamma_i(G) = \llangle [x_1, x_2, \dots, x_{i-1}, x_i] \mid x_1, \dots, x_i \in G \rrangle.
    \]
    Konjugat komutatorja uteži $i$ je komutator uteži $i$, velja namreč $[x_1, \dots, x_i]^g = [x_1^g, \dots, x_i^g]$ za $g \in G$. Tako je podgrupa, generirana s komutatorji uteži $i$, edinka v $G$. S tem je dokaz zaključen. \kljuka
\end{description}
\end{dokaz}

V lemi {\em ne} smemo jemati elementov $x_i$ le iz neke generirajoče množice $X \subseteq G$. V tem primeru namreč enak dokaz kot zgoraj pokaže, da velja
\[
\gamma_i(G) = \llangle [x_1, \dots, x_i] \mid x_1, \dots, x_i \in X \rrangle
\]
in v splošnem podgrupa, generirana s komutatorji uteži $i$ s členi v $X$ {\em ni} nujno edinka v $G$.

\begin{zgled}
Opazujmo diedrsko grupo $D_{2n} = \langle s, t \mid s^n, t^2, t s t^{-1} = s^{-1} \rangle$. Vzemimo $X = \{ s, t \}$. Izračunamo
\[
\gamma_2(D_{2n}) = \llangle [s,t] \rrangle = \llangle s^2 \rrangle =
\langle s^2  \rangle.
\]
Za tem je
\[
\gamma_3(D_{2n}) = [\gamma_2(D_{2n}), G] =
\llangle [s^2, t] \rrangle = \llangle s^4 \rrangle = \langle s^4 \rangle
\]
in nato
\[
\gamma_4(D_{2n}) = [\gamma_3(D_{2n}), G] =
\llangle [s^4, t] \rrangle = \langle s^8 \rangle.
\]
Induktivno velja $\gamma_i(D_{2n}) = \langle s^{2^{i-1}} \rangle$ za $i \geq 2$.

Naj bo $n = 2^a m$ za $a,m \in \NN_0$, pri čemer je $m$ liho število. Velja $\gamma_{a+1}(D_{2n}) = \langle s^{2^a} \rangle = \langle s^{n/m} \rangle$. Če je $m = 1$, potem sledi $\gamma_{a+1}(D_{2n}) = 1$, torej je v tem primeru grupa $D_{2n}$ nilpotentna razreda $a$. Če pa je $m > 1$, potem je $s^{n/m}$ lihega reda $m$, zato je $\gamma_{i}(D_{2n}) = \langle s^{n/m} \rangle$ za vsak $i \geq a+1$. V tem primeru grupa $D_{2n}$ torej {\em ni} nilpotentna.

Podobno velja v primeru $n = \infty$. Spodnja centralna vrsta nikdar ne doseže $1$ in grupa $D_\infty$ {\em ni} nilpotentna.
\end{zgled}

V kvocientu $G/\gamma_i(G)$ so vsi komutatorji uteži $i$ trivialni, zato so vsi komutatorji uteži $i-1$ centralni. Torej je edinka, generirana s komutatorji uteži $i-1$, enaka podgrupi, generirani s komutatorji uteži $i-1$. Skupaj z zgornjo opombo od tod sledi naslednja posledica.

\begin{posledica}
Naj bo $G = \langle X \rangle$. Grupa $\gamma_{i-1}(G)/\gamma_i(G)$ je generirana s komutatorji uteži $i-1$ z vnosi iz množice $X \gamma_i(G)/\gamma_i(G)$.
\end{posledica}

Kadar je grupa $G$ končno generirana, so torej kvocienti spodnje centralne vrste končno generirane abelove grupe. V posebnem zato velja naslednje.

\begin{posledica}
Končno generirana nilpotentna grupa je policiklična.
\end{posledica}

Končno generirana torzijska\footnote{Grupa je {\definicija torzijska}, če je vsak njen element končnega reda.} nilpotentna grupa je torej policiklična z verigo brez neskončnih kvocientov, torej je taka grupa nujno končna. Velja pa še nekaj močnejšega.

\begin{trditev}
Naj bo $G$ {\em nilpotentna} grupa, generirana s končno mnogo torzijskimi elementi. Tedaj je $G$ končna.
\end{trditev}

\begin{dokaz}
Dokažimo, da je za vsak $i$ grupa $\gamma_i(G)/\gamma_{i+1}(G)$ končna. Od tod bo sledilo, da je tudi $G$ končna. Naj bo $G= \langle X \rangle$ z lastnostjo $x^N = 1$ za vse $x \in X$. Grupa $\gamma_i(G)/\gamma_{i+1}(G)$ je generirana s komutatorji uteži $i$ z vnosi iz množice $X \gamma_{i+1}(G)/\gamma_{i+1}(G)$. Za vsak tak komutator $[x_1, \dots, x_i] \gamma_{i+1}$ velja
\[
\left( [x_1, \dots, x_i] \gamma_{i+1} \right)^N = [x_1, \dots, x_{i-1}, x_i^N] \gamma_{i+1}(G) = \gamma_{i+1}(G),
\]
kjer smo v prvi enakosti uporabili drugo komutatorsko identiteto. Tako je abelova grupa $\gamma_i(G)/\gamma_{i+1}(G)$ generirana s končno mnogo elementi reda kvečjemu $N$, torej je končna.
\end{dokaz}

\begin{zgled}
Grupa $D_\infty$ je generirana s torzijskima elementoma $t, st$, a ta grupa ni končna.
\end{zgled}

\begin{odprtproblem}\href{https://en.wikipedia.org/wiki/Burnside_problem}{(Burnside 1902)}
Ali je grupa 
\[
B(2,5) = \langle x, y \mid w^5 = 1 \text{ za vsako besedo $w \in F(\{ x, y \})$} \rangle
\]
končna?
\end{odprtproblem}

\section{Teorija razširitev}

\subsection{Problem rekonstrukcije}

Naj bo $G$ končna grupa. Če $G$ {\em ni} enostavna, potem ima pravo netrivalno edinko $N$. Torej lahko $G$ predstavimo kot razširitev:
\[
1 \to N \to G \to G/N \to 1.
\]
Ta postopek lahko nadaljujemo z grupama $N$ in $G/N$. Torej lahko grupo $G$ dobimo z zaporednim izvajanjem razširitev: začnemo z enostavno grupo $G/N_1$, ki jo razširimo z enostavno grupo $N_1/N_2$, tako dobimo grupo $G/N_2$, ki jo potem razširimo z enostavno grupo $N_2/N_3$, tako dobimo grupo $G/N_3$, \dots, nazadnje dobimo grupo $G/N_k = G$.

V prejšnjem razdelku smo obravnavali policiklične grupe -- to so natanko grupe, pri katerih so vsi kvocienti $N_i/N_{i+1}$ v zgornji konstrukciji ciklični.

V tem razdelku si bomo podrobneje pogledali, do kakšne mere lahko s poznavanjem teh enostavnih kvocientov {\em rekonstruiramo} grupo $G$.

{\vprasanje Kakšne možnosti imamo za enostavne kose? Katere so vse končne enostavne grupe?} Končne enostavne grupe so klasificirane ({\vprasanje res?}), sestojijo iz naslednjih družin:
\begin{itemize}
    \item ciklične grupe: $\ZZ/p\ZZ$,
    \item alternirajoče grupe: $A_n$ za $n \geq 5$,
    \item matrične grupe Liejevega tipa: $\PSL_{n+1}(\field_q)$ za $n \geq 1$, kjer je $q$ potenca praštevila in $\field_q$ končno polje moči $q$; sorodne tej so ortogonalne, simplektične in unitarne grupe,
    \item izjemne grupe Liejevega tipa: te so uresničljive kot podgrupe matrik majhnih dimenzij,
    \item končno mnogo sporadičnih grup.
\end{itemize}

\begin{figure}[t]
\centering
\begin{tikzpicture}
\tikzstyle{vertex}=[circle, fill = oranzna, minimum size = 5pt, inner sep = 0pt]

\draw[<->, thick] (0,-4) node [below] {$q$}  -- (0,0) -- (5,0) node [right] {$n$};
\draw[->, thick] (-2, 0) -- (-2, -4) node [below] {$p$};
\draw[->, thick] (0, 2) -- (5,2) node [right] {$n$};

\draw[rounded corners=1em, draw = vijolicna, thick, fill = vijolicna!10] (-1.9,0) rectangle (-1.1,-4);
\draw[->,thick, draw=vijolicna] (-1.5, -5) -- (-1.5, -3.75);
\node[label = below : {\color{vijolicna} abelove}] at (-1.5, -5) {};
\node[label = above : $\ZZ/p\ZZ$] at (-2, 0) {};
\foreach \p in {1,...,4}
    \node[vertex] at ($(-2, 0) + (0.5, 0.5-\p)$) {};

\draw[rounded corners=1em, draw = vijolicna, thick, fill = vijolicna!10] (0,1.9) rectangle (4.9,1.1);
\draw[->,thick, draw=vijolicna] (6,1.5) -- (4.65, 1.5);
\node[label = right : {\color{vijolicna} alternirajoče}] at (6, 1.5) {};
\node[label = left : $A_n$] at (0,2) {};
\foreach \n in {1,...,5}
    \node[vertex] at ($(0, 2) + (-0.5+\n, -0.5)$) {};

\draw[rounded corners=1em, draw = vijolicna, thick, fill = vijolicna!10] (0.1,-0.1) rectangle (4.9,-3.9);
\draw[->,thick, draw=vijolicna] (6,-2) -- (4.65, -2);
\node[label = right : {\color{vijolicna} Liejevega tipa}] at (6, -2) {};
\foreach \n in {1,...,5}
    \foreach \q in {1,...,4}
        {
        \draw[thick, dashed, draw = rdeca, fill = rdeca!10]  ($(-0.5+\n, 0.5-\q)$) circle (0.35);
        \foreach \angle in {0,90,180,270}
            \node[vertex] at ($(-0.5+\n, 0.5-\q) + (\angle:0.2)$) {};
        };
\draw[->,thick,draw=rdeca] (4.5,-5) to [bend right = 10] (3.6,-2.6);
\node[label = below : {\color{rdeca} $\PSL_{n+1}(\field_q)$ s prijateljicami}] at (4.5,-5) {};
\end{tikzpicture}
\caption{Prikaz klasifikacije končnih enostavnih grup}
\end{figure}

Mi bomo obravnavali le {\em najlažji primer}, ko predpostavimo, da so vsi enostavni kosi kar ciklične grupe. S tem primerom pokrijemo vse končne rešljive grupe, hkrati pa je že ta primer zelo zakompliciran, kot bomo videli.

\subsection{Kategorija razširitev}

Množica vseh razširitev tvori {\em kategorijo}. Njeni objekti so razširitve, {\definicija morfizem med razširitvama} $N \xrightarrow{\iota} G \xrightarrow{\pi} Q$ in $\bar N \xrightarrow{\bar \iota} \bar G \xrightarrow{\bar \pi} \bar Q$ je trojica homomorfizmov $(\alpha, \beta, \gamma)$, tako da velja $\beta \iota = \bar \iota \alpha$ in $\gamma \pi = \bar \pi \beta$.

% \[
%   \begin{tikzcd}
%     N \arrow{r}{\iota} \arrow{d}{\alpha} & G \arrow{r}{\pi} \arrow{d}{\beta} & Q \arrow{d}{\gamma} \\
%     \bar N \arrow{r}{\bar \iota} & \bar G \arrow{r}{\bar \pi} & \bar Q.
%   \end{tikzcd}
% \]

Morfizem, pri katerem je $\alpha = \id_N$ in $\gamma = \id_Q$ (in torej $N = \bar N$, $Q = \bar Q$), imenujemo {\definicija ekvivalenca razširitev}. V tem primeru je $\beta$ nujno izomorfizem grup $G$ in $\bar G$ \vaje.

\subsection{Delovanje v razširitvi z abelovim jedrom}

Predpostavimo, da je $N$ {\em abelova} grupa. V tem primeru je kategorija vseh razširitev {\em blage zahtevnosti} in jo lahko opišemo na naslednji način.

Naj bo $N \xrightarrow{\iota} G \xrightarrow{\pi} Q$ objekt kategorije razširitev. Izberimo {\definicija transverzalo} $\mathcal{T}$: za vsak odsek $N$ v $G$ izberemo en element iz tega odseka. Elementi $\mathcal{T}$ so torej v bijekciji z elementi $Q$ prek preslikave $\pi$. Ekvivalentno, imamo {\definicija prerezno funkcijo}\footnote{Prerezna funkcija v splošnem ni nujno homomorfizem.} $\tau \colon Q \to G$, za katero za vsak $q \in Q$ velja $\tau(q) \in \mathcal{T}$ in $\pi(\tau(q)) = q$. 

Vemo že, da v primeru, ko lahko izberemo prerezno funkcijo tako, da je homomorfizem, dobimo semidirektni produkt. {\vprasanje Kaj pa v primeru, ko $\tau$ ni nujno homomorfizem?}

Elementi transverzale $\mathcal{T}$ delujejo na $N$ s konjugiranjem: za $n\in N$ dobimo ${}^{\tau(q)} n =  \tau(q) \cdot n \cdot \tau(q)^{-1}$. Na ta način dobimo {\em homomorfizem}\footnote{Za $q_1, q_2 \in Q$ velja $\tau(q_1 q_2) \equiv \tau(q_1) \tau(q_2) \mod{N}$. Ker je $N$ {\em abelova}, je torej konjugiranje z $\tau(q_1 q_2)$ enako konjugiranju z $\tau(q_1)\tau(q_2)$. Zato je $\chi$ res homomorfizem.} $\chi \colon Q \to \Aut(N)$, ki $q$ preslika v konjugiranje s $\tau(q)$. Torej dobimo delovanje $Q$ na $N$. To delovanje je {\em neodvisno od izbire transverzale $\mathcal{T}$} oziroma prerezne funkcije $\tau$ \vaje.

Povzemimo.

\begin{trditev}
Vsaka razširitev $N \xrightarrow{\iota} G \xrightarrow{\pi} Q$ z abelovim jedrom $N$ porodi homomorfizem $\chi \colon Q \to \Aut(N)$.
\end{trditev}

\begin{zgled}
Naj bo $G = N \rtimes_\varphi Q$, kjer je $\varphi \colon Q \to \Aut(N)$. Tedaj lahko izberemo transverzalo s prerezno funkcijo $\tau \colon Q \to G, q \mapsto (1,q)$. Velja 
\[
{}^{\tau(q)} n = {}^{(1,q)} (n,1) = (1,q)(n,1)(1,q)^{-1} = (\varphi(q)(n), 1) = \varphi(q)(n).
\]
Torej je $\chi(q) = \varphi(q)$ za vsak $q \in Q$, se pravi $\chi = \varphi$.
\end{zgled}

{\vprasanje Kako se $\chi$ spremeni z ekvivalenco razširitev?}

\begin{trditev}
Ekvivalentni razširitvi z abelovim jedrom porodita enak homomorfizem $\chi \colon Q \to \Aut(N)$.
\end{trditev}

\begin{dokaz}
Imejmo ekvivalentni razširitvi $N \xrightarrow{\iota} G \xrightarrow{\pi} Q$ in $ N \xrightarrow{\bar \iota} \bar G \xrightarrow{\bar \pi} Q$ z izomorfizmom $\beta \colon G \to \bar G$ in s pripadajočima delovanjema $\chi, \bar \chi \colon Q \to \Aut(N)$. Izberimo prerezno funkcijo $\tau \colon Q \to G$ za $\pi$. Tedaj je $\beta \circ \tau \colon Q \to \bar G$ prerezna funkcija za $\bar \pi$. Za $n \in N$ in $q \in Q$ velja
\[
\chi(q)(n) = {}^{\tau(q)} n
\]
in
\[
\bar \chi(q)(n) = {}^{\beta(\tau(q))} n = \beta(\tau(q)) n \beta(\tau(q))^{-1}.
\]
Ker je $\beta |_N = \id_N$, je zadnji element enak
\[
\beta(\tau(q)) \beta(n) \beta(\tau(q))^{-1} = \beta({}^{\tau(q)} n) = \beta(\chi(q)(n)) = \chi(q)(n).
\]
Tako je res $\chi = \bar \chi$.
\end{dokaz}

Ni pa vsa informacija o razširitvi skrita v homomorfizmu $\chi$. Lahko se zgodi, da imata dve neekvivalentni razširitvi enak $\chi$.

\begin{zgled}
Opazujmo dve razširitvi grupe $\ZZ/2\ZZ$ z $\ZZ/2\ZZ$. Prva naj bo kar trivialna razširitev, kjer je srednja grupa enaka $\ZZ/2\ZZ \times \ZZ/2\ZZ$. Druga naj bo razširitev, kjer je srednja grupa enaka $\ZZ/4\ZZ$. Ti dve razširitvi sta jasno neekvivalentni. Delovanji v obeh primerih slikata iz $\ZZ/2\ZZ$ v grupo $\Aut(\ZZ/2\ZZ)$. Slednja grupa je trivialna in vsebuje le $\id_{\ZZ/2\ZZ}$. V vsakem primeru sta torej pripadajoči delovanji trivialni, torej enaki.
\end{zgled}

\subsection{Kohomologija}

Ključ v razumevanju razširitve je v tem, da pogledamo, kako {\em daleč} je prerezna funkcija $\tau$ od homomorfizma.

Naj bo $N \xrightarrow{\iota} G \xrightarrow{\pi} Q$ razširitev z izbrano prerezno funkcijo $\tau$. Ker za vsak $q_1, q_2 \in Q$ velja $\tau(q_1 q_2) \equiv \tau(q_1) \tau(q_2) \mod{N}$, obstaja enoličen $n = n(q_1, q_2) \in N$, za katerega je $n \cdot \tau(q_1q_2) = \tau(q_1)\tau(q_2)$. Dobimo torej funkcijo
\[
\phi \colon Q \times Q \to N, \quad (q_1, q_2) \mapsto n(q_1, q_2).
\]
Ta funkcija ni čisto poljubna. Iz asociativnosti množenja namreč izpeljemo naslednji račun: po eni strani je za poljubne $q_1, q_2, q_3 \in Q$ produkt $\tau(q_1) \tau(q_2) \tau(q_3)$ enak
\begin{align*}
(\tau(q_1) \tau(q_2)) \tau(q_3) &= \phi(q_1, q_2) \tau(q_1 q_2) \tau(q_3) \\
&= \phi(q_1, q_2) \phi(q_1 q_2, q_3) \tau(q_1 q_2 q_3),
\end{align*}
po drugi strani pa je enak
\begin{align*}
\tau(q_1) (\tau(q_2) \tau(q_3)) &= \tau(q_1) \phi(q_2, q_3) \tau(q_2 q_3) \\
&= ^{\tau(q_1)} \phi(q_2, q_3) \cdot \tau(q_1) \tau(q_2 q_3) \\
&= q_1.\phi(q_2, q_3) \cdot \phi(q_1, q_2 q_3) \tau(q_1 q_2 q_3),
\end{align*}
kjer $Q$ deluje na $N$ prek homomorfizma $\chi$. Na ta način dobimo naslednji pogoj, ki mu zadošča funkcija $\phi$:
\[
\forall q_1, q_2, q_3 \in Q \colon \quad
\phi(q_1, q_2) + \phi(q_1 q_2, q_3) = q_1.\phi(q_2, q_3) + \phi(q_1, q_2 q_3)
\]
Tukaj smo operacijo v $N$ zaradi preglednosti pisali aditivno. Vsaki funkciji $\phi \colon Q \times Q \to N$, ki zadošča temu nenavadnemu pogoju, rečemo {\definicija 2-kocikel}. Množico vseh $2$-kociklov označimo z {\definicija $Z^2(Q,N)$}.

\begin{povzetek}
Razširitvi smo torej z nekaj izbire priredili prerezno preslikavo $\tau$, tej pa zdaj še $2$-kocikel $\phi$.
\end{povzetek}

V množici $Z^2(Q,N)$ lahko seštevamo z uporabo operacije na $N$: za $2$-kocikla $\phi_1, \phi_2 \colon Q \times Q \to N$ je $(\phi_1 + \phi_2)(q_1, q_2) = \phi_1(q_1, q_2) + \phi_2(q_1, q_2)$ za $q_1, q_2 \in N$. Na ta način je $Z^2(Q,N)$ abelova grupa.

\begin{zgled}
Če je $G = N \rtimes_\varphi Q$, potem je $\tau$ homomorfizem, zato dobimo $\phi \equiv 0$.
\end{zgled}

{\vprasanje Kako je $\phi$ odvisen od izbire $\tau$?} Naj bo $\tau^\prime$ še en prerez $\pi$. Tako dobimo drug $2$-kocikel $\phi^\prime \in Z^2(Q,N)$. Velja $\tau(q) \equiv \tau^\prime(q) \mod{N}$, zato obstaja enoličen $n = n(q) \in N$, za katerega je $n \cdot \tau^\prime(q) = \tau(q)$. Dobimo torej funkcijo 
\[
\psi \colon Q \to N, \quad q \mapsto n(q).
\]
Funkciji $\phi$ in $\phi^\prime$ sta povezani prek funkcije $\psi$ na naslednji način. Za $q_1, q_2 \in Q$ velja
\begin{align*}
\phi(q_1, q_2) &= \tau(q_1) \cdot \tau(q_2) \cdot \tau(q_1 q_2)^{-1} \\
&= \psi(q_1) \tau^\prime(q_1) \cdot \psi(q_2) \tau^\prime(q_2) \cdot \tau^\prime(q_1 q_2)^{-1} \psi(q_1 q_2)^{-1} \\
&= \psi(q_1) \cdot {}^{\tau^\prime(q_1)}\psi(q_2) \cdot \tau^\prime(q_1) \tau^\prime(q_2) \tau^\prime(q_1 q_2)^{-1} \cdot \psi(q_1 q_2)^{-1} \\
&= \psi(q_1) + q_1.\psi(q_2) + \phi^\prime(q_1, q_2) - \psi(q_1 q_2),
\end{align*}
kjer smo v zadnji vrstici operacijo v $N$ pisali aditivno. Definirajmo funkcijo
\[
\psi^\star \colon Q \times Q \to N, \quad
(q_1, q_2) \mapsto \psi(q_1) + q_1.\psi(q_2) - \psi(q_1 q_2).
\]
Na ta način po zgornjem računu velja $\phi = \phi^\prime + \psi^\star$. Ker sta $\phi, \phi^\prime$ elementa grupe $Z^2(Q,N)$, je tudi $\psi^\star = \phi - \phi^\prime \in Z^2(Q,N)$.

Naj bo {\definicija $B^2(Q,N)$} množica vseh funkcij $f^\star \colon Q \times Q \to N$, za katere obstaja funkcija $f \colon Q \to N$, tako da velja $f^\star(q_1, q_2) = f(q_1) + q_1.f(q_2) - f(q_1 q_2)$. Elementom te množice rečemo {\definicija $2$-korobovi}. Ni težko (je pa sitno) preveriti, da je $B^2(Q,N)$ podgrupa grupe $Z^2(Q,N)$.

Iz zgodnje izpeljave sledi, da je funkcija $\psi^\star = \phi - \phi^\prime$ $2$-korob. 

\begin{povzetek}
Razširitvi smo torej z izbiro prereza $\tau$ priredili $2$-kocikel $\phi$, z izbiro drugega prereza $\tau^\prime$ $2$-kocikel $\phi^\prime$, ta dva kocikla pa se razlikujeta za $2$-korob $\psi^\star$.
\end{povzetek}

Če identificiramo $2$-kocikle, ki se razlikujejo za $2$-korob, je torej dobljen ekvivalenčni razred $2$-kocikla neodvisen od izbire prerezne funkcije. Povzemimo.

\begin{trditev}
Razširitev $N \to G \to Q$ z abelovim jedrom $N$ porodi element $\phi + B^2(Q,N)$ grupe $Z^2(Q,N)/B^2(Q,N)$.
\end{trditev}

Grupo $Z^2(Q,N)/B^2(Q,N)$ označimo s {\definicija $H^2(Q,N)$} in ji pravimo {\definicija druga kohomološka grupa}, prirejena delovanju grupe $Q$ na $N$. Za njeno definicijo potrebujemo le grupi $Q$ in $N$ ter delovanje $\chi \colon Q \to \Aut(N)$.

\begin{povzetek}
Razširitvi smo torej enolično priredili kohomološki element $\phi + B^2(Q,N)$. {\vprasanje Ali vsak kohomološki element izhaja iz neke razširitve?}
\end{povzetek}

Naj bo $\phi \in Z^2(Q,N)$ dan $2$-kocikel. Iz njega lahko sestavimo razširitev na naslednji način. Naj bo grupa $G(\phi)$ kot množica enaka $N \times Q$, množenje na njej pa vpeljemo z naslednjim predpisom:
\[
(n_1, q_1) \cdot (n_2, q_2) = (n_1 + q_1.n_2 + \phi(q_1, q_2), q_1 q_2)
\]
za $n_1, n_2 \in N$, $q_1, q_2 \in Q$. Pri tem smo operacijo na $N$ zopet pisali aditivno. S tem predpisom je $G(\phi)$ grupa. Opremljena je tudi z naravnima preslikavama $\iota \colon N \to G(\phi), n \mapsto (n,1)$ in $\pi \colon G(\phi) \to Q, (n,q) \mapsto q$. {\vprasanje Kateri kohomološki element ustreza razširitvi $N \xrightarrow{\iota} G(\phi) \xrightarrow{\pi} Q$?}

\begin{trditev}
Naj bo $\chi \colon Q \to \Aut(N)$ delovanje na abelovi grupi $N$ in naj bo $\phi \in Z^2(Q,N)$. Razširitev $N \xrightarrow{\iota} G(\phi) \xrightarrow{\pi} Q$ porodi ravno delovanje $\chi$ na $N$, hkrati pa {\em obstaja} prerezna funkcija $\tau \colon Q \to G(\phi)$, da je $2$-kocikel te razširitve ravno $\phi$.
\end{trditev}

\begin{dokaz}
Za $\tau$ lahko vzamemo funkcijo $q \mapsto (1,q)$ za $q \in Q$.
\end{dokaz}

\begin{povzetek}
Razširitvi smo torej enolično priredili kohomološki element. To prirejanje ima {\em prerezno} prirejanje razširitve $2$-kociklu. {\vprasanje Kako je razširitev $G(\phi)$ odvisna od izbire $2$-kocikla $\phi$ v kohomološki grupi?}
\end{povzetek}

Naj bosta $\phi, \phi^\prime \in Z^2(Q,N)$, ki se razlikujeta za $2$-korob $\psi^\star \in B^2(Q,N)$, ki sam izhaja iz preslikave $\psi \colon Q \to N$. Priredimo jima razširitvi $G(\phi), G(\phi^\prime)$. Definirajmo preslikavo
\[
\beta \colon G(\phi) \to G(\phi^\prime), \quad
(n,q) \mapsto (n + \psi(q), q).
\]
Ni težko preveriti, da $\beta$ vzpostavi ekvivalenco razširitev $N \to G(\phi) \to Q$ in $N \to G(\phi^\prime) \to Q$.

Velja tudi obratno. Imejmo ekvivalentni razširitvi $N \xrightarrow{\iota} G \xrightarrow{\pi} Q$ in $N \xrightarrow{\bar \iota} \bar G \xrightarrow{\bar \pi} Q$ z izomorfizmom $\beta \colon G \to \bar G$. Izberimo prerezno funkcijo $\tau$ prve razširitve. Tedaj je $\bar \tau = \beta \circ \tau$ prerezna funkcija druge razširitve. Naj bo $\phi \in Z^2(Q,N)$ prirejen prerezni funkciji $\tau$. Tedaj za $q_1, q_2 \in Q$ velja
\[
\tau(q_1) \tau(q_2) = \phi(q_1, q_2) \tau(q_1 q_2),
\]
od koder po uporabi $\beta$ sledi
\[
\bar \tau(q_1) \bar \tau(q_2) = \phi(q_1, q_2) \bar \tau(q_1 q_2),
\]
torej je $\phi$ ravno $2$-kocikel, prirejen prerezni funkciji $\bar \tau$. Obe razširitvi torej porodita isti kohomološki element $\phi + B^2(Q,N) \in H^2(Q,N)$.

Tako smo nazadnje dokazali naslednji izrek.

\begin{izrek}
Naj grupa $Q$ deluje na abelovi grupi $N$. Tedaj obstaja bijekcija med ekvivalenčnimi razredi razširitev $Q$ z $N$ z danim delovanjem in elementi kohomološke grupe $H^2(Q,N)$. Pri tem razcepna razširitev ustreza enoti kohomološke grupe.
\end{izrek}

\begin{zgled}
Naj bo $N = (\ZZ/2\ZZ)^2$ in $Q = D_8$. Izberimo trivialno delovanje grupe $Q$ na $N$, torej $\chi \colon Q \to \Aut(N), q \mapsto \id_N$. Zahteven, a trivialen račun pokaže, da velja
\[
Z^2(Q,N) \cong (\ZZ/2\ZZ)^8, \quad
B^2(Q,N) \cong (\ZZ/2\ZZ)^2.
\]
Torej je $H^2(Q,N) \cong (\ZZ/2\ZZ)^6$. Imamo torej $2^6 = 64$ neekvivalentnih razširitev pri trivialnem delovanju $Q$ na $N$. Ni pa nujno, da vsaka dva kohomološka elementa porodita {\em neizomorfni} razširitveni grupi. Izkaže se, da v resnici dobimo le $11$ neizomorfnih grup. Konkreten primer netrivialne razširitve je
\[
\ZZ/2\ZZ \times \ZZ/2\ZZ \xrightarrow{\iota} \ZZ/2\ZZ \times D_{16} \xrightarrow{\pi} D_8,
\]
kjer je $\iota(x,y) = (x, \rho^{4y})$ in $\pi(x, \sigma^i \rho^j) = \sigma^i \rho^j$.
\end{zgled}

Če opazujemo le razširitve $2$-grup z $2$-grupami, dobimo število neizomorfnih grup, kot je prikazano v tabeli.

\begin{table}[t]
    \centering
\begin{tabular}{ll}
    $i$ & število grup moči $2^i$ \\ \hline
    1 & 1 \\
    2 & 2 \\
    3 & 5 \\
    4 & 14 \\
    5 & 51 \\
    6 & 267 \\
    7 & 2328 \\
    8 & $\sim 5.6 \cdot 10^4$ \\
    9 & $\sim 10^7$ \\
    10 & $\sim 4.95 \cdot 10^{10}$
\end{tabular}
\caption{Število grup moči $2^i$}
\end{table}

Po drugi strani je število {\em vseh} grup moči kvečjemu $2000$ enako $\sim 4.99 \cdot 10^{10}$. Lahko bi torej rekli, da so {\em skoraj vse končne grupe $2$-grupe}.

\begin{odprtproblem}
Naj bo $A(n)$ število grup moči kvečjemu $n$. Naj bo $B(n)$ število $2$-grup moči kvečjemu $n$. Tedaj je $\lim_{n \to \infty} B(n)/A(n) = 1$.
\end{odprtproblem}

Domnevno je torej skoraj vsa kompleksnost končnih grup skrita v teoriji zaporednih razširitev z grupo $\ZZ/2\ZZ$. Pri tem je $\Aut(\ZZ/2\ZZ) = 1$, torej je vsako delovanje na tej grupi trivialno.

\chapter{Rast}

\section{Funkcija rasti}

\subsection{Besedna metrika}

Naj bo $G$ končno generirana grupa, $G = \langle S \rangle$. Cayleyjev graf $\Cay(G,S)$ je metrični prosto glede na razdaljo
\[
d_S(g,h) = \min \{ n \mid n \in \NN_0, \ g s_1 s_2 \cdots s_n = h, \ s_i \in S \cup S^{-1} \}
\]
za $g,h \in G$. Merimo torej dolžino najkrajše poti v $\Cay(G,S)$ med vozliščema $g$ in $h$. Označimo $\ell_S(g) = d_S(g,1)$, to je {\definicija dolžina} elementa $g \in G$ glede na množico $S$.

Naj bo $B_{G,S}(r) = \{ g \in G \mid \ell_S(g) \leq r \}$ {\definicija krogla} polmera $r$ okoli enote $1$ v $\Cay(G,S)$. Sorodno vpeljemo {\definicija sfero} $S_{G,S}(r) = \{ g \in G \mid \ell_S(g) = r \}$. 

    \begin{figure}[t]
    \centering
    \begin{tikzpicture}
    \tikzstyle{vertex}=[circle, fill = oranzna, minimum size = 5pt, inner sep = 0pt]


    \def\distance{2.5}
    \def\factor{2}
    \def\tinyedge{\distance/(\factor * \factor)}

    % \draw[thick, dashed] (4,0) -- (0,4) -- (-4, 0) -- (0, -4) -- (4,0);

    \node[vertex, fill = vijolicna, minimum size = 10pt, label = below left : {$1$}] (1) at (0,0) {};

    \node[vertex, fill = vijolicna, minimum size = 10pt, label = below left : {$x$}] (x) at (\distance,0) {};
    \draw[line width = 0.25mm] (1) -- (x);
    \node[vertex, fill = vijolicna, minimum size = 10pt, label = below left : {$x^{-1}$}] (x-1) at (-\distance,0) {};
    \draw[line width = 0.25mm] (1) -- (x-1);

    % x2
        \draw[line width = 0.25mm] ({\distance * (1 + 1/\factor)},0) -- ({\distance * (1 + 1/\factor)},{\tinyedge});
        \draw[line width = 0.25mm] ({\distance * (1 + 1/\factor)},0) -- ({\distance * (1 + 1/\factor)},{-\tinyedge});
        \draw[line width = 0.25mm] ({\distance * (1 + 1/\factor)},0) -- ({\distance * (1 + 1/\factor) + \tinyedge},0);

    \node[vertex, label = below left : {$x^2$}] (x2) at ({\distance * (1 + 1/\factor)},0) {};
    \draw[line width = 0.25mm] (x) -- (x2);

    % x-2
        \draw[line width = 0.25mm] ({-\distance * (1 + 1/\factor)},0) -- ({-\distance * (1 + 1/\factor)},{-\tinyedge});
        \draw[line width = 0.25mm] ({-\distance * (1 + 1/\factor)},0) -- ({-\distance * (1 + 1/\factor)},{\tinyedge});
        \draw[line width = 0.25mm] ({-\distance * (1 + 1/\factor)},0) -- ({-\distance * (1 + 1/\factor) - \tinyedge},0);

    \node[vertex, label = below left : {$x^{-2}$}] (x-2) at ({-\distance * (1 + 1/\factor)},0) {};
    \draw[line width = 0.25mm] (x-1) -- (x-2);

    \node[vertex, fill = vijolicna, minimum size = 10pt, label = below left : {$y$}] (y) at (0,\distance) {};
    \draw[line width = 0.25mm] (1) -- (y);
    \node[vertex, fill = vijolicna, minimum size = 10pt, label = below left : {$y^{-1}$}] (y-1) at (0,-\distance) {};    
    \draw[line width = 0.25mm] (1) -- (y-1);

    %y2
        \draw[line width = 0.25mm] (0,{\distance * (1 + 1/\factor)}) -- (0,{\distance * (1 + 1/\factor) + \tinyedge});
        \draw[line width = 0.25mm] (0,{\distance * (1 + 1/\factor)}) -- ({\tinyedge},{\distance * (1 + 1/\factor)});
        \draw[line width = 0.25mm] (0,{\distance * (1 + 1/\factor)}) -- ({-\tinyedge},{\distance * (1 + 1/\factor)});

    \node[vertex, label = below left : {$y^2$}] (y2) at (0,{\distance * (1 + 1/\factor)}) {};
    \draw[line width = 0.25mm] (y) -- (y2);

    % y-2
        \draw[line width = 0.25mm] (0,{-\distance * (1 + 1/\factor)}) -- ({-\tinyedge},{-\distance * (1 + 1/\factor)});
        \draw[line width = 0.25mm] (0,{-\distance * (1 + 1/\factor)}) -- ({\tinyedge},{-\distance * (1 + 1/\factor)});
        \draw[line width = 0.25mm] (0,{-\distance * (1 + 1/\factor)}) -- (0,{-\distance * (1 + 1/\factor) - \tinyedge});

    \node[vertex, label = below left : {$y^{-2}$}] (y-2) at (0,{-\distance * (1 + 1/\factor)}) {};
    \draw[line width = 0.25mm] (y-1) -- (y-2);

    % xy

    \node[vertex, label = below left : {$xy$}] (xy) at (\distance,\distance * 1/\factor) {};
    \draw[line width = 0.25mm] (x) -- (xy);
        \draw[line width = 0.25mm] (xy) -- ($(xy) + (0,{\tinyedge})$);
        \draw[line width = 0.25mm] (xy) -- ($(xy) + ({\tinyedge},0)$);
        \draw[line width = 0.25mm] (xy) -- ($(xy) + ({-\tinyedge},0)$);

    % xy-1
    \node[vertex, label = below left : {$xy^{-1}$}] (xy-1) at (\distance,-\distance * 1/\factor) {};
    \draw[line width = 0.25mm] (x) -- (xy-1);
        \draw[line width = 0.25mm] (xy-1) -- ($(xy-1) + (0,{-\tinyedge})$);
        \draw[line width = 0.25mm] (xy-1) -- ($(xy-1) + ({\tinyedge},0)$);
        \draw[line width = 0.25mm] (xy-1) -- ($(xy-1) + ({-\tinyedge},0)$);

    % x-1y
    \node[vertex, label = below left : {$x^{-1}y$}] (x-1y) at (-\distance,\distance * 1/\factor) {};
    \draw[line width = 0.25mm] (x-1) -- (x-1y);
        \draw[line width = 0.25mm] (x-1y) -- ($(x-1y) + (0,{\tinyedge})$);
        \draw[line width = 0.25mm] (x-1y) -- ($(x-1y) + ({\tinyedge},0)$);
        \draw[line width = 0.25mm] (x-1y) -- ($(x-1y) + ({-\tinyedge},0)$);

    % x-1y-1
    \node[vertex, label = below left : {$x^{-1}y^{-1}$}] (x-1y-1) at (-\distance,-\distance * 1/\factor) {};
    \draw[line width = 0.25mm] (x-1) -- (x-1y-1);
        \draw[line width = 0.25mm] (x-1y-1) -- ($(x-1y-1) + (0,{-\tinyedge})$);
        \draw[line width = 0.25mm] (x-1y-1) -- ($(x-1y-1) + ({\tinyedge},0)$);
        \draw[line width = 0.25mm] (x-1y-1) -- ($(x-1y-1) + ({-\tinyedge},0)$);

    % yx
    \node[vertex, label = below left : {$yx$}] (yx) at (\distance * 1/\factor,\distance) {};
    \draw[line width = 0.25mm] (y) -- (yx);
        \draw[line width = 0.25mm] (yx) -- ($(yx) + (0,{\tinyedge})$);
        \draw[line width = 0.25mm] (yx) -- ($(yx) + (0,{-\tinyedge})$);
        \draw[line width = 0.25mm] (yx) -- ($(yx) + ({\tinyedge},0)$);

    % yx-1
    \node[vertex, label = below left : {$yx^{-1}$}] (yx-1) at (-\distance * 1/\factor,\distance) {};
    \draw[line width = 0.25mm] (y) -- (yx-1);
        \draw[line width = 0.25mm] (yx-1) -- ($(yx-1) + (0,{\tinyedge})$);
        \draw[line width = 0.25mm] (yx-1) -- ($(yx-1) + (0,{-\tinyedge})$);
        \draw[line width = 0.25mm] (yx-1) -- ($(yx-1) + ({-\tinyedge},0)$);

    % y-1x
    \node[vertex, label = below left : {$y^{-1}x$}] (y-1x) at (\distance * 1/\factor,-\distance) {};    
    \draw[line width = 0.25mm] (y-1) -- (y-1x);
        \draw[line width = 0.25mm] (y-1x) -- ($(y-1x) + (0,{\tinyedge})$);
        \draw[line width = 0.25mm] (y-1x) -- ($(y-1x) + (0,{-\tinyedge})$);
        \draw[line width = 0.25mm] (y-1x) -- ($(y-1x) + ({\tinyedge},0)$);

    % y-1x-1
    \node[vertex, label = below left : {$y^{-1}x^{-1}$}] (y-1x-1) at (-\distance * 1/\factor,-\distance) {}; 
    \draw[line width = 0.25mm] (y-1) -- (y-1x-1);
        \draw[line width = 0.25mm] (y-1x-1) -- ($(y-1x-1) + (0,{\tinyedge})$);
        \draw[line width = 0.25mm] (y-1x-1) -- ($(y-1x-1) + (0,{-\tinyedge})$);
        \draw[line width = 0.25mm] (y-1x-1) -- ($(y-1x-1) + ({-\tinyedge},0)$);
    \end{tikzpicture}  
    \caption{$B_{F_2, \{ x, y \}}(1)$}
    \end{figure}

\subsection{Funkcija rasti}

{\definicija Funkcija rasti} je funkcija, ki šteje število elementov v krogli oziroma sferi v odvisnosti od polmera. Natančneje imamo torej dve funkciji rasti:
\[
\beta_{G,S} \colon \NN_0 \to \NN_0, \quad n \mapsto |B_{G,S}(n)|
\]
in
\[
\sigma_{G,S} \colon \NN_0 \to \NN_0, \quad n \mapsto |S_{G,S}(n)|.
\]
Funkciji $\beta_{G,S}$ včasih rečemo tudi kumulativna funkcija rasti. Ponavadi z izrazom funkcija rasti mislimo na slednjo funkcijo.

\begin{zgled}
\begin{itemize}
    \item Opazujmo grupo $\ZZ = \langle 1 \rangle$. Za celo število $a \in \ZZ$ je $\ell_{\{ 1 \}}(a) = |a|$. Velja torej
    \[
    \sigma_{\ZZ, \{ 1 \}}(n) = \begin{cases} 1; & n = 0 \\ 2; & n > 0 {\vprasanje \ (\pm n)} \end{cases}
    \]
    in $\beta_{\ZZ, \{ 1 \}}(n) = 2n+1$. Slednja funkcija je linearna, zato rečemo, da imamo {\vprasanje linearno funkcijo rasti}.

    \item Opazujmo grupo $\ZZ = \langle 2,3 \rangle$. Vsako celo število $a \in \ZZ$ lahko zapišemo v obliki $a = 2r + 3s$ za neka $r,s \in \ZZ$. Tako je $\ell_{\{ 2,3 \}}(a) \leq |r| + |s|$. Če je $r \geq 3$, potem lahko $a$ zapišemo tudi v obliki $a = 2(r-3) + 3(s+2)$, s čimer smo skrajšali dolžino zapisa za $1$. Minimalno dolžino zapisa $a$ dobimo torej z zapisom v obliki, ki je odvisna od ostanka $a$ pri deljenju s $3$. Predpostavimo, da je $a \geq 2$.
    \begin{description}
        \item[Če je $a = 3k$ za nek $k \in \ZZ$:] V tem primeru je $\ell_{\{ 2,3 \}}(a) = k$.

        \item[Če je $a = 3k + 2$ za nek $k \in \ZZ$:] V tem primeru je $\ell(a) = k + 1$.

        \item[Če je $a = 3k + 1$ za nek $k \in \ZZ$:] V tem primeru je $\ell(a) = k + 1$, saj imamo zapis $a = 3(k-1) + 2 \cdot 2$.
    \end{description}
    Jasno velja tudi $\ell_{\{ 2, 3 \}}(1) = 2$ in $\ell_{\{ 2, 3 \}}(0) = 0$. Analogne dolžine dobimo za $a < 0$. Skupaj torej velja:
    \[
    \sigma_{\ZZ, \{ 2,3\}}(n) = \begin{cases}
    1; & n = 0 \\
    4; & n = 1 {\vprasanje \ (\pm 2, \pm 3)} \\
    8; & n = 2 {\vprasanje \ (\pm 4, \pm 6, \pm 5, \pm 1)} \\
    6; & n \geq 3 {\vprasanje \ (\pm 3n, \pm(3(n-1)+2), \pm(3(n-2)+2 \cdot 2)}
    \end{cases}
    \]
    \begin{domacanaloga}
    Dokaži, da za tuji števili $r,s \in \ZZ$, $0 < r < s$, velja $\sigma_{\ZZ, \{ r,s \}}(n) = 2s$ za vse dovolj velike $n$.
    \end{domacanaloga}

    Od tod izračunamo kumulativno funkcijo rasti za $n \geq 2$:
    \[
    \beta_{\ZZ, \{2,3 \}}(n) = \sum_{i = 0}^n \sigma_{\ZZ, \{2,3 \}}(i) = 13 + 6(n-2),
    \]
    torej imamo zopet {\vprasanje linearno funkcijo rasti}.

    \item Opazujmo diedrsko grupo $D_\infty = \langle s, t \mid t^2, tst^{-1} = s^{-1} \rangle$. Vsak element $a \in D_\infty$ lahko zapišemo enolično bodisi kot $a = s^k$ bodisi kot $a = t s^k$ za nek $k \in \ZZ$. Dolžina prvega elementa je $|k|$, dolžina drugega pa $|k| + 1$. Tako je
    \[
    \sigma_{D_\infty, \{ s, t \}}(n) = \begin{cases}
    1; & n = 0 \\
    3; & n = 1 {\vprasanje \ (s^{\pm 1}, t)} \\
    4; & n \geq 2 {\vprasanje \ (s^{\pm n}, t s^{\pm (n-1)})} \\
    \end{cases}
    \]
    in za tem $\beta_{D_\infty, \{ s, t \}}(n) = 4 + 4(n-2)$ za $n \geq 2$. Zopet imamo {\vprasanje linearno funkcijo rasti}.

    \item Opazujmo grupo $\ZZ^d$ za $d \in \NN$, generirano s standardno bazo $e_i = (0,\dots,0,1,0,\dots,0)$ za $i = 1,2,\dots,d$. Vsak element $a \in \ZZ^d$ lahko zapišemo enolično kot $a = \sum_{i = 1}^n \alpha_i e_i$ za neke $\alpha_i \in \ZZ$, dolžina takega elementa pa je enaka $\sum_{i = 1}^n |\alpha_i|$. V posebnem primeru $d = 2$ tako dobimo
    \[
    \sigma_{\ZZ^2, \{ e_1, e_2 \}}(n) = |\{ (\alpha_1, \alpha_2) \in \ZZ^2 \mid |\alpha_1| + |\alpha_2| = n \}|.
    \]
    V splošnem lahko za izračun funkcije rasti z upoštevanjem vrednosti $\alpha_d$ izpeljemo rekurzivno zvezo:
    \[
    \sigma_{\ZZ^d, \{ e_1, \dots, e_d \}}(n) = \sigma_{\ZZ^{d-1}, \{ e_1, \dots, e_{d-1} \}}(n) + \sum_{k = 1}^n 2 \cdot \sigma_{\ZZ^{d-1}, \{ e_1, \dots, e_{d-1}\}}(n-k).
    \]
    V posebnem primeru $d = 2$ od tod izpeljemo
    \[
    \sigma_{\ZZ^2, \{ e_1, e_2\}}(n) = \begin{cases}
    1; & n = 0 \\
    2 + \sum_{k=1}^{n-1} 2 \cdot 2 + 2 = 4n; & n > 0
    \end{cases}
    \]
    in s tem $\beta_{\ZZ^2, \{ e_1, e_2 \}}(n) = 1 + \sum_{k = 1}^n 4k = 2n^2 + 2n + 1$. Tu imamo torej {\vprasanje kvadratno funkcijo rasti}.

    Preprosta indukcija z uporabo gornje rekurzivne zveze dokaže, da je $\beta_{\ZZ^d, \{ e_1, \dots, e_d \}}$ {\vprasanje polinom stopnje $d$}.

    \item Opazujmo prosto grupo $F_d$ ranga $d$ z generatorji $S = \{ x_1^{\pm 1}, \dots, x_d^{\pm 1} \}$. Njeni elementi so okrajšane besede v $S$. Velja torej $\sigma_{F_d, S}(n) = 2d \cdot (2d-1)^{n-1}$ za $n \geq 1$ in zato $\beta_{F_d, S}(n) = 1 + \sum_{k=1}^n 2d (2d-1)^{k-1} = 1 + 2d \cdot \frac{(2d-1)^n - 1}{2d - 2}$. V tem primeru imamo torej {\vprasanje eksponentno funkcijo rasti}.
\end{itemize}
\end{zgled}

\subsection{Ekvivalenca funkcij rasti}

Za dani funkciji $f,g \colon \NN \to \RR$ uvedemo oznako
\[
f \preceq g \quad \Leftrightarrow \quad \exists A \in \RR_{> 0} \ \forall n \in \NN \colon \ f(n) \leq A \cdot g(An).
\]
Rečemo, da sta funkciji $f$ in $g$ {\definicija ekvivalentni}, če velja $f \preceq g$ in $g \preceq f$. V tem primeru pišemo $f \sim g$.

\begin{zgled}
Funkciji $\beta_{\ZZ, \{ 1 \}}(n) = 2n + 1$ in $\beta_{\ZZ, \{ 2, 3 \}}(n) = 6n+1$ sta ekvivalentni.
\end{zgled}

V splošnem je rastna funkcija do ekvivalence natančno neodvisna od izbire generirajoče množice.

\begin{trditev}
Vsaki dve rastni funkciji iste grupe sta ekvivalentni.
\end{trditev}

\begin{dokaz}
Naj bo $G = \langle X \rangle = \langle Y \rangle$ za neki končni množici $X, Y$. Definirajmo
\[
A = \max \left\{ \max \{ \ell_Y(x) \mid x \in X \}, \max \{ \ell_X(y) \mid y \in Y \} \right\}.
\]
Torej za $g \in G$ velja $\ell_X(g) \leq A \cdot \ell_Y(g)$ in $\ell_Y(g) \leq A \cdot \ell_X(g)$. Od tod sledi $\sigma_{G, Y}(n) \leq \sigma_{G,X}(An)$ in $\sigma_{G,X}(n) \leq \sigma_{G,Y}(An)$, torej je $\sigma_{G,Y} \sim \sigma_{G,X}$. Enako velja za $\beta_{G,X}$, $\beta_{G,Y}$.
\end{dokaz}

Torej je rast {\em odvisna le od grupe}, če jo obravnavamo do ekvivalence rasti natančno. V taki situaciji bomo rastno funkcijo grupe $G$ označili kar z $\beta_G$.

\begin{zgled}
Rastna funkcija proste grupe $F_d$ ranga $d \geq 2$ je do ekvivalence natančno enaka
\[
\beta_{F_d}(n) \sim 1 + 2d \cdot \frac{(2d-1)^n - 1}{2d-2} \sim 
\frac{d}{d-1} \cdot (2d-1)^n = \frac{d}{d-1} \cdot e^{n \log(2d-1)}
\sim e^n.
\]
Funkcija $\beta_{F_d}$ je zato ekvivalentna funkciji $\beta_{F_2}$, ni pa ekvivalentna funkciji $\beta_{\ZZ}$. Vse nekomutativne proste grupe imajo torej ekvivalentno rast.
\end{zgled}

\section{Izračun rasti}

Iz zgledov v zadnjem razdelku bi se lahko zdelo, da je izračun funkcije rasti zelo preprosta reč -- a temu ni tako! Kot bomo videli v tem razdelku, je izračun funkcije rasti tesno povezan z rešljivostjo besednega problema.

\subsection{Izračunljivost}

Rečemo, da je funkcija $f \colon A \to \NN$ {\definicija izračunljiva}, če obstaja računalniški algoritem,\footnote{S pojmom računalniški algoritem natančneje mislimo na Turingov stroj.} ki ob vnosu $a \in A$ vrne vrednost $f(a)$.

\begin{zgled}
Besedni problem končno prezentirane grupe $G = \langle S \mid R \rangle$ je rešljiv, če in samo če je karakteristična funkcija
\[
c \colon (S \cup S^{-1})^* \to \{ 0, 1 \}, \quad s \mapsto \begin{cases} 0; & \text{$s=1$ v $G$} \\ 1; & \text{$s\neq 1$ v $G$}
\end{cases}
\]
izračunljiva.
\end{zgled}

\subsection{Izračunljivost rasti}

\begin{trditev}
Bodi $G = \langle S \mid R \rangle$ končno prezentirana grupa. Besedni problem v $G$ je rešljiv, če in samo če je njena funkcija rasti izračunljiva.
\end{trditev}

\begin{dokaz}
\underline{($\Rightarrow$):} Število $\beta_{G,S}(n)$ izračunamo tako, da naštejemo vse besede dolžine kvečjemu $n$ s črkami $S \cup S^{-1}$ in z danim algoritmom za reševanje besednega problema preverimo, katere med njimi so enake. \kljuka

\underline{($\Leftarrow$):} Grupo $G$ predstavimo kot $G = F(S)/\llangle R \rrangle$. Elementi grupe $\llangle R \rrangle$ so produkti konjugatov relacij v $R$ in njihovih inverzov. Teh elementov je števno mnogo in jih lahko uredimo v vrsto. Ko beremo to vrsto, torej naštevamo ravno vse besede v $F(S)$, ki so trivialne v $G$.

Opazujmo besedo $w \in (S \cup S^{-1})^*$. Ločimo dva primera.
\begin{description}
    \item[Če je $w=1$ v $G$:] V tem primeru bomo z naštevanjem elementov $\llangle R \rrangle$ slej kot prej v seznamu naleteli na $w$ in s tem dobili zagotovilo, da je res $w = 1$ v $G$.
    \item[Če je $w \neq 1$ v $G$:] V tem primeru se beseda $w$ nikdar ne pojavi v vrsti $\llangle R \rrangle$. Z naštevanjem elementov vrste se torej ne bomo mogli v končnem času\footnote{Če $w \neq 1$, potem sicer besede res ne bomo našli v vrsti, a bomo za zagotovilo, da se to ne zgodi, morali pregledati celotno neskončno vrsto.} odločiti, ali je $w \neq 1$.
\end{description}
Če je $w = 1$ v $G$, potem je torej postopek naštevanja besed iz $\llangle R \rrangle$ algoritem, ki ugotovi, da je res $w = 1$ v $G$. Če je $w \neq 1$, pa ta postopek ne deluje dobro in ga moramo še malo prilagoditi. Za to uporabimo izračunljivo funkcijo rasti $\beta_{G,S}$.

Naj bo $n$ dolžina besede $w \in (S \cup S^{-1})^*$. Najprej sestavimo množico $\mathcal{B}$ vseh besed dolžine kvečjemu $n$ v $S \cup S^{-1}$. Po tem začnemo postopek naštevanja elementov $\llangle R \rrangle$. Če sta kakšna dva elementa 
$x,y$ množice $\mathcal{B}$ enaka, potem bomo to opazili z naštevanjem elementov $\llangle R \rrangle$, saj bomo slej kot prej naleteli na element $x y^{-1}$. Torej bomo z naštevanjem eleemntov $\llangle R \rrangle$ slej kot prej (v končnem času) identificirali dovolj elementov množice $\mathcal{B}$, da nam bo ostalo natanko $\beta_{G,S}(n)$ besed, ki so vse različne v $G$. Po tem z naštevanjem dodatnih elementov $\llangle R \rrangle$ ne bomo več našli nobene enakosti med besedami, saj bi sicer dobili manj kot $\beta_{G,S}(n)$ elementov dolžine kvečjemu $n$ v $G$. Na tej točki zato pogledamo, kaj se je pri postopku zgodilo z besedo $w \in \mathcal{B}$. Če smo kadarkoli tekom postopka identificirali $w$ s trivialno besedo, potem je seveda $w = 1$ v $G$. Če pa do te zadnje točke besede $w$ nismo identificirali s trivialno besedo $1$, potem imamo zagotovilo, da velja $w \neq 1$ v $G$. 
\end{dokaz}

Ker obstajajo končno prezentirane grupe z nerešljivim besednim problemom, obstajajo torej tudi grupe, katerih funkcija rasti {\em ni} izračunljiva.

\section{Osnovne lastnosti rasti}

\subsection{Tipi rasti}

Naj bo $G$ grupa in $S$ neka njena generirajoča podmnožica. Če staknemo besedi dolžine $n$ in $m$ v $G$, dobimo besedo dolžine kvečjemu $n+m$. Obratno, vsako besedo dolžine $n+m$ lahko zapišemo kot konkatenacijo besed dolžine $n$ in $m$. Velja torej
\[
\sigma_{G,S}(n+m) \leq \sigma_{G,S}(n) \cdot \sigma_{G,S}(m)
\quad
\beta_{G,S}(n+m) \leq \beta_{G,S}(n) \cdot \beta_{G,S}(m).
\]
Zaporedji $\{ \sigma_{G,S}(n) \}_{n \in \NN}$, $\{ \beta_{G,S}(n) \}_{n \in \NN}$ sta torej submultiplikativni. Prirejeni logaritmični zaporedji $\{ \log \sigma_{G,S}(n) \}_{n \in \NN}$, $\{ \log \beta_{G,S}(n) \}_{n \in \NN}$ sta zato subaditivni. Za taka zaporedja velja naslednja lastnost.

\begin{lema}{\sc (Feketejev lema)}
Naj bo $\{ x_n \}_{n \in \NN}$ nenegativno subaditivno zaporedje realnih števil. Tedaj obstaja limita $\lim_{n \to \infty} x_n/n$.
\end{lema}

Dokaz je elementaren \vaje. Iz leme sledi, da obstaja limita
\[
\omega_S(G) = \lim_{n \to \infty} \sqrt[n]{\sigma_{G,S}(n)} < \infty.
\]
Če dodatno predpostavimo, da je grupa $G$ neskončna, potem za vsak $n$ velja $\sigma_{G,S}(n) \geq 1$, zato je $\omega_S(G) \geq 1$.

\begin{lema}
Velja $\lim_{n \to \infty} \sqrt[n]{\beta_{G,S}(n)} = \omega_S(G)$.
\end{lema}

\begin{dokaz}
Označimo limito v lemi z $L$. Ker je $\beta_{G,S}(n) \geq \sigma_{G,S}(n)$, je jasno $L \geq \omega_S(G)$. Za obratno neenakost izberimo poljuben $\epsilon > 0$. Tedaj velja $\sigma_{G,S}(n) \leq (\omega_S(G) + \epsilon)^n$ za vse dovolj velike $n$, od koder sledi $\beta_{G,S}(n) \leq C + n \cdot (\omega_S(G) + \varepsilon)^n$ za neko konstanto $C$ in za vse $n \in \NN$. Tako dobimo $L \leq \omega_S(G) + \varepsilon$. Ker je bil $\varepsilon$ poljuben, je dokaz zaključen.
\end{dokaz}

\begin{zgled}
\begin{itemize}
    \item Velja $\sigma_{\ZZ, \{ 1 \}}(n) = 2$ za vse dovolj velike $n$. Zato je $\omega_{\{ 1 \}}(\ZZ) = 1$. Podobno je $\omega_{\{ 2, 3 \}}(\ZZ) = 1$.
    \item Velja $\sigma_{F_d, S}(n) = 2d \cdot (2d - 1)^{n-1}$ za standardno generirajočo množico proste grupe $S$. Zato je $\omega_S(F_d) = 2d-1$.
\end{itemize}
\end{zgled}

Rečemo, da je grupa $G = \langle S \rangle$ {\definicija eksponentne rasti}, če je $\omega_S(G) > 1$. Kadar je $\omega_S(G) = 1$, je grupa {\definicija podeksponentne rasti}. Kadar velja celo $\beta_{G,S}(n) \leq C \cdot n^s$ za neki konstanti $c,s$ in vse $n \in \NN$, pa rečemo, da je grupa {\definicija polinomske rasti}. V tem zadnjem primeru definiramo še njeno {\definicija stopnjo rasti} kot
\[
\deg_S(G) = \limsup_{n \to \infty} \frac{\log \beta_{G,S}(n)}{\log n}.
\]
Grupe, katerih rast ni niti eksponentna niti polinomska, so {\definicija srednje rasti}.

\begin{zgled}
Grupa $\ZZ = \langle 1 \rangle = \langle 2,3 \rangle$ je polinomske rasti. Grupa $F_d$ s standardno množico generatorjev je za $d \geq 2$ eksponentne rasti. {\vprasanje Kaj pa grupe srednje rasti? Ali sploh obstajajo?}
\end{zgled}

\begin{figure}[t]
\centering
\begin{tikzpicture}
\tikzstyle{vertex}=[circle, fill = oranzna, minimum size = 5pt, inner sep = 0pt]

\draw[line width = 3pt, draw = vijolicna, fill = vijolicna!10] (0,0) circle (1.6);

\node[vertex,label = below : $\color{zelena} \ZZ$] at (0,0) {};
\draw[thick] (0,0) circle (2);
\node[label = below : $\vprasanje \mathbf{?}$] at (1.05,-1.10) {};
\draw[thick] (0,0) circle (3.5);
\node[vertex,label = below : $\color{zelena} F_d$] at (3,0) {};

\draw[->,thick,draw = vijolicna] (-3.5, -2) to [bend left = 30] (-1.35, 0);
\node[label = below : {\color{vijolicna} polinomske rasti}] at (-4.5, -2) {};

\draw[->,thick,draw = vijolicna] (-0.5,-4) to [bend left = 10] (0, -1.75);
\node[label = below : {\color{vijolicna} srednje rasti}] at (-0.5,-4) {};

\draw[->,thick,draw = vijolicna] (3, -2.5) to [bend left = 10] (2, -1.5);
\node[label = right : {\color{vijolicna} eksponentne rasti}] at (3, -2.5) {};
\end{tikzpicture}
\caption{Diagram grup polinomske, srednje in eksponentne rasti}
\end{figure}


Ker sta vsaki dve rastni funkciji dane grupe ekvivalentni, sledi naslednja trditev.

\begin{trditev}
Tip rasti grupe (eksponentna, podeksponentna, polinomska, srednja) je neodvisen od izbire generatorjev. V primeru polinomske rasti je tudi stopnja rasti neodvisna od izbire generatorjev.
\end{trditev}

Od zdaj naprej bomo zato v primeru polinomske rasti namesto $\deg_S(G)$ pisali kar $\deg(G)$. 

V splošnem sama vrednost $\omega_S(G)$ {\em ni} neodvisna od izbire generatorjev. Označimo
\[
\Omega(G) = \inf_{S \subseteq G, \ \langle S \rangle = G} \omega_S(G).
\]
Če za grupo $G$ velja $\Omega(G) > 1$, potem rečemo, da je $G$ {\definicija enakomerno eksponentne rasti}.

\begin{zgled}
Opazujmo prosto grupo $F_2$. Premislimo, da velja $\Omega(F_2) \geq 3$, torej je $F_2$ enakomerno eksponentne rasti. V ta namen naj bo $F_2 = \langle S \rangle$ za neko končno podmnožico $S \subseteq F_2$. V $S$ izberimo dva elementa, ki ne komutirata, recimo jima $x,y$. Ta dva elementa zato generirata prosto podgrupo $H = \langle x, y \rangle \leq F_2$. Tako velja $\beta_{F_2, S}(n) \geq \beta_{H, \{ x, y \}}(n) = 4 \cdot 3^{n-1}$. Od tod sledi $\omega_S(F_2) \geq 3$. Torej je res $\Omega(F_2) \geq 3$.

Ni težko premisliti \vaje, da za prosto grupo $F_d$ ranga $d$ velja celo $\Omega(F_d) = 2d - 1$.
\end{zgled}

\subsection{Rast in podgrupe ter kvocienti}

Pri jemanju podgrup in kvocientov dane grupe se rastni tip ne {\em poveča}.

\begin{trditev}
Naj bosta $G,H$ končno generirani grupi, $H \leq G$, in naj bo $N \unlhd G$. Če je $G$ polinomske ali podeksponentne rasti, sta taki tudi $H$ in $G/N$. V primeru polinomske rasti rasti velja $\deg(H), \deg(G/N) \leq \deg(G)$.
\end{trditev}

\begin{dokaz}
Naj bo $G = \langle S \rangle, H = \langle T \rangle$. Naj bo $C = \max \{ \ell_S(t) \mid t \in T \}$. Tedaj je $\beta_{H,T}(n) \leq \beta_{G,S}(Cn)$. Očitno je tudi $\beta_{G/N, SN/N}(n) \leq \beta_{G,S}(n)$. Trditev sledi.
\end{dokaz}

\begin{zgled}
Opazujmo prosto grupo $F_2 = \langle x,y \rangle$. Ta vsebuje podgrupo $\langle x \rangle \cong \ZZ$. Torej se ob prehodu na podgrupo rast lahko bistveno upočasni -- od eksponentne do polinomske.
\end{zgled}

Pri podgrupah končnega indeksa in kvocientih po končnih podgrupah edinkah imamo več nadzora.

% \begin{trditev}
% Naj bo $G$ končno generirana grupa, $H \leq G$, $|G:H| < \infty$, $N \unlhd G$, $|N| < \infty$.Tedaj so rastne funkcije grup $G$, $H$ in $G/N$ ekvivalentne. V primeru polinomske rasti je zato $\deg(G) = \deg(H) = \deg(G/N)$. Velja še
% $\Omega(G) = \Omega(G/N)$ in $\Omega(H) > 1 \Rightarrow \Omega(G) > 1$.
% \end{trditev}

\begin{trditev}
Naj bo $G$ končno generirana grupa, $H \leq G$, $|G:H| < \infty$, $N \unlhd G$, $|N| < \infty$. Tedaj so rastne funkcije grup $G$, $H$ in $G/N$ ekvivalentne. V primeru polinomske rasti je zato $\deg(G) = \deg(H) = \deg(G/N)$.
Velja še implikacija $\Omega(H) > 1 \Rightarrow \Omega(G) > 1$.
\end{trditev}

\begin{dokaz}
Rastne funkcije obravnavajmo le do ekvivalence natančno. Pišimo jih zato kar kot $\beta_G, \beta_H, \beta_{G/N}$. Vemo že  $\beta_H, \beta_{G/N} \preceq \beta_G$. 
Dokaz trditve sloni na naslednjih dveh pomožnih trditvah.

    \begin{trditev}
    Naj bo $G = \langle S \rangle$. Tedaj lahko izberemo $R \subseteq G$, ki vsebuje natanko enega predstavnika iz vsakega odseka $H$ v $G$. Pri tem lahko dodatno dosežemo, da je $1 \in R$ in da velja $\max \{ \ell_S(r) \mid r \in R \} < |G:H|$.
    \end{trditev}

    \begin{dokaz}
    Dokaz je elementaren. \vaje
    \end{dokaz}

    \begin{trditev}
    Obstaja končna množica $U \subseteq H$ z lastnostjo $H = \langle U \rangle$ in $\forall h \in H \colon \ \ell_U(h) \leq \ell_S(h)$.
    \end{trditev}

    \begin{dokaz}
    Naj bo $h \in H$. Pišimo $h = s_1 s_2 \cdots s_k$ za neke $s_i \in S \cup S^{-1}$, kjer je $k = \ell_S(h)$. Tedaj lahko po vrsti izberemo $1 = u_0, u_1, \dots, u_{k-1}, u_k \in R$, da velja $s_i u_i^{-1} \in u_{i-1}^{-1}H$, se pravi $u_{i-1} s_i u_i^{-1} \in H$ za vsak $i$. Torej je
    \[
    h u_k^{-1}= u_0 s_1 u_1^{-1} \cdot u_1 s_2 u_2^{-1} \cdots u_{k-1} s_k u_k^{-1} \in H,
    \]
    zato je $u_k \in H$ in torej $u_k = 1$. Tako smo zapisali $h$ kot produkt dolžine $k$ elementov iz končne množice $U = \{ x s y \mid x, y \in R \cup R^{-1}, \ s \in S \cup S^{-1} \}$. Ker je bil $h$ poljuben, od tod sledi, da množica $U$ generira $H$ in da pri tem velja $\ell_U(h) \leq k = \ell_S(h)$ z vsak $h \in H$. \kljuka
    \end{dokaz}


\underline{$\beta_G \preceq \beta_H$:} Označimo $C = \max \{ \ell_S(r) \mid r \in R \}$. Naj bo $g \in B_{G,S}(n)$. Pišemo lahko $g = rh$ za neka $r \in R$, $h \in H$. Sledi $\ell_S(h) \leq \ell_S(g) + \ell_S(r) \leq n + C$. Zato velja tudi $\ell_U(h) \leq \ell_S(h) \leq n + C$, torej je $h \in B_{H,U}(n+C)$. Tako smo izpeljali neenakost $\beta_{G,S}(n) \leq |R| \cdot \beta_{H,U}(n+C) \leq |G:H| \cdot \beta_{H,U}((C+1)n)$. Tako je res $\beta_G \preceq \beta_H$. \kljuka

\underline{$\beta_G \preceq \beta_{G/N}:$} Naj bo $G/N = \langle X \rangle$. Za vsak $x \in X$ izberimo dvig $\tilde x \in G$, torej $x = \tilde x N$. Naj bo $Y = \{ \tilde x \mid x \in X \} \cup N$. Velja $G = \langle Y \rangle$. Za vsak $g \in B_{G,Y}(n)$ velja $gN \in B_{G/N, Y}(n)$, torej lahko pišemo $gN = x_1 \cdots x_k$ za nek $k \leq n$. To pomeni $g = \tilde x_1 \cdots \tilde x_k n$ za nek $n \in N$. Tako je $\beta_{G,Y}(n) \leq \beta_{G/N,X}(n) \cdot |N|$, zato je res $\beta_G \preceq \beta_{G/N}$. \kljuka
% \underline{$\Omega(G) = \Omega(G/N):$} Iz zadnjega odstavka sledi, da za $G/N = \langle X \rangle$ velja $\omega_X(G/N) = \omega_Y(G)$. Tako je $\Omega(G/N) = \inf_{X} \omega_X(G/N) \geq \inf_S \omega_S(G) = \Omega(G)$. Velja pa tudi obratno: za $G = \langle S \rangle$ je $\omega_S(G) \geq \omega_{SN/N}(G/N)$, zato je $\Omega(G) \geq \Omega(G/N)$. \kljuka

\underline{$\Omega(H) > 1 \Rightarrow \Omega(G) > 1$:} Naj bo $S \subseteq G$ neka generirajoča množica grupe $G$. Po drugi pomožni trditvi obstaja $U \subseteq H$, tako da je $\langle U \rangle = H$ in za vsak $h \in H$ velja $\ell_U(h) \leq \ell_S(h)$. Po konstrukciji množice $U$ za vsak $u \in U$ velja $\ell_S(u) \leq 2 \cdot \max \{ \ell_S(r) \mid r \in R \} + 1$, zato po prvi pomožni trditvi velja $\ell_S(u) < 2 |G:H| + 1$. Tako velja $\beta_{H,U}(n) \leq \beta_{G,S}(n \cdot (2|G:H| + 1))$. Hkrati je $\beta_{H,U}(n) \geq A c^n$, kjer je $c = \Omega(H) > 1$ odvisna le od $H$, konstanta $A$ pa je morda odvisna od $U$. Od tod izpeljemo, da za vsak $n \in \NN$ velja
\[
\beta_{G,S}(n \cdot (2 |G:H| + 1)) \geq A c^n,
\]
zato za neskončno mnogo naravnih števil $m$ velja
\[
\beta_{G,S}(m) \geq A c^{\frac{m}{2 |G:H| + 1}}.
\]
S tem je $\omega_S(G) \geq c^{\frac{1}{2 |G:H| + 1}}$, kar implicira $\Omega(G) \geq \Omega(H)^{\frac{1}{2|G:H| + 1}} > 1$. \kljuka
\end{dokaz}

\begin{odprtproblem}
Ali v zadnji trditvi velja obrat zadnje implikacije? Se pravi, ali ima podgrupa končnega indeksa v grupi enakomerno eksponentne rasti tudi sama enakomerno eksponentno rast?
\end{odprtproblem}

\begin{zgled}
Opazujmo diedrsko grupo $D_\infty = \langle s, t \mid t^2, tst^{-1}=s^{-1} \rangle$. Ta vsebuje podgrupo edinko $\langle s \rangle \cong \ZZ$ indeksa $2$. Torej je $D_\infty$ virtualno $\ZZ$.  Zato sta rastni funkciji grup $D_\infty$ in $\ZZ$ ekvivalentni. Grupa $D_\infty$ ima tako polinomsko rast stopnje $1$.
\end{zgled}

Nazadnje premislimo še, kaj se zgodi s podgrupami neskončnega indeksa v grupah polinomske rasti. Ta kriterij bo uporaben pri induktivnih sklepih v nadaljevanju.

\begin{trditev}
Bodi $G$ grupa polinomske rasti. Naj bo $H$ končno generirana podgrupa neskončnega indeksa v $G$. Tedaj je $\deg(H) \leq \deg(G) - 1$. Sorodno za vsako končno generirano neskončno podgrupo edinko $N$ v $G$ velja $\deg(G/N) \leq \deg(G) - 1$.
\end{trditev}

\begin{dokaz}
Dokažimo le prvi del trditve glede pogrup, drugi del glede kvocientov je mnogo lažji. 

Naj bo $H = \langle T \rangle$, $G = \langle S \rangle$, privzamemo lahko $T \subseteq S$. Označimo $S = \{ s_1, s_2, \dots, s_m \}$. Opazujmo odseke $H$ v $G$. Začnemo z odsekom $s_1 H$; označimo $x_1 = s_1$, opazujemo torej odsek $x_1 H$. Ker je $|G:H| = \infty$, mora vsaj en generator $s_i$ premakniti $s_1H$ v nek drugo odsek $s_i s_1 H$; označimo $x_2 = s_i s_1$, dobimo torej odsek $x_2 H$. Postopek generiranja novih odsekov nadaljujemo na ta način, tako dobimo paroma različne odseke $x_i H$ za $i \in \NN$. Pri tem po konstrukciji velja $\ell_S(x_i) \leq i$. Tako za vsak $n \in \NN$ dobimo {\em injektivno} preslikavo
\[
\{ x_1, x_2, \dots, x_n \} \times B_{H,T}(n) \to B_{G,S}(2n), \quad
(x_i, h) \mapsto x_i \cdot h.
\]
Od tod sledi $n \cdot \beta_{H,T}(n) \leq \beta_{G,S}(2n)$, kar implicira $1 + \deg(H) \leq \deg(G)$.
\end{dokaz}

\begin{zgled}
Opazujmo naraščajočo verigo grup $\ZZ \leq \ZZ^2 \leq \ZZ^3 \leq \cdots$. Grupa $\ZZ^i$ v tej verigi ima polinomsko rast stopnje $i$.
\end{zgled}

\chapter{Polinomska rast}

\section{Rast nilpotentnih grup}

\subsection{Polinomska rast nilpotentnih grup}

V tem razdelku bomo dokazali, da so nilpotente grupe polinomske rasti. Ker sta rastni funkciji grupe in njene podgrupe končnega indeksa ekvivalentni, bomo s tem dokazali celo naslednje.

\begin{izrek}
Končno generirane virtualno nilpotentne grupe so polinomske rasti.
\end{izrek}

\begin{dokaz}
Naj bo $G$ končno generirana nilpotentna grupa. V posebnem je $G$ policiklična. Z indukcijo na Hirschovo dolžino $h(G)$ bomo dokazali, da je $G$ polinomske rasti.

Baza indukcije je $h(G) = 0$. V tem primeru je grupa $G$ končna, torej je polinomske rasti. \kljuka

Naj ima zdaj $G$ verigo podgrup $G = G_1 \unrhd G_2 \unrhd \cdots \unrhd G_r \unrhd G_{r+1} = 1$ s cikličnimi kvocienti, torej za vsak $i$ lahko pišemo $G_i = \langle G_{i+1}, x_i \rangle$ za nek $x_i \in G$. Naj bo $X = \{ x_1, \dots, x_r \}$. Velja $G = \langle X \rangle$. 

Če je $|G : G_2| < \infty$, potem lahko preidemo na grupo $G_2$, pri čemer sta funkciji rasti grup $G$ in $G_2$ ekvivalentni. Brez škode lahko torej privzamemo, da je $G/G_2 \cong \ZZ$. S tem je $h(G_2) = h(G) - 1$, zato ima po indukciji grupa $G_2$ polinomsko rast.

Ocenimo vrednost $\beta_{G,X}(n)$ za poljuben $n \in \NN$. V ta namen izberimo $x \in B_{G,X}(n)$, ki ga lahko zapišemo v obliki $x = y_1 y_2 \dots y_n$, kjer je $y_i \in X \cup X^{-1}$. Ta element bomo prepisali v obliko $x = x_1^e \cdot z$ za neka $e \in \ZZ, z \in G_2$, in sicer z uporabo naslednjega postopka.

\begin{description}
    \item[Korak 1:] V besedi $y_1 y_2 \dots y_n$ poiščemo pojavitev simbola $x_1$ ali $x_1^{-1}$, ki je desno od nekega drugega simbola iz množice $X \cup X^{-1}$, recimo mu $s$.\footnote{Lahko se na primer zgodi, da imamo nekje v besedi pojavitev oblike $x_2 \cdot x_1$.} Tak par zamenjamo z izrazom $x_1^{\pm 1} s [s, x_1^{\pm 1}]$. Na ta način smo element $x_1^{\pm 1}$ prestavili z desne na levo od $s$. Zdaj element $x_1^{\pm 1}$ še naprej potiskamo na levo, dokler levo od njega ni nobenih drugih generatorjev iz množice $X \cup X^{-1}$.

    \item[Korak 2:] V novi besedi poiščemo še kakšno drugo morebitno pojavitev simbola $x_1$ ali $x_1^{-1}$, ki še ni zbrana na začetku besede. Naj bo $s$ simbol, ki je zapisan levo od $x_1^{\pm 1}$. Pri tem je $s \in X \cup X^{-1} \cup [X^{\pm 1},X^{\pm 1}]$. Tako pojavitev zopet zamenjamo z $x_1^{\pm 1} s [s, x_1^{\pm 1}]$.\footnote{Lahko se na primer zgodi, da imamo nekje v besedi pojavitev oblike $x_3 \cdot x_1$ ali pa $[x_2, x_1] \cdot x_1$. Tako pojavitev zamenjamo z $x_1 x_3 [x_3, x_1]$ oziroma z $x_1 [x_2, x_1] [x_2, x_1, x_1]$.} Zdaj element $x_1^{\pm 1}$ še naprej potiskamo na levo, dokler pred njim ni niti drugih generatorjev niti komutatorjev.

    \item[Korak $i$:] V novi besedi poiščemo še kakšno drugo morebitno pojavitev simbola $x_1$ ali $x_1^{-1}$, ki še ni zbrana na začetku besede. To pojavitev potiskamo na levo z enakim postopkom, kot je opisan v prejšnjih korakih, in sicer zamenjamo vrstni red komutatorja uteži največ $i$ z $x_1^{\pm 1}$, pri čemer uvedemo nov komutator uteži največ $i+1$ z vnosi iz množice $X \cup X^{-1}$.
\end{description}

Te korake izvajamo tako dolgo, dokler niso vse pojavitve $x_1$ in $x_1^{-1}$ zbrane na levi strani besede. Na ta način dobimo želen zapis $x = x_1^e \cdot z$ za neka $e \in \ZZ, z \in G_2$. Pri tem lahko $e$ in $\ell_X(z)$ omejimo na naslednji način.

\begin{itemize}
    \item Ker je $x \in B_{G,X}(n)$, sledi $|e| \leq n$, zato imamo največ $2n+1$ možnosti za $e$.

    \item Po {\em Koraku 1} imamo največ $n$ komutatorjev uteži $2$.
    \item Po {\em Koraku 2} imamo največ $n$ komutatorjev uteži $3$ in največ $n+n = 2n$ komutatorjev uteži $2$.\footnote{Imamo namreč največ $n$ komutatorjev uteži $2$ iz prejšnjega koraka in hkrati uvedemo največ $n$ novih komutatorjev uteži $2$.} 
    \item Po {\em Koraku 3} imamo največ $n$ komutatorjev uteži $4$, največ $n + 2n = 3n$ komutatorjev uteži $3$ in največ $2n + n = 3n$ komutatorjev uteži $2$.
    \item Po {\em Koraku $i$} imamo največ $\binom{i}{k-1} \cdot n$ komutatorjev uteži $k$.
\end{itemize}

Število izvedenih korakov je največ $n$. Hkrati je grupa $G$ nilpotentna, recimo razreda nilpotentnosti $c$, zato so vsi komutatorji uteži vsaj $c+1$ trivialni v $G$. Tako je po zadnjem koraku ({\em Koraku $n$}) element $x = x_1^e z$ zapisan kot produkt $x_1^e$ in produkt največ $\binom{n}{k-1} \cdot n$ komutatorjev uteži $k$ za vse $k = 1, \dots, c$. Velja $\binom{n}{k-1} \cdot n \leq n^k$, zato je $z$ zapisan kot produkt največ $n + n^2 + \cdots + n^c$ komutatorjev uteži kvečjemu $c$ z vnosi iz $X \cup X^{-1}$. Vsak od teh komutatorjev uteži vsaj $2$ je element podgrupe $G_2 = \langle x_2, \dots, x_r \rangle$, saj je $G/G_2$ abelova grupa. Naj bo
\[
A = \max \{ \ell_{\{ x_2, \dots, x_r \}}(k) \mid \text{$k \in G_2$ je komutator uteži $\leq c$ z vnosi iz $X \cup X^{-1}$}\}.
\]
Tako za vsak dovolj velik $n$ velja
\[
\ell_{\{ x_2, \dots, x_r \}}(z) \leq n + A(n^2 + \cdots + n^c) < A n^{c+1}.
\]
Nazadnje torej velja
\[
\beta_{G,X}(n) \leq (2n+1) \cdot \beta_{G_2, \{ x_2, \dots, x_r \}}(A n^{c+1}).
\]
Ker je $\beta_{G_2, \{ x_2, \dots, x_r \}}$ po indukcijski predpostavki polinom, je torej funkcija $\beta_{G,X}$ omejena s polinomom, zato je $G$ polinomske rasti.
\end{dokaz}

\subsection{Rast Heisenbergove grupe}

\begin{zgled}
Naj bo $H$ Heisenbergova grupa. Označimo
\[
x = 
\begin{pmatrix} 
1 & 0 & 0 \\
0 & 1 & 1 \\
0 & 0 & 1
\end{pmatrix}, \quad
y = 
\begin{pmatrix} 
1 & 1 & 0 \\
0 & 1 & 0 \\
0 & 0 & 1
\end{pmatrix}, \quad
z = 
\begin{pmatrix} 
1 & 0 & 1 \\
0 & 1 & 0 \\
0 & 0 & 1
\end{pmatrix}.
\]
Velja $[y,x] = z$. Dolžino elementov grupe $H$ merimo glede na generirajočo množico $\{ x, y, z \}$. Vsak element $h \in H$ lahko enolično zapišemo v obliki
\[
h = 
\begin{pmatrix} 
1 & b & c \\
0 & 1 & a \\
0 & 0 & 1
\end{pmatrix}
= x^a y^b z^c.
\]
Pri zbiranju kot v dokazu zadnjega izreka uporabimo naslednje lastnosti, ki veljajo v grupi $H$:
\begin{align*}
(x^a y^b z^c) \cdot x^{\pm 1} &= x^{a \pm 1} y^b z^{c \pm b}, \\
(x^a y^b z^c) \cdot y^{\pm 1} &= x^1 y^{b \pm 1} z^c, \\
(x^a y^b z^c) \cdot z^{\pm 1} &= x^a y^b z^{c \pm 1}.
\end{align*}
{\vprasanje Kaj lahko povemo o dolžini $\ell(x^a y^b z^c)$?}\footnote{Pozor: v grupi $H$ velja na primer $\ell(xyz) = \ell(yx) = 2$, torej v splošnem {\em ni} res $\ell(x^a y^b z^c) = |a| + |b| + |c|$.} 

\begin{itemize}
    \item Prva lastnost dolžine je:

\begin{trditev}
$\ell(x^a y^b z^c) \leq |a| + |b| + 6 \sqrt{|c|}$
\end{trditev}

\begin{dokaz}
Iz komutatorskih identitet sledi, da je komutator bilinearen v grupah razreda nilpotentnosti $2$. Zato za vsak $i,j$ velja $[y^i, x^j] = [y,x]^{ij} = z^{ij}$. Opazujmo element $z^c$, ki ga s pomočjo zadnjega trika prepišimo na naslednji način. Naj bo $i = \lfloor \sqrt{|c|} \rfloor$ in $j = c - i^2$. Tedaj je $z^c = z^{i^2} z^j = [y^i, x^i] z^j$, torej je $\ell(z^c) \leq 4 i + j \leq 4 \sqrt{|c|} + c - (\sqrt{|c|} - 1)^2 \leq 6 \sqrt{|c|}$. S tem je $\ell(x^a y^b z^c) \leq |a| + |b| + 6 \sqrt{|c|}$. \kljuka
\end{dokaz}


Od tod sledi
\[
\beta_{H}(n) \geq |\{ (a,b,c) \in \ZZ^3 \mid |a| + |b| + 6 \sqrt{|c|} \leq n\}| \geq \left( \frac{n}{8} \right) \cdot \left( \frac{n}{8} \right) \cdot \left( \frac{n}{8} \right)^2,
\]
zato je $n^4 \preceq \beta_H$.

\item Druga lastnost dolžine je:

\begin{trditev}
$\ell(x^a y^b z^c) \leq n \Rightarrow |a| + |b| \leq n \text{ in } |c| \leq n^2$
\end{trditev}

\begin{dokaz}
Predpostavimo $\ell(x^a y^b z^c) \leq n$. 

\underline{$|a| + |b| \leq n$:} Najprej velja
\[
\ell_{H, \{ x, y, z \}}(x^a y^b z^c) \geq \ell_{H / H^{(2)}, \{ x H^{(2)}, y H^{(2)} \}}(x^a y^b H^{(2)}).
\]
Ker je $H / H^{(2)} \cong \ZZ^2$, je zadnja dolžina enaka $|a| + |b|$. Tako je res $n \geq |a| + |b|$. \kljuka

\underline{$|c| \leq n^2$:} Dokazujemo z indukcijo na $n$. V primeru $n = 1$ je seveda $|c| \leq 1$. Predpostavimo zdaj, da trditev že velja za nek $n$. Naj bo $\ell(x^a y^b z^c) \leq n + 1$, torej se $x^a y^b z^c$ lahko zapiše kot beseda $w$ dolžine kvečjemu $n+1$. Ta beseda se konča z $x^{\pm 1}$, $y^{\pm 1}$ ali $z^{\pm 1}$. Torej lahko $w$ skrajšamo za $1$, če jo pomnožimo z desne z elementom $s \in \{ x^{\mp 1}, y^{\mp 1}, z^{\mp 1} \}$. Tako je $\ell(x^a y^b z^c \cdot s) \leq n$. Po indukciji in zgornjih treh lastnostih množenja v grupi $H$ je zato $|c \mp X| \leq n^2$, kjer je $X = 1$ v primeru $b = 0$ in $X = b$ sicer. Ker je $|b| \leq n$, od tod sledi $|c| \leq |X| + n^2 \leq n + n^2 < (n+1)^2$. Indukcijski korak je zaključen. \kljuka
\end{dokaz}


Od tod sledi
\[
\beta_H(n) \leq (2n+1) \cdot (2n+1) \cdot (2n^2 + 1),
\]
zato je $\beta_H \preceq n^4$. 

\end{itemize}

Ko obe lastnosti dolžine združimo, tako sklepamo, da je $\beta_H$ ekvivalentna $n^4$. Pri tem smo zelo podrobno upoštevali strukturo grupe $H$, tako natančnega rezultata o rasti ne bi mogli dobiti le z neposredno uporabo postopka v zadnjem izreku.
\end{zgled}

\subsection{Natančna stopnja rasti nilpotente grupe}

Podobno analizo kot v primeru Heisenbergove grupe lahko izvedemo v splošnem\footnote{A mi tega ne bomo naredili, primer Heisenbergove grupe bo dovolj.} in tako dobimo ekspliciten opis stopnje rasti nilpotentne grupe.

\begin{izrek}
Naj bo $G$ končno generirana nilpotentna grupa. Naj bo $r_i = h(\gamma_i(G) / \gamma_{i+1}(G))$. Tedaj je $\deg(G) = \sum_i i \cdot r_i$.
\end{izrek}

\begin{zgled}
Za Heisenbergovo grupo velja $H \geq \gamma_2(H) \geq \gamma_3(H) = 1$ s kvocientoma $\ZZ^2$ in $\ZZ$. Torej iz izreka sledi $\deg(H) = 1 \cdot 2 + 2 \cdot 1 = 4$, kot smo izračunali v prejšnjem zgledu.
\end{zgled}

\section{Rast rešljivih grup}

\subsection{Rast svetilkarjeve grupe}

Rešljive grupe {\em niso} nujno polinomske rasti.

\begin{zgled}
Naj bo $L$ svetilkarjeva grupa. To je rešljiva grupa, ki jo predstavimo kot $L = (\bigoplus_{i \in \ZZ} \ZZ/2\ZZ) \rtimes \ZZ$. Dokažimo, da je $L$ eksponentne rasti. V ta namen si za vsak $n \in \NN$ oglejmo množico
\[
E_n = \{ (\vec{v}, m) \mid m \in \ZZ, \ m \in [-n, n], \ \vec{v} \in \prod_{i \in \ZZ} \ZZ/2\ZZ, \ j \notin [-n,n] \Rightarrow \vec{v}_j = 0 \} \subseteq L.
\]
Velja $|E_n| = 2^{2n+1} (2n+1)$. Vsak element $(\vec{v}, m) \in E_n$ si lahko predstavljamo kot svetilkarjev sprehod: svetilkar se najprej sprehodi do $- n$, potem izvede več korakov prižiganja svetilke in premikanja za $1$ v levo, nazadnje pa se odpravi do svoje končne lokacije. Dolžina takega elementa je torej kvečjemu $n + 2(2n+1) + 2n = 7n+2$. Tako za standardno množico generatorjev grupe $L$ velja
\[
\beta_L(7n+2) \geq 2^{2n+1}(2n+1).
\]
Grupa $L$ je zato res eksponentne rasti.
\end{zgled}

\subsection{Rešljive grupe podeksponentne rasti}

V zadnjem primeru grupa $L$ {\em ni} policiklična, take rešljive grupe pa nikdar niso niti podeksponentne rasti.

\begin{izrek}
Končno generirane rešljive grupe podeksponentne rasti so policiklične.
\end{izrek}

Izrek bomo izpeljali kot posledico naslednje trditve.

\begin{trditev}{\sc (podeksponentna $\Rightarrow$ izvedena k.g.)}
Bodi $G$ končno generirana grupa podeksponentne rasti. Tedaj je grupa $[G,G]$ končno generirana.
\end{trditev}

\begin{dokaz}[Trditev implicira izrek]
Naj bo $G$ končno generirana rešljiva grupa podeksponentne rasti. Po trditvi je $G^{(2)}$ tudi končno generirana rešljiva grupa podeksponentne rasti, prav take so induktivno zato tudi vse podgrupe $G^{(i)}$ grupe $G$. Torej so kvocienti izvedene vrste $G^{(i)}/G^{(i+1)}$ končno generirane abelove grupe, zato so policiklični. Tako je tudi $G$ policiklična grupa.
\end{dokaz}

\begin{dokaz}[Dokaz trditve {\sc (podeksponentna $\Rightarrow$ izvedena k.g.)}]
Ker je grupa $G/[G,G]$ končno generirana abelova grupa, je policiklična. Zato bo dovolj dokazati, da za vsako podgrupo edinko $N \unlhd G$ s cikličnim kvocientom $G/N$ velja, da je $N$ končno generirana.

Če je $|G:N| < \infty$, potem je jasno $N$ končno generirana. Privzemimo zdaj, da je $G/N = \langle x N \rangle \cong \ZZ$ za nek $x \in G$. Naj bo $G = \langle x_1, \dots, x_d \rangle$. Vsakega od generatorjev lahko zapišemo kot $x_i = x^{e_i} y_i$ za neka $e_i \in \ZZ$, $y_i \in N$. Jasno je $\llangle y_1, \dots, y_d \rrangle \leq N$. Hkrati je kvocientna grupa $G/\llangle y_1, \dots, y_d \rrangle$ generirana z odsekom elementa $x$. Torej je $G/\llangle y_1, \dots, y_d \rrangle$ neskončna ciklična grupa s kvocientom $G/N \cong \ZZ$. To je mogoče le v primeru $N = \llangle y_1, \dots, y_d \rrangle$.

Za vsak $i = 1, 2, \dots, d$ definirajmo $K_i = \langle \{ x^{n} y_i x^{-n} \mid n \in \ZZ \} \rangle \leq N$. Grupa $\langle K_1, \dots, K_d \rangle$ je edinka v $G$, ki je vsebovana v $N$, hkrati pa vsebuje vse $y_1, \dots, y_d$. Zato je $\langle K_1, \dots, K_d \rangle = N$. 

\underline{Za vsak $i$ je grupa $K_i$ končno generirana:}
Opazujmo elemente
\[
x y_i^{\alpha_1} x y_i^{\alpha_2} \cdots x y_i^{\alpha_n} \in G,
\]
kjer je $\alpha_j \in \{ 0, 1 \}$. Vseh takih besed je $2^n$, vsaka od njih je dolžine kvečjemu $2n$. Iz predpostavke o podeksponentni rasti sta zato za dovolj velik $n$ dve taki besedi enaki v $G$. Naj bo $n_0$ najmanjše naravno število, pri katerem se to zgodi. Velja torej
\[
\prod_{j = 1}^{n_0} x y_i^{\alpha_j} = \prod_{j = 1}^{n_0} x y_i^{\beta_j} 
\]
za neke $\alpha_j, \beta_j \in \{ 0, 1 \}$.
Uvedimo oznako $y_i^{[k]} = x^k y_i x^{-k}$. Velja torej
\[
\prod_{j = 1}^{n_0} \left( y_i^{[j]} \right)^{\alpha_j} =
\prod_{j = 1}^{n_0} \left( y_i^{[j]} \right)^{\beta_j}.
\]
Zaradi minimalnosti $n_0$ je pri tem $\alpha_{n_0} \neq \beta_{n_0}$, od koder sledi
\[
y_i^{[n_0]} \in \langle y_i^{[1]}, y_i^{[2]}, \dots, y_i^{[n_0 - 1]} \rangle.
\]
Zdaj lahko izpeljemo
\begin{align*}
y_i^{[n_0 + 1]} \in x \langle y_i^{[1]}, y_i^{[2]}, \dots, y_i^{[n_0 - 1]} \rangle x^{-1} &= \langle  y_i^{[2]}, y_i^{[3]}, \dots, y_i^{[n_0]} \rangle \\
&\leq \langle y_i^{[1]}, y_i^{[2]}, \dots, y_i^{[n_0 - 1]} \rangle.
\end{align*}
Induktivno sledi, da za vsak $n \geq n_0$ velja $y_i^{[n]} \in \langle y_i^{[1]}, y_i^{[2]}, \dots, y_i^{[n_0 - 1]} \rangle$. Analogno lastnost lahko izpeljemo za vse dovolj majhne negativne vrednosti $n$.\footnote{To storimo tako, da v dokazu $x$ zamenjamo z $x^{-1}$.} Torej obstaja $N_0$, da za vsak $n \in \ZZ$ velja
\[
y_i^{[n]} \in \langle y_i^{[- N_0]}, y_i^{[-N_0 + 1]}, \dots, y_i^{[N_0]} \rangle.
\]
S tem je $K_i = \langle \{ y_i^{[n]} \mid n \in \ZZ \} \rangle = \langle y_i^{[- N_0]}, y_i^{[-N_0 + 1]}, \dots, y_i^{[N_0]} \rangle$, torej je $K_i$ res končno generirana grupa.
\end{dokaz}

\subsection{Rast Osinove grupe}

Niso vse policiklične grupe podeksponentne rasti.

\begin{zgled}\href{https://citeseerx.ist.psu.edu/viewdoc/download?doi=10.1.1.520.8970&rep=rep1&type=pdf}{(Osin 2003)}

Naj grupa $\ZZ$ deluje na $\ZZ^2$ s predpisom
\[
\varphi \colon \ZZ = \langle s \rangle \to \Aut(\ZZ^2) = \GL_2(\ZZ), \quad
s \mapsto A := \begin{pmatrix}
2 & 1 \\ 1 & 1
\end{pmatrix}.
\]
Tvorimo semidirektni produkt $\mathcal{O} = \ZZ^2 \rtimes_\varphi \ZZ$. Naj bo $S = \{ e_1, e_2, s \} \subseteq \mathcal{O}$, kjer sta $e_1, e_2$ standardna bazna vektorja v $\ZZ^2$. 

Oglejmo si elemente grupe $\mathcal{O}$ oblike
\[
g_{\epsilon_0, \dots, \epsilon_n} = \sum_{j = 0}^n A^j(\epsilon_j e_1) \in \ZZ^2 \leq \mathcal{O}
\]
za $n \geq 1$ in $\epsilon_i \in \{ 0, 1 \}$. 

\underline{Za fiksno vrednost $n$ so vsi ti elementi različni med sabo:} 
Naj bo $\Lambda = \frac{3 + \sqrt{5}}{2}$ največja lastna vrednost matrike $A$ s pripadajočim lastnim vektorjem $w = (\frac{1 + \sqrt{5}}{2}, 1) \in \RR^2$. Definirajmo linearno preslikavo
\[
L \colon \CC^2 \to \CC, \quad
v \mapsto \left\langle w, v \right\rangle.
\]
Za vsak $v \in \CC^2$ velja račun $LAv = \langle w, Av \rangle = \langle A^T w, v \rangle = \langle \Lambda w, v \rangle = \Lambda \cdot Lv$. Če torej na enem od opazovanih elementov uporabimo $L$, sledi
\[
L(g_{\epsilon_0, \dots, \epsilon_n}) = \sum_{j = 0}^n \Lambda^j \cdot L(\epsilon_j e_1) = \frac{1 + \sqrt{5}}{2} \cdot \sum_{j = 0}^n \Lambda^j \epsilon_j.
\]
Ker je $\Lambda > 2$, lahko iz števila $\sum_{j = 0}^n \Lambda^j \epsilon_j$ enolično izračunamo vrednosti $\epsilon_0, \dots, \epsilon_n$.\footnote{Vrednosti izračunamo induktivno, najprej izračunamo $\epsilon_n$, po tem $\epsilon_{n-1}$, \dots Ta postopek je kot pri izračunu števk števila v dvojiškem zapisu.} \kljuka

Torej je število vseh opazovanih elementov enako
\[
|\{ g_{\epsilon_0, \dots, \epsilon_n} \mid \epsilon_i \in \{ 0, 1 \} \}| = 2^{n+1}.
\]
Hkrati lahko vsak tak element zapišemo v multiplikativni obliki
\[
g_{\epsilon_0, \dots, \epsilon_n} = \prod_{j = 0}^n s^j e_1^{\epsilon_j} s^{-j} = e_1^{\epsilon_0} s e_1^{\epsilon_1} s \cdots s e_1^{\epsilon_n} s^{-n},
\]
kjer smo upoštevali, da v grupi $\mathcal{O}$ velja račun $s e_1 s^{-1} = \varphi(s).e_1 = Ae_1$. 

Velja torej $\ell_S(g_{\epsilon_0, \dots, \epsilon_n}) \leq 1 + 2n + n = 3n + 1$. S tem je $\beta_{\mathcal{O},S}(3n+1) \geq 2^{n+1}$ in grupa $\mathcal{O}$ je zato eksponentne rasti.
\end{zgled}

\subsection{Oddih in razgled}

Do zdaj smo dokazali, da so virtualno nilpotentne grupe (na primer $H$) polinomske rasti in da so rešljive grupe podeksponentne rasti nujno policiklične. Podali smo primer rešljive grupe eksponentne rasti ($L$) in policiklične grupe eksponentne rasti ($\mathcal{O}$). 

\begin{figure}[t]
\centering
\begin{tikzpicture}
\tikzstyle{vertex}=[circle, fill = oranzna, minimum size = 5pt, inner sep = 0pt]

\draw[thick, line width = 3pt, draw = rdeca, fill = rdeca!10, opacity = 0.5] (0.55,1.05) circle (2);
\draw[line width = 3pt, draw = vijolicna, fill = vijolicna!10, opacity = 0.5] (0,0) circle (1.6);

\node[vertex,label = below : $\color{zelena} \ZZ$] at (-0.20,-0.15) {};
\draw[thick] (0,0) circle (1.2);
\draw[thick] (0,0) circle (0.75);
\draw[thick] (0,0) circle (2);
\node[label = below : $\vprasanje \mathbf{?}$] at (1.05,-1.10) {};
\node[label = below : $\vprasanje \mathbf{?}$] at (0.4,-0.95) {};
\draw[thick] (0,0) circle (3.5);
\node[vertex,label = below : $\color{zelena} F_d$] at (3,0) {};

\draw[->,thick,draw = vijolicna] (-3.5, -2) to [bend left = 30] (-1.35, 0);
\node[label = below : {\color{vijolicna} polinomske rasti}] at (-4.5, -2) {};

\draw[->,thick,draw = rdeca] (-3.5, -4) to [bend right = 30] (-0.55, -0.85);
\node[label = below : {\color{rdeca} virtualno nilpotentne}] at (-4, -4) {};

\draw[->,thick,draw = rdeca] (-3.5, 3) to [bend right = 30] (-0.35, 0.35);
\node[label = above : {\color{rdeca} nilpotentne}] at (-3.5, 3) {};
\node[vertex,label = below : $\color{zelena} H$] at (0.20, 0.15) {};

\draw[->,thick,draw = vijolicna] (-0.5,-4) to [bend left = 10] (0, -1.75);
\node[label = below : {\color{vijolicna} srednje rasti}] at (-0.5,-4) {};

\draw[->,thick,draw = vijolicna] (3, -2.5) to [bend left = 10] (2, -1.5);
\node[label = right : {\color{vijolicna} eksponentne rasti}] at (3, -2.5) {};

\draw[->, thick, draw = rdeca] (3.5, 2) to [bend right = 10] (2, 2);
\node[label = right : {\color{rdeca} rešljive}] at (3.5, 2) {};

\draw[thick] (135:2) .. controls (0,3) and (3, 3) .. (-10:2); 
\draw[->, thick, draw = rdeca] (3.75, 0.5) to [bend right = 10] (1.75, 1.5);
\node[label = right : {\color{rdeca} policiklične}] at (3.75, 0.5) {};

\node[vertex,label = right : $\color{zelena} L$] at (0,2.75) {};
\node[vertex,label = right : $\color{zelena} \mathcal{O}$] at (1, 2) {};
\end{tikzpicture}
\caption{Diagram grup različnih rasti in njihove algebraične strukture}
\end{figure}

Pri tem se porodita naslednji naravni vprašanji.

\begin{enumerate}
    \item {\vprasanje {\sc (poli. in ne v. nilp.)} Primer grupe polinomske rasti, ki ni virtualno nilpotentna?}
    \item {\vprasanje {\sc (srednja)} Primer grupe srednje rasti?}
\end{enumerate}

\subsection{Policiklične grupe podeksponentne rasti}

Dokazali bomo\footnote{Dokaz je nekoliko zahteven.}, da v rešljivih grupah ne moremo najti primerov, ki bi odgovorili na zastavljeni vprašanji.



\begin{izrek}
Policiklična grupa podeksponentne rasti je virtualno nilpotentna, torej v posebnem polinomske rasti.
\end{izrek}

\begin{dokaz}
Naj bo $G$ policiklična grupa podeksponentne rasti. Po potrebi $G$ zamenjamo s podgrupo končnega indeksa, tako da bo $G$ imela edinko $H \unlhd G$ z $G/H = \langle x H \rangle \cong \ZZ$ za nek $x \in G$ in torej $h(H) < h(G)$. Po indukciji na Hirschovo dolžino lahko predpostavimo, da je $H$ virtualno nilpotentna. Torej obstaja nilpotentna podgrupa $N \leq H$ končnega indeksa v $H$. Po potrebi $N$ zamenjamo s presekom vseh podgrup indeksa $|H:N|$ v $H$ in zato lahko predpostavimo, da vsak avtomorfizem $H$ ohranja $N$, torej v posebnem velja $N \unlhd G$.

\begin{wrapfigure}{r}{0.4\textwidth}
\centering
\begin{tikzpicture}
\tikzstyle{vertex}=[circle, fill = oranzna, minimum size = 5pt, inner sep = 0pt]

\draw[thick] (0, -2) to [bend right = 15]  node [midway, below right] {$\ZZ$}  (2, -1) to [bend right = 15] node [midway, above right] {$< \infty$} (0,0);
\node[vertex, label = right : $N \langle x \rangle$] at (2,-1) {}; 

\draw[thick] (0,0) to node [midway,right] {$\ZZ$}(0, -1);
\node[vertex, label = left : $G$] at (0,0) {};
\draw[thick] (0,-1) to node [midway,right] {$< \infty$} (0, -2);
\node[vertex, label = left : $H$] at (0, -1) {};
\draw[thick] (0,-2) to node [midway, right] {nilpotentna} (0, -4);
\node[vertex, label = left : $N$] at (0, -2) {};
\node[vertex, label = left : $1$] at (0, -4) {};
\end{tikzpicture}
\end{wrapfigure}

Opazujmo podgrupo $N \langle x \rangle \leq G$. Velja $G/N = H/N \cdot \langle x \rangle N/N$, zato je $|G:N\langle x \rangle| \leq |H:N| < \infty$. Hkrati je $N\langle x \rangle/N \cong \langle x \rangle/(\langle x \rangle \cap N) = \langle x \rangle \cong \ZZ$. Brez škode lahko zato grupo $G$ nadomestimo z grupo $N \langle x \rangle$ in torej predpostavimo, da je $G$ razširitev grupe $\langle x \rangle \cong \ZZ$ z nilpotentno grupo $N$. Radi bi našli nilpotentno podgrupo končnega indeksa v $G$. V ta namen dokažimo naslednjo pomožno trditev.

\begin{trditev}
Naj bo $G$ razširitev grupe $\langle x \rangle \cong \ZZ$ z nilpotentno grupo $N$.
Tedaj obstajata centralna vrsta $N = N_1 \geq N_2 \geq \dots \geq N_\ell = 1$ edink v $G$ in število $k \in \ZZ$, tako da je inducirano delovanje elementa $x^k$ na vsakem kvocientu $N_i/N_{i+1}$ trivialno. 
\end{trditev}

V trditvi element $x^k$ deluje s konjugiranjem na $N_i$, torej deluje tudi na kvocientu $N_i/N_{i+1}$. Trivialnost tega delovanja pomeni, da za vsak $n \in N_i$ velja $x^k n x^{-k} N_{i+1} = n N_{i+1}$, kar je enakovredno $[x^k, N_i] \subseteq N_{i+1}$.

\underline{Pomožna trditev implicira izrek:} Res, naj bo $K = N\langle x^k \rangle$. Tedaj je vrsta $K \unrhd N_1 \unrhd N_2 \unrhd \cdots \unrhd N_\ell = 1$ centralna, saj iz komutatorskih identitet sledi $[K, N_i] \subseteq [N, N_i][x^k, N_i] \subseteq N_{i+1} \cdot N_{i+1} \subseteq N_{i+1}$. Tako je grupa $K$ nilpotentna. Hkrati je $|G:K| = |\langle x \rangle : \langle x^k \rangle| = k$. \kljuka

\underline{Dokaz pomožne trditve:} Grupa $N$ je nilpotentna, zato ima centralno vrsto $N = N_1 \geq N_2 \geq \cdots \geq 1$, za katero velja celo $N_i \unlhd G$,\footnote{Torej vsak člen verige ni le edinka v $N$, ampak celo v $G$.} vzamemo lahko na primer kar $N_i = \gamma_i(N)$. Med vsemi vrstami izberimo tako, ki ima maksimalno število neskončnih kvocientov.\footnote{To število je gotovo kvečjemu $h(N)$.} Dobljeno vrsto lahko po potrebi še pofinimo in predpostavimo, da so vsi kvocienti bodisi končni bodisi proste abelove grupe. Dobljeno vrsto označimo z $N = N_1 \geq N_2 \geq \cdots \geq N_\ell = 1$. Dokazali bomo, da ima slednja vrsta želeno lastnost iz pomožne trditve. Pred tem zabeležimo še eno lastnost te vrste. Po maksimalnosti izbire vrste je za vsako edinko $M \unlhd G$, za katero je $N_{i+1} \leq M \leq N_i$, vsaj ena od grup $M/N_{i+1}$, $N_i/M$ končna. 

Za dokaz pomožne trditve bo za vsak $1 \leq i \leq \ell$ dovolj najti $k_i \in \ZZ$, tako da element $x^{k_i}$ deluje trivialno na $N_i/N_{i+1}$. Potem lahko namreč vzamemo $k = \prod_{i = 1}^\ell k_i$. Pribijmo zato nek $i$ in opazujmo $N_i/N_{i+1}$. Če je slednja grupa končna, potem lahko vzamemo $k_i = |\Aut(N_i/N_{i+1})| < \infty$. Predpostavimo zdaj, da je $N_i/N_{i+1}$ prosta abelova grupa ranga $r$ za nek $r \in \NN$. To grupo obravnavajmo kot modul $M$ nad polinomskim kolobarjem $\ZZ[X]$, kjer naj $X$ deluje kot $x$. Razširimo $M$ do vektorskega prostora $V = \QQ \otimes_\ZZ M$. 

\begin{trditev}
$V$ je enostaven $\QQ[X]$-modul.
\end{trditev}


\begin{dokaz}
Naj bo $0 \neq W \leq V$ podmodul. Za $0 \neq w \in W$ obstaja $n \in \NN$, da je $0 \neq nw \in M$.\footnote{Vsak vektor $v \in V$ je $\QQ$-linearna kombinacija elementov $M$, torej obstaja $n \in \NN$, da je $nv \in M$. Tako je grupa $V/M$ torzijska abelova grupa. Na primer, če je $M = \ZZ$, potem je $V = \QQ$ in $V/M = \QQ/\ZZ$.} Torej je $M \cap W \neq 0$. Zato je $M \cap W$ neskončna abelova grupa. Po zgornjem komentarju o maksimalnosti izbire vrste velja, da je $M/(M \cap W)$ končna grupa. Tako je tudi $V/(M \cap W)$ torzijska grupa. S tem je tudi $V/W$ torzijska abelova grupa, ki pa je hkrati vektorski prostor nad $\QQ$. To je mogoče le, če je $W = V$. \kljuka
\end{dokaz}

\begin{trditev}
{\sc (Schurova lema)} $\End_{\QQ[X]}(V)$ je obseg.  
\end{trditev}


\begin{dokaz}
Naj bo $0 \neq \alpha \in \End_{\QQ[X]}(V)$. Tedaj je $\ker \alpha \neq V$ in $\image \alpha \neq 0$. Ker je $V$ enostaven, od tod sledi $\ker \alpha = 0$ in $\image \alpha = V$. Tako je $\alpha$ bijekcija. \kljuka
\end{dokaz}


Endomorfizme $\End_{\QQ[X]}(V)$ obravnavamo kot podmnožico $\End_\QQ(V)$, kar lahko z izbiro baze prostora $V$ identificiramo s kolobarjem matrik $\Mat_r(\QQ)$.
Naj bo $\alpha \in \End_{\QQ[X]}(V)$, ki ustreza delovanju $x$ na $M = N_i/N_{i+1}$. Torej lahko $\alpha$ vidimo kot matriko v $\Mat_r(\ZZ)$. Ta matrika je ničla svojega karakterističnega polinoma $p \in \ZZ[\lambda]$.\footnote{Algebraičnemu številu $\tau \in \CC$, katerega karakteristični polinom ima celoštevilske koeficiente, rečemo {\definicija algebraično celo število}.} Naj bo $F$ podobseg obsega $\End_{\QQ[X]}(V)$, generiran z elementom $\alpha$. Torej je $F/\QQ$ razširitev polj z elementom $\alpha$, ki je ničla polinoma $p$ s celimi koeficienti. Zato lahko $F$ vložimo v $\CC$, in sicer tako, da izberemo eno ničlo $\alpha_0 \in \CC$ polinoma $p$ in preslikamo $\alpha \mapsto \alpha_0$.\footnote{Ni vsaka izbira $\alpha_0$ dobra, če polinom $p$ ni nerazcepen. V tem primeru moramo obravnavati minimalni polinom elementa $\alpha$. Koeficienti slednjega so po Gaussovi lemi tudi cela števila.}

\begin{trditev}
Naj bo $\tau \in \CC$ algebraično celo število. Predpostavimo, da za vsak $\sigma \in \Gal(\QQ(\tau)/\QQ)$ velja $|\sigma(\tau)| \leq 1$. Tedaj obstaja $k \in \NN$, da je $\tau^k = 1$.
\end{trditev}

\begin{dokaz}
Naj bo $\rho$ poljubna potenca elementa $\tau$. Torej imajo tudi vsi konjugati $\rho$ absolutno vrednost kvečjemu $1$. Naj bo $n = |\QQ(\tau):\QQ|$. Torej je $|\QQ(\rho) : \QQ| \leq n$, zato je minimalni polinom elementa $\rho$ stopnje največ $n$, njegovi koeficienti pa so simetrične funkcije konjugatov $\rho$. Tako je $k$-ti koeficient minimalnega polinoma po absolutni vrednosti enak kvečjemu $\binom{n}{k}$. Ker je $\rho$ algebraično celo število, so vsi koeficienti minimalnega polinoma celi. Zato obstaja le končno mnogo možnosti za minimalni polinom, torej tudi le končno mnogo možnosti za $\rho$. Torej obstaja le končno mnogo različnih potenc elementa $\tau$, zato obstaja $k \in \NN$, da je $\tau^k = 1$. \kljuka
\end{dokaz}


Premislimo, da število $\alpha_0 \in \CC$ zadošča zgornjim pogojem. Od tod bo sledilo, da obstaja $k \in \NN$, da je $\alpha_0^k = 1$, zato je $\alpha^k = \id_V$ in torej $x^k$ deluje trivialno na $N_i/N_{i+1}$, kar smo želeli preveriti. V ta namen predpostavimo, da $\alpha_0$ ne zadošča pogojem leme. Torej je nujno $|\alpha_0| > 1$. Po potrebi nadomestimo $x$ in $\alpha_0$ z njuno potenco, da bo veljalo celo $|\alpha_0| > 2$. Izberimo $0 \neq v \in M$, ki mu ustreza $y \in N_i/N_{i+1}$. Označimo $y^{[n]} = x^n.y$ za $n \in \NN_0$. Iz predpostavke o podeksponentni rasti zdaj kot v dokazu trditve {\sc (podeksponentna $\Rightarrow$ izvedena k.g.)} sledi, da obstaja $n_0 \in \NN$, za katerega je
\[
y^{[n_0]} = \sum_{i = 0}^{n_0 - 1} c_i y^{[i]}
\]
za neke $c_i \in \ZZ$, $|c_i| \leq 1$. To enakost lahko zapišemo kot
\[
\alpha^{n_0}(v) = \left( \sum_{i = 0}^{n_0 - 1} c_i \alpha^i \right)(v).
\]
S tem ima endomorfizem $\alpha^{n_0} - \sum_{i = 0}^{n_0 - 1} c_i \alpha^i \in \End_{\QQ[X]}(V)$ netrivialno jedro. {\sc (Schurova lema)} zato implicira, da je ta element trivialen. Od tod izpeljemo protislovje
\[
|\alpha_0^{n_0}| = \left| \sum_{i = 0}^{n_0 - 1} c_i \alpha_0^i \right|
\leq \sum_{i = 0}^{n_0 - 1} |\alpha_0|^i = \frac{|\alpha_0|^{n_0} - 1}{|\alpha_0| - 1} < |\alpha_0|^{n_0},
\]
kjer smo v zadnji neenakosti upoštevali $|\alpha_0| > 2$. Dokaz pomožne trditve je s tem zaključen.
\end{dokaz}

\section{Rast linearnih grup}

V tem razdelku bomo večino dokazov izpustili in le našteli nekaj težkih orodij, s pomočjo katerih bomo nazadnje podali odgovor na vprašanje {\vprasanje \sc (poli. in ne v. nilp)}.

\subsection{Upodobitve grup}

Naj bo $G$ grupa. {\definicija Upodobitev} grupe $G$ je homomorfizem
\[
\rho \colon G \to \GL_n(F)
\]
za neko naravno število $n$ in polje $F$. Običajno vzamemo kar $F = \CC$ in v tem primeru obravnavamo {\definicija kompleksne upodobitve}. Upodobitvam nad polji neničelne karakteristike rečemo {\definicija modularne upodobitve}.

\begin{zgled}
\begin{itemize}
    \item Diedrsko grupo $D_{2n} = \langle s, t \rangle$ lahko upodobimo pred njenega delovanja na ravnini -- element $s$ predstavlja rotacijo za kot $2 \pi/n$, element $t$ pa zrcaljenje:
    \[
    \rho \colon D_{2n} \to \GL_2(\CC), \quad
    s \mapsto \begin{pmatrix} \cos \frac{2 \pi}{n} & - \sin \frac{2 \pi}{n} \\ \sin \frac{2 \pi}{n} & \cos \frac{2 \pi}{n}  \end{pmatrix}, \quad
    t \mapsto  \begin{pmatrix} 1 & 0 \\ 0 & -1  \end{pmatrix}.
    \]

    \item Svetilkarjevo grupo $L = \langle a, t \rangle$ smo upodobili v matrikah nad poljem racionalnih funkcij nad $\ZZ/2\ZZ$:
    \[
    \rho \colon L \to \GL_2(\ZZ/2\ZZ(x)), \quad
    a \mapsto \begin{pmatrix} 1 & 1 \\ 1 & 0  \end{pmatrix}, \quad
    t \mapsto \begin{pmatrix} x & 0 \\ 0 & 1  \end{pmatrix}.
    \]

    \item Ciklično grupo $\ZZ/20\ZZ$ lahko upodobimo nad končnim poljem $\ZZ/5\ZZ$:
    \[
    \rho \colon \ZZ/20\ZZ \to \GL_1(\ZZ/5\ZZ) = (\ZZ/5\ZZ)^*, \quad
    1 \mapsto 2.
    \]
    Ker je $2^{20} \equiv 1 \pmod {5}$, je s tem res definiran homomorfizem.

    \item Obstajajo grupe, ki nimajo nobene netrivialne upodobitve. Konkreten primer take grupe je {\definicija Tarskijeva pošast}. Naj bo $p$ fiksno praštevilo, večje od $10^{75}$. Tarskijeva pošast je neskončna grupa, v kateri je vsaka prava netrivialna podgrupa izomorfna ciklični grupi $\ZZ/p\ZZ$. Take grupe obstajajo \href{http://www.mathnet.ru/php/archive.phtml?wshow=paper&jrnid=im&paperid=1668&option_lang=eng}{(Olshanskii 1979)}. Ni težko preveriti \vaje, da je Tarskijeva pošast generirana z dvema elementoma in da je enostavna. Vsaka njena upodobitev $\rho \colon T \to \GL_n(F)$ bi zato morala biti bodisi injektivna bodisi trivialna. V primeru $n = 1$ bi tako z netrivialno upodobitvijo dobili vložitev $T$ v $F^*$, kar ni mogoče, ker $T$ ni abelova. {\vprasanje Zakaj $T$ ni mogoče vložiti v $\GL_n(F)$?} V naslednjem razdelku bomo predstavili posebno lastnost podgrup $\GL_n(F)$, ki ji $T$ ne zadošča.
\end{itemize}
\end{zgled}

\subsection{Linearne grupe}

Grupi $G$, ki ima kakšno injektivno upodobitev, rečemo {\definicija linearna grupa}.

\begin{zgled}
\begin{itemize}
    \item $\SL_2(\ZZ)$ se naravno vloži v $\GL_2(\CC)$.
    \item Heisenbergova grupa $H$ se naravno vloži v $\GL_3(\CC)$.
    \item Svetilkarjeva grupa $L$ se vloži v $\GL_2(\ZZ/2\ZZ(x))$.
    \item Če ima enostavna grupa kakšno netrivialno upodobitev, potem je linearna.
\end{itemize}
\end{zgled}

Linearne grupe imajo naslednjo strukturno lastnost.

\begin{izrek}\href{https://people.math.carleton.ca/~jdixon/Titsalt.pdf}{\sc (Titsova alternativa)}
Naj bo $G$ končno generirana linearna grupa. Potem bodisi $G$ vsebuje prosto podgrupo (ranga vsaj $2$) bodisi je $G$ virtualno rešljiva.
\end{izrek}

V dokazu se prostost podgrupe dokaže s pomočjo izreka {\sc (pingpong)}, a je {\em pingpong miza}, ki jo za to potrebujemo, zgrajena iz matrik nad normiranimi napolnitvami polj. Za dokaz bi potrebovali kar veliko ozadja iz te teorije, zato ga izpustimo.

\begin{zgled}
\begin{itemize}
    \item Grupa $\SL_2(\ZZ)$ vsebuje prosto grupo $F_2$.
    \item Grupi $H$ in $L$ sta rešljivi.
    \item Naj bo $T$ Tarskijeva pošast. Če bi bila $T$ linearna, potem bi morala bodisi vsebovati prosto podgrupo (nemogoče, saj ima $T$ torzijske elemente, hkrati pa je vsaka prava podgrupa $T$ ciklična) bodisi bi $T$ morala biti virtualno rešljiva (nemogoče, saj je $T$ enostavna, hkrati pa je indeks vsake prave podgrupe $T$ neskončen). Torej $T$ {\em ni} linearna.
\end{itemize}
\end{zgled}

Iz že dokazane lastnosti o rastnih funkcijah rešljivih grup zdaj sledi enaka lastnost za linearne grupe.

\begin{posledica}
Rastna funkcija končno generirane linearne grupe je bodisi eksponentna bodisi polinomska. Slednji primer je mogoč le, ko je grupa virtualno nilpotentna.
\end{posledica}

Torej tudi v svetu linearnih grup ne moremo najti primerov, ki bi odgovorili na vprašanj {\vprasanje \sc (poli. in ne v. nilp)} in {\vprasanje \sc (srednja)}.

\subsection{Končne linearne grupe}

Vsaka končna grupa $G$ deluje na sebi s Cayleyjevim delovanjem in ima zato upodobitev
\[
G \to \GL_{|G| \times |G|}(\CC), \quad
g \mapsto (e_x \mapsto e_{gx}),
\]
kjer smo z $\{ e_g \mid g \in G \}$ označili bazo prostora $\CC^{|G|}$. Dimenzija upodobitve je v tem primeru seveda odvisna od velikosti grupe.

Kadar imamo dano dimenzijo vektorskega prostora, na njem lahko delujejo je končne grupe, ki so virtualno (v odvisnosti od dimenzije) abelove.

\begin{izrek}\href{https://www.imo.universite-paris-saclay.fr/~breuilla/Jordan.pdf}{(Jordan 1878)}
Naj bo $F$ polje karakteristike $0$. Če je $G$ končna podgrupa $\GL_n(F)$, potem obstaja abelova edinka $A \unlhd G$, da je indeks $|G:A|$ omejen s funkcijo, odvisno le od $n$ in ne od grupe $G$.
\end{izrek}

\subsection{Liejeve grupe}

Naj bo $G \leq \GL_n(\RR)$ ali $G \leq \GL_n(\CC)$. Če je $G$ celo zaprta v ambientalni matrični grupi, potem $G$ podeduje topologijo in jo lahko vidimo kot podmnogoterost v $\RR^{n^2}$ oziroma $\CC^{n^2}$. Množenje in invertiranje v $G$ sta analitični operaciji glede na to topološko strukturo.

Te koncepte lahko posplošimo na naslednji način. Rečemo, da je množica $G$ {\definicija topološka grupa}, če je hkrati grupa in topološki prostor, operaciji množenja in invertiranja pa sta {\em zvezni} preslikavi. {\definicija Liejeva grupa} je topološka grupa, ki je kot topološki prostor {\em analitična mnogoterost},\footnote{To pomeni, da je topološki prostor pokrit z odprtimi množicami $\{ U_i \}_i$, ki so homeomomorfne odprtim podmnožicam v $\RR^n$ oziroma $\CC^n$, in to na {\em analitično usklajen} način: če je $U_i \cap U_j \neq \emptyset$, potem naj bo $\varphi_k \colon U_k \to V_k \subseteq \RR^n$ homeomorfizem za $k \in \{ i,j \}$; zahteva o analitični usklajenosti pomeni, da je prehodna preslikava $\varphi_{ij} \colon \varphi_i(U_i \cap U_j) \to \varphi_j(U_i \cap U_j)$, $\varphi_{ij} = \varphi_j \circ \varphi_i^{-1}$, analitična.} operaciji množenja in invertiranja pa sta {\em analitični}.

\begin{zgled}
\begin{itemize}
    \item $\GL_n(\RR)$ je realna Liejeva grupa, $\GL_n(\CC)$ pa je kompleksna Liejeva grupa.
    \item $S^1 \subseteq \RR^2$ je realna Liejeva grupa.
    \item {\definicija Zvezna Heisenbergova grupa} je grupa
    \[
    H_{\RR} = \left\{ \begin{pmatrix} 1 & b & c \\ 0 & 1 & a \\ 0 & 0 & 1 \end{pmatrix} \mid a,b,c \in \RR \right\}.
    \]
    Ta grupa je realna Liejeva grupa.
    \item Grupa $(\QQ, +)$, opremljena s topologijo na $\RR$, je topološka grupa, ki pa {\em ni} Liejeva grupa.
\end{itemize}
\end{zgled}

Liejeve grupe sicer niso vedno linearne,\footnote{Z nekoliko dodatnega znanja o Liejevih grupah in njihovih prirejenih Liejevih algebrah je mogoče premisliti, da je grupa $H_\RR / \{ I + n E_{3,3} \mid n \in \ZZ \}$ Liejeva grupa, ki {\em ni} linearna.} so pa zelo blizu temu, kot prikaže naslednja trditev.\footnote{V trditvi lahko za $Z$ vzamemo povezano komponento centra $G$, upodobitev grupe $G/Z$ pa izhaja iz delovanja grupe $G$ na svoji Liejevi algebri. Druga lastnost trditve je sorodna znani lastnosti {\em brez malih podgrup}, angleško {\em no small subgroups}.}

\begin{trditev}
Bodi $G$ Liejeva grupa s končno mnogo komponentami.
\vspace{-0.5\baselineskip}
\begin{enumerate}[noitemsep]
    \item Obstaja abelova edinka $Z \unlhd G$, tako da je $G/Z$ kompleksna linearna grupa.
    \item {\vprasanje (brez malih redov)} Za vsak $n \in \NN$ obstaja $U \subseteq_o G$, $1 \in U$, tako da $U$ {\em ne} vsebuje nobenega netrivialnega elementa reda kvečjemu $n$. \vspace{-\baselineskip}
\end{enumerate}
\end{trditev}

\begin{zgled}
Naj bo $G = \GL_n(\RR)$. Poglejmo si idejo dokaza, da ta grupa zadošča drugi lastnosti trditve. Iz ideje ni težko sestaviti natančnega argumenta \vaje.

Naj bo $U$ odprta množica tistih matrik v $\GL_n(\RR)$, ki imajo vse lastne vrednosti {\em zelo blizu} $1$.

Naj bo $1 \neq g \in U$. Če ima $g$ kakšno lastno vrednost, ki je različna od $1$, potem nobena majhna potenca $g$ ni enaka $1$. \kljuka Če pa so vse lastne vrednosti $g$ enake $1$, potem je $g$ podobna netrivialni zgornje trikotni matriki z $1$ po diagonali, zato je $\lim_{m \to \infty} ||1 - g^m|| = \infty$. \kljuka
\end{zgled}  

Liejeve grupe se naravno pojavijo vselej, kjer so dovolj lepi topološki prostori. To nam zagotovi naslednji globok izrek.\footnote{Ta izrek hkrati predstavlja razrešitev znamenitega {\em Hilbertovega petega problema}.}

\begin{izrek}\href{http://www.personal.psu.edu/bvk102/Papers/MZ.pdf}{(Montgomery--Zippin 1955)}
Naj bo $T$ {\em končno dimensionalen}, lokalno kompakten, povezan, lokalno povezan, {\em homogen} metrični prostor. Tedaj lahko grupo izometrij $\Isom(T)$ dodatno opremimo s strukturo Liejeve grupe s končno mnogo komponentami.
\end{izrek}

Spomnimo se na pomen homogenosti in dimenzije.
\begin{itemize}
    \item Rečemo, da je $T$ {\em homogen}, če za vsaka $x,y \in T$ obstaja $\varphi \in \Isom(T)$ z lastnostjo $\varphi(x) = y$.\footnote{Na primer, $\RR^2$ in $\GL_n(\RR)$ sta homogena prostora.} 

    \item Rečemo, da ima $T$ {\em dimenzijo $0$}, če za vsak $x \in T$ in vsako okolico $x \in V \subseteq_o T$ obstaja $U \subseteq_o T$, $\overline{U} \subseteq V$, da je $x \in U$, $\partial U = \emptyset$.\footnote{Na primer, $\RR \backslash \QQ$ s topologijo z $\RR$ ima dimenzijo $0$.} 

    \item Induktivno rečemo, da ima $T$ {\em dimenzijo kvečjemu $n$}, če za vsak $x \in T$ in vsako okolico $x \in V \subseteq_o T$obstaja $U \subseteq_o T$, $\overline{U} \subseteq V$, da je $x \in U$, $\partial U$ pa ima dimenzijo kvečjemu $n-1$.\footnote{Na primer, $\RR^n$ ima dimenzijo kvečjemu $n$.}

    \item Če $T$ nima dimenzije kvečjemu $n$ za vsak $n$, potem rečemo, da ima $T$ {\em neskončno dimenzijo}. V nasprotnem primeru je {\em dimenzija} topološkega prostora $T$ tako število $n$, za katerega ima $T$ dimenzijo kvečjemu $n$, nima pa dimenzije kvečjemu $n-1$.  \footnote{Induktivno sledi, da je dimenzija $\RR^n$ enaka $n$.}
\end{itemize}

\begin{zgled}
Naj bo $T$ običajna sfera v prostoru, $T = S^2 \subseteq \RR^3$. Grupa izometrij $\Isom(T)$ sestoji iz rotacij sfere in zrcaljenj. Imenujemo jo {\definicija ortogonalna grupa} in označimo z $O(3)$. Velja
\[
O(3) = \{ Q \in \GL_3(\RR) \mid Q^T Q = Q Q^T = I \} \subseteq_z \GL_3(\RR).
\]
Grupa $O(3)$ je Liejeva grupa z dvema komponentama, ki ustrezata $\det Q = 1$ oziroma $\det Q = -1$. Komponenti z $\det Q = 1$ rečemo {\definicija specialna ortogonalna grupa}, oznaka $SO(3)$. Elementi grupe $SO(3)$ so rotacije prostora, torej so določeni z osjo rotacije (torej vektorjem v $S^2$) in s kotom rotacije (torej s številom med $- \pi$ in $\pi$). Topološko je $SO(3)$ torej krogla polmera $\pi$, kjer identificiramo antipodne točke. Torej je $SO(3)$ homeomorfna $3$-dimenzionalnemu projektivnemu prostoru $\RR P^3$.
\end{zgled}

\subsection{Izrek Gromova}

Z uporabo izreka {\literatura (Montgomery--Zippin 1955)} je Gromov dokazal naslednji izrek, ki razreši vprašanje {\vprasanje \sc (poli. in ne v. nilp.)}.

\begin{izrek}\href{http://www.numdam.org/article/PMIHES_1981__53__53_0.pdf}{(Gromov 1981)}
Grupe polinomske rasti so virtualno nilpotentne.
\end{izrek}

Izrek bomo dokazali v celoti (modulo nedokazane trditve iz tega razdelka). Ideja Gromova je v tem, da grupi priredi {\em zvezen geometrijski objekt} (imenovan asimptotski stožec),  na katerem grupa deluje. Pod predpostavko polinomske rasti ima ta objekt {\em dobre topološke lastnosti}, zato njegova grupa izometrij tvoji {\em Liejevo grupo}, ki je zelo blizu tega, da bi bila {\em linearna}. V svetu linearnih grup pa zaradi dokazanih rezultatov o rešljivih grupah in Titsovi alternativi že vemo, da {\em polinomska rast implicira virtualno nilpotentnost}.

\section{Asimptotski stožec}

\subsection{Pogled od daleč}

Grupi sicer lahko priredimo Cayleyjev graf, na katerem deluje, a je ta topološki prostor, v katerem so točke vozlišča grafa, {\em diskreten}. Če v ta prostor dodamo še povezave grafa, potem v večini primerov ne dobimo {\em homogenega} metričnega prostora. Ideja Gromova je v tem, da opazujmo Cayleyjev graf grupe {\em od daleč}.

\begin{zgled}
\begin{itemize}
    \item Cayleyjev graf grupe $\ZZ$ sestoji iz vozlišč, indeksiranih z $\ZZ$, ki so med sabo zaporedno povezana. Ko na ta graf pogledamo od daleč, se razdalje med vozlišči skrčijo. V limiti vidimo realno premico $\RR$.

    \item Cayleyjev graf grupe $\ZZ^2$ je mreža v $\RR^2$. Ko skrčimo razdalje med vozlišči, v limiti vidimo ravnino $\RR^2$.
\end{itemize}
\end{zgled}

    \begin{figure}[t]
    \centering
    \begin{tikzpicture}
    \tikzstyle{vertex}=[circle, fill = oranzna, minimum size = 5pt, inner sep = 0pt]

    \foreach \n in {-4,...,3}
        \draw[thick] (\n, 0) -- (\n+1, 0);

    \foreach \n in {-4,...,4}
        \node[vertex] at (\n, 0) {};

    \foreach \n in {-8,...,7}
        \draw[thick] (\n/2, -1) -- (\n/2+1/2, -1);

    \foreach \n in {-8,...,8}
        \node[vertex] at (\n/2, -1) {};

    \foreach \n in {-16,...,15}
        \draw[thick] (\n/4, -2) -- (\n/4+1/4, -2);

    \foreach \n in {-16,...,16}
        \node[vertex] at (\n/4, -2) {};

    \draw[line width = 3pt, draw = oranzna] (-4,-3) -- (4,-3);

    \end{tikzpicture}
    \caption{Ko graf $\Cay(\ZZ, \{ 1 \})$ pogledamo od daleč, vidimo realno premico}
    \end{figure}

Za vsako grupo lahko v Cayleyjev graf uvedemo nove metrike med vozlišči, ki skrajšajo razdalje za faktor $\frac12, \frac13, \dots$ Gromov je uspel konstruirati {\em limitni objekt}, h kateremu ti metrični prostoru {\em konvergirajo}. V tem objektu so razdalje med originalnimi vozlišči grafa enake {\em $0$} in iz diskretnih prostorov dobimo {\em zveznega}. Originalna grupa deluje na tem objektu z izometrijami. Limitni objekt bomo imenovali {\definicija asimptotski stožec}. Za natančno konstrukcijo bomo potrebovali nekaj priprave.

\subsection{Ultrafiltri}

Naj bo $S$ množica. {\definicija Filter} na $S$ je družina podmnožic $\FF \subseteq \PP(S)$, ki zadošča naslednjim pogojem:
\begin{itemize}
    \item $\emptyset \notin \FF$,
    \item $A,B \in \FF \Rightarrow A \cap B \in \FF$, \quad {\vprasanje (zaprt za preseke)}
    \item $A \in F, A \subseteq B \Rightarrow B \in \FF$. \quad {\vprasanje (navzgor zaprt)}
\end{itemize}

\begin{figure}[t]
\centering
\begin{tikzpicture}[scale=1.5]
\draw[thick] (0,0) circle (1);
\node[label = left : $\PP(S)$] at (-1, 0) {};
\draw[thick, draw = vijolicna, fill = vijolicna!10] (50:1) arc (50:130:1) -- (0,-0.5) -- (50:1);
\node[label = right : {\color{vijolicna} $\FF$}] at (1,0) {};
\end{tikzpicture}
\caption{Filter je zaprt za preseke in navzgor zaprt}
\end{figure}

Filtre na $S$ lahko uredimo glede na relacijo inkluzije. Maksimalen filter se imenuje {\definicija ultrafilter}.

\begin{zgled}
Naj bo $s_0 \in S$. Definirajmo $\FF = \{ A \in \PP(S) \mid s_0 \in A \}$. To je ultrafilter na $S$. Če je namreč $\emptyset \neq B \notin \FF$, potem je $B \cap \{ s_0 \} = \emptyset$, zato noben filter, ki je večji od $\FF$, ne vsebuje $B$. Ultrafiltru, konstruiranem na tak način, rečemo {\definicija glavni ultrafilter}.

\begin{figure}[t]
\centering
\begin{tikzpicture}[scale=1.5]
\tikzstyle{vertex}=[circle, fill = oranzna, minimum size = 5pt, inner sep = 0pt]

\draw[thick] (0,0) circle (1);
\node[label = left : $\PP(S)$] at (-1, 0) {};
\draw[thick, draw = vijolicna, fill = vijolicna!10] (50:1) arc (50:130:1) -- (0,-0.9) -- (50:1);
\node[label = right : {\color{vijolicna} $\FF$}] at (1,0) {};
\node[vertex, label = below : {$\{ s_0 \}$}] at (0,-0.9) {};
\end{tikzpicture}
\caption{Glavni ultrafilter}
\end{figure}
\end{zgled}

Kadar je $S$ končna množica, je vsak ultrafilter na njej glavni.\footnote{Če je namreč $\FF$ neglavni ultrafilter, potem za vsak $s_0 \in S$ obstaja $A_{s_0} \in \FF$, da je $s_0 \notin A_{s_0}$. Zato je $\emptyset = \bigcap_{s_0 \in S} A_{s_0} \in \FF$, kar je nemogoče.} Nasprotno pa v primeru, ko je $S$ neskončna množica, vselej obstaja neglavni ultrafilter, ki ga lahko dobimo na naslednji način. Naj bo
\[
\FF = \{ A \in \PP(S) \mid |S \backslash A| < \infty \}, 
\]
to je {\definicija kokončni filter}. Po Zornovi lemi obstaja maksimalen filter $\UU$, ki vsebuje $\FF$. Filter $\UU$ je ultrafilter, ki pa {\em ni} glavni.\footnote{Za vsak $s_0 \in S$ imamo množico $S \backslash \{ s_0 \} \in \FF \subseteq \UU$, ki ne vsebuje $s_0$.} Ni težko premisliti \vaje, da velja tudi obratno: vsak neglavni ultrafilter vsebuje kokončni ultrafilter in presek vseh neglavnih ultrafiltrov je ravno kokončni ultrafilter. Poznana ni {\em nobena eksplicitna konstrukcija} neglavnega ultrafiltra. Vemo le, da ti objekti obstajajo.


\begin{figure}[t]
\centering
\begin{tikzpicture}
\tikzstyle{vertex}=[circle, fill = oranzna, minimum size = 5pt, inner sep = 0pt]

\foreach \n in {-3,...,3}
    {
    \draw[thick] (\n, 1) -- (0,0);
    \node[vertex] at (\n, 1) {};
    };

\node[vertex, label = below : $\FF$] at (0,0) {};
\node[vertex, label = below : $\UU$] at (3,1) {};
\end{tikzpicture}
\caption{Presek neglavnih ultrafiltrov je kokončni filter}
\end{figure}


Ultrafiltre uporabljamo, kadar želimo razdeliti vse podmnožice množice $S$ na bodisi {\em velike} (tiste v ultrafiltru) bodisi {\em majhne} (tisti izven ultrafiltra), pri čemer uporabljamo bolj natančen pomen majhnega kot le {\em končno mnogo elementov} (ta definicija majhnega ustreza kokončnemu filtru).

\begin{trditev}{\sc (osnovne lastnosti ultrafiltrov)}
\vspace{-0.5\baselineskip}
\begin{enumerate}[noitemsep]
    \item Filter $\FF$ je ultrafilter, če in samo če za vsak $A \in \PP(S)$ velja $A \in \FF$ ali $S \backslash A \in \FF$.
    \item Če je $\FF$ ultrafilter in $A_1 \cup \cdots \cup A_n \in \FF$, potem je $A_i \in \FF$ za nek $i$.
    \item Če je $G \subseteq \PP(S)$ z lastnostjo, da je vsak presek končno mnogo množic v $G$ neprazen, potem je $G$ vsebovan v nekem filtru. \vspace{-\baselineskip}
\end{enumerate}
\end{trditev}

\begin{dokaz}
Dokažimo le 1. točko, dokaz 2. in 3. točke je podobne zahtevnosti \vaje.

\underline{$(\Leftarrow)$:}  Če je $\FF^\prime$ filter z lastnostjo $\FF^\prime \supsetneq \FF$, potem obstaja $A \subseteq S$, da je $A \in \FF^\prime \backslash \FF$. Zato je po predpostavki $S \backslash A \in \FF$, s tem pa velja $\emptyset = (S \backslash A) \cap A \in \FF^\prime$. Protislovje. Torej je $\FF$ res maksimalen filter.

\underline{$(\Rightarrow)$:} Naj za neko množico $A \subseteq S$ velja $A \notin \FF$ in $S \backslash A \notin \FF$. Naj bo $\GG = \{ B \subseteq S \mid \exists F \in \FF \colon \ F \cap A \subseteq B \}$. 

\begin{figure}[h]
\centering
\begin{tikzpicture}[scale=2]
\tikzstyle{vertex}=[circle, fill = oranzna, minimum size = 5pt, inner sep = 0pt]
\draw[thick] (0,0) circle (1);
\node[label = left : $\PP(S)$] at (-1, 0) {};
\draw[thick, draw = rdeca, fill = rdeca!10] (50:1) arc (50:150:1) -- (-0.1,-0.7) -- (50:1);
\draw[thick, draw = vijolicna, fill = vijolicna!10] (50:1) arc (50:130:1) -- (0,-0.5) -- (50:1);
\node[label = right : {\color{vijolicna} $\FF$}] at (1,0.2) {};
\node[label = right : {\color{rdeca} $\GG$}] at (1,-0.2) {};
\draw[very thick] (-0.6, 0.1) -- (-0.28,-0.4) -- (0.0, 0.2);
\draw[very thick] (-0.28, -0.4) -- (-0.28, 0.5);
\node[vertex, label = left : $A$] at (-0.6, 0.1) {};
\node[vertex, label = left : $A \cap F$] at (-0.28, -0.4) {};
\node[vertex, label = above : $B$] at (-0.28, 0.5) {};
\node[vertex, label = above : $F$] at (0.0, 0.2) {};
\end{tikzpicture}
\end{figure}

Jasno je $\GG \supseteq \FF$ in $A \in \GG \backslash \FF$. Dokažimo, da je $\GG$ filter. To bo protislovje z maksimalnostjo filtra $\FF$.

\begin{itemize}
    \item Če je $\emptyset \in \GG$, potem je $F \cap A = \emptyset$ za nek $F \in \FF$. Od tod sledi $F \subseteq S \backslash A$, zato je $S \backslash A \in \FF$. Protislovje. \kljuka

    \item Naj bosta $G_1, G_2 \in \FF$. Potem je $F_1 \cap A \subseteq G_1$ in $F_2 \cap A \subseteq G_2$ za neka $F_1, F_2 \in \FF$. Sledi $F_1 \cap F_2 \cap A \subseteq G_1 \cap G_2$ in $F_1 \cap F_2 \in \FF$, zato je $G_1 \cap G_2 \in \GG$. \kljuka

    \item Naj bo $G \in \GG$ in $H \supseteq G$. Potem je $F \cap A \subseteq G \subseteq H$ za nek $F \in \FF$, zato je $H \in \GG$. \kljuka
\end{itemize}\vspace{-1.5\baselineskip}
\end{dokaz}

% \subsection{Ultrapotence}

% \todo{Tale razdelek lahko najbrž odstranimo.}

% Naj $\FF$ filter na $S$ in naj bo $A$ neka množica. Opazujmo množico $A^S$ vseh funkcij iz $S$ v $A$. Na to množico uvedemo relacijo: za $f,g \in A^S$ naj bo
% \[
% f \sim g \quad \Leftrightarrow \quad \{ x \in S \mid f(x) = g(x) \} \in \FF.
% \]
% Če si predstavljamo, da $\FF$ vsebuje vse {\em velike} podmnožice $S$, torej z relacijo $\sim$ identificiramo funkcije, ki se ujemajo na veliki množici. Kvocientna množica $A^S/\sim$ je {\definicija reducirana potenca} množice $A$. Kadar je $\FF$ celo ultrafilter, reducirani potenci rečemo tudi {\definicija ultrapotenca}.

\subsection{Ultralimite}

Izberimo nek neglavni ultrafilter $\UU$ na množici $\NN$. Naj bo $T$ nek topološki prostor, v katerem opazujmo zaporedje $\{ x_n \}_{n \in \NN}$. Za $x \in T$ in okolico $x \in U \subseteq_o T$ označimo $I(U) = \{ n \in \NN \mid x_n \in U \}$. Točka $x$ je {\definicija $\UU$-limita} zaporedja $\{ x_n \}_{n \in \NN}$, če za vsako odprto okolico $U$ točke $x$ velja $I(U) \in \UU$.\footnote{Za vsako okolico $U$ torej indeksi členov zaporedja, ki so v $U$, tvorijo {\em veliko} podmnožico $\NN$. Pri tem smo bolj natančni kot pri običajni definiciji limite, kjer zahtevamo le, da so vsi razen končno mnogo členov zaporedja v $U$.} V tem primeru pišemo $x = \lim_{\UU} x_n$.

\begin{zgled}
Naj bo $\FF$ kokončni filter. Tedaj je $x = \lim_{n \to \infty} x_n$, če in samo če je $x = \lim_{\FF} x_n$.
\end{zgled}

Spomnimo se, da so neglavni ultrafiltri ravno ultrafiltri nad kokončnim filtrom in njihov presek je kokončni filter. Od tod izpeljemo, da limite vzdolž vseh neglavnih ultrafiltrov hkrati podajajo enako informacijo kot običajne limite. 

\begin{trditev}
Velja $x = \lim_{n \to \infty} x_n$, če in samo če za vsak neglavni ultrafilter $\UU$ na $\NN$ velja $x = \lim_{\UU} x_n$.
\end{trditev}

V tem smislu si lahko predstavljamo, da z ultrafiltrom $\UU$ definiramo pojem $\UU$-limite, ki je {\em bolj specifičen} od pojma običajne limite. V dovolj lepem topološkem prostoru $T$ te specifične limite {\em vedno obstajajo} in so enolične.

\begin{trditev}
\vspace{-0.5\baselineskip}
\begin{enumerate}[noitemsep]
\item Če je $T$ Hausdorffov prostor, potem je $\UU$-limita enolična.
\item Če je $T$ kompakten, potem ima vsako zaporedje $\UU$-limito. \vspace{-\baselineskip}
\end{enumerate}
\end{trditev}

\begin{dokaz}
1. Naj bosta $x \neq y$ različni $\UU$-limiti zaporedja $\{ x_n \}_{n \in \NN}$. Po predpostavki o Hausdorffovosti obstajata $U,V \subseteq_o T$, da velja $U \cap V = \emptyset$, $x \in U$, $y \in V$. Sledi $I(U) \cap I(V) = \emptyset$, zato  množici $I(U), I(V)$ ne moreta obe pripadati $\UU$. Protislovje.

2. Denimo, da zaporedje $\{ x_n \}_{x \in \NN}$ nima $\UU$-limite. Torej ima vsak $y \in T$ odprto okolico $U_y \subseteq_o T$, da je $I(U_y) \notin \UU$. Po predpostavki o kompaktnosti obstajajo $y_1, \dots, y_k \in T$, da je $T = \bigcup_{i = 1}^k U_{y_i}$. Hkrati za vsak $i$ velja $I(U_{y_i})^c = \{ n \in \NN \mid x_n \notin U_{y_i} \} \in \UU$, zato je $\bigcap_{i = 1}^k I(U_{y_i})^c \in \UU$. V posebnem je zadnji presek neprazen, torej obstaja $i_0 \in \NN$, da je $x_{i_0} \notin U_{y_i}$ za vsak $i$. Protislovje, saj množice $U_{y_i}$ pokrijejo ves $T$.
\end{dokaz}

\begin{zgled}
Naj bo $\{ x_n \}_{n \in \NN}$ zaporedje v $[0,1] \subseteq \RR$. Tedaj to zaporedje $\UU$-konvergira za vsak neglavni ultrafilter $\UU$ in ima enolično $\UU$-limito.
\end{zgled}

Za {\em omejena zaporedja realnih števil} ima $\UU$-limita običajne lastnosti limite. Lahko je preveriti linearnost in monotonost:
\begin{itemize}
    \item $\lim_\UU (x_n + y_n) = \lim_\UU x_n + \lim_\UU y_n$,
    \item $\lim_\UU (c_n) = c \lim_\UU x_n$,
    \item če je $x_n \leq y_n$ za vsak $n$, potem je $\lim_\UU x_n \leq \lim_\UU y_n$.
\end{itemize}

\begin{trditev}
Naj bo $\{ x_n \}_{n \in \NN}$ zaporedje v topološkem prostoru $T$. Naj bo $x \in T$. Predpostavimo, da $T$ zadošča prvemu separacijskemu aksiomu $T_1$ in da je $1$-števen. Tedaj je $x = \lim_\UU x_n$ za nek neglavni ultrafilter $\UU$, če in samo če je $x$ (običajna) limita nekega podzaporedja $\{ x_n \}_{n \in \NN}$.
\end{trditev}

Ultralimita torej {\em izbere} neko stekališče zaporedja.

\begin{dokaz}
\underline{$(\Rightarrow)$:} Naj bo $x = \lim_\UU x_n$. Naj bo $\{ U_i \}_{i \in \NN}$ padajoča baza odprtih okolic točke $x$. Velja $I(U_1) \in \UU$, zato je v posebnem $I(U_1) \neq \emptyset$, torej obstaja $n_1 \in \NN$, da je $x_{n_1} \in U_1$. Ker ultrafilter $\UU$ vsebuje kokončni filter, induktivno za vsak $i > 1$ velja $I(U_i) \cap \{ 1,2,\dots,n_{i-1} \}^c \in \UU$, zato obstaja $n_i \in \NN$, da je $n_i > n_{i-1}$ in $x_{n_i} \in U_i$. Torej je $\lim_{i \to \infty} x_{n_i} = x$. \kljuka

\underline{$(\Leftarrow)$:} Naj bo $x$ stekališče zaporedja $\{ x_n \}_{n \in \NN}$. Naj bo $\{ U_i \}_{i \in \NN}$ padajoča baza odprtih okolic točke $x$. Ker je $x$ stekališče, je $I(U_1) \neq \emptyset$, torej obstaja $n_1 \in \NN$, da je $x_{n_1} \in U_1$. Induktivno za vsak $i > 1$ obstaja $n_i \in \NN$, da je $x_{n_i} \in U_i$ in $n_i > n_{i-1}$. Naj bo $G = \{ I(U_i) \backslash \{ 1, 2, \dots, n_{i-1} \} \mid i \in \NN \}$. Uporabimo {\sc (osnovne lastnosti ultrafiltrov)} in sklepamo, da obstaja ultrafilter $\UU \supseteq G$. S tem je $x = \lim_\UU x_n$. Hkrati ultrafilter $\UU$ ni glavni, saj za vsak $k \in \NN$ za dovolj velik $i$ velja $k \notin I(U_i) \backslash \{ 1,2,\dots,n_{i-1} \}$.
\end{dokaz}

\subsection{Asimptotski stožec}

Fiksirajmo neglavni ultrafilter $\UU$ na $\NN$. Opazujmo metrični prostor $T$ z odlikovano točko $e \in T$.\footnote{Nas bo nazadnje zanimal primer, ko je $T = \Cay(G,S)$ in $e = 1 \in G$.} Asimptotski stožec prostora $T$ bomo konstruirali podobno kot konstruiramo $\RR$ iz $\QQ$: realna števila so Cauchyjeva zaporedja racionalnih števil modulo zaporedja z limito $0$. Pri tem bomo iz tehničnih razlogov obravnavali le {\em posebna} zaporedja v $T$.

    \begin{figure}[t]
    \centering
    \begin{tikzpicture}[scale = 0.5]
    \tikzstyle{vertex}=[circle, fill = oranzna, minimum size = 5pt, inner sep = 0pt]

    \foreach \y in {1,...,5}
        {
        \draw[thick, fill = black!10, draw = black] (0,-2*\y) ellipse ({4/\y} and {0.6/\y});
        \node[label = left : {razdalje v $T$ skrčimo za $\frac{1}{\y}$}] at (-5,-2*\y) {};
        \node[vertex, fill = vijolicna] at (-2/\y, -2*\y) {};
        }
    
    \end{tikzpicture}
    \caption{Zaporedje v $T$, ki ga opazujemo od daleč -- razdalje v $T$ na vsakem koraku skrčimo}
    \end{figure}

{\definicija Zaporedje zmerne rasti} je zaporedje $\{ x_n \}_{n \in \NN}$ v $T$, ki zadošča pogoju
\[
\forall n \in \NN \colon \quad d(x_n, e) \leq An
\]
za neko konstanto $A$, odvisno le od zaporedja $\{ x_n \}_{n \in \NN}$.\footnote{Z drugimi besedami, zaporedje zmerne rasti pomeni, da je $\limsup_{n \to \infty} d(x_n, e)/n < \infty$.} {\definicija Razdalja} med zaporedjema zmerne rasti $\alpha = \{ x_n \}_{n \in \NN}$ in $\beta = \{ y_n \}_{n \in \NN}$ je
\[
d(\alpha, \beta) = \lim_{\UU} \left( \frac{d(x_n, y_n)}{n} \right).\footnote{Pri tem je $d(x_n, y_n) \leq d(x_n, e) + d(e, y_n) \leq Cn$ za neko konstanto $C$, zato je zaporedje $\{ d(x_n, y_n)/n \}_{n \in \NN}$ omejeno, torej za njegovo $\UU$-limito veljajo običajne lastnosti limite.}
\]

\begin{zgled}
Naj bo $T = \Cay(\ZZ, \{ 1 \})$.

Izberimo najprej $\alpha = \{ 0 \}_{n \in \NN}$ in $\beta = \{ 1 \}_{n \in \NN}$. Velja $d(0, 1)/n = 1/n$ za vsak $n$, zato je $d(\alpha, \beta) = \lim_\UU 1/n = 0$. V asimptotskem stožcu bomo ti dve zaporedji torej identificirali.

Izberimo zdaj $\alpha = \{ 0 \}_{n \in \NN}$ in $\beta = \{ n \}_{n \in \NN}$. Velja $d(0,n)/n = 1$ za vsak $n$, zato je $d(\alpha, \beta) = 1$. V asimptotskem stožcu bo torej zaporedje $\alpha$ predstavljalo število $0 \in \RR$, zaporedje $\beta$ pa število $1 \in \RR$.
\end{zgled}

\begin{figure}[t]
    \centering
    \begin{tikzpicture}[scale = 0.5]
    \tikzstyle{vertex}=[circle, fill = oranzna, minimum size = 5pt, inner sep = 0pt]

    \foreach \y in {1,...,5}
        {
        \foreach \n in {-5,...,5}
            \draw[thick] (\n, -2*\y) -- (\n+1, -2*\y);
        \foreach \n in {-4,...,5}
            \node[vertex] at (\n, -2*\y) {};
        \node[label = left : {razdalje skrčimo za $\frac{1}{\y}$}] at (-6,-2*\y) {};
        \node[vertex, fill = rdeca, minimum size = 10pt] at (0, -2*\y) {};
        \node[vertex, fill = vijolicna, minimum size = 10pt] at (\y, -2*\y) {};
        }
    
    \end{tikzpicture}
    \caption{Razdalja med zaporedjema $\{ 0 \}_{n \in \NN}$ in $\{ n \}_{n \in \NN}$ v grafu $\Cay(\ZZ, \{ 1 \})$, ki ga opazujemo od daleč, je enaka $1$}
\end{figure}

Recimo, da sta zaporedji $\alpha, \beta$ {\definicija ekvivalentni}, če velja $d(\alpha, \beta) = 0$. Množica vseh ekvivalenčnih razredov zaporedij zmerne rasti se imenuje {\definicija asimptotski stožec} prostora $T$ z odlikovano točko $e$ glede na $\UU$, oznaka $\Cone_\UU(T,e)$. Razdalja $d$ na zaporedjih se prenese na stožec, zato je $\Cone_\UU(T,e)$ {\em metrični prostor}.

\begin{zgled}
\begin{itemize}
    \item Naj bo $T = \RR$. Ta prostor se ne spremeni s krčenjem razdalj, zato pričakujemo, da je $\Cone_\UU(\RR, 0)$ izometričen $\RR$. Pričakovanje potrdita inverzni izometriji
    \[
    f \colon \RR \to \Cone_\UU(\RR, 0), \quad x \mapsto [\{ n x \}_{n \in \NN}]
    \]
    in
    \[
    g \colon \Cone_\UU(\RR, 0) \to \RR, \quad [\{ x_n \}_{n \in \NN}] \to d(\{ x_n \}_{n \in \NN}, \{ 0 \}_{n \in \NN}) = \lim_\UU \left(\frac{x_n}{n}\right).
    \]

    \item Naj bo $T = \ZZ$, opremljen z metriko iz $\Cay(\ZZ, \{ 1 \})$. Trdimo, da je $\Cone_\UU(\ZZ, 0)$ izometričen $\RR$. Definirajmo preslikavi
    \[
    f \colon \RR \to \Cone_\UU(\ZZ, 0), \quad
    x \mapsto [\{ \lfloor n x_n \rfloor \}_{n \in \NN}]
    \]
    in
    \[
    g \colon \Cone_\UU(\ZZ, 0) \to \RR, \quad
    [\{ x_n \}_{n \in \NN}] \mapsto d(\{ x_n \}_{n \in \NN}, \{ 0 \}_{n \in \NN}) = \lim_\UU \left( \frac{x_n}{n} \right).
    \]
    Preprosto je preveriti, da $f$ in $g$ ohranjata razdalji na obeh metričnih prostorih. Preverimo, da sta ti dve preslikavi inverzni. Za zaporedje zmerne rasti $\{ x_n \}_{n \in \NN}$ velja
    \[
    (f \circ g)([\{ x_n \}_{n \in \NN}]) = [ \{ \lfloor \lim_\UU \frac{x_n}{n} \cdot m \rfloor  \}_{m \in \NN} ] = [ \{ \lfloor \lim_\UU ( x_n \frac{m}{n}) \rfloor \}_{m \in \NN} ].
    \]
    Preverimo, da je zadnji ekvivalenčni razred enak $[\{ x_m \}_{m \in \NN}]$. V ta namen izračunajmo razdaljo med njima. Velja
    \begin{align*}
    d \left( \{ \lfloor \lim_\UU ( x_n \frac{m}{n}) \rfloor \}_{m \in \NN}, \{ x_m \}_{m \in \NN} \right) &= \lim_\UU \frac{1}{m} \cdot d \left( \lfloor \lim_\UU ( x_n \frac{m}{n}) \rfloor, x_m \right) \\
    &= \lim_\UU \frac{1}{m} \cdot | \lfloor \lim_\UU ( x_n \frac{m}{n}) \rfloor - x_m | \\ 
    &= |\lim_\UU \frac{x_n}{n} - \lim_\UU \frac{x_m}{m}| = 0,
    \end{align*}
    kjer smo v predzadnji enakosti uporabili $\lim_{m \to \infty} \frac{1}{m} \lfloor c m \rfloor = c$ za vsak $c \in \RR$. Torej je res $f \circ g = \id_{\Cone_\UU(\ZZ, 0)}$. Druga enakost, $g \circ f = \id_\RR$, sledi neposredno iz $\lim_{m \to \infty} \frac{1}{m} \lfloor c m \rfloor = c$. 
\end{itemize}
\end{zgled}

\subsection{Asimptotski stožec grupe}

Naj bo $G$ končno generirana grupa, $G = \langle S \rangle$. Opazujmo Cayleyjev graf $\Cay(G,S)$, opremljen z metriko $d(x,y) = \ell_S(x^{-1} y)$. Za odlikovano točko vzemimo $1 \in G$. Dobimo asimptotski stožec $K = \Cone_\UU(\Cay(G,S), 1)$. Množico vseh zaporedij zmerne rasti označimo z $\ZR$. Množica $\ZR$ je grupa z enoto $1 = \{ 1 \}_{n \in \NN}$. Naj bo $\ZR_0$ množica tistih zaporedij $\alpha \in \ZR$, za katera je $d(\alpha, 1) = 0$. Množica $\ZR_0$ je podgrupa $\ZR$. Asimptotski stožec $K$ lahko zato identificiramo z množico odsekov $\ZR/\ZR_0$.

V primeru konstrukcije iz grup ima asimptotski stožec veliko dobrih topoloških lastnosti.

\begin{trditev}
Asimptotski stožec $K$ grupe $G$ je homogen, lokalno povezan, s potmi povezan in poln metrični prostor.
\end{trditev}

\begin{dokaz}
\underline{Homogenost:} Asimptotski stožec identificiramo z množico odsekov $\ZR/\ZR_0$. Na tej množici deluje grupa $\ZR$ z levim množenjem: $\{ z_n \}_{n \in \NN}. \{ x_n \}_{n \in \NN}\ZR_0 = \{ z_n \cdot x_n \}_{n \in \NN} \ZR_0$. To delovanje je tranzitivno, saj je $\{ y_n x_n^{-1} \}_{n \in \NN}. \{ x_n \}_{n \in \NN}\ZR_0 = \{ y_n \}_{n \in \NN} \ZR_0$. \kljuka

\underline{Povezanost s potmi:} 

\begin{trditev}
Za vsak $\alpha \in K$ obstaja zvezna pot $\gamma \colon [0,1] \to K$ z $\gamma(0) = 1$ in $\gamma(1) = \alpha$.
\end{trditev}

\begin{dokaz}
Naj bo $\alpha = [\{ x_n \}_{n \in \NN}]$. Vsak člen $x_n$ zapišimo kot besedo dolžine $\ell_S(x_n)$, torej $x_n = s_1 s_2 \cdots s_{\ell_S(x_n)}$. Naj bo $0 \leq t \leq 1$ parameter in naj $x_n(t)$ označi predpono izbrane besede za $x_n$ dolžine $\lfloor t \cdot \ell_S(x_n) \rfloor$, se pravi $x_n(t) = s_1 s_2 \cdots s_{\lfloor t \cdot \ell_S(x_n) \rfloor}$. Definirajmo $\gamma(t) = [\{ x_n(t) \}_{n \in \NN}] \in K$. Jasno je $\gamma(0) = 1$ in $\gamma(1) = \alpha$. Za vsaka $t_1 < t_2$ velja
\begin{align*}
d(x_n(t_1), x_n(t_2)) &= \left| \lfloor t_2 \cdot \ell_S(x_n) \rfloor - \lfloor t_1 \cdot \ell_S(x_n) \rfloor \right| \\
&\leq (t_2 - t_1) \cdot \ell_S(x_n) + 1.
\end{align*}
Od tod sledi
\[
d(\gamma(t_1), \gamma(t_2)) = \lim_\UU \frac{d(x_n(t_1), x_n(t_2))}{n} 
\leq d(\alpha, 1) \cdot (t_2 - t_1).
\]
Preslikava $\gamma$ je torej Lipschitzovo zvezna. \kljuka
\end{dokaz}


\underline{Lokalna povezanost:} Naj bo $\alpha = [\{ x_n \}_{n \in \NN}] \in K$ in naj bo $\gamma$ pot kot v pomožni trditvi. Po konstrukciji za vsak $t \in [0,1]$ velja $d(\gamma(t), 1) \leq d(\alpha, 1) \cdot t$, zato je slika $\gamma$ v celoti vsebovana v krogli $B_K(d(\alpha, 1))$\footnote{$B_K(r)$ je krogla polmera $r$ v metričnem prostoru $K$ s središčem $1 = [\{ 1 \}_{n \in \NN}] \in K$.}. Iz pomožne trditve zato sledi, da je notranjost krogle $B_K(d(\alpha, 1))$ povezana. Ker je prostor $K$ homogen, od tod sledi lokalna povezanost prostora $K$. \kljuka

\underline{Polnost:} \vaje
\end{dokaz}

\subsection{Delovanje grupe na asimptotskem stožcu}

Naj bo $G$ grupa in $K = \Cone_\UU(\Cay(G,S), 1)$ njen asimptotski stožec. Grupa $G$ naravno deluje z levo translacijo na zaporedjih v $G$:
\[
g.\{ x_n \}_{x \in \NN} = \{ g x_n \}_{n \in \NN} \quad 
\text{za $g, x_n \in G$}.
\]
Če je zaporedje $\{ x_n \}_{x \in \NN}$ zmerne rasti, je pri tem tudi $g. \{ x_n \}_{x \in \NN}$ zmerne rasti.\footnote{Za vsak $n$ velja $d(g x_n, 1) = \ell_S(gx_n) \leq \ell_S(x_n) + \ell_S(g) \leq Cn$ za neko konstanto $C$.} Ker $G$ deluje z izometrijami na $\Cay(G,S)$, velja tudi 
\[
d(g.\{ x_n \}_{n \in \NN}, g.\{ y_n \}_{n \in \NN}) = d(\{ x_n \}_{n \in \NN}, \{ y_n \}_{n \in \NN}).\]
Tako $G$ deluje z izometrijami na grupi zmernih zaporedij $\ZR$ in zato deluje tudi na kvocientni množici odsekov $\ZR / \ZR_0 \equiv K$. Dobimo torej homomorfizem
\[
\Phi \colon G \to \Isom(K), \quad
g \mapsto ([\{ x_n \}_{x \in \NN}] \mapsto [\{ g x_n \}_{x \in \NN}]).
\]

Grupa izometrij $\Isom(K)$ je {\em topološka grupa}: opremljena je s topologijo, pri kateri baza okolic izometrije $\id_K$ sestoji iz množic
\[
O_{k, \epsilon} = \{ \sigma \in \Isom(K) \mid d(\sigma.\alpha, \alpha) < \epsilon \text{ za vsak $\alpha \in B_K(k)$} \}
\]
za $k \in \NN$ in $\epsilon > 0$. To je {\definicija topologija enakomerne konvergence po kompaktih}.

Pri dokazu izreka Gromova bomo ključno uporabili naslednjo lastnost delovanja $G$ na $K$.\footnote{Dokaz je {\em precej} zahteven.}

\begin{izrek}{\sc (o delovanju $\Phi$)}
Predpostavimo, da je slika $\image \Phi$ končna in da jedro $\ker \Phi$ {\em ni} virtualno abelova grupa. Tedaj za vsako okolico $O$ identitete $\id_K \in \Isom(K)$ obstaja homomorfizem $\varphi \colon \ker \Phi \to \Isom(K)$ z lastnostjo $|\image \varphi \cap O| > 1$.
\end{izrek}

\begin{figure}
\centering
\begin{tikzpicture}
\tikzstyle{vertex}=[circle, fill = oranzna, minimum size = 5pt, inner sep = 0pt]

\draw[thick, fill = rdeca!10, draw = rdeca, opacity = 0.5] (3,0) circle (1);
\draw[thick, fill = vijolicna!10, draw = vijolicna, opacity = 0.5] (2.5, 0.5) circle (1);

\draw[thick] (-2, 0) circle (1);
\node[label = left : $G$] at (-3, 0) {};
\draw[thick, draw = vijolicna, fill = vijolicna!10] (-2, 0) circle (0.6);
\node at (-2, 0) {$\color{vijolicna} \ker \Phi$};
\draw[->, thick] (-0.5,0) to node [midway, above] {$\Phi$} (0.5,0);
\draw[thick] (3,0) circle (2);
\node at (4,-1) {\color{rdeca} $O$};
\node[vertex, label = above : $\id_K$] at (3,0) {};
\node[vertex, fill = vijolicna] at (2.35,0.15) {};
\draw[thick, draw = vijolicna, ->] (-2,0.4) to [bend left = 30, midway, above] node {\color{vijolicna} $\varphi$} (2,1);
\node[label = right : $\Isom(K)$] at (5,0) {};
\end{tikzpicture}
\caption{Za vsako okolico $O$ identitete $\id_K \in \Isom(K)$ obstaja homomorfizem $\varphi \colon \ker \Phi \to \Isom(K)$, katerega slika vsebuje vsaj en netrivialen element v $O$}
\end{figure}

\begin{dokaz}
Za $g \in G$ in $r \in \RR$ označimo največji možni premik\footnote{Angleško {\em displacement}.} elementa $g$ na krogli polmera $r$ kot
\[
D(g,r) = \max \{ d(a,g \cdot a) \mid a \in B_{G,S}(r) \} = \max \{ \ell_S(a^{-1} g a) \mid a \in B_{G,S}(r) \}.
\]

\begin{trditev}
Za vsaka $g \in \ker \Phi$ in $C > 0$ velja $\lim_\UU \frac{D(g,Cr)}{r} = 0$.
\end{trditev}

\begin{dokaz} 
Za vsak $r \in \NN$ izberimo $a_r \in B_{G,S}(Cr)$ z lastnostjo $D(g,Cr) = \ell_S(a_r^{-1} g a_r)$. Zaporedje $\alpha = \{ a_r \}_{r \in \NN}$ je zmerne rasti. Ker je $g \in \ker \Phi$, velja $g.[\alpha] = [\alpha]$, se pravi $d(g.[\alpha],[\alpha]) = 0$, kar je ekvivalentno
\[
\lim_\UU \frac{D(g, Cr)}{r} = \lim_\UU \frac{d(ga_r, a_r)}{r} = 0.
\]
Dokaz je zaključen. \kljuka
\end{dokaz}


\begin{trditev}
Za fiksen $g \in G$ je funkcija $D(g, \cdot) \colon \NN \to \RR$ omejena, če in samo če je $|g^G| < \infty$. V tem primeru velja $g \in \ker \Phi$.
\end{trditev}


\begin{dokaz}
Očitno iz definicije $D$ in zadnjega dokaza. \kljuka
\end{dokaz}


Elementi $g \in G$ z lastnostjo $|g^G| < \infty$ se imenujejo {\definicija FC-elementi}.\footnote{Iz angleškega izraza {\em finitely many conjugates}.} Ti elementi tvorijo podgrupo edinko v $G$.

% Za podgrupo $H \leq G$ označimo $D_H(g,r) = \{ d(a, g \cdot a) \mid a \in B_{G,S}(r) \cap H \}$. Če je $|G:H| < \infty$, potem za $g \in H$ velja $|g^G| < \infty$, če in samo če je $|g^H| < \infty$, kar pomeni, da je funkcija $D(g, \cdot)$ omejena, če in samo če je funkcija $D_H(g, \cdot)$ omejena.

\begin{trditev}
Za vse $g,h \in G$ in $r,s \in \ZZ$ velja:
\vspace{-0.5\baselineskip}
\begin{enumerate}[noitemsep]
    \item $D(g,r+s) \leq D(g,r) + 2s$
    \item $D(h^{-1} g h, r) \leq D(g,r) + 2 \ell_S(h)$ \vspace{-\baselineskip}
\end{enumerate}
\end{trditev}

\begin{dokaz}
1. Naj bo $a \in B_{G,S}(r+s)$. Zapišimo $a = b \cdot c$ za neka $b \in B_{G,S}(r)$ in $c \in B_{G,S}(s)$. Tedaj je 
\begin{align*}
d(ga,a) = d(gbc,bc) 
&\leq d(gbc, gb) + d(gb, b) + d(b, bc) \\
&= \ell_S(c) + d(gb,b) + \ell_S(c) \\
&\leq D(g,r) + 2s,
\end{align*}
kar smo želeli dokazati. \kljuka

2. Naj bo $a \in B_{G,S}(r)$. Velja
\[
d(h^{-1}gha,a) = d(gha, ha) \leq D(g, r + \ell_S(h)) \leq D(g,r) + 2 \ell_S(h),
\]
kjer smo v zadnji neenakosti uporabili prejšnjo točko. \kljuka
\end{dokaz}


\begin{trditev}
Nobena grupa {\em ni} končna unija odsekov pogrup neskončnega indeksa.
\end{trditev}

\begin{dokaz}
Elementarno. \vaje \kljuka
\end{dokaz}


Z vsem naštetim lahko nazadnje dokažemo izrek {\sc (o delovanju $\Phi$)}. Po predpostavki je $|\image \Phi| = |G : \ker \Phi| < \infty$. Torej je jedro $\ker \Phi$ končno generirana grupa, označimo $\ker \Phi = \langle y_1, \dots, y_d \rangle$. Naj bo $O_{C^\prime, \epsilon^\prime} \subseteq \Isom(K)$ poljubna bazna okolica identitete $\id_K$ in naj bo $C = 2 C^\prime$, $\epsilon = \epsilon^\prime/2$. Dokaz bomo izvedli v zaporedju večih korakov.

\begin{itemize}
    \item Predpostavimo najprej, da je vsak generator $y_j$ FC-element. V tem primeru je $|y_j^{\ker \Phi}| < \infty$, zato je centralizator $C_{\ker \Phi}(y_j)$ končnega indeksa v $\ker \Phi$ za vsak $j$. Od tod sledi, da je tudi $Z(\ker \Phi) = \bigcap_{j = 1}^d C_{\ker \Phi}(y_j)$ končnega indeksa v $\ker \Phi$. To pomeni, da je grupa $\ker \Phi$ virtualno abelova. Protislovje. Torej FC-elementi tvorijo pravo podgrupo v $\ker \Phi$, zato lahko po morebitni menjavi generatorjev grupe $\ker \Phi$ predpostavimo, da celo\footnote{Če $y_{j_0}$ ni FC-element in $y_j$ je FC-element, potem lahko $y_j$ zamenjamo z $y_j y_{j_0}$, ki ni FC-element.}
    \[
    \text{noben od $y_j$ ni FC-element.}
    \]

    \item Fiksirajmo $r \in \NN$.  Opazujmo množico
    \begin{align*}
    Y_t &= \{ y \in \ker \Phi \mid D(y^{-1} y_t y, Cr) \leq \epsilon r \} \\
    &= \{ y \in \ker \Phi \mid \forall a \in B_{G,S}(Cr) \colon \ \ell_S(a^{-1} y^{-1} y_t y a) \leq \epsilon r \} \\
    &= \{ y \in \ker \Phi \mid \forall a \in B_{G,S}(Cr) \colon \ y_t^{ya} \in B_{G,S}(\epsilon r) \}.
    \end{align*}
    Množica $\{ y_t^{ya} \mid y \in Y_t, \ a \in B_{G,S}(Cr) \}$ je torej končna, saj je vsebovana v $B_{G,S}(\epsilon r)$. Zato je tudi množica odsekov $C_t = \{ C_{\ker \Phi}(y_t) ya \mid y \in Y_t, \ a \in B_{G,S}(Cr) \}$ končna. Ker $y_t$ ni FC-element, je vsak centralizator $C_{\ker \Phi}(y_t)$ neskončnega indeksa v $\ker \Phi$, zato po zadnji pomožni trditvi $\ker \Phi$ ni unija odsekov iz $C_1 \cup \cdots \cup C_d$. Tako obstaja element $z_r \in \ker \Phi$, ki ne pripada nobenemu odseku iz $C_t$. V posebnem od tod sledi, da za vsak $t$ velja $z_r \notin Y_t$, se pravi 
    \[
    \text{za vsak $t$ velja $D(z_r^{-1} y_t z_r, Cr) > \epsilon r$.}
    \]

    \item Zapišimo element $z_r$ kot besedo v $y_1, \dots, y_d$ in naj bo $x_r$ najkrajša predpona besede, za katero za {\em nek} $t$ velja $D(x_r^{-1} y_t x_r, Cr) > \epsilon r$. Izberimo en tak $t = t(r)$.\footnote{Velja torej $D(x_r^{-1} y_{t(r)} x_r, Cr) > \epsilon r$.} Za vsak $i = 1, \dots, d$ naj bo $S(i) = \{ r \mid t(r) = i \}$. Velja $S(1) \cup \cdots \cup S(d) = \NN$. Torej obstaja $i_0$, da je $S(i_0) \in \UU$. Torej velja
    \[
    \text{$D(x_r^{-1} y_{i_0} x_r, Cr) > \epsilon r$ za vse $r$ iz neke množice v $\UU$.}
    \]

    \item Po prvi pomožni trditvi za vse $r$ iz neke množice v $\UU$ velja $D(y_t, r) < \epsilon r$ za vsak $t$. Od tod sledi, da je $x_r \neq 1$, torej lahko pišemo $x_r = w_r \cdot y$ za nek $y \in \{ y_1^{\pm 1}, \dots, y_d^{\pm 1} \}$ in krajšo besedo $w_r$. Iz tretje pomožne trditve zdaj sledi
    \[
    D(x_r^{-1} y_t x_r, Cr) = D(y^{-1} w_r^{-1} y_t w_r y, Cr) \leq D(w_r^{-1} y_t w_r, Cr) + 2 \ell_S(y).
    \]
    Naj bo $\ell = \max \{ \ell_S(y_i) \mid i = 1, \dots, d \}$. Torej\footnote{Tu upoštevamo minimalnost dolžine elementa $x_r$, od koder sledi $D(w_r^{-1} y_t w_r, Cr) \leq \epsilon r$ za vsak $t$.} 
    \[
    \text{za vse $r$ iz neke množice v $\UU$ je }
    D(x_r^{-1} y_t x_r, Cr) 
    \leq \epsilon r
    + 2 \ell \text{ za vsak $t$.}
    \]

    \item Za vse $r$ iz neke množice v $\UU$ torej velja neenakost
    \[
    \epsilon r < D(x_r^{-1} y_{i_0} y_r, Cr) \leq \epsilon r + 2 \ell, 
    \]
    od koder sledi
    \[
    \lim_\UU \frac{D(x_r^{-1} y_{i_0} x_r, Cr)}{r} = \epsilon. \quad (*)
    \]

    \item Za poljuben element $w \in \ker \Phi$ opazujmo zaporedje $\{ x_r^{-1} w x_r \}_{r \in \NN}$ v grupi $G$. Zapišimo $w$ kot besedo v generatorjih $y_1, \dots, y_d$ dolžine $m$. Velja
    \[
    \ell_S(x_r^{-1} w x_r) \leq D(x_r^{-1} w x_r, Cr) \leq m \cdot \max \{ D(x_r^{-1} y_i x_r, Cr) \mid i = 1, \dots, d \}.
    \]
    Za vse $r$ iz neke množice $U \in \UU$ zato velja
    \[
    \ell_S(x_r^{-1} w x_r) \leq m \epsilon r + 2 m \ell.
    \]
    Definirajmo zaporedje $\{ X_r(w) \}_{r \in \NN}$ na naslednji način: 
    \[
    X_r(w) =
    \begin{cases}
    x_r^{-1} w x_r; & r \in U \\
    1; & r \in \NN \backslash U.
    \end{cases}
    \]
    Zaporedji, ki se na množici v $\UU$ ujemata, imata enako $\UU$-limito, zato velja
    \[
    d(\{ X_r(w) \}_{r \in \NN}, \{ x_r^{-1} w x_r \}_{r \in \NN}) = 0
    \quad
    \text{in}
    \quad
    \{ X_r(w) \}_{r \in \NN} \in \ZR.
    \]
    
    \item Zaporedje zmerne rasti $\{ X_r(w) \}_{r \in \NN}$ lahko z desne pomnožimo z drugimi zmernimi zaporedji in na ta način dobimo izometrijo asimptotskega stožca $\ZR/\ZR_0 = K$. To izometrijo označimo s $\varphi(w)$. Imamo torej homomorfizem grup
    \[
    \varphi \colon \ker \Phi \to \Isom(K).
    \]

    \item Kot v dokazu prve pomožne trditve zdaj iz lastnosti $(*)$ sledi, da zaporedje $\{ X_r(y_{i_0}) \}_{r \in \NN}$ deluje netrivialno z levim množenjem na $K$. Torej je $\varphi(y_{i_0}) \neq \id_K$.

    \item Naj bo $\alpha = \{ a_r \}_{r \in \NN} \in \ZR$ z $[\alpha] \in B_K(C/2)$. Za vse $r$ iz neke množice v $\UU$ velja torej $a_r \in B_{G,S}(Cr)$. Od tod sledi
    \begin{align*}
    d(\varphi(y_{i_0}).[\alpha], [\alpha])
    &= d(\{ X_r(y_{i_0})\}_{r \in \NN}, \{ a_r \}_{r \in \NN}) \\
    &= \lim_\UU \frac{d(X_r(y_{i_0}) a_r, a_r)}{r} \\
    &\leq \lim_\UU \frac{D(X_r(y_{i_0}), Cr)}{r} \\
    &= \lim_\UU \frac{D(x_r^{-1} y_{i_0} x_r, Cr)}{r} \\
    &= \epsilon.
    \end{align*}
    Torej je $\varphi(y_{i_0}) \in O_{C/2, 2\epsilon} = O_{C^\prime, \epsilon^\prime}$. S tem je dokaz zaključen. 
\end{itemize}\vspace{-1.5\baselineskip}
\end{dokaz}

\section{Izrek Gromova}

V tem razdelku bomo končno dokazali izrek Gromova.

\begin{izrek}{\literatura (Gromov 1981)}
Grupe polinomske rasti so virtualno nilpotentne.
\end{izrek}

\subsection{Hausdorffova dimenzija}

Naš prvi cilj je, da dokažemo, da je asimptotski stožec grupe polinomske rasti končno dimenzionalen. V ta namen bomo dokazali nekoliko močnejšo trditev, v kateri bomo {\em merili dimenzijo} s pokrivanjem objekta s kroglami.

\begin{zgled}
Opazujmo {\em dovolj lepo} podmnožico v $\RR^n$. Njen volumen lahko izračunamo tako, da jo pokrijemo s kroglami polmera $\epsilon$, seštejemo volumne krogel in pošljemo $\epsilon$ proti $0$. Volumen krogle polmera $\epsilon$ v $\RR^n$ je $\frac{\pi^{n/2}}{\Gamma(\frac{n}{2} + 1)} \cdot \epsilon^n$, kjer je $\Gamma$ Eulerjeva funkcija. Če odmislimo prvi faktor, potem vsaka krogla prispeva k vsoti člen $\epsilon^n$. Pri tem je $\epsilon$ polmer krogle, $n$ pa dimenzija prostora $\RR^n$.
\end{zgled}

Naj bo $M$ metrični prostor in $X \subseteq M$. Opazujmo vsa {\em števna} pokritja množice $X$ z množicami premera $< \epsilon$ za nek $\epsilon$. Za vsako tako pokritje $\{ U_i \}_i$ izračunamo $\sum_i (\diam U_i)^d$ za nek fiksen $d > 0$. Izlimitiramo $\epsilon \to 0$ in dobimo
\[
\mu_d(X) = \lim_{\epsilon \to 0} \ \inf_{\{ U_i \}_i \text{ pokritje $X$ z $\diam U_i < \epsilon$}} \ \sum_i (\diam U_i)^d,
 \]
to je {\definicija $d$-dimenzionalna Hausdorffova mera množice $X$}. Bolj kot mera sama nas zanima, kakšne so njene vrednosti, ko se $d$ spreminja. Po motivaciji iz evklidskih prostorov pričakujemo, da {\em pravilen volumen} izračunamo ravno z mero $\mu_d$, kjer je $d$ dimenzija prostora $X$.

\begin{zgled}
Naj bo $X = [0,1] \subseteq \RR$. Velja:
\begin{itemize}
    \item $\mu_1([0,1]) = 1$,
    \item $\mu_2([0,1]) \leq \lim_{n \to \infty, \epsilon := \frac{1}{n}} \sum_{i = 1}^n \left( \frac{1}{n} \right)^2 = \lim_{n \to \infty} \frac{1}{n} = 0$,
    \item $\mu_{\frac{1}{2}}([0,1]) = \lim_{n \to \infty} \sum_{i = 1}^n \frac{1}{n^{1/2}} = \infty$.
\end{itemize}
Dimenzija prostora $[0,1]$ je $1$. Pravilen volumen smo izračunali z uporabo $\mu_1$. Pri $\mu_2$ in $\mu_{\frac12}$ smo izračunalni nesmiselna volumna, $0$ in $\infty$.
\end{zgled}

Ni težko premisliti \vaje, da kot v zadnjem zgledu za vsak $X$ obstaja število $d_0 \in \RR_{\geq 0} \cup \{ \infty \}$, tako da je
\[
\mu_d(X) = 
\begin{cases}
\infty; & d < d_0 \\
m_0; & d = d_0 \\
0; & d > d_0
\end{cases}
\]
za nek $m_0 \in \RR_{\geq 0} \cup \{ \infty \}$. 

\begin{figure}[t]
\centering
\begin{tikzpicture}[scale = 2]
\tikzstyle{vertex}=[circle, fill = oranzna, minimum size = 5pt, inner sep = 0pt]

\draw[<->, thick] (3,0) node [right] {$d$} -- (0,0) -- (0,1) node [left] {$\mu_d(X)$};
\draw[->, thick, draw=vijolicna] (0,1.1) to node [midway, above] {\color{vijolicna} $\infty$} (1,1.1);
\draw[<-, thick, draw=vijolicna] (1,0) node [below] {$d_0$} to node [midway, above] {\color{vijolicna} $0$} (3,0);
\node[vertex, fill = vijolicna] at (1, 0.5) {};
\end{tikzpicture}
\caption{Hausdorffova mera $\mu_d(X)$ za različne vrednosti $d$}
\end{figure}

Bolj kot mera $m_0$ nas zanima število $d_0$, ki ga imenujemo {\definicija Hausdorffova dimenzija} množice $X$, oznaka $\dim_H(X)$. Velja torej
\[
d_0 = \dim_H(X) = \inf \{ d \geq 0 \mid \mu_d(X) = 0 \}.
\]
Hausdorffova dimenzija ni nujno celo število.

\begin{zgled}
\begin{itemize}
    \item Hausdorffova dimenzija števnega metričnega prostora z diskretno topologijo je enaka $0$.

    \item Opremimo množico $\RR$ z diskretno metriko, kjer je razdalja med vsakima različnima številoma enaka $1$. Odprta množica premera manj kot $1$ vsebuje le eno točko. Števnih pokritij $\RR$ z množicami takega premera torej ni in zato je Hausdorffova dimenzija v tem primeru enaka $\inf \emptyset = \infty$.

    \item Naj bo $C \subseteq [0,1]$ Cantorjeva množica, konstruirana kot presek $\bigcap_{k = 1}^\infty C_k$, kjer je $C_k$ unija $2^k$ intervalov dolžine $\frac{1}{3^k}$. Vsaka množica $C_k$ nam torej podaja odprto pokritje množice $C$ s podmnožicami premera $\frac{1}{3^k}$. Ko gre $k$ čez vse meje, dobimo oceno za Hausdorffovo mero množice $C$:
\[
\mu_d(C) \leq \lim_{k \to \infty} 2^k \cdot \left( \frac{1}{3^k} \right)^d = \lim_{k \to \infty} \left( \frac{2}{3^d} \right)^k.
\]
Če je $2 < 3^d$, se pravi $d > \frac{\log 2}{\log 3}$, je torej $\mu_d(C) = 0$. Od tod sledi $\dim_H(C) \leq \frac{\log 2}{\log 3} \approx 0.63$. Dokazati je mogoče, da velja celo enakost $\dim_H(C) = \frac{\log 2}{\log 3}$.
\end{itemize}

\end{zgled}

Hausdorffova dimenzija je definirana za metrične prostore, običajna topološka dimenzija pa je definirana za vse topološke prostore. Za dan metrični prostor lahko obstaja kak homeomorfen prostor, ki je tudi metrični, a ima drugačno Hausdorffovo dimenzijo. Najmanjša možna Hausdorffova dimenzija homeomorfnih modelov prostora je ravno topološka dimenzija.

\begin{izrek}{\literatura (Szpilrajn 1937)}
Naj bo $X$ metrični prostor. Tedaj je topološka dimenzija $X$ enaka najmanjši možni Hausdorffovi dimenziji $\dim_H(Y)$, kjer $Y$ preteče metrične prostore, homeomorfne $X$.
\end{izrek}

\begin{dokaz}
Podajmo le idejo dokaza, da je topološka dimenzija prostora $X$ vselej manjša ali enaka Hausdorffovi dimenziji.\footnote{V nadaljevanju bomo uporabili le to lastnost dimenzij, saj bomo z uporabo Hausdorffove dimenzije dokazali, da je asimptotski stožec končne dimenzije.} Ekvivalentno, če je $X$ topološki prostor z $\mu_{d+1}(X) = 0$, potem je $X$ dimenzije kvečjemu $d$. Dokazujemo z indukcijo na $d$.

Naj bo $x \in X$ poljubna točka. Za vsak $r > 0$ opazujmo rob krogle okoli $x$, se pravi sfero $S(r) \subseteq X$ s središčem v $x$ in polmerom $r$. Dovolj bo dokazati, da za nek $r$ velja $\mu_d(S(r)) = 0$, izrek od tod sledi induktivno.

\begin{trditev}
Za vsako podmnožico $A \subseteq X$ velja
\[
\int_0^\infty \diam(S(r) \cap A)^d \ dr \leq \diam(A)^{d+1}.
\]
\end{trditev}

\begin{dokaz}
Naj bo $r_1 = \inf_{a \in A} d(a, x)$ in $r_2 = \sup_{a \in A} d(a, x)$. Velja $r_2 - r_1 \leq \diam(A)$, od koder sledi
\begin{align*}
\int_0^\infty \diam(S(r) \cap A)^d \ dr &= \int_{r_1}^{r_2} \diam(S(r) \cap A)^d \ dr \\
&\leq \diam(A)^d \int_{r_1}^{r_2} \ dr \leq \diam(A)^{d+1}.
\end{align*}
\end{dokaz}


Ker velja $\mu_{d+1}(X) = 0$, obstaja zaporedje števnih pokritij $\{ U_i^n \}_i$ množice $X$ za $n = 1, 2, \dots$, tako da je 
\[
\lim_{n \to \infty} \sum_i \diam(U_i^n)^{d+1} = 0.
\]
Iz pomožne trditve zdaj sledi
\[
\lim_{n \to \infty} \int_{0}^\infty \sum_i \diam(S(r) \cap U_i^n)^d \ dr = 0.
\]
Torej obstaja podzaporedje\footnote{Obstoj takega podzaporedja nam zagotovi Fatoujev lema.} zaporedja pokritij $\{ U_i^{n_k} \}_i$ za $k = 1, 2, \dots$, da za nek\footnote{Natančneje, ne le za nek $r$, temveč za vsak $r$ izven množice z Lebesgueovo mero $0$.} $r > 0$ velja
\[
\lim_{k \to \infty} \sum_i \diam(S(r) \cap U_i^{n_k})^d = 0.
\]
Od tod sledi $\mu_d(S(r)) = 0$.
\end{dokaz}

Končno dimenzionalnost asimptotskega stožca bomo izpeljali iz končnosti njegove Hausdorffove dimenzije. Pri tem bomo uporabili naslednjo metodo.

\begin{trditev}{\sc (krogle in Hausdorffova dimenzija)}
Predpostavimo, da lahko metrični prostor $X$ za vsak $\epsilon > 0$ pokrijemo s $k = k(\epsilon)$ kroglami premera kvečjemu $\epsilon$, kjer je $\limsup_{\epsilon \to 0} k \cdot \epsilon^d < \infty$. Tedaj je $\dim_H(X) \leq d$.
\end{trditev}

\begin{dokaz}
Naj bo $k(\epsilon) \cdot \epsilon^d < C$ za nek $C > 0$ neodvisen od $\epsilon$. Izberimo $e > d$. Tedaj je
\[
\mu_e(X) = \lim_{\epsilon \to 0} \inf_{\{ U_i \}_i } \sum_i (\diam U_i)^e
\leq \lim_{\epsilon \to 0} k(\epsilon) \cdot (\epsilon)^e \leq \lim_{\epsilon \to 0} C \epsilon^{e-d} = 0.
\]
\end{dokaz}

\subsection{Asimptotski stožec grupe polinomske rasti}

Pokažimo najprej, kako polinomska rast grupe vpliva na pokritja podmnožic asimptotskega stožca s kroglami in s tem na Hausdorffovo dimenzijo.

\begin{trditev}
Naj bo $G$ grupa polinomske rasti stopnje $d = \deg(G)$. Tedaj obstajata $\epsilon > 0$ in ultrafilter $\UU$ na $\NN$, za katera za asimptotski stožec $K$ grupe $G$ velja naslednje. Če krogla $B \subseteq K$ polmera $1$ vsebuje natanko $k$ točk in če sta vsaki dve točki pri tem oddaljeni za več kot $2 \epsilon$, potem je $k \leq \left( 4/\epsilon \right)^{d+1}$.
\end{trditev}

\begin{figure}[t]
\centering
\begin{tikzpicture}[scale = 2]
\tikzstyle{vertex}=[circle, fill = oranzna, minimum size = 5pt, inner sep = 0pt]

\draw[thick, fill = vijolicna!10, draw = vijolicna] (0,0) circle (1);
\foreach \r in {1,...,10}
\foreach \phi in {1,...,\r}
    {
    % \draw (0,0) circle (\r/5);
    \coordinate (C) at (10*\r^2+360/\r*\phi:\r/10);
    \draw[thick, fill = rdeca!10, draw = rdeca] (C) circle (0.045);
    % \node[vertex] at (C) {};
    };
\end{tikzpicture}
\caption{V kroglo $B \subseteq K$ ne moremo vnesti veliko malih disjunktnih kroglic}
\end{figure}

\begin{dokaz}
Naj bo $\epsilon > 0$ in $\UU$ ultrafilter na $\NN$, oba zaenkrat še nedoločena. Zaradi homogenosti $K$ lahko predpostavimo, da je $B = B_K(1)$. 

Naj bodo $x_1, \dots, x_k \in B$ elementi z $d(x_i, x_j) > 2\epsilon$ za $i \neq j$. Naj bo $x_i = [\{ g_{i,n} \}_{n \in \NN}]$ za neke $g_{i,n} \in G$. Ker je $x_i \in B$, velja $\lim_\UU d(g_{i,n}, 1)/n \leq 1$, zato za vse $n$ iz neke množice $M \in \UU$ velja $d(g_{i,n}, 1) \leq (1+\epsilon)n$.

Za vsaka $i \neq j$ in naj bo
\[
N(i,j,\epsilon) = \{ n \in \NN \mid d(g_{i,n}, g_{j,n}) > 2 \epsilon n \}.
\]
Ker je $d(x_i, x_j) > 2 \epsilon$, velja $N(i,j,\epsilon) \in \UU$, zato je tudi
\[
N(\epsilon) := \bigcap_{i \neq j} U(i,j,\epsilon) \in \UU.
\]

Izberimo $r \in \NN$ z lastnostjo $2^{-r} < \epsilon \leq 2^{-r+1}$. Naj bo $n \in \NN$, $n > r$, z lastnostjo $2^n \in N(\epsilon)$. V {\em globini} $2^n$ velja
\[
d(g_{i, 2^n}, g_{j, 2^n}) > 2^{n+1} \epsilon > 2^{n+1-r} = 2 \cdot 2^{n-r},
\]
zato so krogle $g_{i,2^n} \cdot B_{G,S}(2^{n-r})$ polmera $2^{n-r}$ okoli točk $g_{i, 2^n}$ disjunktne za $i = 1, \dots, k$. Od tod sledi
\[
\left| \bigcup_{i = 1}^k g_{i,2^n} \cdot B_{G,S}(2^{n-r}) \right| = k \cdot \beta_{G,S}(2^{n-r}).
\]
Hkrati za vsak $i$ velja
\[
g_{i,2^n} \cdot B_{G,S}(2^{n-r}) \subseteq B_{G,S}(2^{n-r} + d(g_{i, 2^n}, 1)) \subseteq B_{G,S}(2^{n-r} + (1+\epsilon) 2^n),
\]
slednja krogla pa je vsebovana v $B_{G,S}(2^{n+1})$ za dovolj majhen $\epsilon$.\footnote{Potrebujemo $(1+\epsilon) + 2^{-r} \leq 2$.} Od tod sledi neenakost
\[
k \cdot \beta_{G,S}(2^{n-r}) \leq \beta_{G,S}(2^{n+1})
\]
za vse $n > r$, za katere je $2^n \in N(\epsilon)$, pri tem pa je $r = r(\epsilon)$ fiksen.

\begin{trditev}
Za neskončno mnogo naravnih števil $a$ velja
\[
\forall b < a \colon \ \log_2 \beta_{G,S}(2^a) \leq \log_2 \beta_{G,S}(2^{a-b}) + b (d+1).
\]
\end{trditev}

\begin{dokaz}
Označimo $L(a) = \log_2 \beta_{G,S}(2^a)$. Ker je $d = \deg(G)$, za vse dovolj velike $a \in \NN$ velja $L(a)/\log_2(2^a) < d + \frac12$, se pravi $L(a) < a(d+1) - \frac{a}{2}$. V posebnem je $\lim_{a \to \infty} (L(a) - a(d+1)) = - \infty$. Za vsak $k \in \ZZ$ naj bo $a_k$ najmanjše naravno število, za katerega je $L(a_k) - a_k(d+1) < k$. Tedaj za vse $b < a_k$ velja
\[
L(a_k) - a_k(d+1) < k \leq L(a_k - b) - (a_k - b)(d+1),
\]
zato je $L(a_k) < L(a_k - b) + b(d+1)$. \kljuka
\end{dokaz}


Iz pomožne trditve in zadnje neenakosti pred njo izpeljemo\footnote{Uporabimo pomožno trditev z $a = n+1$ in $b = r+1$. Pri tem moramo biti nekoliko previdni glede tega, če pomožno trditev sploh lahko uporabimo z $a = n+1$. Naj bo $S$ neskončna množica naravnih števil $a$, za katere velja pomožna trditev. Naj bo $T = \{ 2^{a-1} \mid a \in S \}$. Ultrafilter $\UU$ izberimo tako, da vsebuje množico $T$. Za tem izberimo $n \in \NN$ z lastnostjo $2^n \in \NN_{> 2^r} \cap M \cap N(\epsilon) \cap T \in \UU$. S tako izbiro $n$ celoten dokaz deluje.}
\[
k \cdot \beta_{G,S}(2^{n-r}) \leq \beta_{G,S}(2^{n+1}) \leq 2^{(r+1)(d+1)} \cdot \beta_{G,S}(2^{n-r}),
\]
kar implicira $k \leq 2^{(r+1)(d+1)} \leq \left( \frac{4}{\epsilon} \right)^{d+1}$. Dokaz je s tem zaključen.
\end{dokaz}

S pomočjo zadnje trditve lahko omejimo dimenzijo asimptotskega stožca.

\begin{izrek}
Naj bo $G$ grupa polinomske rasti in $K$ njen asimptotski stožec glede na ultrafilter iz zadnje trditve. Tedaj ima $K$ končno dimenzijo in je lokalno kompakten.
\end{izrek}

\begin{dokaz}
\underline{Končna dimenzija:} Dimenzija je lokalna topološka lastnost, zato je dovolj dokazati, da je dimenzija krogle $B = B_K(1)$ končna. V ta namen izberimo maksimalno število točk $x_1, \dots, x_k \in B$ z lastnostjo $d(x_i, x_j) > 2 \epsilon$ za vse $i \neq j$. Iz zadnje trditve sledi $k \leq ( 4/\epsilon )^{d+1}$. Torej lahko $B$ pokrijemo z $(4/\epsilon)^{d+1}$ kroglami polmera $2 \epsilon$.\footnote{Vsaka točka v $B$ je po maksimalnosti izbire točk $x_1, \dots, x_k$ namreč največ $2\epsilon$ stran od kakšne od teh točk.} Iz trditve {\sc (krogle in Hausdorffova dimenzija)} sledi $\dim_H(B) \leq d+1$, s tem pa iz izreka {\literatura (Szpilrajn 1937)} sklepamo, da je $B$ končne dimenzije. \kljuka

\underline{Lokalna kompaktnost:} Dovolj je dokazati, da je vsaka krogla $B \subseteq K$ kompaktna. Naj bo $\{ x_n \}_{n \in \NN}$ zaporedje v $B$. Za vsak $i$ pokrijmo $B$ s $k_i$ kroglami polmera kvečjemu $2^{-i}$.\footnote{Tako pokritje res obstaja po zadnji trditvi.} Izberimo podzaporedje, ki v celoti leži v eni od krogel polmera kvečjemu $1$. Po tem izberimo podzaporedje, ki v celoti leži v eni od krogel polmera kvečjemu $\frac12$. Po tem izberimo podzaporedje, ki v celoti leži v eni od krogel polmera kvečjemu $\frac14$. \dots Nazadnje vzemimo podzaporedje, ki sestoji iz prvih členov vseh teh podzaporedij. Po konstrukciji je to Cauchyjevo zaporedje. Iz polnosti $K$ torej sledi, da ima $\{ x_n \}_{n \in \NN}$ konvergentno podzaporedje. \kljuka
\end{dokaz}

\subsection{Preko stožca do matrik}

Asimptotski stožec grupe polinomske rasti ima zdaj veliko dobrih topoloških lastnosti, od koder izpeljemo naslednjo posledico. 

\begin{posledica}
Bodi $G$ neskončna grupa polinomske rasti. Tedaj obstaja edinka $C \unlhd G$ končnega indeksa v $G$, da velja vsaj ena od naslednjih trditev.
\vspace{-0.5\baselineskip}
\begin{enumerate}[noitemsep]
    \item Za nek $k \in \NN$ ima $C$ epimorfno sliko $D$ za neko neskončno grupo $D \leq \GL_k(\CC)$.
    \item Obstaja Liejeva grupa $\Gamma$ s končno mnogo komponentami, tako da za vsak $n \in \NN$ obstaja homomorfizem $\varphi_n \colon C \to \Gamma$ z lastnostjo $|\image \varphi_n| \geq n$. \vspace{-\baselineskip}
\end{enumerate}
\end{posledica}

\begin{dokaz}
Uporabimo izrek {\literatura (Montgomery--Zippin 1955)} za asimptotski stožec $K$. Naj bo $\Gamma = \Isom(K)$, opremljena s strukturo Liejeve grupe s končno mnogo komponentami. Torej obstaja abelova edinka $Z \unlhd \Gamma$, da se $\Gamma/Z$ vloži v $\GL_k(\CC)$ za nek $k$. Naj bo $\Phi \colon G \to \Gamma$ delovanje grupe $G$ na asimptotskem stožcu in naj bo $L = \Phi^{-1}(Z)$. Velja $L \unlhd G$ in $L/\ker \Phi$ je abelova grupa. Obravnavajmo več primerov glede na indeks $L$ oziroma $\ker \Phi$ v $G$.

\begin{description}
    \item[Če je $|G : L| = \infty$:] V tem primeru ima $G$ epimorfno sliko $G/L$, ki se s $\Phi$ vloži v $\Gamma/Z$, ta pa se vloži v $\GL_k(\CC)$. Vzamemo lahko torej $C = G$ (1. točka). \kljuka

    \item[Če je $|G : L| < \infty$ in $|G : \ker \Phi| = \infty$:] V tem primeru ima $L$ epimorfno sliko $L/\ker\Phi$, ki je končno generirana neskončna abelova grupa, zato ima epimorfno sliko $\ZZ$. Slednja grupa se vloži v $\GL_1(\CC)$. Vzamemo lahko torej $C = L$ (1. točka). \kljuka

    \item[Če je $|G : \ker \Phi| < \infty$:] V tem primeru je $\image \Phi$ končna. Če je $\ker \Phi$ virtualno abelova grupa, ima torej edinko končnega indeksa z epimorfno sliko $\ZZ$, ki se vloži v $\GL_1(\CC)$. Vzamemo lahko torej $C = \ker \Phi$ (1. točka). \kljuka Sicer pa po izreku {\sc (o delovanju $\Phi$)} sledi, da za vsako okolico $O$ identitete $\id_K$ najdemo homomorfizem $\varphi_O \colon \ker \Phi \to \Isom(K)$ z lastnostjo $|\image \varphi_O \cap O| > 1$. Po lastnosti {\vprasanje (brez malih redov)} lahko za vsak $n \in \NN$ izberemo okolico $O_n$, ki ne vsebuje elementov reda kvečjemu $n$. Slika $\image \varphi_{O_n}$ tako vsebuje netrivialen element reda večjega od $n$, zato je $|\image \varphi_{O_n}| > n$. Vzamemo lahko torej $C = \ker \Phi$ (2. točka). \kljuka
\end{description}\vspace{-1.5\baselineskip}
\end{dokaz}

\subsection{Dokaz izreka Gromova}

Iz zadnje posledice lahko izluščimo naslednjo lastnost grup polinomske rasti, ki nam bo služila kot odskočna deska za induktiven dokaz izreka Gromova.

\begin{trditev}
Bodi $G$ neskončna grupa polinomske rasti. Tedaj obstaja pogrupa $H \leq G$ končnega indeksa v $G$, ki ima epimorfno sliko $\ZZ$.
\end{trditev}

\begin{dokaz}
Naj bo $C$ kot v zadnji posledici.

1. Naj bo $\alpha \colon C \to D$ epimorfizem za neko neskončno grupo $D \leq \GL_k(\CC)$. Grupe $G$, $C$ in $D$ so polinomske rasti, zato je $D$ virtualno nilpotentna. Torej obstaja podgrupa $E \leq D$ končnega indeksa v $D$ in zato tudi končnega indeksa v $G$, za katero je $E/\ker \alpha$ nilpotentna in neskončna podgrupa $\GL_k(\CC)$. Naj bo $k \in \NN$ največje število, za katero je $|E : E^{(k)}| < \infty$. Tedaj je grupa $E^{(k)}$ končnega indeksa v $G$, njen kvocient $E^{(k)}/E^{(k+1)}$ pa je neskončna končno generirana abelova grupa. Vzamemo lahko torej $H = E^{(k)}$. \kljuka

2. Naj bo $\varphi_n \colon C \to \Gamma$ homomorfizem z $|\image \varphi_n| > n$ za vsak $n \in \NN$. Naj bo $Z \unlhd \Gamma$ abelova edinka, za katero se $\Gamma/Z$ vloži v $\GL_k(\CC)$. Naj bo $L_n = \varphi_n^{-1}(Z)$. Torej se grupa $C/L_n$ vloži v $\Gamma/Z$, slednja pa se vloži v $\GL_k(\CC)$. 

Izrek {\literatura (Jordan 1878)} zagotovi obstoj podgrupe $K_n \leq C$ omejenega indeksa\footnote{Indeks $|C : K_n|$ je omejen s funkcijo, odvisno le od $k$ in ne od $n$.} v $C$, za katero je $K_n \geq L_n$ in $K_n/L_n$ je abelova grupa. Ker ima grupa $C$ le končno mnogo podgrup danega indeksa, je torej za neskončno mnogo vrednosti $n$ grupa $K_n$ enaka neki fiksni podgrupi $K \leq G$. Za neskončno mnogo naravnih števil $n$ je torej grupa $K/L_n$ abelova. 

Če so velikosti grup $|K/L_n|$ neomejene, ima grupa $K$ poljubno velike abelove kvociente. Ker je tudi končno generirana, sledi $|K/K^{(2)}| = \infty$. V tem primeru lahko vzamemo $H = K$. \kljuka 

Predpostavimo zdaj, da so velikosti grup $|K/L_n|$ omejene, ko $n$ preteče neko neskončno podmnožico naravnih števil. Ker ima $K$ le končno mnogo podgrup danega indeksa, je za neskončno mnogo vrednosti $n$ grupa $L_n$ enaka neki fiksni podgrupi $L \leq K$. Ker je $|\image \varphi_n| \geq n$ in $|C : L| < \infty$, sledi $\limsup_{n \to \infty} |L/\ker \varphi_n| = \infty$. Grupa $L/ \ker \varphi_n$ je abelova, saj se vloži v $Z$. Torej ima $L$ poljubno veliko abelove kvociente. Ker je tudi končno generirana, sledi $|L:L^{(2)}| = \infty$. Vzamemo lahko torej $H = L$. \kljuka
\end{dokaz}

Od tod lahko končno izpeljemo izrek Gromova.

\begin{dokaz}[Dokaz izreka Gromova]
Naj bo $G$ grupa polinomske rasti stopnje kvečjemu $d$. Dokazujemo z indukcijo na $d$. Bazni primer $d = 0$ je mogoče le, ko je grupa $G$ končna in v tem primeru izrek seveda velja. Predpostavimo zdaj, da izrek velja za vse grupe stopnje kvečjemu $d-1$. Po zadnji trditvi obstaja podgrupa $H \leq G$ končnega indeksa v $G$ in $N \unlhd H$, da je $\ZZ \cong H/N = \langle hN \rangle$ za nek $h \in H$. Po dokazu trditve {\sc (podeksponentna $\Rightarrow$ izvedena k.g.)} je $N$ končno generirana, hkrati pa je $\deg(N) \leq d-1$. Po indukciji zato obstaja nilpotentna podgrupa $K \leq N$ končnega indeksa v $N$. Po potrebi $K$ nadomestimo s presekom vseh podgrup v $N$ indeksa $|N:K|$, zato lahko predpostavimo, da je $K \unlhd H$. Naj bo $C = \langle K, h \rangle \leq H$. Velja $H/K = C/K \cdot N/K$, zato je $C$ končnega indeksa v $H$ in s tem tudi v $G$. Grupa $C$ je razširitev $\ZZ$ z nilpotentno grupo $K$, torej je $C$ rešljiva. Ker je $C$ polinomske rasti, od tod sledi, da je $C$ virtualno nilpotentna. Dokaz je s tem zaključen.
\end{dokaz}

\chapter{Srednja rast}

V tem razdelku si bomo ogledali konkreten primer grupe srednje rasti. S tem bomo odgovorili na vprašanje {\sc \vprasanje (srednja)}. Pokazali bomo tudi, kako so te grupe tesno povezane z (ne)obstojem paradoksalnih dekompozicij.

\section{Grigorčukova grupa}

Prvi primer grupe srednje rasti najdemo v \href{http://www.mathnet.ru/links/7f1a24e826f4c70ec48d9c3a6dd9ef75/faa1772.pdf}{(Grigorčuk 1980)}. Grupa je konstruirana rekurzivno kot grupa nekih posebnih transformacij intervala, mi pa si bomo pogledali njeno ekvivalentno rekurzivno konstrukcijo prek avtomorfizmov drevesa. Videli bomo, da je ravno ta rekurzivnost definicije osrednja pri srednjosti njene rasti.

\subsection{Avtomorfizmi drevesa}

Naj bo $V = \{ 0, 1 \}^*$ množica {\em končnih} nizov s črkama $0$ in $1$, vključivši prazen niz $\emptyset$. Naj bo $E = \{ \{ v, w \} \mid v,w \in V, \ \text{$w = v0$ ali $w = v1$} \}$. Na ta način dobimo {\definicija dvojiško drevo} $T$ z vozlišči $V$ in povezavami $E$. Rečemo, da je drevo $T$ {\definicija zakoreninjeno} v vozlišču $\emptyset$. Za vozlišče $v \in V$ naj bo $T_v$ poddrevo, zakoreninjeno v $v$. Drevesi $T_v$ in $T$ sta naravno izomorfni.

\begin{figure}[t]
\centering
\begin{tikzpicture}[level distance=5mm, edge from parent/.style={draw,thick}, scale = 1.5]

\tikzstyle{vertex}=[circle, fill = oranzna, minimum size = 5pt, inner sep = 0pt]

\tikzstyle{level 1}=[sibling distance=26mm]
\tikzstyle{level 2}=[sibling distance=15mm]
\tikzstyle{level 3}=[sibling distance=5mm]
\coordinate (root)node [vertex] {}
child {node [vertex] {}
       child {node [vertex] {}
               child {node [vertex] {}}
               child {node [vertex] {}}
             }
       child {node [vertex] {}
               child {node [vertex] {}}
               child {node [vertex] {}}
             }
      }
child {node [vertex] {}
       child {node [vertex] {}
               child {node [vertex] {}}
               child {node [vertex] {}}
             }
       child {node [vertex] {}
               child {node [vertex, label = left : $110$] (110) {} }
               child {node [vertex] {}}
             }
      };

\node[above] at (root) {$\emptyset$};
\draw[thick, draw = vijolicna, fill = vijolicna!10] (110) -- ($(110) + (-60:0.45)$) -- ($(110) + (-120:0.45)$) -- (110);
\end{tikzpicture}
\caption{Drevo $T$ in njegovo poddrevo $T_{110}$}
\end{figure}

Opazujmo grupo avtomorfizmov $\Aut(T)$. Vsak avtomorfizem fiksira koren. Naj bosta $T_0$ in $T_1$ levo in desno poddrevo drevesa $T$. Definirajmo avtomorfizem, ki zamenja ti dve poddrevesi:
\[
a \in \Aut(T), \quad 
\emptyset \mapsto \emptyset, \ 0v \mapsto 1v, \ 1v \mapsto 0v \quad (v \in \{0,1\}^*).
\]
Jasno je $a^2 = 1$. 

Za poljubno vozlišče $v$ naj $\Aut(T, T_v) \leq \Aut(T)$ sestoji iz tistih avtomorfizmov drevesa $T$, ki ohranjajo drevo $T_v$ in fiksirajo vozlišča izven $T_v$. Iz izomorfizma grafov $T \to T_v$ dobimo izomorfizem grup
\[
\prenos_v \colon \Aut(T) \to \Aut(T,T_v),
\]
ki avtomorfizem drevesa $T$ prenese na poddrevo $T_v$. Tako lahko avtomorfizem $a$ prenesemo na poddrevesa in dobimo $a_v = \prenos_v(a) \in \Aut(T,T_v)$ za poljubno vozlišče $v$.

S pomočjo preslikave $\prenos_v$ lahko sestavimo iz dveh avtomorfizmov drevesa enega in dobimo {\definicija sestavljajoči homomorfizem}
\[
\sestavi \colon \Aut(T) \times \Aut(T) \to \Aut(T), \quad
(\alpha, \beta) \mapsto \prenos_0(\alpha) \cdot \prenos_1(\beta).
\]
S tem lahko rekurzivno definiramo avtomorfizme $b,c,d \in \Aut(T)$: 
\[
b = \sestavi(a,c), \quad
c = \sestavi(a,d), \quad
d = \sestavi(1, b).
\]
Z drugimi besedami, velja
\[
b = \prod_{i = 0}^\infty \left( a_{1^{3i} 0} \cdot a_{1^{3i+1}0} \right), \quad
c = \prod_{i = 0}^\infty \left( a_{1^{3i} 0} \cdot a_{1^{3i+2}0} \right), \quad
d = \prod_{i = 0}^\infty \left( a_{1^{3i+1} 0} \cdot a_{1^{3i+2}0} \right),
\]
kjer smo z $1^i$ označili niz enic dolžine $i$.
{\definicija Grigorčukova grupa} je grupa $\Gamma = \langle a,b,c,d \rangle \leq \Aut(T).$

\begin{figure}[t]
\centering
\begin{tikzpicture}[level distance=5mm, edge from parent/.style={draw,thick}, scale = 1.5]

\tikzstyle{vertex}=[circle, fill = oranzna, minimum size = 5pt, inner sep = 0pt]

\tikzstyle{level 1}=[sibling distance=26mm]
\tikzstyle{level 2}=[sibling distance=15mm]
\tikzstyle{level 3}=[sibling distance=5mm]
\coordinate (root)node [vertex] {}
child {node [vertex] (1) {}}
child {node [vertex] {}
       child {node [vertex] (10) {}}
       child {node [vertex] {}
               child {node [vertex] (110) {} }
               child {node [vertex] (111) {}}
             }
      };

\draw[thick, draw = vijolicna, fill = vijolicna!10] (1) -- ($(1) + (-60:0.45)$) -- ($(1) + (-120:0.45)$) -- (1);
\node [label = below : {\color{vijolicna} $a$}] at (1) {};

\draw[thick, draw = vijolicna, fill = vijolicna!10] (10) -- ($(10) + (-60:0.45)$) -- ($(10) + (-120:0.45)$) -- (10);
\node [label = below : {\color{vijolicna} $a$}] at (10) {};

\draw[thick, draw = vijolicna, fill = vijolicna!10] (110) -- ($(110) + (-60:0.45)$) -- ($(110) + (-120:0.45)$) -- (110);
\node [label = below : {\color{vijolicna} $1$}] at (110) {};

\draw[thick, draw = vijolicna, fill = vijolicna!10] (111) -- ($(111) + (-60:0.45)$) -- ($(111) + (-120:0.45)$) -- (111);
\node [label = below : {\color{rdeca} $b$}] at (111) {};

% \node [label = below : {\color{rdeca} $b$}] at (root) {};
\end{tikzpicture}
\caption{Delovanje elementa $b \in \Gamma$ na drevesu $T$}
\end{figure}

% \subsection{Transformacije intervala}

% Za poljuben odprt interval $I \subseteq \RR$ s srednjo točko $i \in I$ definirajmo naslednji bijekciji:
% \[
% E_I \colon I \backslash \{ i \} \to I \backslash \{ i \}, \quad x \mapsto x,
% \quad \text{in} \quad
% P_I \colon I \backslash \{ i \} \to I \backslash \{ i \}, \quad x \mapsto \begin{cases}
% x - \frac{|I|}{2}; & x > i \\
% x + \frac{|I|}{2}; & x < i.
% \end{cases}
% \]
% Za funkciji $f \colon I_1 \to I_1$ in $g \in \colon I_2 \to I_2$, kjer sta $I_1, I_2$ neka disjunktna intervala, pišemo $f \oplus g \colon I_1 \cup I_2 \to I_1 \cup I_2$ za funkcijo, ki na $I_1$ slika kot $f$ in na $I_2$ kot $g$. 

% Opazujmo množico $J = (0,1) \backslash \{ \frac{a}{2^n} \mid a \in \ZZ, \ n \in \NN_0 \} \subseteq \RR$. Definirajmo naslednje transformacije množice $J$:
% \begin{align*}
% a \colon J \to J, \quad & a = P_J, \\
% b \colon J \to J, \quad & b = P_{J \cap (0,\frac12)} \oplus P_{J \cap (\frac12, \frac34)} \oplus E_{J \cap (\frac34, \frac78)} \oplus b, \\
% c \colon J \to J, \quad & c = P_{J \cap (0,\frac12)} \oplus E_{J \cap (\frac12, \frac34)} \oplus P_{J \cap (\frac34, \frac78)} \oplus c, \\
% d \colon J \to J, \quad & b = E_{J \cap (0,\frac12)} \oplus P_{J \cap (\frac12, \frac34)} \oplus P_{J \cap (\frac34, \frac78)} \oplus d,
% \end{align*}
% kjer zadnji člen transformacij $b,c,d$ razumemo rekurzivno na intervalu $(\frac78, 1)$.

% \begin{zgled}
% Velja $b \colon \frac{1}{3} \mapsto \frac{1}{3} - \frac{1}{4}$, $\frac{65}{80} \mapsto \frac{65}{80}$, $\frac{71}{80} \mapsto \frac{71}{80} + \frac{1}{16}$.
% \end{zgled}

% {\definicija Grigorčukova grupa} je $\Gamma = \langle a,b,c,d \rangle \leq \Sym(J)$. Pri tem ni zares pomembno, da grupa deluje na konkretni množici $J$. Lahko bi enakovredno definirali generatorje $a,b,c,d$ na množici $J \cap (0, \frac12)$ in dobili izomorfno grupo. 

\subsection{Osnovne lastnosti}

V grupi $\Gamma$ jasno velja $a^2 = 1$. Ker elementi $a_{1^i0}$ komutirajo za različne vrednosti $i$, velja tudi $b^2 = c^2 = d^2 = 1$ in $bc = cb = d$, $cd = dc = b$, $bd = db = c$. Torej je $\langle b,c,d \rangle \cong (\ZZ/2\ZZ)^2$.

Vsak element grupe $\Gamma$ lahko zapišemo kot besedo s črkami $a,b,c,d$, pri čemer lahko predpostavimo, da ni zaporednih ponovitev nobene črke in da ni zaporednih ponovitev črk iz množice $\{ b,c,d \}$. Tako se v zapisu vsakega elementa grupe $\Gamma$ izmenjuje črka $a$ in črka iz $\{ b,c,d \}$. Taki besedi proste grupe $F(\{ a,b,c,d \})$ pravimo {\definicija alternirajoča beseda}.

\begin{zgled}
Velja $a^3 b^{-1} c a^2 = a b c = ad$.
\end{zgled}

Najkrajše besede, s katerimi lahko zapišemo elemente v grupi $\Gamma$, so alternirajoče.

\begin{lema}
Naj bo $g \in \Gamma$, zapisan z besedo $w \in F(\{ a,b,c,d \})$ dolžine $\ell(g)$. Potem je $w$ alternirajoča beseda. 
\end{lema}

\begin{dokaz}
Predpostavimo, da $w = x_1 x_2 \dots x_k$ ni alternirajoča. To je možno le v dveh primerih.

\begin{description}
    \item[Če za nek $i$ velja $x_i = x_{i+1}$:] V tem primeru je $x_i x_{i+1} = x_i^2 = 1$. Beseda $w^\prime = x_1 \cdots x_{i-1} x_{i+2} \cdots x_k$ zato predstavlja element $g$ in je krajše dolžine kot $\ell(g)$. Protislovje.

    \item[Če za nek $i$ velja $x_i, x_{i+1} \in \{b,c,d\}$:] Ker velja $bc = d, bd=c, cd = b$, je v tem primeru $y := x_i x_{i+1} \in \{b,c,d\}$. Beseda $w^\prime = x_1 \cdots x_{i-1} y x_{i+2} \cdots x_k$ zato predstavlja element $g$ in je krajše dolžine kot $\ell(g)$. Protislovje.
\end{description}\vspace{-1.5\baselineskip}
\end{dokaz}

Grupa $\Gamma$ inducirano deluje na množici $\{ T_0, T_1 \}$.\footnote{Če je $x \in T_0$, potem je $a.x \in T_1$, $b.x \in T_0$, $c.x \in T_0$, $d.x \in T_0$. Podobno velja za $x \in T_1$.} Torej dobimo homomorfizem
\[
\Gamma \to \Sym(\{ T_0, T_1 \}).
\]
Jedro tega delovanja označimo s $H$. Grupa $H$ sestoji iz besed, v katerih se element $a$ pojavi sodokrat. Velja torej
\[
H = \langle b,c,d, aba, aca, ada \rangle \leq \Gamma \quad
\text{in} \quad |\Gamma:H| = 2.
\]
Elementi grupe $H$ delujejo na vsakem od poddreves $T_0$ in $T_1$. Za element $x \in H$ označimo z $x_0$ delovanje $x$ na $T_0$ in z $x_1$ delovanje $x$ na $T_1$. Po identifikaciji $T_0$ in $T_1$ z drevesom $T$ lahko vidimo $x_0, x_1$ kot avtomorfizma drevesa $T$.

\begin{zgled}
Za element $b \in H$ je $b_0 = a$ in $b_1 = c$.
\end{zgled}

Na ta način dobimo {\definicija razstavljajoči homomorfizem}\footnote{Ta homomorfizem je v bistvu inverz homomorfizma $\sestavi$.}
\[
\razstavi \colon H \to \Aut(T) \times \Aut(T),
\quad
x \mapsto (x_0, x_1).
\]
Generatorji grupe $H$ se pri tem preslikajo na naslednji način:
\[
b \mapsto (a,c), \ c \mapsto (a,d), \ d \mapsto (1,b), \
aba \mapsto (c,a), \ aca \mapsto (d,a), \ ada \mapsto (b,1).\footnote{Ne spreglejmo, da je element $d$ rahlo poseben. To bomo kasneje dobro izkoristili.}
\]
Od tod sledi, da je $\image \razstavi \leq \Gamma \times \Gamma$. Še več, ker je vsak element grupe $H$ enolično določen s tem, kako deluje na obeh polovicah drevesa $T$, je preslikava $\razstavi$ injektivna. S tem smo dobili vložitev grupe $H$ v $\Gamma \times \Gamma$. Ta vložitev ima še dodatno lastnost, da sta preslikavi $\pi_i \circ \razstavi$ surjekciji za $i=1,2$, kjer smo s $\pi_i \colon \Gamma \times \Gamma \to \Gamma$ označili projekcijo na $i$-to komponento. Nazadnje ima torej grupa $\Gamma$ podgrupo indeksa $2$, ki ima epimorfno sliko $\Gamma$. Zaradi tega je $\Gamma$ nujno {\em neskončna} grupa.

\subsection{Samopodobnost}

Kadar imata grupi $X$ in $Y$ izomorfni podgrupi $Z \cong W$, $Z \leq X$, $|X : Z| < \infty$, $W \leq Y$, $|Y:W| < \infty$, potem rečemo, da sta grupi $X$ in $Y$ {\definicija primerljivi}.\footnote{Angleško {\em commensurable}.}

\begin{trditev}
Grupi $\Gamma$ in $\Gamma \times \Gamma$ sta primerljivi.
\end{trditev}

\begin{dokaz}
Opazujmo vložitev $\razstavi \colon H \to \Gamma \times \Gamma$. Za vsak $g \in \Gamma$ obstaja $x \in H$, da je $x_0 = g$, se pravi $\razstavi(x) = (g, x_1)$. Hkrati je $\razstavi(ada) = (b,1)$, kar po konjugiranju z $x$ postane $\razstavi((ada)^x) = (b,1)^{\razstavi(x)} = (b^g, 1)$. Naj bo $B = \llangle b \rrangle \leq G$. Torej slika $\image \razstavi$ gotovo vsebuje grupo $B \times 1 \leq \Gamma \times \Gamma$. Podobno dobimo $\image \razstavi \geq 1 \times B$ in s tem $\image \razstavi \geq B \times B$.

\begin{trditev}
Velja $\langle a,d \rangle \cong D_8$ in $|\Gamma : B| \leq 8$.
\end{trditev}

\begin{dokaz}
Velja $a^2 = d^2 = 1$. Ker je $(ad)^2 = ada \cdot d \in H$, je $\razstavi((ad)^2) = (b, b)$, ki je reda $2$. Ker je $\razstavi$ vložitev, je zato $ad$ reda $4$. Tako po univerzalni lastnosti dobimo surjektiven homomorfizem $\langle s,t \mid s^4, t^2, tst = s^{-1} \rangle \to \langle a,d \rangle$, $s \mapsto ad$, $t \mapsto a$. Ker je $\langle a,d \rangle > \langle ad \rangle \cong \ZZ/4\ZZ$, je zadnji homomorfizem nujno izomorfizem. \kljuka

Ker je $\Gamma = \langle a,b,c \rangle$, je grupa $\Gamma/B$ generirana z odsekoma elementov $a$ in $d$. Ker je $\langle a,d \rangle \cong D_8$, je res $|\Gamma : B| \leq 8$. \kljuka
\end{dokaz}


Iz pomožne trditve sledi $|\Gamma \times \Gamma : \image \razstavi| \leq 64$. Tako ima grupa $\Gamma$ podgrupo končnega indeksa $H$, ki je izomorfna podgrupi končnega indeksa $\image \razstavi$ grupe $\Gamma \times \Gamma$.
\end{dokaz}

Grupa $\Gamma$ je torej {\em samopodobna}, kot neke vrste fraktal. Če se ji malce ($|\Gamma : H| < \infty$) približamo, vidimo objekt ($H$), ki skoraj ($|\Gamma \times \Gamma : \image \razstavi| < \infty$) izgleda kot dve kopiji ($\Gamma \times \Gamma$) začetnega objekta. Samopodobne grupe nikdar niso polinomske rasti.

\begin{posledica}
$\Gamma$ {\em ni} polinomske rasti.
\end{posledica}

\begin{dokaz}
Predpostavimo, da je $\Gamma$ polinomske rasti. Potem je tudi $\Gamma \times \Gamma$ polinomske rasti. Iz zadnje trditve zdaj sledi $\deg(\Gamma) = \deg(\Gamma \times \Gamma)$. Toda $\deg(\Gamma \times \Gamma) = 2 \cdot \deg(\Gamma)$ \vaje. Od tod sledi $\deg(\Gamma) = 0$, kar je mogoče le, če je $\Gamma$ končna. Protislovje.
\end{dokaz}

\subsection{Rast}

Za analizo rasti grupe $\Gamma = \langle a,b,c,d \rangle$ bomo uporabili preslikavo $\razstavi$, ki skrajšuje dolžine elementov.

\begin{lema}
Bodi $x \in H$ in $\razstavi(x) = (x_0, x_1)$. Velja $\ell(x_0), \ell(x_1) \leq \frac12 (\ell(x) + 1)$.
\end{lema}

\begin{dokaz}
Zapišimo $x$ kot najkrajši možen alternirajoč produkt elementov $u$ in $ava$ za $u,v \in \{ b,c,d \}$. Naj bo število uporabljenih faktorjev enako $2k+r$ za $r \in \{0, 1\}$. Torej je $x$ produkt $k$ parov oblike $u \cdot ava$ ali $ava \cdot u$ z morebitnim dodatnim faktorjem na koncu. Vsak tak par je dolžine kvečjemu $4$, s preslikavo $\razstavi$ pa se preslika v dva elementa dolžine kvečjemu $2$. Torej  v primeru $r = 0$ velja $\ell(x) = 4k$ in $\ell(x_0), \ell(x_1) \leq 2k$, v primeru $r = 1$ pa velja $\ell(x_0), \ell(x_1) \leq 2k+1$.
\end{dokaz}

Skupna dolžina $x_0$ in $x_1$ je torej $\ell(x_0) + \ell(x_1) \leq \ell(x) + 1$. Če ima $x$ veliko pojavitev črke $d$, potem je dolžina $x_0$ oziroma $x_1$ še krajša, ker je $\razstavi(d) = (1,b)$ in $\razstavi(ada) = (b,1)$. Če je v $x$ malo pojavitev črke $d$, pa morda lahko uporabimo $\razstavi$ na komponentah $x_0$, $x_1$ in pretvorimo pojavitve $c$ v $d$ s prvo uporabo $\razstavi$, z drugo pa potem $d$ v $1$. Če je malo pojavitev $c$ in $d$, morda lahko uporabimo $\razstavi$ trikrat in pretvorimo pojavitve $b$ v $c$, $c$ v $d$ in nazadnje $d$ v $1$ ter s tem še bolj skrajšamo dolžino.

Naj bo $K = \razstavi^{-1}(H \times H)$. Velja $|H : K| = |\razstavi(H) : \razstavi(H) \cap (H \times H)| \leq |\Gamma \times \Gamma : H \times H| = 4$, zato je $|\Gamma : K| \leq 8$. Naj bo še $L = \razstavi^{-1}(K \times K)$. Kot prej velja $|\Gamma : L| \leq 128$. Za $x \in L$ je $x_0, x_1 \in K$, zato je $x_{00}, x_{01}, x_{10}, x_{11} \in H$ in nazadnje dobimo elemente $x_{000}, x_{001}, \dots, x_{111} \in \Gamma$. Trdimo, da je skupna dolžina po trikratni uporabi $\razstavi$ bistveno krajša od originalne dolžine.

\begin{lema}{\sc (o krajšanju dolžin)}
Bodi $x \in L$. Velja
\[
\ell(x_{000}) + \ell(x_{001}) + \cdots + \ell(x_{111}) \leq \frac56 \ell(x) + 8.
\] \vspace{-\baselineskip}
\end{lema}

\begin{dokaz}
Vemo že, da velja $\ell(x_0) + \ell(x_1) \leq \ell(x) + 1$. Naj bo $\ell_b(x)$ število pojavitev črke $b$ v besedi $x$, zapisani alternirajoče z dolžino $\ell(x)$. Analogno definiramo $\ell_c(x), \ell_d(x)$.

\begin{itemize}
    \item Vsaka pojavitev črke $d$ ali $ada$ v besedi $x$ postane $1$ v natanko enem od $x_0, x_1$. Od tod sklepamo\footnote{Na primer, velja $\razstavi(aba \cdot d \cdot aca) = (c,a)(1,b)(d,a) = (cd, aba) = (b, aba)$.}
\[
\ell(x_0) + \ell(x_1) \leq \ell(x) + 1 - \ell_d(x)
\]
in zatorej
\begin{align*}
\ell(x_{000}) + \cdots + \ell(x_{111}) &\leq (\ell(x_{00}) + 1) + \cdots + (\ell(x_{11}) + 1) \\
&\leq (\ell(x_0) + 1) + (\ell(x_1) + 1) + 4 \\
&\leq \ell(x) + 7 - \ell_d(x). 
\end{align*}
    
    \item Vsaka pojavitev $c$ ali $aca$ v besedi $x$ postane $d$ v natanko enem od $x_0, x_1$.\footnote{Pri tem se lahko zgodi, da po eni uporabi $\razstavi$ dobimo $x_0, x_1$, ki se še poenostavita, kot v primeru $\razstavi(aba \cdot d \cdot aca)$.} Od tod sklepamo
\begin{align*}
\ell(x_{00}) + \cdots + \ell(x_{11}) &\leq \ell(x_0) + 1 + \ell(x_1) + 1 - \ell_c(x) \\
&\leq \ell(x) + 3 - \ell_c(x)
\end{align*}
in zatorej
\begin{align*}
\ell(x_{000}) + \cdots + \ell(x_{111}) &\leq (\ell(x_{00}) + 1) + \cdots + (\ell(x_{11}) + 1) \\
&\leq \ell(x) + 7 - \ell_c(x).
\end{align*}

    \item Vsaka pojavitev $b$ ali $aba$ v besedi $x$ postane $c$ v natanko enem od $x_0, x_1$. Od tod sklepamo
\begin{align*}
\ell(x_{000}) + \cdots + \ell(x_{111}) &\leq (\ell(x_{00}) + 1) + \cdots + (\ell(x_{11}) + 1) - \ell_b(x) \\
&\leq \ell(x) + 7 - \ell_b(x). 
\end{align*}
\end{itemize}

Velja $\ell_b(x) + \ell_c(x) + \ell_d(x) \geq \frac12 (\ell(x) - 1)$, zato je $\ell_u(x) \geq \frac16 \ell(x) - 1$ za nek $u \in \{ b,c,d\}$. Iz zgornjih neenakosti zato v vsaki od treh možnosti za $u$ sledi
\[
\ell(x_{000}) + \cdots + \ell(x_{111}) \leq \ell(x) + 7 - \ell_u(x) \leq \frac56 \ell(x) + 8.
\]
\end{dokaz}

S pomočjo krajšanja dolžin lahko nazadnje dokažemo, da $\Gamma$ ni eksponentne rasti. Vemo tudi že, da $\Gamma$ ni polinomske rasti, torej s tem odgovorimo na vprašanje {\vprasanje \sc (srednja)}.

\begin{izrek}
$\Gamma$ je srednje rasti.
\end{izrek}

\begin{dokaz}
Naj bo $\omega = \omega_{\{ a,b,c,d \}}(\Gamma)$. Dokazati želimo, da je $\omega = 1$.

Najprej prevedimo problem na grupo $L$. Naj bo $R$ množica predstavnikov odsekov $L$ v $\Gamma$. Torej je $\Gamma = L \cdot R$. Naj bo $M = \max \{ \ell(r) \mid r \in R \}$. Izberimo poljuben $x \in B_{\Gamma}(n)$. Zapišemo ga lahko kot $x = yz$ za neka $y \in L, z \in R$. Pri tem velja $\ell(y) \leq \ell(x) + \ell(z) \leq n + M$. Torej je
\[
\beta_\Gamma(n) \leq |\Gamma : L| \cdot |\{ y \in L \mid \ell(y) \leq n+M \}| 
\leq 128 \cdot |L \cap B_\Gamma(n+M)|.
\]

Oceniti moramo torej še $|L \cap B_\Gamma(n+M)|$. V ta namen izberimo poljuben $x \in L \cap B_\Gamma(m)$ za nek $m \in \NN$. Naj bo $m_{ijk} = \ell(x_{ijk})$ za $i,j,k \in \{0,1\}$. Število možnosti za 8-terico elementov $(x_{000}, \dots, x_{111})$ je enako kvečjemu $\beta_\Gamma(m_{000}) \cdots \beta_\Gamma(m_{111})$. Po definiciji $\omega$ za vsak $\epsilon > 0$ velja $\beta_\Gamma(n) \leq A \cdot (\omega + \epsilon)^n$ za vsak $n$ in neko konstanto $A$. Torej je število možnosti za $(x_{000}, \dots, x_{111})$ in s tem za $x \in L \cap B_\Gamma(m)$ kvečjemu
\[
A^8 \cdot (\omega + \epsilon)^{m_{000} + \cdots + m_{111}} 
\leq A^8 \cdot (\omega + \epsilon)^{\frac56 m + 8} \\
\leq C \cdot (\omega + \epsilon)^{\frac56 m}
\]
za neko konstanto $C$, kjer smo v prvi neenakosti uporabili lemo {\sc (o krajšanju dolžin)}. 

Za vsak $n \in \NN$ nazadnje torej velja
\[
\beta_\Gamma(n) \leq 128 \cdot C \cdot (\omega + \epsilon)^{\frac56 (n+M)}.
\]
Hkrati po definiciji $\omega$ velja $\beta_\Gamma(n) \geq C^\prime \cdot (\omega - \epsilon)^n$ za neko konstanto $C^\prime$. Uporabimo $n$-ti koren, izlimitiramo $n \to \infty$ in dobimo
\[
\omega - \epsilon \leq (\omega + \epsilon)^{\frac56}.
\]
Ker je bil $\epsilon$ poljuben, sledi $\omega = \omega^{\frac56}$, zato je $\omega = 1$.
\end{dokaz}

Rastna funkcija $\Gamma$ ni torej niti polinomska niti eksponentna. {\vprasanje Ali je ekvivalentna kateri znani elementarni funkciji?}

\begin{izrek}
Obstajata števili $0 < p, q < 1$, da je $e^{n^p} \preceq \beta_\Gamma \preceq e^{n^q}$.
\end{izrek}

\begin{dokaz}
\underline{Zgornja meja:}
Dokaz je v tem primeru bolj natančna različica dokaza zadnjega izreka.
Iz leme {\sc (o krajšanju dolžin)} sledi, da za nek $x_{ijk}$ velja $\ell(x_{ijk}) \leq \frac{5}{48} \ell(x) + 1$. Za element $x_{ijk}$ imamo torej največ $\beta_\Gamma(\frac{5}{48} \ell(x) + 1)$ možnosti. Za vse druge elemente množice $\{ x_{000}, \dots, x_{111} \} \backslash \{ x_{ijk} \}$ pa imamo največ $\beta_\Gamma(\frac18 \ell(x) + 1)$ možnosti.\footnote{Za $x \in L$ namreč velja $\ell(x_{000}), \dots, \ell(x_{111}) \leq \frac18 \ell(x) + 1$, saj za vsak $h \in H$ po prvi lemi v tem razdelku velja $\ell(h_0), \ell(h_1) \leq \frac12 (\ell(h) + 1)$.} Po submultiplikativnosti rastne funkcije velja
\[
\beta_\Gamma \left( \frac18 \ell(x) + 1 \right) \leq \beta_\Gamma \left( \frac{1}{48} \ell(x) + 1 \right)^6.
\]
Vseh možnosti za $8$-terico $(x_{000}, \dots, x_{111})$ je torej kvečjemu\footnote{Izbrati moramo element $x_{ijk}$, za katerega velja strožja neenakost glede dolžine. Za to imamo na voljo $8$ možnosti.}
\[
8 \cdot \beta_\Gamma \left( \frac{1}{48} \ell(x) + 1 \right)^{6 \cdot 7 + 5}.
\]
Kot v dokazu zadnjega izreka od tod sledi
\[
\beta_\Gamma(n) \leq 128 \cdot |L \cap B_\Gamma(n+M)| \leq C \cdot \beta_\Gamma\left(\frac{1}{48} n + C\right)^{47}
\]
za neko konstanto $C$. To neenakost zdaj iteriramo. V naslednjem koraku dobimo
\begin{align*}
\beta_\Gamma(n) &\leq C \cdot \left( C \cdot \beta_\Gamma\left(\frac{1}{48} \left( \frac{1}{48} n + C \right) + C\right)^{47}\right)^{47} \\
&= C^{1 + 47} \cdot \beta_\Gamma \left( \frac{1}{48^2} n + C \left(1 + \frac{1}{48}\right) \right)^{47^2},
\end{align*}
po $t$ korakih pa dobimo neenakost
\[
\beta_\Gamma(n) \leq C^{1 + 47 + 47^2 + \cdots + 47^{t-1}} \cdot \beta_\Gamma \left( \frac{1}{48^t} n + C \left( 1 + \frac{1}{48} + \frac{1}{48^2} + \cdots + \frac{1}{48^{t-1}} \right) \right)^{47^t}.
\]
Izberimo $t = \lceil \log_{48} n \rceil$. Tedaj je $n/48^t \leq 1$, zato iz zadnje neenakosti dobimo
\[
\beta_\Gamma(n) \leq C^{47^t} \cdot \beta_\Gamma(D)^{47^t} = E^{47^t}
\]
za neki konstanti $D, E$. Po izbiri $t$ od tod nazadnje sledi
\[
\beta_\Gamma(n) \leq E^{47^{\log_{48} n + 1}} 
= e^{47 \cdot \log E \cdot 47^{\frac{\log n}{\log 48}}}
\sim e^{n^{\log_{48} 47}},
\]
torej v izreku lahko vzamemo $q = \log_{48} 47 < 1$. \kljuka

\underline{Spodnja meja:}
Dokaz je v tem primeru kombinacija samopodobnosti grupe $\Gamma$ in zgornjega iterativnega argumenta. Funkcija $\beta_\Gamma$ je ekvivalentna funkciji $\beta_{\Gamma \times \Gamma}$. Hkrati za $S = \{ a,b,c,d \}$ velja $\beta_{\Gamma \times \Gamma, S \times 1 \cup 1 \times S \cup S \times S} = \beta_{\Gamma}^2$. Torej obstaja konstanta $C > 1$, da je
\[
\beta_\Gamma(n)^2 \leq C \cdot \beta_\Gamma(Cn).
\]
To neenakost zdaj iteriramo. V naslednjem koraku dobimo
\[
\beta_\Gamma(n) \geq \frac{1}{C} \beta_\Gamma \left(\frac{n}{C}\right)^2,
\]
po $t$ korakih pa dobimo neenakost
\[
\beta_\Gamma(n) \geq \frac{1}{C^{1 + 2 + 4 + \cdots + 2^{t-1}}} \cdot \beta_\Gamma \left( \frac{n}{C^t} \right)^{2^t} \geq \left( \beta_\Gamma \left( \frac{n}{C^t} \right) / C \right)^{2^t}.
\]
Izberimo $t = \lfloor \log_C \frac{n}{m} \rfloor$ za nek dovolj velik $m$, za katerega je $\beta_\Gamma(m) > C$. Tedaj je $C^t < \frac{n}{m}$, zato je $\beta_\Gamma(n/C^t) > \beta_\Gamma(m)$. Iz zadnje neenakosti tako dobimo
\[
\beta_\Gamma(n) \geq \left( \beta_\Gamma(m)/C \right)^{2^t} = D^{2^t}
\]
za neko konstanto $D > 1$. Po izbiri $t$ od tod nazadnje sledi
\[
\beta_\Gamma(n) \geq D^{2^{\log_C{\frac{n}{m}} - 1}}
\geq e^{\frac12 \cdot \log D \cdot 2^{-\log_C m} \cdot n^{\log_C 2}}
\sim e^{n^{\log_C 2}},
\]
torej v izreku lahko vzamemo $q = \log_C 2$. \kljuka
\end{dokaz}

V zvezi s tem omenimo še sodoben rezultat \href{https://link.springer.com/article/10.1007/s00222-019-00922-0}{(Erschler--Zheng 2020)}, v katerem avtorja z opazovanjem {\em roba}\footnote{Poissonov rob grupe $G$, opremljene z verjetnostno porazdelitvijo $\mu$, je množica vseh omejenih $\mu$-harmoničnih funkcij na $G$. Znana je povezava med rastjo grupe in asimptotiko repa porazdelitve $\mu$ s končno entropijo in netrivialnim Poissonovim robom. Obstoj take porazdelitve $\mu$ nam torej pove nekaj o rasti grupe $G$.} grupe $\Gamma$ dokažeta, da velja celo
\[
\lim_{n \to \infty} \log_n \log \beta_\Gamma(n) = p_0,
\]
kjer je $p_0 = \frac{\log 2}{\log \lambda_0} \approx 0.77$, kjer je $\lambda_0$ pozitivna ničla polinoma $X^3 - X^2 - 2X - 4$.

\begin{odprtproblem}
Znano je, da Grigorčukova grupa {\em ni} končno prezentirana. Ali obstaja končno prezentirana grupa srednje rasti?
\end{odprtproblem}

\section{Amenabilnost}

Amenabilnost je grupno-teoretična lastnost, ki ima veliko ekvivalentnih definicij in ki povezuje rešljive grupe, podeksponentno rast in paradoksalne dekompozicije.

\subsection{Obstoj invariantne mere}

Grupa $G$ je {\definicija amenabilna}, če ima netrivialno, končno, končno aditivno in translacijsko invariantno mero. Z drugimi besedami, obstaja funkcija $\mu \colon \PP(G) \to \RR_{\geq 0}$ z naslednjimi lastnostmi:
\begin{itemize}
    \item $\mu(G) > 0$, \quad {\vprasanje (netrivialnost)}
    \item $\forall A,B \subseteq G \colon \ A \cap B = \emptyset \Rightarrow \mu(A \cup B) = \mu(A) + \mu(B)$, \quad {\vprasanje (končna aditivnost)}
    \item $\forall A \subset G \ \forall x \in G \colon \ \mu(A \cdot x) = \mu(A)$. \quad {\vprasanje (desna translacijska invarianca)}
\end{itemize}
Kadar taka mera obstaja, jo lahko vselej normaliziramo in zato predpostavimo $\mu(G) = 1$. Iz desno translacijsko invariantne mere $\mu$ lahko preprosto dobimo levo translacijsko invariantno mero $\nu$, in sicer $\nu(A) = \mu(A^{-1})$ za $A \subseteq G$. {\vprasanje Ali lahko dobimo tudi mero, ki je hkrati levo in desno translacijsko invarianta?}

Amenabilne grupe so torej ravno grupe, v katerih lahko  na smiseln način izmerimo velikosti vseh podmnožic grupe.

\begin{zgled}
\begin{itemize}
    \item Naj bo $G$ končna grupa. Za $A \subseteq G$ definirajmo $\mu(A) = |A|/|G|$. To je {\em edina} mera na $G$, ki zadošča vsem pogojem za amenabilnost in je hkrati normalizirana.

    \item Naj bo $G = \ZZ$. Za $A \subseteq \ZZ$ in $n \in \NN$ definirajmo $A_n = A \cap [-n, n] \subseteq \ZZ$. Izberimo neglavni ultrafilter $\UU$ na $\NN$. Definirajmo $\mu(A) = \lim_\UU |A_n|/n$. S tem dobimo mero, ki zadošča vsem pogojem za amenabilnost. Ta mera pa je daleč od enolične, saj temelji na izbiri ultrafiltra $\UU$.

    \item Naj bo $G = F_2$ prosta grupa ranga $2$ z generatorjema $x,y$. Dokažimo, da $F_2$ {\em ni} amenabilna. Zavoljo prostislovja predpostavimo, da je $\mu$ {\em levo} translacijsko invariantna normalizirana mera, ki opazi amenabilnost $F_2$. Velja $\mu(F_2 \backslash \{ 1 \}) = 1$.\footnote{Če je namreč $\mu(\{ 1 \}) = p > 1$, potem je $\mu(\{ g \}) = p$ za vsak $g \in F_2$ po translacijski invarianci, zato je $\mu(F_2) \geq \sum_g \mu(\{ g \})$, kjer vsota teče po poljubni končni množici. To je protislovno z omejenostjo $\mu(F_2)$.}

    Naj bo $X_{+} \subseteq F_2$ množica okrajšanih besed s prvo črko $x$, in naj bo $X_{-}$ množica okrajšanih besed s prvo črko $x^{-1}$. Sorodno definiramo $Y_{+}, Y_{-}$. Velja $x \cdot Y_+, x \cdot Y_-, x \cdot X_+ \subseteq X_+$ in celo
    \[
    X_+ = \{ x \} \oplus x \cdot Y_+ \oplus x \cdot Y_- \oplus x \cdot X_+.
    \]
    Od tod sledi $\mu(X_+) = \mu(Y_+) + \mu(Y_-) + \mu(X_+)$, zato je $\mu(Y_+) = \mu(Y_-) = 0$. Podobno izpeljemo $\mu(X_+) = \mu(X_-) = 0$. Od tod nazadnje sledi protislovje $\mu(F_2) = 0$.
\end{itemize}
\end{zgled}

    \begin{figure}[t]
    \centering
    \begin{tikzpicture}
    \tikzstyle{vertex}=[circle, fill = oranzna, minimum size = 5pt, inner sep = 0pt]

    \def\distance{2.5}
    \def\factor{2}
    \def\tinyedge{\distance/(\factor * \factor)}

    \draw[thick, dashed, draw = vijolicna, rounded corners=1em] ($(xy) + (-0.7,0.7)$) rectangle ($(xy-1) + (2,-0.7)$);
    \node[label = right : {\color{vijolicna} $X_+$}] at ($(xy-1) + (2,-0.7)$) {};

    \draw[thick, dashed, draw = vijolicna, rounded corners=1em] ($(x-1y) + (0.7,0.7)$) rectangle ($(x-1y-1) + (-2,-0.7)$);
    \node[label = left : {\color{vijolicna} $X_-$}] at ($(x-1y-1) + (-2,-0.7)$) {};

    \draw[thick, dashed, draw = vijolicna, rounded corners=1em] ($(yx-1) + (-0.7,-0.7)$) rectangle ($(yx) + (0.7,2)$);
    \node[label = right : {\color{vijolicna} $Y_+$}] at ($(yx) + (0.7,2)$) {};

    \draw[thick, dashed, draw = vijolicna, rounded corners=1em] ($(y-1x-1) + (-0.7,0.7)$) rectangle ($(y-1x) + (0.7,-2)$);
    \node[label = right : {\color{vijolicna} $Y_-$}] at ($(y-1x) + (0.7,-2)$) {};

    \node[vertex, label = below left : {$1$}] (1) at (0,0) {};

    \node[vertex, label = below left : {$x$}] (x) at (\distance,0) {};
    \draw[line width = 0.25mm] (1) -- (x);
    \node[vertex, label = below left : {$x^{-1}$}] (x-1) at (-\distance,0) {};
    \draw[line width = 0.25mm] (1) -- (x-1);

    % x2
        \draw[line width = 0.25mm] ({\distance * (1 + 1/\factor)},0) -- ({\distance * (1 + 1/\factor)},{\tinyedge});
        \draw[line width = 0.25mm] ({\distance * (1 + 1/\factor)},0) -- ({\distance * (1 + 1/\factor)},{-\tinyedge});
        \draw[line width = 0.25mm] ({\distance * (1 + 1/\factor)},0) -- ({\distance * (1 + 1/\factor) + \tinyedge},0);

    \node[vertex, label = below left : {}] (x2) at ({\distance * (1 + 1/\factor)},0) {};
    \draw[line width = 0.25mm] (x) -- (x2);

    % x-2
        \draw[line width = 0.25mm] ({-\distance * (1 + 1/\factor)},0) -- ({-\distance * (1 + 1/\factor)},{-\tinyedge});
        \draw[line width = 0.25mm] ({-\distance * (1 + 1/\factor)},0) -- ({-\distance * (1 + 1/\factor)},{\tinyedge});
        \draw[line width = 0.25mm] ({-\distance * (1 + 1/\factor)},0) -- ({-\distance * (1 + 1/\factor) - \tinyedge},0);

    \node[vertex, label = below left : {}] (x-2) at ({-\distance * (1 + 1/\factor)},0) {};
    \draw[line width = 0.25mm] (x-1) -- (x-2);

    \node[vertex, label = below left : {$y$}] (y) at (0,\distance) {};
    \draw[line width = 0.25mm] (1) -- (y);
    \node[vertex, label = below left : {$y^{-1}$}] (y-1) at (0,-\distance) {};    
    \draw[line width = 0.25mm] (1) -- (y-1);

    %y2
        \draw[line width = 0.25mm] (0,{\distance * (1 + 1/\factor)}) -- (0,{\distance * (1 + 1/\factor) + \tinyedge});
        \draw[line width = 0.25mm] (0,{\distance * (1 + 1/\factor)}) -- ({\tinyedge},{\distance * (1 + 1/\factor)});
        \draw[line width = 0.25mm] (0,{\distance * (1 + 1/\factor)}) -- ({-\tinyedge},{\distance * (1 + 1/\factor)});

    \node[vertex, label = below left : {}] (y2) at (0,{\distance * (1 + 1/\factor)}) {};
    \draw[line width = 0.25mm] (y) -- (y2);

    % y-2
        \draw[line width = 0.25mm] (0,{-\distance * (1 + 1/\factor)}) -- ({-\tinyedge},{-\distance * (1 + 1/\factor)});
        \draw[line width = 0.25mm] (0,{-\distance * (1 + 1/\factor)}) -- ({\tinyedge},{-\distance * (1 + 1/\factor)});
        \draw[line width = 0.25mm] (0,{-\distance * (1 + 1/\factor)}) -- (0,{-\distance * (1 + 1/\factor) - \tinyedge});

    \node[vertex, label = below left : {}] (y-2) at (0,{-\distance * (1 + 1/\factor)}) {};
    \draw[line width = 0.25mm] (y-1) -- (y-2);

    % xy

    \node[vertex, label = below left : {}] (xy) at (\distance,\distance * 1/\factor) {};
    \draw[line width = 0.25mm] (x) -- (xy);
        \draw[line width = 0.25mm] (xy) -- ($(xy) + (0,{\tinyedge})$);
        \draw[line width = 0.25mm] (xy) -- ($(xy) + ({\tinyedge},0)$);
        \draw[line width = 0.25mm] (xy) -- ($(xy) + ({-\tinyedge},0)$);

    % xy-1
    \node[vertex, label = below left : {}] (xy-1) at (\distance,-\distance * 1/\factor) {};
    \draw[line width = 0.25mm] (x) -- (xy-1);
        \draw[line width = 0.25mm] (xy-1) -- ($(xy-1) + (0,{-\tinyedge})$);
        \draw[line width = 0.25mm] (xy-1) -- ($(xy-1) + ({\tinyedge},0)$);
        \draw[line width = 0.25mm] (xy-1) -- ($(xy-1) + ({-\tinyedge},0)$);

    % x-1y
    \node[vertex, label = below left : {}] (x-1y) at (-\distance,\distance * 1/\factor) {};
    \draw[line width = 0.25mm] (x-1) -- (x-1y);
        \draw[line width = 0.25mm] (x-1y) -- ($(x-1y) + (0,{\tinyedge})$);
        \draw[line width = 0.25mm] (x-1y) -- ($(x-1y) + ({\tinyedge},0)$);
        \draw[line width = 0.25mm] (x-1y) -- ($(x-1y) + ({-\tinyedge},0)$);

    % x-1y-1
    \node[vertex, label = below left : {}] (x-1y-1) at (-\distance,-\distance * 1/\factor) {};
    \draw[line width = 0.25mm] (x-1) -- (x-1y-1);
        \draw[line width = 0.25mm] (x-1y-1) -- ($(x-1y-1) + (0,{-\tinyedge})$);
        \draw[line width = 0.25mm] (x-1y-1) -- ($(x-1y-1) + ({\tinyedge},0)$);
        \draw[line width = 0.25mm] (x-1y-1) -- ($(x-1y-1) + ({-\tinyedge},0)$);

    % yx
    \node[vertex, label = below left : {}] (yx) at (\distance * 1/\factor,\distance) {};
    \draw[line width = 0.25mm] (y) -- (yx);
        \draw[line width = 0.25mm] (yx) -- ($(yx) + (0,{\tinyedge})$);
        \draw[line width = 0.25mm] (yx) -- ($(yx) + (0,{-\tinyedge})$);
        \draw[line width = 0.25mm] (yx) -- ($(yx) + ({\tinyedge},0)$);

    % yx-1
    \node[vertex, label = below left : {}] (yx-1) at (-\distance * 1/\factor,\distance) {};
    \draw[line width = 0.25mm] (y) -- (yx-1);
        \draw[line width = 0.25mm] (yx-1) -- ($(yx-1) + (0,{\tinyedge})$);
        \draw[line width = 0.25mm] (yx-1) -- ($(yx-1) + (0,{-\tinyedge})$);
        \draw[line width = 0.25mm] (yx-1) -- ($(yx-1) + ({-\tinyedge},0)$);

    % y-1x
    \node[vertex, label = below left : {}] (y-1x) at (\distance * 1/\factor,-\distance) {};    
    \draw[line width = 0.25mm] (y-1) -- (y-1x);
        \draw[line width = 0.25mm] (y-1x) -- ($(y-1x) + (0,{\tinyedge})$);
        \draw[line width = 0.25mm] (y-1x) -- ($(y-1x) + (0,{-\tinyedge})$);
        \draw[line width = 0.25mm] (y-1x) -- ($(y-1x) + ({\tinyedge},0)$);

    % y-1x-1
    \node[vertex, label = below left : {}] (y-1x-1) at (-\distance * 1/\factor,-\distance) {}; 
    \draw[line width = 0.25mm] (y-1) -- (y-1x-1);
        \draw[line width = 0.25mm] (y-1x-1) -- ($(y-1x-1) + (0,{\tinyedge})$);
        \draw[line width = 0.25mm] (y-1x-1) -- ($(y-1x-1) + (0,{-\tinyedge})$);
        \draw[line width = 0.25mm] (y-1x-1) -- ($(y-1x-1) + ({-\tinyedge},0)$);
    \end{tikzpicture}  
    \caption{Razčlenitev netrivialnih besed v $F_2$}
    \end{figure}

\subsection{Integral po meri}

Na amenabilnih grupah lahko, kot bomo pojasnili, {\em integriramo} funkcije s pomočjo mere $\mu$. 

Naj bo $f \colon G \to \RR$ omejena funkcija, se pravi $a \leq f(x) < b$ za vsak $x \in G$. Razdelimo interval $[a,b]$ na podintervale z izbiro delilnih točk $a = a_0 < a_1 < \cdots < a_k = b$. Označimo to izbiro delilnih točk z $\Delta$. Naj bo
\[
A_i = \{ x \in G \mid a_{i-1} \leq f(x) < a_i \}.
\]
Velja $G = \bigcup_{i = 1}^k A_i$. 

Funkcijo $f$ lahko {\em stopničasto aproksimiramo} s funkcijami, ki so konstantne na vsaki od množic $A_i$ razčlenitve $G = \bigcup_{i = 1}^k A_i$. Za vsako od takih aproksimacij $g \colon G \to [a,b]$ lahko izračunamo njeno {\em delno vsoto} $\sum_{i = 1}^k \mu(A_i) \cdot g(A_i)$.
Za definicijo vrednosti integrala bosta posebej pomembni delni vsoti
\[
S_\Delta = \sum_{i = 1}^k \mu(A_i) \cdot a_i \quad
\text{in} \quad
s_\Delta = \sum_{i = 1}^k \mu(A_i) \cdot a_{i-1}.
\]
Jasno velja $a \leq s_\Delta, S_\Delta \leq b$ in $S_\Delta - s_\Delta \leq \max_i (a_i - a_{i-1})$.

Če je $\Delta^\prime$ finejša izbira delilnih točk kot $\Delta$, potem je $S_{\Delta^\prime} \leq S_\Delta$ in $s_{\Delta^\prime} \geq s_\Delta$. Ko torej izbiramo vedno bolj fine delitve,\footnote{Pri tem mislimo, da izbiramo delitve, pri katerih gre $\max_i (a_i - a_{i-1})$ proti $0$. Vsaki dve delitvi imata seveda skupno pofinitev.} vrednosti $S_\Delta$ in $s_\Delta$ konvergirata k skupni vrednosti $\inf_\Delta S_\Delta = \sup_\Delta s_\Delta$. To vrednost imenujemo {\definicija integral} funkcije $f$ po grupi $G$, označimo pa kot $\int_G f \ d \mu$ ali včasih tudi kot $\int_G f(x) \ d \mu(x)$.

\begin{zgled}
Naj bo $G$ končna grupa, opremljena z mero $\mu(A) = |A|/|G|$ za $A \subseteq G$. Naj bo $f \colon G \to \RR$ funkcija. Tedaj je $\int_G f \ d \mu = \sum_{x \in G} f(x) \cdot \mu(x) = \frac{1}{|G|} \sum_{x \in G} f(x)$, to je povprečje $f$. 
\end{zgled}

V posebnem lahko s pomočjo integrala povprečimo desno-invariantno mero, ki opazi amenabilnost grupe, in dobimo obojestransko invarianco.

\begin{trditev}
Bodi $G$ amenabilna. Potem ima $G$ mero, ki opazi njeno amenabilnost in je hkrati levo in desno translacijsko invariantna.
\end{trditev}

\begin{dokaz}
Naj bo $\mu$ desno translacijsko invariantna mera, ki opazi amenabilnost $G$. Definirajmo novo mero $\nu$ s predpisom $\nu(A) = \int_G \mu(x \cdot A) \ d\mu(x)$. Trdimo, da je $\nu$ iskana mera. Preverimo le, da je $\nu$ levo in desno translacijsko invariantna. Z uporabo desne invariantnosti mere $\mu$ izpeljemo, da za vsak $y \in G$ velja
\begin{align*}
\nu(y \cdot A) &= \int_G \mu(x \cdot y \cdot A) \ d \mu(x) \\
&= \inf_\Delta \sum_i \mu(\{ x \in G \mid a_{i-1} \leq \mu(x \cdot y \cdot A) < a_i \}) a_i \\
&= \inf_\Delta \sum_i \mu(\{ x \in G \mid a_{i-1} \leq \mu(x \cdot A) < a_i \} \cdot y^{-1}) a_i \\
&= \inf_\Delta \sum_i \mu(\{ x \in G \mid a_{i-1} \leq \mu(x \cdot A) < a_i \}) a_i \\
&= \int_G \mu(x\cdot A) \ d \mu(x) = \nu(A)
\end{align*}
in 
\[
\nu(A \cdot y) = \int_G \mu(x \cdot A \cdot y) \ d \mu(x) = \int_G \mu(x \cdot A) \ d\mu(x) = \nu(A).
\]
\end{dokaz}

\subsection{Osnovne lastnosti amenabilnosti}

Amenabilnost je zaprta za vse osnovne konstrukcije grup.

\begin{izrek}
Podgrupe in kvocienti amenabilnih grup so amenabilni. Razširitev amenabilne grupe z amenabilno je amenabilna. Končne in abelove grupe so amenabilne. 
\end{izrek}

\begin{dokaz}
\underline{Zaprtost za podgrupe:} Naj bo $H \leq G$, kjer je $G$ amenabilna. Izberimo neko množico predstavnikov odsekov $H$ v $G$, recimo ji $R$. Naj bo $A \subseteq H$. Definirajmo $\mu_H(A) = \mu(R \cdot A)$. Na ta način dobimo želeno mero na $H$. \kljuka

\underline{Zaprtost za kvociente:} Naj bo $N \unlhd G$, kjer je $G$ amenabilna. Naj bo $A \subseteq G/N$, torej $A = \{ x_{\alpha}N \}_{\alpha}$. Definirajmo $\mu_{G/N}(A) = \mu(\bigcup_{\alpha} x_\alpha N)$. Na ta način dobimo želeno mero na $G/N$. \kljuka

\underline{Zaprtost za razširitve:} Naj bo $N \unlhd G$, kjer sta $N$ in $G/N$ amenabilni z merama $\nu$ in $\sigma$. Naj bo $A \subseteq G$. Definirajmo $f_A \colon G/N \to \RR$, $xN \mapsto \nu(N \cap xA)$. Definirajmo $\mu(A) = \int_{G/N} f_A \ d \sigma$. Na ta način dobimo želeno mero na $G$. Preverimo le levo translacijsko invariantnost. Za $y \in G$ velja\footnote{Kot v dokazu prejšnje trditve izpeljemo, da je $\int_{G/N} g(xy) \ d \sigma(x) = \int_{G/N} g(x) \ d \sigma(x)$ za vsako funkcijo $g \colon G/N \to \RR$.}
\begin{align*}
\mu(yA) &= \int_{G/N} f_{yA}(xN) \ d \sigma(xN) \\
&= \int_{G/N} \nu(N \cap xyA) \ d \sigma(xN) \\
&= \int_{G/N} \nu(N \cap xA) \ d \sigma(xN) \\
&= \int_{G/N} f_A(xN) \ d \sigma(xN) = \mu(A).
\end{align*}

\underline{Končne in abelove grupe:} Končne grupe so jasno opremljene z ustrezno mero. Prav tako velja za grupo $\ZZ$. Od tod in že dokazanih zaprtosti amenabilnosti sledi, da so končno generirane abelove grupe tudi amenabilne. Za neskončno generirane grupe se moramo malo bolj potruditi.

\begin{trditev}
Naj bo $G$ grupa. Če je vsaka končno generirana podgrupa grupe $G$ amenabilna, potem je tudi $G$ amenabilna.
\end{trditev}

\begin{dokaz}[Dokaz pomožne trditve]
Opazujmo prostor funkcij $F = [0,1]^{\PP(G)}$, opremljen s produktno topologijo. Končno aditivne normalizirane invariantne mere na $G$ tvorijo zaprto podmnožico v $F$. Naj bo $H \leq G$ končno generirana podgrupa z mero $\mu_H$. Mero razširimo na $G$ z definicijo $\mu(A) = \mu_H(A \cap H)$ za $A \subseteq G$. Ta razširitev je končno aditivna in normalizirana, translacijsko invariantna pa je le glede na elemente v $H$. Tudi množica vseh takih mer $F_H$ je zaprta v prostoru $F$ in vemo, da je celo neprazna. Denimo, da je presek vseh množic $F_H$, ko $H$ preteče končno generirane podgrupe grupe $G$, prazen. Zaradi kompaktnosti $F$ od tod sledi, da obstajajo $H_1, \dots, H_k$ končno generirane podgrupe grupe $G$, za katere je $F_{H_1} \cap \cdots \cap F_{H_k} = \emptyset$. Toda množice $F_{H_i}$ vsebujejo $F_K$, kjer je $K = \langle H_1, \dots, H_k \rangle$, torej je njihov presek neprazen. Protislovje. Torej res najdemo mero $\mu$ v preseku vseh $F_H$, ki je zato $G$-invariantna. \kljuka
\end{dokaz}

\end{dokaz}

Iz izreka neposredno sledi, da so rešljive grupe amenabilne. Torej je amenabilnost posplošitev rešljivosti.

\begin{zgled}
Heisenbergova grupa in svetilkarjeva grupa sta amenabilni.
\end{zgled}

\subsection{Følnerjeva zaporedja}

Naj bo $G = \langle S \rangle$. Za množico $A \subseteq G$ definiramo njen {\definicija rob} kot
\[
\partial A = \{ x \in G \mid \min_{a \in A} d_S(x,a) = 1 \}.
\]

\begin{zgled}
V grupi $\ZZ^2 = \langle (0,1), (1,0) \rangle$ naj bo $A = \{ (0,0), (0,1), (1,0) \}$. Rob te množice je $\partial A = \{ (-1, 0), (-1, 1), (0, 2), (1, 1), (2, 0), (1, -1), (0, -1) \}$.
\end{zgled}

\begin{figure}
\centering
\begin{tikzpicture}[scale = 1.3]
    \tikzstyle{vertex}=[circle, fill = oranzna, minimum size = 5pt, inner sep = 0pt]

    \foreach \n in {-2,...,2}
        \foreach \m in {-1,...,2}
            \draw[thick] (\n, \m) to (\n + 1, \m);

    \foreach \n in {-1,...,2}
        \foreach \m in {-2,...,2}
            \draw[thick] (\n, \m) to (\n, \m + 1);

    \foreach \n in {-1,...,2}
        \foreach \m in {-1,...,2}
            \node[vertex, label = -100 : {$(\n , \m)$}] at (\n, \m) {};

    \foreach \t in {(0,0), (0,1), (1,0)}
        \node[vertex, fill = vijolicna, minimum size = 10pt] at \t {};
    \foreach \t in {(-1, 0), (-1, 1), (0, 2), (1, 1), (2, 0), (1, -1), (0, -1)}
        \node[vertex, fill = rdeca, minimum size = 10pt] at \t {};
\end{tikzpicture}
\caption{Rob množice $A = \{ (0,0), (0,1), (1,0) \} \subseteq \ZZ^2$}
\end{figure}

Rečemo, da grupa $G$ zadošča {\definicija Følnerjevemu pogoju}, če obstaja zaporedje $\{ X_n \}_{n = 1}^\infty$ končnih podmnožic grupe $G$ z lastnostjo $\lim_{n \to \infty} |\partial X_n|/|X_n| = 0$.\footnote{Rob množic $X_n$ je torej v limiti relativno gledano zanemarljiv. Taka grupa je torej {\em počasne} rasti. {
\em Ortogonalno} definiciji Følnerjevega pogoja so grupe $G$, za katere velja $\inf_X |\partial X|/|X| > 0$, kjer $X$ preteče vse končne podmnožice grupe $G$. Take grupe so {\definicija ekspanzivne} in so eksponentne rasti.}

\begin{zgled}
Opazujmo grupo $\ZZ^2$. Naj bo $X_n = \{ (x,y) \in \ZZ^2 \mid |x|, |y| \leq n \}$. Velja $|\partial X_n| = 4(2n+1)$ in $|X_n| = (2n+1)^2$, zato je $\lim_{n \to \infty} |\partial X_n|/|X_n| = 0$. Grupa $\ZZ^2$ torej ima Følnerjevo zaporednje.
\end{zgled}

\begin{izrek}
Končno generirana grupa zadošča Følnerjevemu pogoju, če in samo če je amenabilna.
\end{izrek}

\begin{dokaz}
Dokazali bomo le implikacijo iz leve v desno. Za dokaz obratne implikacije bi potrebovali kar nekaj funkcionalne analize, zato ga izpustimo. 

Naj bo $X_n$ Følnerjevo zaporedje v $G$. Amenabilnost dokažemo kot v zgledu grupe $\ZZ$. Izberimo neglavni ultrafilter $\UU$ na $\NN$. Za $A \subseteq G$ naj bo $A_n = A \cap X_n$. Definirajmo $\mu(A) = \lim_\UU |A_n|/|X_n|$. To je želena mera na $G$. Preverimo le desno translacijsko invarianco z elementom $s \in S$. Velja
$|As \cap X_n| = |A \cap X_n s^{-1}|$ in $A \cap X_n s^{-1} \subseteq A_n \cup \partial X_n$,\footnote{Če $a \in A \cap X_n s^{-1}$ ni v $\partial X_n$, potem je $a \in X_n$, zato je $a \in A_n$.} od koder sledi
\[
\frac{|As \cap X_n|}{|X_n|} \leq \frac{|A_n|}{|X_n|} + \frac{|\partial X_n|}{|X_n|}.
\]
V limiti je zato $\mu(As) \leq \mu(A) + 0$. Od tod zdaj izpeljemo še drugo neenakost $\mu(A) = \mu(Ass^{-1}) \leq \mu(As)$. S tem je res $\mu(As) = \mu(A)$.
\end{dokaz}
 
Zabeležimo pomembno posledico, ki natančneje poveže koncept rasti z amenabilnostjo.

\begin{posledica}
Grupe podeksponentne rasti so amenabilne.
\end{posledica}

\begin{dokaz}
Opazujmo množice $X_n = B_{G,S}(n)$. Velja $|X_n| = \beta_{G,S}(n)$ in $|\partial X_n| = \beta_{G,S}(n+1) - \beta_{G,S}(n)$. Ker je $G$ podeksponentne rasti, velja
\[
1 = \lim_{n \to \infty} \beta_{G,S}(n)^{1/n} \geq\footnote{Konvergenčni radij vrste $\sum_n \beta_{G,S}(n) x^n$ je enak $1$.} \liminf_{n \to \infty} \frac{\beta_{G,S}(n+1)}{\beta_{G,S}(n)} \geq 1,
\]
zato je $\liminf_{n \to \infty} |\partial X_n|/|X_n| = 1 - 1 = 0$. Torej ima $\{ X_n \}_{n \in \NN}$ Følnerjevo podzaporedje.
\end{dokaz}

\begin{zgled}
Grigorčukova grupa $\Gamma$ in svetilkarjeva grupa $L$ sta amenabilni. Iz njiju lahko sestavimo še posebej zanimivo grupo $\left( \bigoplus_{i \in \Gamma} \ZZ/2\ZZ \right) \rtimes \Gamma$,\footnote{To grupo označimo krajše kot $\ZZ/2\ZZ \wreath \Gamma$ in jo imenujemo {\definicija venčni produkt} grup $\ZZ/2\ZZ$ in $\Gamma$, angleško {\em wreath product}.} ki {\em ni} rešljiva (ima nerešljivi kvocient $\Gamma$\footnote{Grigorčukova grupa je končno generirana in srednje rasti, zato po dihotomiji med eksponentno in polinomsko rastjo rešljivih grup ni rešljiva.}), je eksponentne rasti (premislimo podobno kot za grupo $L$) in je amenabilna (razširitev amenabilne z amenabilno).
\end{zgled}

\begin{figure}[t]
\centering
\begin{tikzpicture}
\tikzstyle{vertex}=[circle, fill = oranzna, minimum size = 5pt, inner sep = 0pt]

\draw[line width = 3pt, draw = modra, fill = modra!10, opacity = 0.5] (0,0.6) circle (2.8);
\draw[thick, line width = 3pt, draw = rdeca, fill = rdeca!10, opacity = 0.5] (0.55,1.05) circle (2);
\draw[line width = 3pt, draw = vijolicna, fill = vijolicna!10, opacity = 0.5] (0,0) circle (2);


\node[vertex,label = below : $\color{zelena} \ZZ$] at (-0.20,-0.15) {};
% \draw[thick] (0,0) circle (1.2);
% \draw[thick] (0,0) circle (0.75);
% \draw[thick] (0,0) circle (2);
\node[vertex, label = left : $\color{zelena} \Gamma$] at (0.9,-1.3) {};

\node[vertex, label = below : {$\color{zelena} \ZZ/2\ZZ \wreath \Gamma$}] at (-2.6,0.9) {};

% \node[label = below : $\vprasanje \mathbf{?}$] at (0.4,-0.95) {};
\draw[thick] (0,0) circle (3.5);
\node[vertex,label = below : $\color{zelena} F_d$] at (3,0) {};
\node[vertex,label = below : $\color{zelena} \SL_2(\ZZ)$] at (0,-2.5) {};


\draw[->,thick,draw = vijolicna] (-3.5, -2) to [bend left = 30] (-1.5, -0.5);
\node[label = below : {\color{vijolicna} podeksponentne rasti}] at (-4.5, -2) {};

\draw[->,thick,draw = modra] (-3.5, 2) to [bend left = 30] (-2, 1.5);
\node[label = left : {\color{modra} amenabilne}] at (-3.5, 2) {};

% \draw[->,thick,draw = rdeca] (-3.5, -4) to [bend right = 30] (-0.55, -0.85);
% \node[label = below : {\color{rdeca} virtualno nilpotentne}] at (-4, -4) {};

% \draw[->,thick,draw = rdeca] (-3.5, 3) to [bend right = 30] (-0.35, 0.35);
% \node[label = above : {\color{rdeca} nilpotentne}] at (-3.5, 3) {};
\node[vertex,label = below : $\color{zelena} H$] at (0.20, 0.15) {};

% \draw[->,thick,draw = vijolicna] (-0.5,-4) to [bend left = 10] (0, -1.75);
% \node[label = below : {\color{vijolicna} podeksponentne rasti}] at (-0.5,-4) {};

\draw[->,thick,draw = vijolicna] (2.5, -3) to [bend right = 10] (2.25, -2);
\node[label = below right : {\color{vijolicna} eksponentne rasti}] at (2, -3) {};

\draw[->, thick, draw = rdeca] (3.5, 2) to [bend right = 10] (2, 2);
\node[label = right : {\color{rdeca} rešljive}] at (3.5, 2) {};

% \draw[thick] (135:2) .. controls (0,3) and (3, 3) .. (-10:2); 
% \draw[->, thick, draw = rdeca] (3.75, 0.5) to [bend right = 10] (1.75, 1.5);
% \node[label = right : {\color{rdeca} policiklične}] at (3.75, 0.5) {};

\node[vertex,label = right : $\color{zelena} L$] at (1,2.25) {};
% \node[vertex,label = right : $\color{zelena} \mathcal{O}$] at (1, 2) {};
\end{tikzpicture}
\caption{Diagram amenabilnosti, rešljivosti in podeksponentne rasti}
\end{figure}

\subsection{Paradoksalne dekompozicije}

Grupa $G$ je {\definicija paradoksalna}, če obstajajo množice $A_1, \dots, A_n, B_1, \dots, B_m \subseteq G$ in elementi $x_1, \dots, x_n, y_1, \dots, y_m \in G$ z lastnostjo
\[
G = \bigoplus_{i=1}^n A_i \oplus \bigoplus_{j = 1}^m B_j, \quad 
G = \bigoplus_{i=1}^n x_i A_i = \bigoplus_{j = 1}^m y_j B_j.
\]

\begin{zgled}
Trdimo, da je prosta grupa $F_2$ paradoksalna. Za to le malo popravimo množice, ki so opazile, da $F_2$ ni amenabilna. Naj bo $X_+ \subseteq F_2$ množica okrajšanih besed s prvo črko $x$, in naj bo $X_-$ množica okrajšanih besed s prvo črko $x^{-1}$. Naj bo $Y_+$ množica okrajšanih besed s prvo črko $y$ skupaj z nepozitivnimi potencami $y$. Naj bo $Y_- = F_2 \backslash (X_+ \cup X_- \cup Y_+)$. Velja $x^{-1}\cdot X_+ \oplus X_- = F_2$ in $y^{-1} \cdot Y_+ \oplus Y_- = F_2$. Torej je $F_2$ res paradoksalna.
\end{zgled}

Paradoksalne grupe torej obstajajo in v njih ni nič protislovnega. Težave se pojavijo, če želimo {\em meriti} velikosti množic v paradoksalni dekompoziciji.

\begin{izrek}{\literatura (Tarski 1949)}
Grupa je paradoksalna, če in samo če {\em ni} amenabilna.
\end{izrek}

\begin{dokaz}
Dokazali bomo le implikacijo iz leve v desno, ki pojasni, da so paradoksalne grupe res protislovne v primeru, ko lahko merimo velikosti podmnožico z neko mero.

Naj bo $G$ paradoksalna grupa, torej
\[
G = \bigoplus_{i=1}^n A_i \oplus \bigoplus_{j = 1}^m B_j, \quad 
G = \bigoplus_{i=1}^n x_i A_i = \bigoplus_{j = 1}^m y_j B_j.
\]
Predpostavimo, da je $G$ amenabilna z normalizirano mero $\mu$. Velja
\[
1 = \mu(G) = \sum_{i = 1}^n \mu(A_i) + \sum_{j = 1}^m \mu(B_j)
\]
in hkrati
\[
1 = \mu(G) = \sum_{i = 1}^n \mu(x_i A_i) = \sum_{i = 1}^n \mu(A_i),
\quad
1 = \mu(G) =  \sum_{j = 1}^m \mu(y_j B_j) = \sum_{j = 1}^m \mu(B_j),
\]
od koder izpeljemo protislovje $1 = 1+1$.
\end{dokaz}

Delovanja paradoksalnih grup vodijo v paradoksalne dekompozicije množic. Naj grupa $G$ deluje na množici $X$. Podmnožica $Y \subseteq X$ je {\definicija $G$-paradoksalna}, če obstajajo množice $A_1, \dots, A_n, B_1, \dots, B_m \subseteq Y$ in elementi $x_1, \dots, x_n, y_1, \dots, y_m \in G$ z lastnostjo
\[
Y = \bigoplus_{i=1}^n A_i \oplus \bigoplus_{j = 1}^m B_j, \quad 
Y = \bigoplus_{i=1}^n x_i.A_i = \bigoplus_{j = 1}^m y_j.B_j.
\]

\begin{trditev}
Naj paradoksalna grupa $G$ deluje {\em prosto} na množici $X$. Množica $X$ je tedaj $G$-paradoksalna.
\end{trditev}

\begin{dokaz}
Naj bo $G$ paradoksalna grupa, torej
\[
G = \bigoplus_{i=1}^n A_i \oplus \bigoplus_{j = 1}^m B_j, \quad 
G = \bigoplus_{i=1}^n x_i A_i = \bigoplus_{j = 1}^m y_j B_j.
\]
Izberimo en element iz vsake orbite delovanja $G$ na $X$, vsi ti predstavniki naj tvorijo množico $R$. Tedaj zaradi prostosti delovanja velja
\[
X = \bigoplus_{i = 1}^n A_i.R \oplus \bigoplus_{j = 1}^m B_j.R = \bigoplus_{i = 1}^n x_iA_i . R = \bigoplus_{j = 1}^m y_jB_j . R,
\]
torej je $X$ res $G$-paradoksalna.
\end{dokaz}

\begin{zgled}(Hausdorffov paradoks)\footnote{Hausdorffov paradoks je soroden bolj znanemu paradoksu \href{http://matwbn.icm.edu.pl/ksiazki/fm/fm6/fm6127.pdf}{(Banach-Tarski 1924)}.}
Opazujmo delovanje grupe $SO(3)$ na prostoru $\RR^3$ z običajnim množenjem matrik z vektorji. Na ta način dobimo inducirano delovanje $SO(3)$ na sferi $S^2$.

Grupa $SO(3)$ vsebuje prosto grupo $F_2$: z uporabo izreka {\sc (pingpong)} ni težko dokazati \vaje, da je
\[
G := \left\langle
\begin{pmatrix}
\frac35 & \frac45 & 0 \\
 -\frac45 & \frac35 & 0 \\
 0 & 0 & 1
\end{pmatrix}, \
\begin{pmatrix}
1 & 0 & 0 \\
0 & \frac35 & -\frac45 \\
0 & \frac45 & \frac35
\end{pmatrix}
\right\rangle \cong F_2.
\]
Grupa $G$ je torej paradoksalna. Opazujmo njeno delovanje na $S^2$. To {\em ni} prosto delovanje, saj je vsak element $1 \neq g \in SO(3)$ rotacija in ima natanko dve fiksni točki na $S^2$. Naj bo
\[
D = \bigcup_{g \in G \backslash \{ 1 \}} (S^2)^g \subseteq S^2.
\]
Ker je $G$ končno generirana grupa, je števna, zato je tudi $D$ števna množica. Grupa $G$ deluje na $S^2 \backslash D$.\footnote{Če je namreč $x \in S^2 \backslash D$ in $g \in G$ z lastnostjo $g.x \in D$, potem obstaja $h \in G$, da je $h.g.x = g.x$, zato je $x \in (S^2)^{g^{-1}hg}$, torej je $x \in D$. Protislovje.} To delovanje je prosto. Iz zadnje trditve zato sledi, da je $S^2 \backslash D$ paradoksalna glede na delovanje $G$.

Od tod izpeljemo, da {\em ne} obstaja števno aditivna funkcija $\mu$ na $\PP(S^2)$, za katero je $\mu(S^2) = 1$ in $\mu(g \cdot A ) = \mu(A)$ za vsaka $A \subseteq S^2, g \in SO(3)$.\footnote{Za tako funkcijo bi namreč zaradi translacijske invariance nujno veljalo $\mu(d) = 0$ za vsak $d \in D$, torej po števni aditivnosti velja $\mu(D) = 0$ in zato taka funkcija inducira normalizirano funkcijo na $S^2 \backslash D$, ki kot v dokazu izreka {\literatura (Tarski 1949)} vodi v protislovje.} {\em Vsem} podmnožicam sfere $S^2$ torej {\em ne} moremo na smiseln način prirediti njihove {\em velikosti} oziroma {\em mere}. Ta rezultat predstavlja začetno točko {\em teorije mere}.
\end{zgled}

\chapter{Eksponentna rast}


V tem poglavju bomo preleteli nekaj znanih rezultatov glede enakomernosti eksponentne rasti. Podali bomo tudi primer grupe eksponentne rasti, ki ni enakomerno eksponentne rasti.

\section{Enakomerno eksponentna rast}

Spomnimo se, za da grupo $G = \langle S \rangle$ eksponentnost rasti merimo z
\[
\omega_S(G) = \lim_{n \to \infty} \sqrt[n]{\beta_{G,S}(n)}
\quad
\text{in}
\quad
\Omega(G) = \inf_{S \subseteq G, \ \langle S \rangle = G} \omega_S(G).
\]
Vemo že, da za prosto grupo $F_2$ velja $\Omega(F_2) \geq 3$, torej je ta grupa enakomerno eksponentne rasti. V tem razdelku si bomo pogledali, da so v določenih naravnih družinah grup vse grupe eksponentne rasti avtomatično tudi enakomerno eksponentne rasti.

\subsection{Rešljive grupe}

\begin{izrek}\href{https://citeseerx.ist.psu.edu/viewdoc/download?doi=10.1.1.520.8970&rep=rep1&type=pdf}{(Osin 2003)}
Rešljiva grupa eksponentne rasti je enakomerno eksponentne rasti.
\end{izrek}

Dokaz je sicer elementaren, a zahteven. Mi si bomo tu pogledali le en značilen primer izreka,\footnote{Poseben primer tega primera smo že videli v razdelku o policikličnih grupah.} ki je v resnici zelo pomembna sestavina Osinovega dokaza. 

\begin{zgled}
Opazujmo rešljivo grupo $G = \ZZ^d \rtimes_\varphi \ZZ$, kjer je
\[
\varphi \colon \ZZ \to \Aut(\ZZ^d) = \GL_d(\ZZ),
\quad
1 \mapsto A \in \GL_d(\ZZ).
\]
{\em Predpostavimo}, da je $\det(A) = \pm 1$ in da je karakteristični polinom matrike $A$ nerazcepen. Naj bo $\Spec(A)$ množica lastnih vrednosti matrike $A$.\footnote{Vse lastne vrednosti so med sabo različne, ker je karakteristični polinom enostaven.} Označimo
\[
\Lambda = \max \{ |\lambda| \mid \lambda \in \Spec(A) \}.
\]
Ker je po predpostavki o determinanti produkt vseh absolutnih vrednosti lastnih vrednosti matrike $A$ enak $1$, je $\Lambda \geq 1$. {\em Predpostavimo}, da je $\Lambda > 1$.\footnote{Če je $\Lambda = 1$, potem ni težko premisliti, da je $A$ nujno končnega reda.} Trdimo, da tedaj velja neenakost
\[
\Omega(G) \geq 2^{\frac{1}{2 \log_\Lambda 2 + 4}}.
\]
Torej ima grupa $G$ enakomerno eksponentno rast in spodnja meja je odvisna le od $\Lambda$.

K dokazu neenakosti pristopimo kot v posebnem primeru $A = \left( \begin{smallmatrix} 2 & 1 \\ 1 & 1 \end{smallmatrix} \right)$, ki smo ga že obravnavali. Naj bo $G = \langle S \rangle$ za poljubno končno množico $S$. Izberimo $s \in S$ z lastnostjo $m := \pi(s) > 0$, kjer je $\pi \colon G \to G/\ZZ^d \cong \ZZ$ naravna projekcija. Izberimo še $t \in S$ z lastnostjo $u := [s,t] \in \ZZ^d \backslash \{ 0 \} \subseteq G$. Velja torej $s u s^{-1} = A^m(u) \in \ZZ^d$.\footnote{V posebnem primeru smo imeli $u = e_1 \in \ZZ^2$.}

Naj bo $N = \lfloor \log_\Lambda 2 \rfloor$. Oglejmo si elemente grupe $G$ oblike
\[
g_{\epsilon_0, \dots, \epsilon_n} = \sum_{j = 0}^n A^{jNm}(u^{\epsilon_j}) \in \ZZ^d \leq G
\]
za $n \geq 1$ in $\epsilon_i \in \{0,1\}$. Za fiksno vrednost $n$ so vsi ti elementi različni med sabo.\footnote{Kot v dokazu rasti Osinove grupe definiramo linearno formo $L \colon \CC^d \to \CC$ z lastnostjo, da za vsak $v \in \CC^2$ velja $LAv = \lambda \cdot Lv$, kjer je $\lambda \in \Spec(A)$ z lastnostjo $|\lambda| = \Lambda$. To formo uporabimo na elementih $g_{\epsilon_0, \dots, \epsilon_n}$.} Torej je število opazovanih elementov enako $2^{n+1}$. Hkrati lahko vsak tak element zapišemo v multiplikativni obliki
\[
g_{\epsilon_0, \dots, \epsilon_n} = \prod_{j = 0}^n s^{jN} u^{\epsilon_j} s^{-jN}
= u^{\epsilon_0} s^N u^{\epsilon_1} s^N \cdots s^N u^{\epsilon_n} s^{-nN}.
\]
Velja torej
\[
\ell_S(g_{\epsilon_0, \dots, \epsilon_n}) \leq (n+1) \cdot \ell_S(u) + 2 n N
\leq n (2N + 4) + 4.
\]
S tem je $\beta_{G,S}(n(2N + 4) + 4) \geq 2^{n+1}$. Uvedimo $x = n(2N + 4) + 4$, s tem velja
\[
\sqrt[x]{\beta_{G,S}(x)} \geq 2^{\frac{1}{x} \left( \frac{x - 4}{2N + 4} + 1 \right)},
\]
torej v limiti $n \to \infty$ dobimo
\[
\omega_S(G) \geq 2^{\frac{1}{2N+4}} \geq 2^{\frac{1}{2 \log_\Lambda 2 + 4}}.
\]
\end{zgled}

\subsection{Linearne grupe}

V linearnih grupah imamo na voljo Titsovo alternativo, ki pove, da je linearna grupa $G$ bodisi virtualno rešljiva bodisi vsebuje prosto grupo. 

Če linearna grupa vsebuje prosto grupo, je eksponentne rasti. 
Z natančno analizo dokaza Titsove alternative je Breuillard izpeljal {\em enakomerno} različico alternative, ki nam pove, da prosto podgrupo vselej najdemo v {\em omejenem} številu korakov v odvisnosti od dimenzije matrik.

\begin{izrek}{\href{https://arxiv.org/abs/0804.1395}{(Breuillard 2008)}} {\sc (enakomerna Titsova alternativa)}

Za vsak $d \in \NN$ obstaja $N = N(d) \in \NN$, da velja naslednje. Za vsako polje $K$ in vsako končno množico $S \subseteq \GL_d(K)$ z lastnostjo $1 \in S = S^{-1}$ velja:
\vspace{-0.5\baselineskip}
\begin{itemize}[noitemsep]
\item bodisi je $\langle S \rangle$ virtualno rešljiva
\item bodisi množica $S^N = B_{\langle S \rangle,S}(N)$ vsebuje elementa, ki generirata prosto grupo $F_2$. \vspace{-\baselineskip}
\end{itemize}
\end{izrek}

Dokaz je precej zahteven in poleg teorije normiranih polj uporabi orodja iz logike, diofantske geometrije in algebraičnih grup. V celoti ga bomo izpustili. Bomo pa pokazali, kako lahko z njegovo uporabo hitro izpeljemo, da so linearne grupe, ki niso virtualno rešljive, nujno {\em enakomerno} eksponentne rasti, ki je odvisna le od dimenzij matrik.

\begin{posledica}
Za vsak $d \in \NN$ obstaja $\epsilon = \epsilon(d) > 0$, da velja naslednje. Za vsako polje $K$ in vsako končno generirano linearno grupo $G \leq \GL_d(K)$, ki {\em ni} virtualno rešljiva, velja $\Omega(G) \geq 1 + \epsilon$.
\end{posledica}

\begin{dokaz}
Naj bo $G =\langle S \rangle$. Izrek {\sc (enakomerna Titsova alternativa)} zagotovi $N = N(d)$, za katerega obstajata $a,b \in S^N$, da je $\langle a,b \rangle \cong F_2$. Torej je
\[
\omega_S(G) = \lim_{n \to \infty} \sqrt[n]{\beta_{G,S}(n)}
= \lim_{n \to \infty} \sqrt[nN]{\beta_{G,S}(nN)}
\]
Ker je $\beta_{G,S^N}(n) = \beta_{G,S}(nN)$, je zadnja limita enaka
\[
\lim_{n \to \infty} \sqrt[nN]{\beta_{G, S^N}(n)} =
\sqrt[N]{\omega_{S^N}(G)} 
\geq \sqrt[N]{\omega_{\{a,b \}}(G)} 
\geq \sqrt[N]{3} 
> 1+ \epsilon.
\]
\end{dokaz}

Iz tega rezultata lahko s {\em projekcijo} potisnemo eksponentno rast krogel iz neskončnih linearnih grup v končne grupe. 

\begin{zgled}
Opazujmo grupo $\SL_2(\ZZ)$. Izberimo neko njeno generirajočo množico z dvema elementoma, torej $\langle a,b \rangle = \SL_2(\ZZ)$.\footnote{Vzamemo lahko na primer $a = \left( \begin{smallmatrix} 1 & 1 \\ 0 & 1 \end{smallmatrix} \right)$, $b = \left( \begin{smallmatrix} 1 & 0 \\ 1 & 1 \end{smallmatrix} \right)$.} Velja $\omega_{\SL_2(\ZZ), \{a,b\}} \geq 1 + \epsilon$, kjer je $\epsilon > 0$ univerzalna konstanta iz zadnje posledice. Za vsak $n \in \NN$ velja torej
\[
\beta_{\SL_2(\ZZ), \{ a, b \}}(n) \geq c \cdot \left( 1 + \frac{\epsilon}{2} \right)^n
\]
za neko konstanto $c$, ki je lahko odvisna od izbire generirajoče množice.

Naj bo $w \in B_{\SL_2(\ZZ), \{ a, b \}}(n)$. Element $w$ reduciramo po modulu $p$ za neko praštevilo $p$ s preslikavo $\pi \colon \SL_2(\ZZ) \to \SL_2(\ZZ/p\ZZ)$. Naj bo $A$ največja absolutna vrednost vnosov matrik $a,b,a^{-1},b^{-1} \in \SL_2(\ZZ)$. Tedaj je največja absolutna vrednost vnosov matrike $w$ kvečjemu $2^{n-1} A^n$.\footnote{Iz definicije množenja matrik sledi, da je največja absolutna vrednost vnosov matrike $ab$ je kvečjemu $A^2 + A^2 = 2 A^2$, matrike $aba$ kvečjemu $2A^2 \cdot A + 2A^2 \cdot A = 4 A^3$, \dots} Če je torej $2^{n-1} A^n < \frac{p}{2}$, potem redukcija $\pi$ inducira {\em injektivno} preslikavo
\[
B_{\SL_2(\ZZ), \{a,b\}}(n) \to B_{\SL_2(\ZZ/p\ZZ), \{ \pi(a),\pi(b) \}}(n).
\]
Za neko konstanto $c$, odvisno od generirajoče množice, torej za vsak $n \leq c \cdot \log p$ velja
\[
\beta_{\SL_2(\ZZ/p\ZZ), \{ \pi(a), \pi(b) \}}(n) \geq c \cdot \left( 1 + \frac{\epsilon}{2} \right)^n,
\]
od koder sledi
\[
\beta_{\SL_2(\ZZ/p\ZZ), \{ \pi(a), \pi(b) \}}(c \cdot \log p) \geq c \cdot p^{c \log (1 + \epsilon/2)} \geq c^{\prime} \cdot |\SL_2(\ZZ/p\ZZ)|^{c^{\prime}}
\]
za neko konstanto $c^\prime$, odvisno od generirajoče množice, a neodvisno od $p$. V zelo malem številu korakov, primerljivem z $\log |\SL_2(\ZZ/p\ZZ)| \sim \log p$, torej z generatorjema $\pi(a), \pi(b)$ pokrijemo ogromen del grupe $\SL_2(\ZZ/p\ZZ)$.

Na podlagi teh idej je z orodji iz nekomutativne aditivne kombinatorike \href{https://annals.math.princeton.edu/2008/167-2/p06}{(Helfgott 2008)} dokazal, da za vsako generirajočo množico $H$ grupe $\SL_2(\ZZ)$ obstaja konstanta $C > 0$, da za vsako praštevilo $p$ velja 
\[
B_{\SL_2(\ZZ/p\ZZ), \pi(H)}(C \log p) = \SL_2(\ZZ/p\ZZ).
\]
Poenostavljena različica tega dokaza je predstavljena v \href{https://repozitorij.uni-lj.si/IzpisGradiva.php?id=140815&lang=eng}{(Cerar 2022)}.
V le parih korakih, primerljivih z $\log |\SL_2(\ZZ/p\ZZ)|$, lahko torej pokrijemo {\em celotno} grupo $\SL_2(\ZZ/p\ZZ)$. Ta rezultat je bil kasneje ustrezno posplošen na druge generirajoče množice grup $\SL_n(\field_q)$ in je vodil v velik razvoj študija rasti v {\em končnih} grupah.
\end{zgled}

{\vprasanje Kako je pa z rastjo linearne grupe, ki je virtualno rešljiva?}
Če je virtualno rešljiva grupa $G$ eksponentne rasti, potem je tudi njena rešljiva podgrupa $H$ eksponentne rasti, zato je po prejšnjem razdelku $H$ enakomerno eksponentne rasti, torej\footnote{Spomnimo se, da za $H \leq G$, $|G:H|< \infty$, velja implikacija $\Omega(H) > 1 \Rightarrow \Omega(G) > 1$.} je tudi grupa $G$ enakomerno eksponentne rasti.
Ni pa jasno, če lahko tudi za virtualno rešljive linearne grupe eksponentne rasti najdemo spodnjo mejo za njihovo enakomerno rast, ki je odvisna le od dimenzij matrik. Domnevno naj bi to bilo res.

\begin{odprtproblem}{\sc (Breuillardova domneva)}
Za vsak $d \in \NN$ obstaja $\epsilon = \epsilon(d) > 0$, da velja naslednje. Za vsako končno generirano linearno grupo $G \leq \GL_d(\CC)$, ki {\em ni} virtualno {\em nilpotentna}, velja $\Omega(G) \geq 1 + \epsilon$.
\end{odprtproblem}

V zvezi s tem omenimo še sodoben rezultat \href{https://www.repository.cam.ac.uk/handle/1810/287816}{(Breuillard--Varjú 2020)}, kjer avtorja dokažeta, da je zgornja domneva v primeru $G \leq \GL_2(\bar{\QQ})$ {\em ekvivalentna} znani domnevi \href{https://en.wikipedia.org/wiki/Lehmer%27s_conjecture}{(Lehmer 1933)} iz teorije števil. Obe domnevi sta še vedno odprti. 

% \begin{comment}

\section{Bartholdijeva grupa}

Če izstopimo iz sveta rešljivih in linearnih grup, lahko najdemo grupe neenakomerno eksponentne rasti. Tukaj si bomo ogledali en primer take grupe. Prvi primer je konstuiral \href{https://link.springer.com/article/10.1007/s00222-003-0321-8}{(Wilson 2004)}, mi pa si bomo ogledali poenostavljeno\footnote{Kot bomo videli, je poenostavljena različica sicer elementarna, vsekakor pa ni preprosta.} različico tega, ki jo je odkril \href{https://www.sciencedirect.com/science/article/pii/S1631073X03001316}{(Bartholdi 2003)}.

\subsection{Konstrukcija}

Naj bo $A = \GL_3(\ZZ/2\ZZ)$, grupa $3 \times 3$ matrik nad končnim poljem $\ZZ/2\ZZ$. Grupa $A$ naravno deluje na neničelnih vektorjih v prostoru $(\ZZ/2\ZZ)^3$, ki jih označimo na naslednji način:
\[
\one = \begin{pmatrix} 1 \\ 0 \\ 0\end{pmatrix}, \quad 
\two = \begin{pmatrix} 0 \\ 1 \\ 0\end{pmatrix}, \quad
\three = \begin{pmatrix} 1 \\ 1 \\ 0\end{pmatrix}, \quad
\four = \begin{pmatrix} 0 \\ 0 \\ 1\end{pmatrix}, \quad 
\five = \begin{pmatrix} 1 \\ 0 \\ 1\end{pmatrix}, \quad 
\six = \begin{pmatrix} 0 \\ 1 \\ 1\end{pmatrix}, \quad 
\seven = \begin{pmatrix} 1 \\ 1 \\ 1\end{pmatrix}.
\]
Te vektorje lahko uredimo v Fanovo ravnino, kjer enakoležnost vektorjev na neki premici v tej ravnini pomeni, da so ti vektorji linearno odvisni.
Delovanje grupe $A$ ohranja premice v tej ravnini. 

\begin{figure}[t]
\centering
\begin{tikzpicture}[scale = 5]
\tikzstyle{vertex}=[circle, fill = oranzna, minimum size = 5pt, inner sep = 0pt]

\draw[thick]
  (0,0) coordinate (1) --
  (1,0) coordinate (2) --
  ($ (1)!.5!(2) ! {sin(60)*2} ! 90:(2) $) coordinate (4) -- cycle;
\coordinate (7) at
  (barycentric cs:1=1,2=1,4=1);
\draw[thick] (7) circle [radius=1.717/6];
\draw[thick] (4) -- ($ (1)!.5!(2) $) coordinate (3); 
\draw[thick] (1) -- ($ (2)!.5!(4) $) coordinate (6); 
\draw[thick] (2) -- ($ (4)!.5!(1) $) coordinate (5); 

\node[vertex, label = below left : $\one$] at (1) {};
\node[vertex, label = below right : $\two$] at (2) {};
\node[vertex, label = below : $\three$] at (3) {};
\node[vertex, label = above : $\four$] at (4) {};
\node[vertex, label = above left : $\five$] at (5) {};
\node[vertex, label = above right : $\six$] at (6) {};
\node[vertex, label = 80: $\seven$] at (7) {};

\draw[<->, thick, draw=vijolicna, shorten >= 5pt, shorten <= 5pt] (1) to [bend left = 30] node [midway, above left] {\color{vijolicna} $x$} (5);
\draw[<->, thick, draw=vijolicna, shorten >= 5pt, shorten <= 5pt] (3) to [bend left = 30] (7);

\draw[<->, thick, draw=rdeca, shorten >= 5pt, shorten <= 5pt] (2) to [bend left = 30] node [midway, below] {\color{rdeca} $y$} (3);
\draw[<->, thick, draw=rdeca, shorten >= 5pt, shorten <= 5pt] (6) to [bend left = 30] (7);

\draw[<->, thick, draw=modra, shorten >= 5pt, shorten <= 5pt] (4) to [bend left = 30] node [midway, above right] {\color{modra} $z$} (6);
\draw[<->, thick, draw=modra, shorten >= 5pt, shorten <= 5pt] (5) to [bend left = 30] (7);
\end{tikzpicture}
\caption{Fanova ravnina z delovanji elementov $x,y,z \in A$}
\end{figure}

Opazujmo naslednje elemente grupe $A$:
\[
x = (\one \ \five) (\three \ \seven), \quad
y = (\two \ \three) (\six \ \seven), \quad
z = (\four \ \six) (\five \ \seven).
\]
Ni težko preveriti, da je $A = \langle x, y, z \rangle$.\footnote{Za poljubno matriko $a \in A$ lahko najdemo besedo $w$ v $x,y,z$, da bo veljalo $w \cdot a.\one = \one$. Po tem lahko najdemo še besedo $w^\prime$, da bo veljalo $w^\prime \cdot w \cdot a . \one = \one$ in $w^\prime \cdot w \cdot a . \two = \two$. Nazadnje lahko najdemo še besedo $w^{\prime \prime}$, da $w^{\prime \prime} \cdot w^\prime \cdot w \cdot a$ fiksira vse točke. Torej je $a = (w^{\prime \prime} \cdot w^\prime \cdot w)^{-1}$.}

Naj bo $P = \{ \one, \two, \dots, \seven \} = (\ZZ/2\ZZ)^3 \backslash \{ 0 \}$. Naj bo $P^*$ množica vseh končnih zaporedij točk iz $P$. Grupa $A$ deluje {\em z desne} na $P^*$ na naslednja dva načina:
\begin{description}
    \item[1. način:] Naj bo $a \in A$. Definirajmo $(p_1 p_2 \cdots p_n).a = (p_1.a)p_2\cdots p_n$.\footnote{Element $a$ torej deluje na zaporedju tako, da deluje le na prvi točki.}
    \item[2. način:] Naj bo $a \in A$. Definirajmo $(p_1 p_2 \cdots p_m p_{m+1} \cdots p_n).a$ kot
    \[
    \begin{cases}
    p_1 \cdots p_m (p_{m+1}.a) p_{m+2} \cdots p_n; & p_1 = \cdots = p_{m-1} = \one, \ p_m = \two \\
    p_1 \cdots p_n; & \text{sicer.}
    \end{cases}
    \]
\end{description}
Naj bo $\bar A$ kopija grupe $A$. Naj $A$ deluje na $P^*$ na 1. način, $\bar A$ pa na 2. način. Imamo torej vložitvi $\varphi_1 \colon A \to \Sym(P^*)$ in $\varphi_2 \colon \bar A \to \Sym(P^*)$. {\definicija Bartholdijeva grupa} je $W = \langle \image \varphi_1, \image \varphi_2 \rangle \leq \Sym(P^*)$.  

\begin{izrek}
Grupa $W$ je eksponentne rasti, ni pa enakomerno eksponentne rasti.
\end{izrek}

Predstavili bomo le glavne poteze dokaza. Kot pri Grigorčukovi grupi dokaz sloni na samopodobnosti grupe $W$, a na nekoliko drugačen način.

\subsection{Samopodobnost}

Naj bo $G$ poljubna grupa. Opazujmo množico preslikav $P \to G$, ki jo lahko identificiramo z $G^7$. Ker $A$ deluje na $P$, dobimo inducirano delovanje $A$ na $G^7$ z desne. Torej lahko tvorimo semidirektni produkt $G^7 \rtimes A$, ki ga krajše označimo z $G \wreath A$. Elemente tega semidirektnega produkta zaradi lažje preglednosti pišemo kot ${\textbf [} g_1, \dots, g_7 {\textbf ]} a$ za $g_i \in G, a \in A$.

Kadar grupa $G$ ni čisto poljubna, temveč celo deluje na $P^*$, dobimo inducirano delovanje grupe $G \wreath A$ na $P^*$ na naslednji način:
\[
(p_1p_2 \cdots p_n).{\textbf [} g_1, \dots, g_7 {\textbf ]} a = (p_1.a) \left( (p_2p_3\cdots p_n).g_{p_1} \right).
\]
V tem smislu lahko vidimo grupo $G \wreath A$ kot podgrupo $\Sym(P^*)$.

Vzemimo zdaj $G = W$. Torej je $W \wreath A \leq \Sym(P^*)$, hkrati pa je $W \leq \Sym(P^*)$ prek delovanj $\varphi_1, \varphi_2$.


\begin{zgled}
Za $a \in A$ velja
\[
(p_1 p_2 \cdots p_n).{\textbf [} \bar a, a, 1, 1, 1, 1, 1 {\textbf ]} =
\begin{cases}
p_1 \cdots p_n; & p_1 \neq \one, \two \\
p_1 (p_2.a) p_3 \cdots p_n; & p_1 = \two \\
p_1 (p_2 \cdots p_n.\bar a); & p_1 = \one.
\end{cases}
\]
Torej element ${\textbf [} \bar a, a, 1, 1, 1, 1, 1 {\textbf ]}$ deluje kot $\bar a$ na $P^*$. Znotraj grupe $\Sym(P^*)$ lahko torej identificiramo $\bar a = {\textbf [} \bar a, a, 1, 1, 1, 1, 1 {\textbf ]}$.
\end{zgled}


\begin{trditev}
Identifikaciji $W$ in $W \wreath A$ kot podgrupi $\Sym(P^*)$ inducirata izomorfizem $W \cong W \wreath A$.
\end{trditev}

\begin{dokaz}
Obravnavajmo $W$ in $W \wreath A$ kot podgrupi $\Sym(P^*)$. Iz zadnjega zgleda je razvidno, da velja $\varphi_1(A), \varphi_2(\bar A) \leq W \wreath A$, zato je $W \leq W \wreath A$. Dokazati moramo še obratno vsebovanost. 

Izberimo poljubna $a,b \in A$. Tedaj je $[\bar a, \bar b^y] \in \langle A, \bar A \rangle = W$. Velja
\[
(p_1p_2\cdots p_n).[\bar a, \bar b^y] = (p_1 \cdots p_n).\bar a^{-1} \cdot y^{-1} \bar b^{-1} y \cdot \bar a \cdot y^{-1} \bar b y.
\]
Zadnji element je enak
\[
\begin{cases}
\one \cdots \one \two (p_{m+1}.a^{-1})p_{m+2}\cdots p_n. y^{-1} \bar b^{-1} y \bar a y^{-1} \bar b y; & p_1 = \cdots = p_{m-1} = \one, \ p_m = \two \\
(p_1\cdots p_n).y^{-1} \bar b^{-1} y \cdot \bar a \cdot y^{-1} \bar b y; & \text{sicer,}
\end{cases}
\]
kar se poenostavi do
\[
\begin{cases}
\one \cdots \one \two (p_{m+1}.[a,b])p_{m+2}\cdots p_n; & p_1 = \cdots = p_{m-1} = \one, \ p_m = \two, \ m > 1 \\
\two p_2 p_3 \cdots p_n; & p_1 = \two \\
p_1\cdots p_n; & \text{sicer.}
\end{cases}
\]
Torej je ${\textbf [} [\bar a, \bar b], 1, 1, 1, 1, 1, 1 {\textbf ]} = [\bar a, \bar b^y] \in W$. Ni težko preveriti, da velja $A = [A,A]$, zato od tod izpeljemo ${\textbf [} \bar A, 1, 1, 1, 1, 1, 1 {\textbf ]} \leq W$. Podobno z opazovanjem elementa $[\bar a, \bar b^x]$ dobimo vsebovanost ${\textbf [} 1, A, 1, 1, 1, 1, 1 {\textbf ]} \leq W$.

Naj bo $t \in A$ in označimo $p = t.\one \in P$. Tedaj je
\[
(p p_2 \cdots p_n).t^{-1} {\textbf [} \bar a, 1, 1, 1, 1, 1, 1 {\textbf ]} t
= \one (p_2 \cdots p_n.\bar a).t
= p(p_2 \cdots p_n.\bar a), 
\]
torej je $t^{-1} {\textbf [} \bar a,  1, 1, 1, 1, 1, 1 {\textbf ]} t = {\textbf [} 1, \dots, 1, \bar a, 1, \dots, 1 {\textbf ]}$, kjer v zadnjem elementu $\bar a$ stoji na poziciji $p$.

S konjugiranjem z ustreznim elementom $A$ torej izpeljemo, da $W$ vsebuje ${\textbf [} 1, \dots, 1, A, 1, \dots, 1 {\textbf ]}$ in ${\textbf [} 1, \dots, 1, \bar A, 1, \dots, 1 {\textbf ]}$, od koder sledi, da $W$ vsebuje celo ${\textbf [} W, W, \dots, W {\textbf ]}$. Ker seveda $W$ vsebuje tudi $A$, nazadnje izpeljemo želeno vsebovanost ${\textbf [} W, W, \dots, W {\textbf ]} \rtimes A = W \wreath A \leq W$.
\end{dokaz}


\subsection{Generatorji}

Naštejmo par lastnosti glede generatorjev Bartholdijeve grupe.\footnote{Preverjanje teh lastnosti je tehnično zahtevno, a konceptualno podobno premisleku $W \cong W \wreath A$, zato dokaze izpustimo.} Prva lastnost je ta, da je Bartholdijeva grupa generirana s tremi elementi reda $2$, ki jih dobimo iz generatorjev $x,y,z$ grupe $A$. Naslednja lastnost pa nam pove, da lahko iz generatorjev poljubne grupe $G$ dobimo generatorje grupe $G \wreath A$.

\begin{trditev}
\vspace{-0.5\baselineskip}
\begin{enumerate}[noitemsep]
\item Grupa $W$ je generirana z naslednjimi tremi elementi reda $2$:
\[
\begin{aligned}
a &= {\textbf [} 1, \bar x, 1, x ,1 ,1, 1 {\textbf ]} x, \\
b &= {\textbf [} y, 1, 1, \bar y, 1, 1, 1 {\textbf ]} y, \\
c &= {\textbf [} \bar z, z, 1, 1, 1, 1, 1 {\textbf ]} z.
\end{aligned}
\]

    \item Če je grupa $G$ generirana z elementi $a,b,c$ reda $2$ in $G = [G,G]$, potem je grupa $G \wreath A$ generirana z naslednjimi tremi elementi reda $2$:
\[
\begin{aligned}
a^\prime &= {\textbf [} 1, 1, 1, a ,1, 1, 1 {\textbf ]} x, \\
b^\prime &= {\textbf [} b, 1, 1, 1, 1, 1, 1 {\textbf ]} y, \\
c^\prime &= {\textbf [} 1, c, 1, 1, 1, 1, 1 {\textbf ]} z.
\end{aligned}
\]
\vspace{-2\baselineskip}
\end{enumerate}
\end{trditev}

\subsection{Rast}

Preverimo najprej, da je Bartholdijeva grupa eksponentne rasti.

\begin{trditev}
$W$ je eksponentne rasti.
\end{trditev}

\begin{dokaz}[Ideja dokaza]
Trdimo, da grupa $W$ vsebuje prosti monoid na dveh generatorjih.
Izberimo $u,v \in A$, $u \neq v$, z lastnostjo $\one.u = \one.v = \two$, $\two.u = \two.v = \one$. Naj bo $a = \bar u u$ in $b = \bar v v$. Tedaj $a$ in $b$ generirata prosti monoid, saj nobena netrivialna beseda v $a,b$ ne deluje trivialno na $P^*$. Dokaz tega dejstva izpustimo.
\end{dokaz}

Zdaj bi radi premislili še, da $W$ ni enakomerno eksponentne rasti. Iz prejšnjega razdelka lahko iterativno dobimo tri generatorje grupe $(((W \wreath A) \wreath A ) \cdots ) \wreath A$, ki je zaradi samopodobnosti izomorfna grupi $W$. Pri tem postopku se stopnja rasti glede na dobljene generatorje hitro manjša.

\begin{trditev}
Naj grupa $G$ deluje na $P$. Predpostavimo, da je $G$ generirana z elementi $a,b,c$ reda $2$. Naj bo $S = \{ a,b,c \}$ in naj bo $S^\prime = \{ a^\prime, b^\prime, c^\prime \}$ iz druge trditve prejšnjega razdelka. Tedaj velja
\[
\omega_{S^\prime}(G \wreath A) \leq \inf_{\eta \in (0,1)} \max \left\{ \omega_S(G)^{1-\eta}, \frac{30^\eta}{\eta^\eta (1-\eta)^{1-\eta}} \right\}.
\]\vspace{-\baselineskip}
\end{trditev}

\begin{dokaz}[Ideja dokaza]
Naj bo $w$ beseda s črkami v $S^\prime$. Predstavimo jo v obliki ${\textbf [} w_1, \dots, w_7 {\textbf ]} \sigma$ za neke besede $w_i$ s črkami v $S$ in $\sigma \in \Sym(P)$. Podobno kot pri Grigorčukovi grupi zdaj analiziramo, kakšne so dolžine besed $w_i$ v odvisnosti od strukture $w$ in kakšna krajšanja se zgodijo v vsaki od besed $w_i$.
\end{dokaz}

Nazadnje lahko izpeljemo, da Bartholdijeva grupa {\em ni} enakomerno eksponentne rasti.

\begin{dokaz}[Dokaz $\Omega(W) = 1$]
Naj bo $S_0 = \{ a, b, c \}$ generirajoča množica grupe $W$ iz prve trditve prejšnjega razdelka. Ker so vsi elementi $S_0$ reda $2$, je $\beta_{W,S_0}(n) \leq 3 \cdot 2^{n-1}$, zato je $\omega_{S_0}(W) \leq 2$.

Naj bo $S_{i+1} = S_i^\prime$ za $i \geq 0$, kjer uporabljamo drugo lastnost iz trditve v zadnjem razdelku. Velja
\[
\langle S_0 \rangle = W, \ \langle S_1 \rangle = W \wreath A \cong W, \ \langle S_2 \rangle = (W \wreath A) \wreath A \cong W \wreath A \cong W, \dots
\]
Po ustrezni identifikaciji v $\Sym(P^*)$ lahko torej množice $S_i$ vidimo kot generirajoče množice grupe $W$.

Naj bo $\Lambda_0 = 2$. Za $i \geq 0$ naj bo $\eta_i \in (0, \frac12)$ enolična rešitev enačbe
\[
\frac{30^{\eta_i}}{\eta_i^{\eta_i} (1 - \eta_i)^{1 - \eta_i}} = \Lambda_i^{1 - \eta_i}.
\]
Za tem definirajmo še $\Lambda_{i+1} = \Lambda_i^{1 - \eta_i} > 1$. Ni težko preveriti, da za vsak $i$ velja $0 < \eta_{i+1} < \eta_i$, zato obstaja limita $\eta = \lim_{i \to \infty} \eta_i \geq 0$. Podobno za vsak $i$ velja $1 < \Lambda_{i+1} < \Lambda_i$, zato obstaja limita $\Lambda = \lim_{i \to \infty} \Lambda_i \geq 1$. 

Iz definicije $\eta_i$ in $\Lambda_i$ sledi enakost $\Lambda^{1-\eta} = \Lambda$, kar implicira $\Lambda = 1$ ali $\eta = 0$. V slednjem primeru sledi $\Lambda = 1$, torej v obeh primerih sklepamo $\Lambda = 1$. Hkrati po zadnji trditvi za vsak $i$ velja neenakost $\omega_{S_i}(W) \leq \Lambda_i$. Od tod sledi $\Omega(W) \leq \lim_{i \to \infty} \omega_{S_i}(W) = 1$. Dokaz je zaključen.
\end{dokaz}

\end{document}
